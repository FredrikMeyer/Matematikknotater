\documentclass[11pt, english]{article}
%\usepackage[latin1]{inputenc}
\usepackage[T1]{fontenc}
\usepackage[utf8]{inputenc}
\usepackage[english]{babel}   % S P R A A K


% \usepackage{graphicx}    % postscript graphics
\usepackage{amssymb, amsmath, amsthm, amssymb} % symboler, osv
\usepackage{mathrsfs}
\usepackage{url}
\usepackage{thmtools}
\usepackage{enumerate}  % lister $  
\usepackage{float}
\usepackage{tikz}
\usetikzlibrary{calc}
\usepackage[all]{xy}   % for comm.diagram
\usepackage{wrapfig} % for float right
\usepackage{hyperref}
\usepackage{mystyle} % stilfilen      

\begin{document}
\title{Triangulations of $\C \PP^2$ and deformations}
\author{Fredrik Meyer}
\maketitle 

\section{Approach 1}

Let $X \subseteq \PP^5$ we a smooth pfaffian cubic hypersurface, and let $F_1(X)$ be its Fano variety of lines. The latter is a reducible subvariety of the Grassmannian $\Gr(2,6)$.

If we assume that $X$ is smooth, then it is known [[ref]] that $F_1(X)$ is deformation equivalent to the Hilbert scheme $S^{[2]}$ of pairs of points on $S$, where $S$ is a K3 surface of degree $14$ in $\PP^8$. By standard theory, a Stanley-Reisner degeneration of a K3 surface is a triangulated sphere, or just $\PP_\C^1$. Thus, a Stanley-Reisner degeneration of $S^{[2]}$ should give a triangulation of  $\PP^1_\C \ast \PP_\C^1 \approx \PP^2_\C$, the complex projective plane, as topological spaces.

Since $F_1(X)$ is embedded in $\PP^{14}$ as a closed subscheme of the Grassmannian, finding such a triangulation is equivalent to finding a square-free initial ideal of the ideal of this embedding. 

Since smooth hypersurfaces have too computationally consuming equations, we start naïvely with a singular hypersurface $X =V(f)= V(x_0x_2x_4-x_1x_3x_5)$. This is both a toric variety and a pfaffian hypersurface. However, because of the form of $f$, $F_1(X)$ is reducible:

\begin{prop}
If $X=V(x_0x_2x_4-x_1x_3x_5)$, the variety of lines $F_1(X)$ is reducible. It decomposes into $15$ components: $9$ of them are copies of $\Gr(2,4)$, and the $6$ other are toric varieties, all of them isomorphic.
\end{prop}
\begin{proof}
The Grassmannian components are easy to see: if one of the even numbered coordinates are zero, then one of the odd must be too. Thus $X$ contains $3 \times 3$ special copies of $\PP^3$. Lines in $\PP^3$ are parametrized by Grassmannian $\Gr(2,4)$. This have dimension $4$. 

The other components are not that easy: Consider the  rational map $\psi:\PP^2_{abc} \times \PP^2_{xyz} \rmap X$ given by sending $(a:b:c) \times (x:y:z)$ to $(ax:bx:by:cy:cz:az)$. Then this map is well-defined outside the points $(1:0:0) \times (0:1:0)$, $(0:1:0)\times (0:0:1)$ and $(0:0:1)\times(1:0:0)$. Resolving the locuf of indeterminacy give an isomorphism of a desingularization of $X$ with the blowup of $\PP^2 \times \PP^2$ in $3$ points. There are six ways of choosing the map $\psi$, which gives the six different toric components of $X$. 

Since the map is of bidegree $(1,1)$, we have a well-defined induced map $(\PP^2)^\wedge \times \PP^2 \rmap F_1(X)$ given by
\[
\ell \times p \mapsto \ell_p := \big[\psi(x,p),\psi(y,p)\big]_{x,y \in \ell},
\]
where $\ell_p$ denotes the line connecting $\psi(x,p)$ and $\psi(y,p)$ in $\PP^5$. This gives one of the components in $F_1(X)$.


That these are all components can either be checked on a computer, or by degree considerations. [[unsure of the latter]]
\end{proof}

The equations of $F_1(X)$ inside $\PP^{14}$ are rather messy. The embedding is of degree $108$ and there are $70$ equations ($15$ quadrics corresponding to the inclusion in $\Gr(2,6)$ and $55$ cubics, of which $6$ are monomials). 

However, the Hilbert polynomial is $h(t) =\frac 92 t^4 + \frac {15}2 i^2  + 3$, giving a topological Euler characteristic of $3$, which agrees with the Euler characteristic of (the topological space) $\C \PP^2$.

Thus, it is at least not totally unreasonable to predict that a Stanley-Reisner degeneration of $F_1(X)$ would correspond to a triangulation of $ \C \PP ^ 2$. 


\section{Bad er good news}

The article [[ref]] presents a triangulation $T$ of $\C \PP^2$ having $15$ vertices, \emph{and} the same $f$-vector as the triangulation we're looking for. This gives us a Stanley-Reisner scheme $X_T$ living in the same Hilbert scheme as $F_1(X)$. [[see the reference (for now) for a description]]

Now, the Hilbert scheme is a terribly lonely place, so it could be that this new triangulation lives in a completely different component than the one $F_1(X)$ lives in. However, if they do - we could hope for one particularly nice situation: $F_1(X)$ degenerates to $X_T$, and we would have another ``algebro-geometric'' explanation of the existence of $T$.

One way to find a degeneration is to try to deform $X_T$. First: $\CC \PP^2$ have a cell structure consisting of three $4$-cells, where their intersections are $3$ different filled tori. The triple intersection is a $2$-torus. Here's a surprising result:

\begin{prop}
 There exists a deformation of $X_T$, the Stanley-Reisner scheme of the triangulation of $T$, to the union of three toric varieties. Their pairwise intersections are Stanley-Reisner schemes of affine dimension $4$, corresponding to the product of two tori. [[... FILL IN THIS...]] 
\end{prop}

The question is now of course if this union of toric varieties is smoothable, and if so, does it smooth to $F_1(X)$? 

Here is some information about the equations of the deformation: its degree is of course $108$. The ideal is generated by $37$ elements, a lot less than the $70$ of $F_1(X)$. There are $15$ quadrics (as with $F_1(X)$) and $22$ cubics. The natural question now is of course: what can we say about the number of defining equations for deformation equivalent objects?

\begin{remark}
 The quadric equations have four terms, whereas the Grassmannian terms in $F_1(X)$ have three terms. Unless there is some change of coordinate to decrease the number of terms, there is no way (?) this can deform to the Grassmannian.
\end{remark}
\end{document}
