\documentclass[11pt, english]{article}
%\usepackage[latin1]{inputenc}
\usepackage[T1]{fontenc}
\usepackage[utf8]{inputenc}
\usepackage[english]{babel}   % S P R A A K


% \usepackage{graphicx}    % postscript graphics
\usepackage{amssymb, amsmath, amsthm, amssymb} % symboler, osv
\usepackage{mathrsfs}
\usepackage{url}
\usepackage{thmtools}
\usepackage{enumerate}  % lister $  
\usepackage{float}
\usepackage{tikz}
\usetikzlibrary{calc}
\usepackage[all]{xy}   % for comm.diagram
\usepackage{wrapfig} % for float right
\usepackage{hyperref}
\usepackage{mystyle} % stilfilen      

\begin{document}
\title{Triangulations of $\C \PP^2$ and deformations}
\author{Fredrik Meyer}
\maketitle 

\section{Approach 1}

Let $X \subseteq \PP^5$ we a smooth pfaffian cubic hypersurface, and let $F_1(X)$ be its Fano variety of lines. The latter is a subvariety of the Grassmannian $\Gr(2,6)$.

If we assume that $X$ is smooth, then it is known \cite{cp2_beauville} that $F_1(X)$ is deformation equivalent to the Hilbert scheme $S^{[2]}$ of pairs of points on $S$, where $S$ is a K3 surface of degree $14$ in $\PP^8$. By standard theory, a Stanley-Reisner degeneration of a K3 surface is a triangulated sphere, or just $\PP_\C^1$. Thus, a Stanley-Reisner degeneration of $S^{[2]}$ should give a triangulation of  $\PP^1_\C \ast \PP_\C^1 \approx \PP^2_\C$, the complex projective plane, as topological spaces.

Since $F_1(X)$ is embedded in $\PP^{14}$ as a closed subscheme of the Grassmannian, finding such a triangulation is equivalent to finding a square-free initial ideal of the ideal of this embedding. 

Since smooth hypersurfaces have too computationally consuming equations, we start naïvely with a singular hypersurface $X =V(f)= V(x_0x_2x_4-x_1x_3x_5)$. This happens to be both a toric variety and a pfaffian hypersurface. However, because of the form of $f$, $F_1(X)$ is reducible:

\begin{prop}
If $X=V(x_0x_2x_4-x_1x_3x_5)$, the variety of lines $F_1(X)$ is reducible. It decomposes into $15$ components: $9$ of them are copies of $\Gr(2,4)$, and the $6$ other are toric varieties, all of them isomorphic.
\end{prop}
\begin{proof}
The Grassmannian components are easy to see: if one of the even numbered coordinates are zero, then one of the odd must be too. Thus $X$ contains $3 \times 3$ special copies of $\PP^3$. Lines in $\PP^3$ are parametrized by Grassmannian $\Gr(2,4)$. This have dimension $4$. 

The other components are not that easy: Consider the  rational map $\psi:\PP^2_{abc} \times \PP^2_{xyz} \rmap X$ given by sending $(a:b:c) \times (x:y:z)$ to $(ax:bx:by:cy:cz:az)$. Then this map is well-defined outside the points $(1:0:0) \times (0:1:0)$, $(0:1:0)\times (0:0:1)$ and $(0:0:1)\times(1:0:0)$. Resolving the locus of indeterminacy give an isomorphism of a desingularization of $X$ with the blowup of $\PP^2 \times \PP^2$ in $3$ points. There are six ways of choosing the map $\psi$.

Since the map is of bidegree $(1,1)$, we have a well-defined induced map $(\PP^2)^\vee \times \PP^2 \rmap F_1(X)$ given by
\[
\ell \times p \mapsto \ell_p := \big[\psi(x,p),\psi(y,p)\big]_{x,y \in \ell},
\]
where $\ell_p$ denotes the line connecting $\psi(x,p)$ and $\psi(y,p)$ in $\PP^5$. This gives one of the components in $F_1(X)$.


That these are all components can either be checked on a computer, or by degree considerations. [[unsure of the latter]]
\end{proof}

The equations of $F_1(X)$ inside $\PP^{14}$ are rather messy. The embedding is of degree $108$ and there are $70$ equations ($15$ quadrics corresponding to the inclusion in $\Gr(2,6)$ and $55$ cubics, of which $6$ are monomials). 

However, the Hilbert polynomial is $h(t) =\frac 92 t^4 + \frac {15}2 i^2  + 3$, giving a topological Euler characteristic of $3$, which agrees with the Euler characteristic of (the topological space) $\C \PP^2$.

Thus, it is at least not totally unreasonable to predict that a Stanley-Reisner degeneration of $F_1(X)$ would correspond to a triangulation of $ \C \PP ^ 2$. 


\section{Bad er good news}

The article \cite{gaifullin_triangulation} presents a triangulation $T$ of $\C \PP^2$ having $15$ vertices, \emph{and} the same $f$-vector as the triangulation we're looking for. This gives us a Stanley-Reisner scheme $X_T$ living in the same Hilbert scheme as $F_1(X)$.

Now, the Hilbert scheme is a terribly lonely place, so it could be that this new triangulation lives in a completely different component than the one $F_1(X)$ lives in. However, if they do -- we could hope for one particularly nice situation: $F_1(X)$ degenerates to $X_T$, and we would have another ``algebro-geometric'' explanation of the existence of $T$.

One way to find a degeneration is to try to deform $X_T$.

\begin{remark}
The complex projective plane $\C \PP^2$ can be divided into three four-dimensional cells as follows:
\begin{align*}
  B_1 &= \{ (z_1:z_2:z_3) \mid \lvert z_1 \rvert \geq \lvert z_2 \rvert, \lvert z_1 \rvert \geq  \lvert z_3 \rvert \} \\
  B_2 &= \{ (z_1:z_2:z_3) \mid \lvert z_2 \rvert \geq \lvert z_1 \rvert, \lvert z_2 \rvert \geq  \lvert z_3 \rvert \} \\
  B_3 &= \{ (z_1:z_2:z_3) \mid \lvert z_3 \rvert \geq \lvert z_1 \rvert, \lvert z_3 \rvert \geq  \lvert z_2 \rvert \}.
\end{align*}
The intersections $B_i \cap B_j$ are solid tori, and the intersection $B_1 \cap B_2 \cap B_3$ is a $2$-dimensional torus. A triangulation of $\C \PP^2$ is an \emph{equilibrium triangulation} if the complexes $B_i$ are subcomplexes of the triangulation. See the introduction of \cite{gaifullin_triangulation} for details and references therein.
\end{remark} 

Here's a surprising result. Recall that (nice) toric varieties degenerates to Stanley-Reisner spheres: their degenerations are triangulations of polytopes. 

\begin{prop}
\label{defT}
 There exists a deformation of $X_T$, the Stanley-Reisner scheme of the triangulation of $T$, to the union of three toric varieties $T_1,T_2,T_3$. Their intersections $T_{ij}$ are the Stanley-Reisner schemes of solid tori, and the triple intersection $T_{123}$ is a solid torus. 
\end{prop}
Thus it seems that this particular triangulation somehow remembers the cell structure.

The question is now of course if this union of toric varieties is smoothable, and if so, does it smooth to $F_1(X)$? It could of course be that if it smooths, it smooths to $F_1(Y)$ for some other cubic hypersurface. To answer the question of the existence of a smoothing, it would be nice to be able to calcute $T^1$ of the deformed $X_T$, but that requires other methods than by computer. 

Here is some information about the equations of the deformation: its degree is of course $108$. The ideal is generated by $37$ elements, a lot less than the $70$ of $F_1(X)$. There are $15$ quadrics (as with $F_1(X)$) and $22$ cubics. The natural question now is of course: what can we say about the number of defining equations for deformation equivalent objects?

\begin{remark}
 The quadratic equations have four terms, whereas the Grassmannian terms in $F_1(X)$ have three terms. Unless there is some change of coordinate to decrease the number of terms, there is no way (?) this can deform to the Grassmannian.
\end{remark}

\section{Calabi-Yau stuff}

The triangulation $T$ of $\C \PP^2$ above  have three special vertices. Their links are joins of hectagons, $D_6 \ast D_6$, having their own Stanley-Reisner schemes. In particular, by a standard result in Stanley-Reisner theory, we have that the homogeneous localization of $\PP(T)$ at $x_0$ (one of the special vertices) correspond to the Stanley-Reisner scheme of $(D_6 \ast D_6) \ast \{ x_0 \}$. In particular, $\PP(D_6 \ast D_6)$ is embedded as a hyperplane cut in $\PP(T)_{x_0}$. 

Now, it turns out the whole deformation from Proposition \ref{defT} is reducible, and the decomposition respects the localization above. [[to be made more precise]] In particular, it induces a deformation of $\PP(T)_{x_0}$ into one of the three toric components. This is great news, because that in turn induces a deformation of $\PP(D_6 \ast D_6)$.

\begin{lemma}
There is a deformation $Y$ of $\PP(D_6 \ast D_6)$ to a hyperplane cut in a toric variety, which is a deformation of $\PP(D_6 \ast D_6 \ast \{x_0 \})$.

In fact, $Y$ has isolated singularities.  
\end{lemma}

Now, $D_6 \ast D_6$ is a sphere, an by general results, it follows that $Y$ is in fact a Calabi-Yau variety. This is often called the \emph{Batyrev-Borisov construction}.

It can be checked that this give rise to a Calabi-Yau variety with Hodge numbers $44,8$, giving an Euler characteristic of $2(44-8)=72$. 

\bibliographystyle{plain}
\bibliography{bibliografi} 

\end{document}
