\documentclass[11pt, norsk]{article}
%\usepackage[latin1]{inputenc}
\usepackage[T1]{fontenc}
\usepackage[utf8x]{inputenc}
\usepackage[norsk]{babel}   % S P R A A K
% \usepackage{graphicx}    % postscript graphics
\usepackage{amssymb, amsmath, amsthm, amssymb} % symboler, osv
\usepackage{mathrsfs}
\usepackage{url}
\usepackage{thmtools}
\usepackage{enumerate}  % lister $  
\usepackage{float}
\usepackage{tikz}
\usepackage{tikz-cd}
\usetikzlibrary{calc}
%\usepackage{tikz-3dplot}
\usepackage{subcaption}
\usepackage[all]{xy}   % for comm.diagram
\usepackage{wrapfig} % for float right
\usepackage{hyperref}
\usepackage{mystyle} % stilfilen      
\usepackage{booktabs}
\usepackage{resizegather}
\setcounter{MaxMatrixCols}{48}
\usepackage{listings}
\lstset{language=Macaulay2}
\definecolor{mygreen}{rgb}{0,0.6,0}
\definecolor{mygray}{rgb}{0.5,0.5,0.5}
\definecolor{mymauve}{rgb}{0.58,0,0.82}

\lstset{ %
  backgroundcolor=\color{white},   % choose the background color; you must add \usepackage{color} or \usepackage{xcolor}
  basicstyle=\footnotesize,        % the size of the fonts that are used for the code
  breakatwhitespace=false,         % sets if automatic breaks should only happen at whitespace
  breaklines=true,                 % sets automatic line breaking
  captionpos=b,                    % sets the caption-position to bottom
  commentstyle=\color{mygreen},    % comment style
  deletekeywords={...},            % if you want to delete keywords from the given language
  escapeinside={\%*}{*)},          % if you want to add LaTeX within your code
  extendedchars=true,              % lets you use non-ASCII characters; for 8-bits encodings only, does not work with UTF-8
  frame=single,	                   % adds a frame around the code
  keepspaces=true,                 % keeps spaces in text, useful for keeping indentation of code (possibly needs columns=flexible)
  keywordstyle=\color{blue},       % keyword style
  language=Macaulay2,                 % the language of the code
  otherkeywords={*,...},           % if you want to add more keywords to the set
  numbers=left,                    % where to put the line-numbers; possible values are (none, left, right)
  numbersep=5pt,                   % how far the line-numbers are from the code
  numberstyle=\tiny\color{mygray}, % the style that is used for the line-numbers
  rulecolor=\color{black},         % if not set, the frame-color may be changed on line-breaks within not-black text (e.g. comments (green here))
  showspaces=false,                % show spaces everywhere adding particular underscores; it overrides 'showstringspaces'
  showstringspaces=false,          % underline spaces within strings only
  showtabs=false,                  % show tabs within strings adding particular underscores
  stepnumber=2,                    % the step between two line-numbers. If it's 1, each line will be numbered
  stringstyle=\color{mymauve},     % string literal style
  tabsize=2,	                   % sets default tabsize to 2 spaces
  title=\lstname                   % show the filename of files included with \lstinputlisting; also try caption instead of title
}

%\usepackage[a5paper,margin=0.5in]{geometry}

\begin{document}
\title{Dagbok}
\author{Fredrik Meyer}
\maketitle

\section{31 januar 2017}

\begin{itemize}
  \item Veiledning. Fikk arbeidsrutinetips fra Jan. Han foreslo to timer effektiv skriving hver dag. Han skrev ned tidspunkt for hver avbrytelse. Helst tidlig på morgen (han var veldig trøtt da, og få avbrytelser). Han påsto at man blir overrasket over hvor mye man får skrevet med en slik rutine. Skal prøve.

\item Fikk skrevet i circa to timer i dag. Skal prøve det samme i morgen!

Fikk epost om at jeg er invitert på jobbintervju til Microsoft University. Må lese litt om dette er en jobb jeg vil ha.
\end{itemize}

\section{1 februar 2017} % (fold)
\label{sec:1_februar_2017}

\begin{itemize}
  \item Nettopp vært på andregangsintervju hos inmeta. Virket som om det gikk veldig bra - han snakket om et fagintervju ganske snart. Han var ikke 100\% klar i talen, men virket som om jeg skulle få et fagintervju ganske snart (dvs presentere innholdet i to forskningsartikler!). Venter fortsatt på bekreftelsesepost, da.

  Skal forøvrig også på intervju på førstkommende tirsdag på Microsoft University, også de hos Glasspaper. Men inmeta-jobben virker mye morsommere.
\end{itemize}

\section{3 februar 2017} % (fold)
\label{sec:3_februar_2017}

\begin{itemize}
  \item Venter fortsatt på bekreftelsesepost fra inmeta om eventuelt tredje intervju. Tror han sa ``i løpet av et par dager'', og det er to dager siden nå (og klokken er bare 13:13), så det er strengt tatt fremdeles innenfor marginene til at jeg ikke burde være bekymret. Dessverre virker ikke hjernen slik, så jeg er ganske stressa - sjekker epost konstant, og sliter litt med å konsentrere meg i dag. Håper \emph{virkelig} på svar snart.
  \item I går: rusreguleringssamtale på Chateu Neuf som gikk veldig bra. Var veldig morsomt, og artig å henge igjen å samtale etterpå.
  \item Møter kanskje Emil ikveld igjen.
\end{itemize}

\section{8 februar 2017} % (fold)
\label{sec:8_februar_2017}

\begin{itemize}
  \item Fagintervju overstått! Gjennomgikk artikkelen om AlexNet og viste dem min analyse av et datasett. Sistnevnte var de tydelig fornøyd med, men vanskeligere å lese reaksjonen på presentasjonen av artikkelen. Tror det gikk rimelig greit, men de er klar over at jeg ikke har så mye kunnskap om temaet (så kan hende de klarer å finne andre med mer kunnskap om sånt!?). Skulle høre noe ``i løpet av et par dager eller forhåpentligvis allerede i dag''. 

  Må prøve å tenke på andre ting nå.
  \item Middag hos JV+Stian i dag. Blir interessant. 
  \item Var på intervju hos Glasspaper igjen i går for Microsoft University. Gikk bra. Hun skulle anbefale meg videre til ``parterbedriftene''.
\end{itemize}
% section 8_februar_2017 (end)

\section{9 februar} % (fold)
\label{sec:9_februar}

\begin{itemize}
  \item To minutter inne hos Jan i dag. Han mener å ha funnet ca passende dato for disputas. 14 juni, og mener fortsatt vi er i rute. Han prøvde å be om Nathan Ilten kunne være opponent (men det skal vel være to til?).  Uansett - skrikende nært i tid.
  \item Har fremdeles ikke hørt noe fra inmeta (klokken er nå 16:21, og det er kanskje ikke vanlig med telefoner etter klokken 16?).
\end{itemize}
% section 9_februar (end)

\section{10 febraur} % (fold)
\label{sec:10_febraur}

\begin{itemize}
  \item Ble ringt opp av Pål Hellum fra Glasspaper nettopp. Han hadde snakket med Lars Joakim Nilsson fra inmeta, og Lars mente altså at personligheten min kanskje ikke passet til konsulentjobben. Pål hadde ikke det inntrykket, og jeg sa at jeg trodde ikke konsulentjobb ville bli noe problem, og han skulle videreformidle det til Lars. Så skulle Lars antakelig ringe meg i løpet av dagen og prate litt. Snakk om å tilspisse situasjonen!
  \item Rar slitsom dag. Har ikke fått konsentrert meg noe. Ikke hørt noe mer fra inmeta etter samtalen i morges. Men ble derimot ringt opp av en headhunter som hadde funnet profilen min på LinkedIn. Han synes profilen så interessant ut, og allerede nå virker det som om han har sikret meg to jobbintervjuer neste uke. Dette går unna dette. 
\end{itemize}

% section 10_febraur (end)

\section{13 februar} % (fold)
\label{sec:13_februar}

\begin{itemize}
  \item Bursdag i dag. Ble ringt opp av Pål Hellum i dag, og han lurte på om Lars Joakim fra inmeta hadde ringt meg ennå, noe han ikke hadde. Så venter ennå på telefon. Føler altså at sjansene for denne jobben minker hele tiden. Skal dog på intervju i morgen hos Auka, og det er jo muligens interessant.
  \item Spekulerer i om jeg skal feire bursdag og hvor mange jeg skal be. Ulempen med mange: Bezzerwisser blir vanskelig.
  \item Fikk telefon fra Pål Hellum nå igjen: inmeta takket nei til meg, fordi de "ville ha noen med mer konkret maskinlæringerfaring". Å vel. Da blir det mer jobbsøkerhelvete framover. På den annen side: skal på jobbintervju både på tirsdag (i morgen) og torsdag (PWC).
  \item Forsåvidt veldig fin dag: jeg ble spandert øl på av lesesalsgjengen på Jekylls pub. Planlegger også en liten feiring på lørdag. 
\end{itemize}

% section 13_februar (end)

\section{14 februar} % (fold)
\label{sec:14_februar}

\begin{itemize}
  \item Intervju hos Auka i dag. Var ganske hyggelig. Virker som en ganske fin jobb, men han var litt usikker på behovene. Vi skulle snakkes i begynnelsen av mars igjen for et teknisk intervju (litt nervøs!). Eneste ulempen er at det kanskje er litt ustabil framtidsutsikt for dem?
  \item Ellers: skrev et par søknader til diverse analyse-jobber på Finn. For eksempel capgemini og NorgesGruppen. Fikk også epost om PWC-intervjuet på torsdag: kleskode ``jobbklær''. Hva i all verden betyr det?
  \item Hørte fra rekrutteren Charlie Tiplady igjen. Han skal prøve å få ordnet intervju med bWise (Visma) snart, og han har kanskje en interessant maskinlæringjobb på lager til meg også. Dette begynner å bli spennende (og ganske mye!).
  \item Ikke fullt så mye skriving i dag, men har fått gjort endel, og lest meg litt opp igjen på disse hyper-Kähler-greiene.
\end{itemize}

% section 14_februar (end)

\section{15 februar} % (fold)
\label{sec:15_februar}

\begin{itemize}
  \item Karoline fortalte at disputas i juni antakelig blir helt umulig på grunn av tidsbegrensninger. Får ta dette opp med Jan.
\end{itemize}
% section 15_februar (end)

\section{18 februar} % (fold)
\label{sec:18_februar}

\begin{itemize}
  \item Arrangerer ``bursdag'' i dag. Baker kake, som muligens blir pittelitt mislykket?
  \item Søkte jobb hos Bekk i går. De svarte raskt, og jeg skulle få tilbakemelding i løpet av en ukes tid. 
\end{itemize}

\section{27 februar (mandag)} % (fold)
\label{sec:27_februar}

\begin{itemize}
  \item Intervju hos både Bekk og Capgemini på fredag. Ble oppringt av Capgemini i dag, og de vil ha andreintervju. Tror Bekk-intervjuet gikk bra, og håper på svar snart (inntrykket mitt av Bekk er hakket bedre enn av Capgemini).
  \item Var hos Marius i helga, og vi så litt på det batteri-datasettet. Var ganske morsomt, og potensielt noe jeg kan leke meg med. (han snakket også om sine inntrykk av konsulentbransjen: han likte ikke Capgemini/Accenture/McKinsey, men hadde greit/ok+ inntrykk av Bekk)
  \item Doktorgrad: oppe i 40 pdf-sider nå. Synes fremdeles det er altfor lite. Tror kapitlet om $\mathrm{dP6}$-singularitetene blir litt lengre, og mulig jeg kan inkludere noe kode og sånt. Må få snakket med Jan i morgen om \textbf{disputas-dato}. 
\end{itemize}

% section 27_februar (end)

\section{2. mars} % (fold)
\label{sec:2_mars}

\begin{itemize}
  \item Var på interjvu hos Capgemini i dag igjen. Følte det gikk bra - var en veldig avslappet og ganske uformell samtale. De solgte seg ganske bra - for nå har jeg plutselig blitt usikker på hva jeg ønsker mest av dem og Bekk. Skal på andreintervju hos Bekk neste onsdag, og omtrent samtidig skal jeg høre fra Capgemini om de vil ha meg.
  \item Jobbet \emph{svært} lite på doktorgraden denne uken. Tiden flyr og motivasjonen også.
  \item Var i Framtida.no i går. Veldig morsomt med all oppmerksomheten.
\end{itemize}

% section 2_mars (end)

\section{4. mars} % (fold)
\label{sec:4_mars}

\begin{itemize}
  \item Lærte fra Vegard Fjellbo i dag at jeg ligger på lønnstrinn 55, noe som betyr at lønna er 470.000 i året. Altså mye mer enn det jeg trodde jeg tjente :S. Det gjør jo eventuelle lønnstilbud etterhvert litt vanskeligere å ta stilling til. Hva betyr ``betydelig økning'' for eksempel...
\end{itemize}
% section 4_mars (end)

\section{6. mars} % (fold)
\label{sec:6_mars}

\begin{itemize}
  \item Ringt opp av Charlie Tiplady igjen. Jeg skal visst på intervju hos bWise i morgen klokken 13, og det skulle være et ``relativt tek nisk intervju''. Spent på hva de har å fortelle (men jeg håper egentlig mer på Capgemini eller Bekk...).
\end{itemize}

% section 6_mars (end)

\section{8. mars} % (fold)
\label{sec:8_mars}

\begin{itemize}
  \item Var på andre intervju hos Bekk i dag. Første intervuet gikk ganske bra, tror jeg. Var veldig hyggelig og god stemning. Andreintervjuet (det tekniske) var hakket mer ubehagelig, men jeg tror det gikk greit.
  \item Var på bWise-intervjuet i går. De holder til samme sted som EY. Stedet virker hyggelig, men kanskje litt vel ensrettet mhp teknologien de bruker.
\end{itemize}

% section 8_mars (end)

\section{11. mars} % (fold)
\label{sec:11_mars}

\begin{itemize}
  \item Fredag i går. Skulle høre fra Capgemini ``i midten av denne uka'', så da tenkte jeg at jeg garantert ville høre fra dem i går. Ingenting. Heller ingenting fra Bekk. Stress. Da er sannsynligheten stor for at jeg hører noe fra begge på mandag.
\end{itemize}

% section 11_mars (end)

\section{16 mars} % (fold)
\label{sec:16_mars}

\begin{itemize}
  \item Ble ringt opp av Bekk på tirsdag. De vil ha meg på et siste intervju, hvor jeg denne gangen skal møte avdelingsleder. Julie Brunsvik (rekrutteringsansvarlig) sa alle ``bare hadde lovord'' om meg, så det virket som om kanskje har sikret meg denne jobben. Jeg er litt usikker på hva formålet med det siste intervjuet er. Sissel fikk jobb rett etter det tekniske intervjuet, og hun mistenkte dette var for å ta en ekstra sjekk på om jeg passer inn. Så litt nervøs for dette.
  \item Har ennå ikke hørt noe fra Capgemini.
  \item Snakket med Biljana i går. Hun mente at om jeg leverer i juni, kan jeg disputere i september. Jeg videreformidlet dette til Jan, og han spurte nå om vi ikke kan utsette leveringen til september, og heller ha disputas i desember. Jeg sa det var forsåvidt greit, men at det vil krasje litt med eventuell ny jobb (og at jeg er litt lei av dette...).
\end{itemize}

% section 16_mars (end)

\section{20 mars (mandag)} % (fold) 
\label{sec:20_mars}

\begin{itemize}
  \item Fest hos Øyvind på lørdag. Lars Andreas fortalte at Capgemini hadde fortalt ham at det var tre aktuelle kandidater og to plasser. Det betyr at om jeg og han er blant disse tre aktuelle, så får minst én av oss jobbtilbud der. Sissel hadde god tro på oss begge hos Bekk (drømte at jeg fikk jobbtilbud i natt...)
  \item Ble ringt opp av Bekk i dag, og fikk jobben! Skal møte opp i morgen for kontraktsgjennomgang! 
\end{itemize}
% section 20_mars (end)

\section{21. mars} % (fold)
\label{sec:21_mars}

\begin{itemize}
  \item Var hos Bekk i dag og skrev under kontrakt! Endte opp med 535.000 i lønn pluss en liten haug med perks. Begynner første august! Så nå er det bare å raske på med doktorskriving.
\end{itemize}
% section 21_mars (end)

\section{27. mars} % (fold)
\label{sec:27_mars}

\begin{itemize}
  \item Fått til mer jobbing på PhD i det siste. Ikke like effektivt som i januar og februar, men still. Akkurat nå sliter jeg med et veldig rart kompileringsproblem i Sublime.
\end{itemize}

% section 27_mars (end)

\section{28. mars} % (fold)
\label{sec:28_mars}

\begin{itemize}
  \item Fikk epost fra Bekk i går hvor jeg ble spurt om hva slags teknologier jeg kunne tenket meg å jobbe med (.NET, Java, frontend/backend, Javascript). Jeg svarte at jeg har veldig svake preferanser, men at jeg kanskje foretrekker Java foran .NET, og backend foran frontend. Tar nok en god stund før jeg får vite eksakt.
  \item Kjøpte jakke og sko i går. Må også få kjøpt flere skjorter. Byttet det trådløse headsettet jeg kjøpte nylig pga dårlig signal. Viser seg at også dette muligens lider av dårlig signal...
  \item Emil kommer på besøk i kveld. :--)
\end{itemize}

% section 28_mars (end)

\section{1. april} % (fold)
\label{sec:1_april}

\begin{itemize}
  \item Ang. eposten fra Bekk. Hun jeg snakket med (Guro Seternes) la merke til at jeg skrev at jeg kan HTML/CSS på CV-en. Jeg refererte til hjemmesiden min og Ole Martins hjemmeside. Jeg har egentlig ikke lyst til å jobbe med kun webutvikling (men dette sa jeg ikke så veldig klart).
  \item Binget litt bøker på Amazon nylig. Mission ``bli flink programmerer'' starter nå.
  \item Om avhandlingen: jeg kom over en tabell i en artikkel av Kapustka, hvor han lister Calabi--Yau-mangfoldigheter med $\mathrm{Pic} X = \mathbb Z$. Èn av elementene i listen hadde $H^3=36$, $\chi=-72$ og $|H|=12$, noe som stemmer med min $X_1$. Denne var bare spådd å eksistere i en artikkel av Duco van Straten, basert på løsninger av noen differensialligninger av ``Calabi--Yau type''. Disse skal formodentligvis svare til Picard--Fuchs-ligninger til 1-parameter-familier av Calabi--Yau-mangfoldigheter.

  Jeg sendte epost til Duco og fortalte om dette, og etter litt ettertanke var han overbevist om at min konstruksjon faktisk svarte til hans differensialligning. Også de to andre jeg har konstruert passer med differensialligninger i databasen over Calabi--Yau-ligninger.
  \item Tror forresten jeg har klart å framstille $X_1$ og $X_2$ med små nok Gröbner-baser til at vi kan regne ut $T^1$ i grad null. Må dobbeltsjekke at forenklingene jeg gjør ikke lager singulariteter.
\end{itemize}

% section 1_april (end)

\section{6. april} % (fold)
\label{sec:6_april}

\begin{itemize}
  \item Tror forenklingene jeg har gjort faktisk lager singulariteter. Det er interessant hvordan en generisk matrise gir en Gröbner-base på $49$ elementer, mens en radredusert matrise gir en Gröbner-base på $49$ elementer. Det er som om glatthet automatisk krever mange elementer i Gröbner-basen.
  \item Beeks boot-camp krasjer visst med Øya. Så jeg tror jeg må selge Øya-billetten min =/.
  \item Ikke skrevet så mye siste uka. Har vært ganske forkjølet, og heller brukt tiden på å forstå konsekvensene av at mine konstruksjoner er på Ducos liste. Har også brukt litt tid på å regne på deformasjoner av sfærer med åtte hjørner. Kan jo inkludere utregningene i et appendiks eller noe.
\end{itemize}

% section 6_april (end)

\end{document}
