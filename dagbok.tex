\documentclass[11pt, norsk]{article}
%\usepackage[latin1]{inputenc}
\usepackage[T1]{fontenc}
\usepackage[utf8x]{inputenc}
\usepackage[norsk]{babel}   % S P R A A K
% \usepackage{graphicx}    % postscript graphics
\usepackage{amssymb, amsmath, amsthm, amssymb} % symboler, osv
\usepackage{mathrsfs}
\usepackage{url}
\usepackage{thmtools}
\usepackage{enumerate}  % lister $  
\usepackage{float}
\usepackage{tikz}
\usepackage{tikz-cd}
\usetikzlibrary{calc}
%\usepackage{tikz-3dplot}
\usepackage{subcaption}
\usepackage[all]{xy}   % for comm.diagram
\usepackage{wrapfig} % for float right
\usepackage{hyperref}
\usepackage{mystyle} % stilfilen      
\usepackage{booktabs}
\usepackage{resizegather}
\setcounter{MaxMatrixCols}{48}
\usepackage{listings}
\lstset{language=Macaulay2}
\definecolor{mygreen}{rgb}{0,0.6,0}
\definecolor{mygray}{rgb}{0.5,0.5,0.5}
\definecolor{mymauve}{rgb}{0.58,0,0.82}

\lstset{ %
  backgroundcolor=\color{white},   % choose the background color; you must add \usepackage{color} or \usepackage{xcolor}
  basicstyle=\footnotesize,        % the size of the fonts that are used for the code
  breakatwhitespace=false,         % sets if automatic breaks should only happen at whitespace
  breaklines=true,                 % sets automatic line breaking
  captionpos=b,                    % sets the caption-position to bottom
  commentstyle=\color{mygreen},    % comment style
  deletekeywords={...},            % if you want to delete keywords from the given language
  escapeinside={\%*}{*)},          % if you want to add LaTeX within your code
  extendedchars=true,              % lets you use non-ASCII characters; for 8-bits encodings only, does not work with UTF-8
  frame=single,	                   % adds a frame around the code
  keepspaces=true,                 % keeps spaces in text, useful for keeping indentation of code (possibly needs columns=flexible)
  keywordstyle=\color{blue},       % keyword style
  language=Macaulay2,                 % the language of the code
  otherkeywords={*,...},           % if you want to add more keywords to the set
  numbers=left,                    % where to put the line-numbers; possible values are (none, left, right)
  numbersep=5pt,                   % how far the line-numbers are from the code
  numberstyle=\tiny\color{mygray}, % the style that is used for the line-numbers
  rulecolor=\color{black},         % if not set, the frame-color may be changed on line-breaks within not-black text (e.g. comments (green here))
  showspaces=false,                % show spaces everywhere adding particular underscores; it overrides 'showstringspaces'
  showstringspaces=false,          % underline spaces within strings only
  showtabs=false,                  % show tabs within strings adding particular underscores
  stepnumber=2,                    % the step between two line-numbers. If it's 1, each line will be numbered
  stringstyle=\color{mymauve},     % string literal style
  tabsize=2,	                   % sets default tabsize to 2 spaces
  title=\lstname                   % show the filename of files included with \lstinputlisting; also try caption instead of title
}

%\usepackage[a5paper,margin=0.5in]{geometry}

\begin{document}
\title{Dagbok}
\author{Fredrik Meyer}
\maketitle

\section{31 januar 2017}

\begin{itemize}
  \item Veiledning. Fikk arbeidsrutinetips fra Jan. Han foreslo to timer effektiv skriving hver dag. Han skrev ned tidspunkt for hver avbrytelse. Helst tidlig på morgen (han var veldig trøtt da, og få avbrytelser). Han påsto at man blir overrasket over hvor mye man får skrevet med en slik rutine. Skal prøve.

\item Fikk skrevet i circa to timer i dag. Skal prøve det samme i morgen!

Fikk epost om at jeg er invitert på jobbintervju til Microsoft University. Må lese litt om dette er en jobb jeg vil ha.
\end{itemize}

\section{1 februar 2017} % (fold)
\label{sec:1_februar_2017}

\begin{itemize}
  \item Nettopp vært på andregangsintervju hos inmeta. Virket som om det gikk veldig bra - han snakket om et fagintervju ganske snart. Han var ikke 100\% klar i talen, men virket som om jeg skulle få et fagintervju ganske snart (dvs presentere innholdet i to forskningsartikler!). Venter fortsatt på bekreftelsesepost, da.

  Skal forøvrig også på intervju på førstkommende tirsdag på Microsoft University, også de hos Glasspaper. Men inmeta-jobben virker mye morsommere.
\end{itemize}

\section{3 februar 2017} % (fold)
\label{sec:3_februar_2017}

\begin{itemize}
  \item Venter fortsatt på bekreftelsesepost fra inmeta om eventuelt tredje intervju. Tror han sa ``i løpet av et par dager'', og det er to dager siden nå (og klokken er bare 13:13), så det er strengt tatt fremdeles innenfor marginene til at jeg ikke burde være bekymret. Dessverre virker ikke hjernen slik, så jeg er ganske stressa - sjekker epost konstant, og sliter litt med å konsentrere meg i dag. Håper \emph{virkelig} på svar snart.
  \item I går: rusreguleringssamtale på Chateu Neuf som gikk veldig bra. Var veldig morsomt, og artig å henge igjen å samtale etterpå.
  \item Møter kanskje Emil ikveld igjen.
\end{itemize}

\section{8 februar 2017} % (fold)
\label{sec:8_februar_2017}

\begin{itemize}
  \item Fagintervju overstått! Gjennomgikk artikkelen om AlexNet og viste dem min analyse av et datasett. Sistnevnte var de tydelig fornøyd med, men vanskeligere å lese reaksjonen på presentasjonen av artikkelen. Tror det gikk rimelig greit, men de er klar over at jeg ikke har så mye kunnskap om temaet (så kan hende de klarer å finne andre med mer kunnskap om sånt!?). Skulle høre noe ``i løpet av et par dager eller forhåpentligvis allerede i dag''. 

  Må prøve å tenke på andre ting nå.
  \item Middag hos JV+Stian i dag. Blir interessant. 
  \item Var på intervju hos Glasspaper igjen i går for Microsoft University. Gikk bra. Hun skulle anbefale meg videre til ``parterbedriftene''.
\end{itemize}
% section 8_februar_2017 (end)

\section{9 februar} % (fold)
\label{sec:9_februar}

\begin{itemize}
  \item To minutter inne hos Jan i dag. Han mener å ha funnet ca passende dato for disputas. 14 juni, og mener fortsatt vi er i rute. Han prøvde å be om Nathan Ilten kunne være opponent (men det skal vel være to til?).  Uansett - skrikende nært i tid.
  \item Har fremdeles ikke hørt noe fra inmeta (klokken er nå 16:21, og det er kanskje ikke vanlig med telefoner etter klokken 16?).
\end{itemize}
% section 9_februar (end)

\section{10 febraur} % (fold)
\label{sec:10_febraur}

\begin{itemize}
  \item Ble ringt opp av Pål Hellum fra Glasspaper nettopp. Han hadde snakket med Lars Joakim Nilsson fra inmeta, og Lars mente altså at personligheten min kanskje ikke passet til konsulentjobben. Pål hadde ikke det inntrykket, og jeg sa at jeg trodde ikke konsulentjobb ville bli noe problem, og han skulle videreformidle det til Lars. Så skulle Lars antakelig ringe meg i løpet av dagen og prate litt. Snakk om å tilspisse situasjonen!
  \item Rar slitsom dag. Har ikke fått konsentrert meg noe. Ikke hørt noe mer fra inmeta etter samtalen i morges. Men ble derimot ringt opp av en headhunter som hadde funnet profilen min på LinkedIn. Han synes profilen så interessant ut, og allerede nå virker det som om han har sikret meg to jobbintervjuer neste uke. Dette går unna dette. 
\end{itemize}

% section 10_febraur (end)

\section{13 februar} % (fold)
\label{sec:13_februar}

\begin{itemize}
  \item Bursdag i dag. Ble ringt opp av Pål Hellum i dag, og han lurte på om Lars Joakim fra inmeta hadde ringt meg ennå, noe han ikke hadde. Så venter ennå på telefon. Føler altså at sjansene for denne jobben minker hele tiden. Skal dog på intervju i morgen hos Auka, og det er jo muligens interessant.
  \item Spekulerer i om jeg skal feire bursdag og hvor mange jeg skal be. Ulempen med mange: Bezzerwisser blir vanskelig.
  \item Fikk telefon fra Pål Hellum nå igjen: inmeta takket nei til meg, fordi de "ville ha noen med mer konkret maskinlæringerfaring". Å vel. Da blir det mer jobbsøkerhelvete framover. På den annen side: skal på jobbintervju både på tirsdag (i morgen) og torsdag (PWC).
  \item Forsåvidt veldig fin dag: jeg ble spandert øl på av lesesalsgjengen på Jekylls pub. Planlegger også en liten feiring på lørdag. 
\end{itemize}

% section 13_februar (end)

\section{14 februar} % (fold)
\label{sec:14_februar}

\begin{itemize}
  \item Intervju hos Auka i dag. Var ganske hyggelig. Virker som en ganske fin jobb, men han var litt usikker på behovene. Vi skulle snakkes i begynnelsen av mars igjen for et teknisk intervju (litt nervøs!). Eneste ulempen er at det kanskje er litt ustabil framtidsutsikt for dem?
  \item Ellers: skrev et par søknader til diverse analyse-jobber på Finn. For eksempel capgemini og NorgesGruppen. Fikk også epost om PWC-intervjuet på torsdag: kleskode ``jobbklær''. Hva i all verden betyr det?
  \item Hørte fra rekrutteren Charlie Tiplady igjen. Han skal prøve å få ordnet intervju med bWise (Visma) snart, og han har kanskje en interessant maskinlæringjobb på lager til meg også. Dette begynner å bli spennende (og ganske mye!).
  \item Ikke fullt så mye skriving i dag, men har fått gjort endel, og lest meg litt opp igjen på disse hyper-Kähler-greiene.
\end{itemize}

% section 14_februar (end)

\section{15 februar} % (fold)
\label{sec:15_februar}

\begin{itemize}
  \item Karoline fortalte at disputas i juni antakelig blir helt umulig på grunn av tidsbegrensninger. Får ta dette opp med Jan.
\end{itemize}
% section 15_februar (end)

\end{document}
