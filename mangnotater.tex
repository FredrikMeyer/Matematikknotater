\documentclass[11pt, english]{article}
%\usepackage[latin1]{inputenc}
\usepackage[T1]{fontenc}
\usepackage[utf8]{inputenc}
\usepackage[english]{babel}   % S P R A A K
% \usepackage{graphicx}    % postscript graphics
\usepackage{amssymb, amsmath, amsthm, amssymb} % symboler, osv
\usepackage{mathrsfs}
\usepackage{url}
\usepackage{thmtools}
\usepackage{enumerate}  % lister $  
\usepackage{float}
\usepackage{tikz}
\usepackage{tikz-cd}
\usetikzlibrary{calc}
%\usepackage{tikz-3dplot}
\usepackage{subcaption}
\usepackage[all]{xy}   % for comm.diagram
\usepackage{wrapfig} % for float right
\usepackage{hyperref}
\usepackage{mystyle} % stilfilen      

%\usepackage[a5paper,margin=0.5in]{geometry}


\begin{document}
\title{Mangfoldigheter}
\author{Fredrik Meyer}
\maketitle 

\section{Mangfoldigheter}

Mangfoldigheter er topologiske rom som er konstruert ved å lime sammen kopier av $\R^n$ på et kontinuerlig vis.

Presist:

\begin{defi}
Et Hausdorff, parakompakt, topologisk rom $M$ er en \textbf{topologisk mangfoldighet} hvis det for hver $p \in M$ finnes en åpen mengde $U \ni p$ og en homeomorfi $\varphi_{p,U}:U \to \R^n$. 
\end{defi}

Vi kan utstyre mangfoldigheter med mer struktur. I dette kurset er vi interessert i "differensiable strukturer". Èn måte å introdusere dette på er ved hjelp av et \emph{maksimalt atlas}. 

[kan dette ekvivalent gjøres vha et knippe av funksjoner?]

\begin{defi}[Glatt maksimalt atlas]
Et \textbf{kart} er en homeomorfi $\varphi:U \to V$ fra en åpen mengde $U \subset \R^n$ til en åpen delmengde $V \subset M$. Et \textbf{atlas} er en mengde kompatible kart, i betydningen at hvis $x:U \to M$ og $y:V \to M$ er to kart, så er $y^{-1} \circ x : U \to V$ glatt. Atlaset skal dekke mangfoldigheten, slik at hvert punkt $p \in M$ er dekket av et kart. 

Et \textbf{maksimalt atlas} er et atlas som ikke kan utvides med flere kompatible kart.
\end{defi}

Av tekniske grunner definerer vi så en \textbf{differensiabel mangfoldighet} (heretter bare kalt "mangfoldighet") til å være en topologisk mangfoldighet utstyrt med et glatt maksimalt kart. 

\end{document}