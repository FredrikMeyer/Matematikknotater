\documentclass[11pt, english]{article}
%\usepackage[latin1]{inputenc}
\usepackage[T1]{fontenc}
\usepackage[utf8]{inputenc}
\usepackage[english]{babel}   % S P R A A K
% \usepackage{graphicx}    % postscript graphics
\usepackage{amssymb, amsmath, amsthm, amssymb} % symboler, osv
\usepackage{mathrsfs}
\usepackage{url}
\usepackage{thmtools}
\usepackage{enumerate}  % lister $  
\usepackage{float}
\usepackage{tikz}
\usepackage{tikz-cd}
\usetikzlibrary{calc}
%\usepackage{tikz-3dplot}
\usepackage{subcaption}
\usepackage[all]{xy}   % for comm.diagram
\usepackage{wrapfig} % for float right
\usepackage{hyperref}
\usepackage{mystyle} % stilfilen      

%\usepackage[a5paper,margin=0.5in]{geometry}


\begin{document}
\title{Mangfoldigheter}
\author{Fredrik Meyer}
\maketitle 

\section{Mangfoldigheter}

Mangfoldigheter er topologiske rom som er konstruert ved å lime sammen kopier av $\R^n$ på et kontinuerlig vis.

Presist:

\begin{defi}
Et metrisk rom $M$ er en \textbf{topologisk mangfoldighet} hvis det for hver $p \in M$ finnes en åpen mengde $U \ni p$ og en homeomorfi $\varphi_{p,U}:U \to \R^n$. 
\end{defi}


\textbf{Merk:} Her krever vi at $n$ er \emph{ekvidimensjonal}, nemlig at dimensjonen er konstant. Merk også at den åpne mengden alltid kan velges til å være homeomorf med en åpen ball i $\R^n$.

\begin{example}
Alle $\R^n$ og alle åpne delmengder av $\R^n$.
\end{example}
\begin{example}
Produkter av mangfoldigheter er mangfoldigheter og disjunkte unioner også.
\end{example}

\begin{example}
Sfærene $S^n$: La $N=(0,\ldots,0,1)$ og $S=(0,\ldots,0,-1)$, "nord- og sørpolen". Vi definerer \emph{stereografisk projeksjon} fra $N$ (og $S$) fra $S^n \bs N \to \R^n$. Om $(a^1,\ldots,a^{n+1}) \in S^n \subset R^{n+1}$, sendes denne til
$$
\left( \frac{a^1}{1-a^{n+1}}, \ldots, \frac{a^n}{1-a^{n+1}} \right).
$$
Dette gir en homeomorfi med $\R^n$ med invers
$$
(a^1, \ldots, a^n) \mapsto \left( \frac{2a^1}{1+\lvert a \rvert}, \ldots, \frac{2 a^n}{1+\lvert a \rvert}, \frac{\lvert a \rvert - 1}{1+\lvert a \rvert}\right).
$$
DEtte gir oss en differensiabel struktur på alle sfærene $S^n$. Vi kan spørre oss om det finnes en unik differensiabel struktur på $S^n$. Dette er sant for $n \leq 6$, men Milnor viste at det er 28 differensiable strukturer på $S^7$, de såkalte "eksotiske sfærene".
\end{example}
\begin{example}[Projektive rom]

La $\PP^n$ være mengden av linjer gjennom origo i $\R^{n+1}$. Merk at hver linje kan gis ved et $n+1$-tuppel, men dette tuplet er bare unikt opp til multiplikasjon fra $\R \bs \{ 0\}$. Vi kan definere et atlas for $\PP^n$ på følgende hvis ($\PP^n$ arver automatisk også topologien på dette viset). La $U_i$ være mengden av linjer hvor den $i$'te koordinaten er ulik null. Da definerer vi en homeomorfi til $\R^n$ ved 
$$
\varphi_j: [a_1,\ldots, a^{n+1}] \mapsto \left(a_1/a^i, \ldots, a^{i-1}/a^i, a^{i+1}/a^i,\ldots, a^{n+1}\right) \in \R^n
$$
Inversen er gitt ved $(a^1,\ldots, a^n) \mapsto [a^1, \ldots, 1, \ldots, a^n]$. Her er koordinatene i brackets en vektor som spenner linjen. Merk at avbildningen er veldefinert. 

Anta nå $i \neq j$. Vi ser på den induserte avbildningen $\R^n \to \R^n$ gitt ved $\varphi_j \circ \varphi_i^{-1}$:
$$
(a^1,\ldots, a^n) \mapsto \left( \frac{a^1}{a^j}, \ldots, \frac{1}{a^j}, \ldots, \frac{a^n}{a^j} \right).
$$
Her er $1/a^j$  posisjon $i$.
\end{example}


Vi kan utstyre mangfoldigheter med mer struktur. I dette kurset er vi interessert i "differensiable strukturer". Èn måte å introdusere dette på er ved hjelp av et \emph{maksimalt atlas}. 

[kan dette ekvivalent gjøres vha et knippe av funksjoner?]

\begin{defi}[Glatt maksimalt atlas]
Et \textbf{kart} er en homeomorfi $\varphi:U \to V$ fra en åpen mengde $U \subset \R^n$ til en åpen delmengde $V \subset M$. Et \textbf{atlas} er en mengde kompatible kart, i betydningen at hvis $x:U \to M$ og $y:V \to M$ er to kart, så er $y^{-1} \circ x : U \to V$ glatt. Atlaset skal dekke mangfoldigheten, slik at hvert punkt $p \in M$ er dekket av et kart. 

Et \textbf{maksimalt atlas} er et atlas som ikke kan utvides med flere kompatible kart.
\end{defi}

Av tekniske grunner definerer vi så en \textbf{differensiabel mangfoldighet} (heretter bare kalt "mangfoldighet") til å være en topologisk mangfoldighet utstyrt med et glatt maksimalt kart. 

\subsection{Differensiabilitet og konsekvenser}

En avbildning $f:M^n \to N^m$ mellom mangfoldigheter er differensiabel om den lokalt er differensiabel. Dette gir mening å si, siden vi vet hva differensiabilitet er for avbildninger $\R^n \to \R^m$. 

Noe av det spesielle med differensialgeometri er eksistensen av glatte funksjoner med fine egenskaper (bytt ut "differensiabel" med det sterkere kravet "holomorf" for eksempel, og du er på langt dypere vann). 

\begin{enumerate}
\item Definer $h:\R \to \R$ ved 
\[
h(x) = \begin{cases} e^{-1/x^2} & x \neq 0 \\
0 & x=0.
\end{cases}
\]
Da er $h$ glatt og $h^{(n)}(0)=0$ for alle $n$.
\item Funksjonen $j:\R \to \R$ definert ved
$$
j(x) = \begin{cases} e^{-(x-1)^{-2}}e^{-(x+1)^{-2}} & x \in (-1,1) \\
0 & x \not \in (-1,1)
\end{cases}
$$
er glatt og strengt positiv innenfor $(-1,1)$. Ved skalering og translasjon kan vi få en funksjon $k:\R \to \R$ som er positiv på $(0,\delta)$ og null utenfor.
\item Definer 
$$
l(x) = \left( \int_0^x k(x) dx \right) / \left( \int_0^\delta k(x) dx \right). 
$$
Den er $0$ for $x \leq 0$, stigende på $(0,\delta)$, og $1$ for $x \geq \delta$.
\item Funksjonen $g:\R^n \to \R$ definert ved
$$
g(x) = \prod_{i=1}^n j(x^i/\epsilon)
$$
er positiv på $(-\epsilon, \epsilon)^n$ og null ellers.
\end{enumerate}

Dette er det vi trenger for å bevise eksistensen av såkalte "bump"-funksjoner:

\begin{prop}
La $K \subset U \subset M$ være en kompakt mengde. Da finnes det en funksjon $g:M \to [0,1]$ som er $1$ på $K$ og $0$ utenfor $U$. 
\end{prop}
\begin{proof}
For hver $p \in U$ kan vi finne et koordinatsystem $x:V \to \R^n$ med $x(U) \supset (-\epsilon,\epsilon)^n$. Funksjonen $x \circ g$ er glatt på $V$. La $\overline{g}$ betegne samme funksjonen definert på $M$, men $0$ utenfor $V$. Denne er klart kontinuerlig og glatt. 

Dette kan gjøres for hver $p \in U$. Siden $K$ er kompakt, vil endelig mange $V$ dekke $K$. Kall disse $V_1,\ldots,V_p$ med tilhørende funksjoner $g_1,\ldots,g_p$ som over. 

Funksjonen $g_1+\ldots + g_p$ er positiv på $K$, si større enn $\delta$. La $l$ være definert som i punkt 3 over. Da vil $l \circ(g_1 + \ldots + g_p)$ være $1$ på $K$ og $0$ utenfor $U$. 
\end{proof}

\subsection{Derivarberhet og sånn}

Vi skal skille mellom når vi deriverer \emph{i et koordinatsystem} og når vi deriverer på mangfoldigheten. For å gjøre dette må koordinatsystemet være bakt inn i notasjonen. 

La $f:\R^n \to \R$ være en funksjon (la oss heretter anta alle funksjoner er glatte, med mindre vi mot formodning skulle finne på å snakke om monstre). Vi definerer
$$
D_if(a) \stackrel{\Delta}{=} \lim_{h \to 0} \frac 1h \left( f(a^1, \ldots, a^i +h, \ldots, a^n)-f(a) \right).
$$

Om $g:\R^m \to \R^n$ og $f: \R^n \to \R$, så sier kjerneregelen at
\[
D_j(f \circ g)(a) = D_i(f(g(a))) D_jg^i(a).
\]
(vi prøver oss på Einstein-notasjonen)

Anta nå at $f:M \to \R$ er en funksjon og at $(x,U)$ er et koordinatsystem på $M$. Da definerer vi
\[
\dd{f}{x^i} (p) \stackrel{\Delta}{=} \restr{\dd{f}{x^i}}{p} \stackrel{\Delta}{=} D_i(f \circ x^{-1})(x(p)).
\]
Notasjonen referer altså til koordinatsystemet $(x,U)$, og måler "veksthastigheten" til funksjonen $f$ langs den i'te koordinataksen.

Ved å fjerne punktet $p$ fra notasjonen kan vi betrakte ligningene som ligninger mellom funksjoner. 

Den deriverte av en funksjon i et koordinatsystem er relatert til den deriverte i et annet koordinatsystem:
\begin{prop}
Om $(x,U)$ og $(y,V)$ er koordinatsystemer og $f: M \to \R$ er deriverbar, så gjelder på $U \cap V$ at
$$
\dd{f}{y^i} = \dd{x^j}{y^i} \dd{f}{x^j}.
$$
\end{prop}
\begin{proof}
\begin{align*}
\dd{f}{y^i}(p) &= D_i(f \circ y^{-1})(y(p)) \\
&= D_i([f \circ x^{-1}] \circ[ x \circ y^{-1}] )(y(p)) \\
&= D_i(f' \circ g')(y(p)) \\
&= D_j(f'(g'(y(p)))) D_i g^{'j}(y(p)) \\
&= D_j(f \circ x^{-1}(p)) D_i [x \circ y^{-1}]^j (y(p)) \\
&= \dd{f}{x^j}(p)  \dd{x^j}{y^i}(p).
\end{align*}
Legg merke til bruken av kjerneregelen.
\end{proof}

\begin{remark}
Det er lett å sjekke at operatoren $\restr{\dd{}{x^i}}{p}$ som sender funksjonskimer til $\R$ er en derivasjon. Formelt, la $\OO_p$ betegne ekvivalensklasser av funksjoner definert nær $p$. Da er $\ell=\restr{\dd{}{x^i}}{p}$ en lineær funksjon $\OO_p \to \R$ som tilfredstiller Leibniz-regelen: $\ell(fg) = f(p)\ell(g) + g(p) \ell(f)$.
\end{remark}

\subsection{Glatte avbildninger}

Avbildninger har egenskaper. La $f:M^n \to N^m$ være en avbildning mellom glatte mangfoldigheter og la $p \in M$. La $(x,U)$ være en kartomegn om $p$ og la $(y,V)$ være en kartomegn om $f(p)$. Da kan vi betrakte avbildningen $y \circ f \circ x^{-1}$, som er en avbildning $\R^n \to \R^m$. Matrisen
$$
J = \left( \restr{\dd{y^i \circ f}{x^j}}{p} \right)
$$
er Jacobi-matrisen til $f$ i punktet $p$. Da kan $J$ enten ha rang lik $m$ eller mindre enn $m$. Om rangen er mindre enn $m$, kaller vi $p$ et "\textbf{kritisk punkt}" for $f$, og om rangen er lik $m$ kaller vi $p$ for et "\textbf{regulært punkt}" for $f$. 

Om $p$ er kritisk, kaller vi $f(p)$ for en "\textbf{kritisk verdi}", og hvis ikke, kaller vi $f(p)$ for en "\textbf{regulær verdi}". 

Det er et faktum at bildet av de kritiske verdiene til $f$ alltid er "lite":

\begin{prop}
Om $f:M \to N$ er en glatt avbildning av mangfoldigheter. Da er mengden av kritiske verdier for $f$ en mengde av mål null i $N$.
\end{prop}

Dette er Sards "sterke" teorem slik det er beskrevet i Spivak. 

Om vi vet rangen til en avbildning i et punkt $p$, kan vi beskrive avbildningen lokalt.

\begin{prop}
\begin{enumerate}
\item Om $f: M^n \to N^m$ har rang $k$ i $p$, så finnes det koordinatsystemer $(x,U)$ rundt $p$ og $(y,V)$ rundt $f(p)$ slik at 
$$
y \circ f \circ x^{-1}(a^1,\ldots, a^n) = (a^1, \ldots, a^k, \psi^{k+1}(a), \ldots, \psi^m(a)).
$$
\item Om $f$ har rang $k$ i \emph{et område} rundt $p$ kan koordinatsystemene velges slik at
$$
y \circ f \circ x^{-1}(a^1,\ldots, a^n) = (a^1, \ldots, a^k, 0,\ldots,0).
$$
\end{enumerate}
\end{prop}
\begin{proof}
1. Dette er litt grisete å skrive opp, så jeg skisserer ideen. La $(u,U)$ være et koordinatsystem rundt $p$ og la $(y,V)$ være et system rundt $f(p)$. Siden $f$ har rang $k$ i $p$, finnes det en $k \times k$-undermatrise med determinant ulik null i $J$. Vi kan permutere søyler og rader slik at denne undermatrisen er øverst til venstre i $J$. Dette betyr at de $k$ første koordinatene til $f$ (uttrykt i $(y,V)$) kan brukes som koordinater.

Eksplisitt, definer et nytt koordinatsystem $(x,U)$ ved $x^\alpha(a) = y^\alpha(f(a))$ for $\alpha \leq k$ og $x^r(a) = u^r(a)$ for $r > k$. Da er det lett å se at $x$ definerer et koordinatsystem (følger fra Invers Funksjonsteoremet). La $q= x^{-1}(a^1, \ldots,a^n)$. Da er per definisjon $x^i(q) = a^i$. Da er $y^\alpha \circ f(q) = a^\alpha$ for $\alpha \leq k$ og $u^r(q)=a^r$ for $r > k$, per definisjon. Men da har vi at 
$$
y f x^{-1}(a^1,\ldots, a^n) = y \circ f(q) = (a^1,\ldots, a^k, ......).
$$
Som er det vi skulle vise.

2. Denne er teknisk mer infløkt. 
\end{proof}

\begin{prop}
\begin{enumerate}
\item Om $m \leq n$ og $f:M^n \to N^m$ har rang $m$ i $p$, og $(y,V)$ er et vilkårlig koordinatsystem rundt $f(p)$, så finnes et koordinatsystem $(x,U)$ rundt $p$ med
$$
y \circ f \circ x^{-1}(a^1, \ldots, a^n) = (a^1, \ldots, a^m).
$$
Med andre ord, om $f$ har maksimal rang, ser $f$ lokalt ut som en projeksjon.
\item Om $n \leq m$, og $(x,U)$ er et vilkårlig koordinatsystem rundt $p$, kan vi finne koordinatsystem rundt $(y,V)$ rundt $f(p)$ slik at
$$
y \circ f \circ x^{-1}(a^1,\ldots, a^n ) = (a^1, \ldots, a^n, 0,\ldots,0).
$$
Med andre ord, om $f$ har maksimal rang, så er $f$ lokalt en injeksjon.
\end{enumerate}
\end{prop}

Det er flere måter en mangfoldighet kan inkluderes i en annen mangfoldighet. 

Vi sier at en avbildning $f:M^n \to N^m $ er en \textbf{immersjon} om $n \leq m$ og $f$ har maksimal rang (lik $n$) overalt.

Immersjon er likevel ikke alltid nok - bildet trenger ikke nødvendigvis være en mangfoldighet. Ta for eksempel $8$. Dette kan realiseres som et bilde av en avbildning $S^1 \to \R^2$, men $8$-tallet er ikke homeomorf med en sirkel. 

En immersjon er alltid lokalt injektiv, men det er ikke nok å kreve injektivitet for å få "pene" undermangfoldigheter. Eksemplet $\R \to S^1 \times S^1$ med irrasjonal stigning gir en injektiv immersjon av den reelle linja inn i torusen. Vi kaller disse for en "\textbf{immersed submanfold}" (norsk ord?) 

\subsection{Eksistens av partisjon av enheten}

Her tror jeg vi kun beskriver.

La $\OO$ være en åpen overdekning av $M$. Da eksisterer det en samling av glatte funksjoner $\phi_i:M \to [0,1]$ slik at
\begin{enumerate}
\item Familien $\{ p \mid \phi_i(p) \neq 0 \}$ er lokalt endelig. (lokalt endelig støtte)
\item $\sum_i \phi_i(p) = 1$ for alle $p \in M$.
\item For hver $i$ finnes $U \in \OO$ med $\supp \phi_i \subset U$.
\end{enumerate}

\subsection{Viktigste ting i dette kapitlet}

\begin{enumerate}
\item Definisjon av mangfoldighet.
\item Prop 1.8.
\item Kritiske og regulære verdier.
\item Struktur til avbildninger med gitt rang. 
\item Embedding.
\item Partisjon av enheten.
\end{enumerate}


\section{Tangentbunten}

\subsection{Viktige ting i dette kapitlet}

\begin{itemize}
\item Definisjon av tangentbunten.
\item Definisjon av vektorbunter.
\item Basis for $T_pM$.
\item $\mathscr F$-struktur på $\Gamma(TM)$.
\item Orientering.
\end{itemize}

%%%%%%%%%%%%
\section{Tensorer}

\subsection{Viktige ting}

\begin{itemize}
\item Kotangentbunten og basis for denne.
\item Pullback.
\item 1-1 korrespondanse Theorem 2.
\item 
\end{itemize}

\section{Vektorfelt og differensiallikninger}

\begin{itemize}
\item Lokal eksistens av integralkurver.
\item Lie-derivert av $X$ og $\omega$.
\item Bracket.
\end{itemize}


\section{Differensialformer}

\subsection{de Rham-kohomologi}

%%%
%% -def
% -homotopiinvarians
%%%

La $M$ og $N$ være mangfoldigheter og la $I=[0,1]$ betegne enhetsintervallet. Vi har lyst på et begrep som beskriver hvordan en avbildning kan ``kontinuerlig forandres''. Helt presist:
\begin{defi}
La $f:M \to N$ og $g:M \to N$ være to avbildninger. Da er en \textbf{homotopi} mellom dem en avbildning
$$
F:M \times I \to N
$$
slik at $f_0 = f$ og $f_1=g$, hvor $f_t(p) \stackrel{\Delta}{=} F(p,t)$.
\end{defi}

Nå kan vi formulare nok et begrep. Vi sier at en mangfoldighet $M$ er \textbf{kontraktibel} om identitetsavbildningen og den konstante avbildningen $p \mapsto p_0$ (for en $p_0 \in M$) er homotope. 


\subsection{Poincaré-lemma}

Vi sier at en åpen mengde $U \subset \R^n$ er \textbf{stjerneformet} om det finnes et punkt $p \in U$ slik at for alle andre punkter $q \in U$ er linjestykekt mellom $p$ og $q$ inneholdt i $U$.\footnote{Merk at dette er svakere enn \emph{konveks}, siden i definisjonen av konveks er både $p$ og $q$ frie variable.}

\begin{lemma}
 En stjerneformet mengde $U$ er kontraktibel.
\end{lemma}
\begin{proof}
Velg $p \in U$ som stjernens ``sentrum''. Da kan vi definere en homotopi ved
$$
(q,t) \mapsto q(1-t) + tp
$$
for $q \in M$ og $t \in [0,1]$.
\end{proof}

Poincaré-lemmaet er et lemma som egentlig burde vært et teorem om en tar konsekvensene i betrakning. Det er enkelt å formulere:

\begin{thm}[Poincare-lemma]
 Om $U \subset \R^n$ er en kontraktibel åpen mengde, er $H^k_{dR}(U)=0$ for $k \geq 0$.
\end{thm}

Størstedelen av beviset vil gå med på å analyse avbildningen $i_\alpha:M \to M \times [0,1]$ gitt ved $p \mapsto (p,\alpha)$. 

Merk først at om det er en naturlig funksjon $t$ på $M \times [0,1]$, nemlig projeksjonen $\pi_2$. Om $(x,U)$ er et lokalt koordinatsystem på $M$, er
$$
( x^1 \circ \pi_1, \ldots, x^n \circ \pi_1, t)
$$
et koordinatsystem på $M \times [0,1]$. Vi forkorter $x^i \circ \pi_1$ med $\overline x ^i$. Da kan vi regne ut at
\[
i_\alpha^\ast \left( \sum_{i=1}^n \omega_i d \overline x^i + f dt \right) = \sum_{i=1}^n \omega_i(-,\alpha) dx^i.
\]
En generell $1$-form på $M \times [0,1]$ ser ut som
\[
\omega(p,t) = \sum_{i=1}^n \omega_i(p,t) d \overline x ^i + f(p,t) dt.
\]
Da er $d \omega$ gitt som 
\begin{align*}
dw &= \sum_{i=1}^n \left[ \sum_{j=1}^n   \dd{\omega_i}{\overline x^j} d \overline x^j \wedge d \overline x ^i + \dd{\omega_i}{t} dt \wedge d \overline x^i \right] + \sum_{i=1}^n  \dd{f}{\overline x^k} d \overline x^k \wedge dt \\
&= \text{(ingen dt)}+ \sum_{i=1}^n \left( -\dd{\omega_i}{t}  d \overline x^i \wedge dt  + \dd{f}{\overline x^i} d \overline x^i \wedge dt \right).
\end{align*}

Om vi krever at $d \omega = 0$, ser vi at 
\[
\dd{\omega_i}{t} = \dd{f}{\overline x^i}
\]
for alle $i$. Dermed
\[
\omega_i(p,1)-\omega_i(p,0) = \int_0^1 \dd{\omega_i}{t} dt = \int_0^1 \dd{f}{\overline x^i } dt.
\]

Dermed er
\[
\sum_{i=1}^n \left( \omega_i(p,1) - \omega_i(p,0) \right) = \sum_{i=1}^n\left( \int_0^1 \dd{f}{\overline x ^i } (p,t) dt\right) dx^i.
\]

Om vi definerer
\[
g(p) = \int_0^1 f(p,t) dt,
\]
sier dette at
\[
i_1^\ast \omega - i_0^\ast \omega = dg,
\]
siden $\dd{g}{x^i}(p) = \int_0^1 \dd{f}{\overline x^i} (p,t) dt$. Om vi kaller $g$ for $I\omega$ (det er klart at $g$ er en funksjon av $\omega$), ser formelen slik ut:
\[
i_1^\ast \omega - i_0 ^\ast \omega = d(I\omega). 
\]

Omtrent samme bevis, bare med mer notasjon fungerer mer generelt. Legg først merke til at $\Omega^k(M \times [0,1])$ splitter som en direkte sum. Enhver $\omega \in \Omega^k(M \times [0,1])$ kan deles opp i $\omega_1 \in \ker \pi_1$ og $\omega_2 \in \ker \pi_2$. Bruker man den kanoniske koordinaten $t$ på $M \times [0,1]$, ser vi at $\omega_1$ kan skrives som $dt \wedge \eta$ for en $\eta \in \Omega^{k-1}(M \times [0,1])$. Med andre ord kan enhver $\omega$ skrives som
\[
\omega = \omega_2 + dt \wedge \eta.
\]
Vi definerer dermed
\[
I(\omega)(p) \stackrel{\Delta}{=} \int_0^1 \eta(p,t) dt
\]
For å være heeelt presis, for formelen over gir ikke uten godvilje mening, betyr dette:
\[
I(\omega)(p)(v_1,\ldots,v_{k-1}) = \int_0^1 \eta(p,t)(i_{t\ast}v_1,\ldots, i_{t\ast}v_{k-1}) dt.
\]

I utregningene over ville dette altså vært $g$ (mens $f=\eta$).

Vi har:

\begin{prop}
  For hver $k$-form $\omega \in \Omega^k(M \times [0,1])$ har vi at
\[
i_1^\ast  \omega - i_0^\ast \omega = d(I\omega) + I(d\omega).
\]
I homologisk algebra-språk sier dette at formene $i_1^\ast \omega$ og $i_0^\ast \omega$ er kjedehomotope. De har altså samme verdi på homologi.
\end{prop}

\begin{proof}
 La oss jobbe i et koordinatsystem.

Siden $I$ er lineær har vi to tilfeller å sjekke.
\begin{enumerate}
\item Anta $\omega = f dx^I$, altså at det ikke er noen $dt$ inolvert. Da er $I\omega = 0$. Så høyresiden er bare
\begin{align*}
I(d \omega)(p) &= I\left(\sum_{i=1}^n \dd{f}{d \overline x^i} dx^i \wedge dx^I + \dd{f}{t} dt \wedge dx^I \right)(p) \\
&= \left(\int_0^1 \dd{f}{t} dt  \right) dx^I(p) \\
&= (f(p,1)-f(p,0)) dx^I(p) \\
&= i_1^\ast \omega(p) - i_0^\ast \omega(p).
\end{align*}
Som var det vi skulle vise.
\item Anta $\omega = f dt \wedge dx^I$. Da er $i_1^\ast \omega=i_0^\ast \omega = 0$ siden $i_1^\ast dt = i_0^\ast dt = 0$. Vi har at
  \begin{align*}
d(I\omega) &= d((\int_0^1 f dt) \wedge dx^I) \\
&= \sum_{i=1}^n \dd{}{x^i} \left( \int_0^1 f   dt\right) dx^i \wedge dx^I.
  \end{align*}
Men også
\begin{align*}
I(d \omega) &= I(\sum_{i=1}^n \dd{f}{x^i} dx^i \wedge dt \wedge dx^I) \\
&= - I(\sum_{i=1}^n \dd{f}{x^i} dt \wedge  dx^i \wedge dx^I) \\
&= -\sum_{i=1}^n \left( \int_0^1 \dd{f}{x^i} dt\right) dx^i \wedge dx^I .
\end{align*}
Og vi ser at $I(d\omega) + d(I\omega) =0$.
\end{enumerate}
\end{proof}



\subsection{Viktige punkter i dette kapitlet}
\begin{itemize}
\item Overgangsformler.
\item Kriterie for orienterbarhet.
\item $d$
\item Poincare-lemma
\end{itemize}


\section{Integrasjon}

\begin{itemize}
\item Definisjon.
\item $\partial$
\item Stoke
\item de Rham-kohomologi
\end{itemize}

\section{Alg-top}


%%%%%%%%%%%%%%%%%%%

\section{Obligen}

\begin{exc}
Let $G$ be a Lie group and let $\g$ be its Lie algebra at the identity.

A smooth vector field $X$ on $G$ is called \emph{left-invariant} if $(d_hl_g)(X_h)=X_{gh}$ for all $g,h \in G$. Here $l_g$ is left-multiplication by $g$. A left-invariant vector field is completely determined by its value at $e \in G$. We denote the corresponding vector field by $X^v_g := (d_el_g)(v)$. Every element $v \in T_e G=\g$ arises this way. 

The commutator of two left-invariant vector fields is again left-invariant (a short computation), and hence we get a Lie bracket on $\g$ given by $[v,w] := [X^v, X^w]_e$. 

\begin{enumerate}
\item For $v \in \g$, let $\gamma_v$ be the maximal integral curve of $X^v$ such that $\gamma_v(0)=e$. Show that for every $g \in G$ the curve $\gamma(t)=g \gamma_v(t)$ is an integral curve of $X^v$ such that $\gamma(0)=g$. Conclude that $\gamma_v(t)$ is defined for all $t \in \R$ and the flow $(\phi_t^v)_t$ defined by $X^v$ is given by $\phi_t^v(g)=g\gamma_v(t)$. 

\item Choose a coordinate chart $x:U \to \R^n$ on $G$ containing $e \in G$ such that $x(e)=0$. Let $f:V \times V \to \R^n$, where $V$ is a small neighbourhood of $0 \in \R^n$, and $f$ be the map describing the group law of $G$ in these coordinates, so $f(a,b)=x(x^{-1}(a)x^{-1}(b))$. Consider the Taylor expansion of $f$ at $0 \in \R^{2n}$. Vis at Taylor-rekka har formen
$$
f(a,b) = a+b+B(a,b)+h.o.t.
$$
where $B:\R^n \times \R^n \to \R^n$ is a bilinear map.

\item Show that in the chosen local coordinates the Lie bracket on $\g$ is given by
\[
[v,w] = B(v,w) - B(w,v).
\]
\item Take $G=\GL_n(\R)$. Identify $\g$ with $\mathrm{Mat}_n(\R)$. Show that $[A,B] = AB-BA$.
\end{enumerate}
\end{exc}

\begin{sol}
  \begin{enumerate}
  \item It is clear that $\gamma(0)=g$. We need to see that $\gamma'(t)=X_{\gamma(t)}^v$. The key observation is that $\gamma = l_g \circ \gamma_v$, so that
\begin{align*}
(d\gamma)_t(1) &= d(l_g \circ \gamma_v)_t(1) \\
&= (d l_g)_{\gamma_v(t)} \circ (d \gamma_v)_t(1) \\
&= (d l_g)_{\gamma_v(t)}(X^v_{\gamma_v(t)}) \\
&\stackrel{!}{=} X^v_{g \gamma_v(t)} \\
&= X_{\gamma(t)},
\end{align*}
hvor vi i den merkede likheten har brukt at $X$ er venstreinvariant.

 \item Husk at Taylor-utviklingen er gitt ved å summere alle mulige ledd $\dd{f}{x^{a}} x^{a}$, hvor $a$ er med i $\N^n$. 

Førsteleddet er verdien i origo, som er null. At neste ledd er $a+b$ er ekvivalent med at $\restr{\dd{f}{a^i}}{0} = \restr{\dd{f}{b^i}}{0}= 1$. 

Vi kan regne ut $\dd{f}{a^i}(0)$ fra definisjonen:
\begin{align*}
\dd{f}{a^i}(0) &\stackrel{\Delta}{=} \lim_{h \to 0} \frac 1h \left( x(x^{-1}(he^i)x^{-1}(0))-x(x^{-1}(0)x^{-1}(0)) \right) \\
&=  \lim_{h \to 0} \frac 1h  (he^i-0) = e^i
\end{align*}
siden $x(x^{-1}(0)x^{-1}(0))=x(ee)=x(e)=0$.

Grad 2-leddene kan skrives på formen
\[
\sum \frac{\partial^2 f}{\partial a^i \partial a^j}(0) a^i a^j + 
\sum \frac{\partial^2 f}{\partial a^i \partial b^j}(0) a^i b^j +
\sum \frac{\partial^2 f}{\partial a^i \partial a^j}(0) b^i b^j.
\]
For at dette skal være bilineært i $\vec a$ og $\vec b$, må første og siste-leddet være null. Så vi må vise at 
$$
\restr{\frac{\partial^2 f}{\partial a^i \partial a^j}}{0}= 0.
$$
HER ER JEG LITT STØKK. 
\item 
  \end{enumerate}
\end{sol}


\end{document}