\documentclass[11pt, english]{article}
%\usepackage[latin1]{inputenc}
\usepackage[T1]{fontenc}
\usepackage[utf8]{inputenc}
\usepackage[english]{babel}   % S P R A A K
% \usepackage{graphicx}    % postscript graphics
\usepackage{amssymb, amsmath, amsthm, amssymb} % symboler, osv
\usepackage{mathrsfs}
\usepackage{url}
\usepackage{thmtools}
\usepackage{enumerate}  % lister $  
\usepackage{float}
\usepackage{tikz}
\usepackage{tikz-cd}
\usetikzlibrary{calc}
%\usepackage{tikz-3dplot}
\usepackage{subcaption}
\usepackage[all]{xy}   % for comm.diagram
\usepackage{wrapfig} % for float right
%\usepackage{hyperref}
 \usepackage{mystyle} % stilfilen      
\usepackage{wasysym}

\begin{document}
\title{Seminarnotater}
\author{Fredrik Meyer}
\maketitle 

\section{Innledning}

Vi husker litt notasjon.
\begin{itemize}
\item $\Gamma$ er en gruppe som virker på $\Hh$, det øvre halvplanet.
\item Vanligste eksemplene er $\Gamma(1)=\SL_2(\Z)$, og kongruensundergruppene av denne: $\Gamma(N)$.
\end{itemize}


\textbf{HUSK}

\begin{enumerate}
\item \textbf{Modulær funksjon}: modulære funksjoner er funksjoner invariant under $\Gamma$. Det er ikke nødvendigvis spesielt mange av disse. $f(\gamma z)=f(\gamma)$ for alle $\gamma \in \Gamma$. Vi krever at de er meromorfe på $\Hh$ og på "køspene".
\item \textbf{Modulær form}: modulære former er som "brøker", evt. som homogene polynomer på $\PP^N$. Så gitt en gruppe $\Gamma$, så er en modulær form for $\Gamma$ av vekt $2k$ gitt ved en funksjon på $\Hh$ slik at 1) $f(\gamma z)=(cz+d)^{2k}f(z)$ for $z \in \Hh$ og $\gamma=\begin{pmatrix} a & b \\ c & d \end{pmatrix} \in \Gamma$. Vi krever at $f$ er holomorf på $\Hh$ og på køspene. 
\item Litt notasjon: $\mathcal M_k(\Gamma)$ er vektorrommet av modulære former av vekt $2k$ for $\Gamma$. $\mathcal S_k(\Gamma)$ er underrommet av køspformer (=null på køspene). Ved multiplikasjon av modulære former ser vi at 
$$
\mathcal M( \Gamma) = \bigoplus_{k \geq 0} \mathcal M_k(\Gamma)
$$
er en gradert ring. Kristian nevnte sist at 
\[
\dim \mathcal M_k(\Gamma) = \begin{cases} 0 & k \leq -1 \\
1 & k = 0 \\
(2k-1)(g-1) + \nu_\infty k + \sum_P k\left[1-\frac{1}{e_P}\right] & k \geq 1
\end{cases}
\]
Her er $\nu_\infty$ antall ikke-ekvivalente køsper. Summen går over representanter for elliptiske punkter $P$ av $\Gamma$. $e_P$ er orden til en eller annen stabilisator... 
\item Det viste seg at
\[
\mathcal M (\Gamma(1)) \simeq \C[T^2,T^3].
\]
(hehe, køsp på nytt)
\end{enumerate}

På dette tidspunktet regner Milne ut Fourier-koffesientene for Eisenstein-serien til $\Gamma(1)$. Jeg tror vi hopper over dette.

Vi kan vel nevne resultatet og \textbf{betrakte} sammenhengen med tallteori. La 
$$
\sigma_k(n) = \sum_{d \mid n} d^k.
$$
Da er (PROPOSISJON!!)
\[
G_k(z) = 2\zeta(2k) + 2\frac{(2\pi i) ^{2k}}{(2k-1)!} \sum_{n=1}^\infty \sigma_{2k-1}(n) q^n.
\]
(trommevirvel)


\section{Modulære former som seksjoner av linjebunter}

\textbf{TERMINOLOGI:}

La $X$ være en kompleks mangfoldighet. Da er en \textbf{linjebunt} på $X$ gitt ved en avbildning $\pi:L \to X$ slik at for en overdekning $\{ U_i \}$ av $X$, har vi at $\pi^{-1}(U_i) \simeq U_i \times \C$. 

For $U \subset X$, la  $\Gamma(U,L)$ betegne mengden av seksjoner av $\pi$ over $U$. For den trivielle linjebunten er dette bare holomorfe funksjoner.

Betrakt følgende situasjon:

La $\Gamma$ være en diskret gruppe som virker fritt og "ekte diskontinuerlig" på en Riemann-flate $H$. La $X = \Gamma \bs H$.

La $\pi:L \to X$ være en linjebunt på $X$. Da er 
$$
p^\ast (L) = \{ (h,l) \subset H \times L \mid p(h) = \pi(l) \}
$$
en linjebunt på $H$ (pullback).
\[
\xymatrix{
p^\ast (L) \pullbackcorner \ar[r] \ar[d] & H \ar[d]^p \\
L \ar[r]_\pi & X
}
\]

Dette kan sjekkes lokalt på en overdekning som trivialiserer både $\pi$ og $p$ (finnes det en mer kategorisk metode???). 

Anta gitt en isomorfi $i: H \times \C \to p^\ast (L)$. Da kan vi overføre virkningen av $\Gamma$ på $p^\ast (L)$ til en virkning av $\Gamma$ på $H \times \C$ over $H$.  La $(t,z) \in H \times \C$. Vi skriver:

$$
\gamma \cdot (t,z) = \left( \gamma t , j_\gamma(t)z \right)
$$
hvor $j_\gamma(t) \in \C^\X$. 

Da er 
$$
\gamma \gamma' (t,z) = \gamma(\gamma' t, j_{\gamma '}(t)z ) = (\gamma \gamma' t, j_{\gamma}(\gamma' t) j_{\gamma'}(t) z).
$$

Så
$$
j_{\gamma \gamma'}(t) = j_\gamma(\gamma' t) j_{\gamma'}(t).
$$

En funksjon $j:\Gamma \times \Hh \to \C^\X$ som dette som er holomorf kalles for en \textbf{"automorfisk faktor"}.

\begin{example}
Enhver åpen delmengde av $\C$ med en gruppevirkning fra $\Gamma$ kommer med en kanonisk automorfisk faktor $j_\gamma(t)$, nemlig:
\[
\Gamma \times H \to \C, \qquad (\gamma, t) \mapsto (d\gamma)_t \in \C.
\]
I ord: $\gamma$ induserer en avbildning. Tangentrommet til $\C$ er $\C$ selv, så differensialen er bare gitt ved å multiplisere med et komplekst tall. 

At dette er en automorfisk faktor følger fra kjerneregelen! Prøv selv :)

EKSEMPEL: Se på $\Gamma(1)$ som virker på $\Hh$. Om $\gamma$ sender $z$ til $\frac{az+b}{cz+d}$ følger det at 
$$
d\gamma = \frac{1}{(cz+d)^2} dz,
$$
så $j_\gamma(t)= (cz+d)^{-2}$ og $j_\gamma(t)^k= (cz+d)^{-2k}$. 
\end{example}

Vi har følgende:
\begin{prop}
Det er en 1-1-korrespondanse mellom par $(L,i)$ hvor $L$ er en linjebunt på $\Gamma \bs H$ og $i$ er en isomorfi $H \times \C \simeq p^\ast L$ og mengden av automorfiske faktorer.
\end{prop}
\begin{proof}
Vi har sett hvordan vi går fra $(L,i) \mapsto j_\gamma(t)$. 

Gitt en automorfisk faktor $j$, bruk denne til å definere en virkning av $\Gamma$ på $H \times \C$, og la $L$ være gitt ved $\Gamma \bs H \times \C$.
\end{proof}

Siden alle linjebunter på $\Hh$ er trivielle (finnes det et kort bevis for dette?), har vi en "klassifikasjon" av linjebunter på $\Gamma \bs \Hh$. Den trivielle linjebunten svarer til $j=1$. 


--

Merk at
\[
\Gamma(X,L) = \{ s \in \Gamma(H, p^\ast L) \mid \text{ s kommuterer med $\Gamma$} \}.
\]
Anta gitt en isomorfo $p^\ast L \simeq H \times \C$. Dermed identifiserer vi $p^\ast L$ med $H \times \C$. En seksjon $s:H \to H \times \C$ kan skrives $s(t)=(t,f(t))$. $\Gamma$ virker på $H \times \C$ ved $\gamma(t,z) = (\gamma t, j_\gamma(t)z)$ for en automorfisk faktor $j_\gamma$. Da er kravet om kommutativitet gitt ved $s(\gamma t)=\gamma s(t)$. Eksplisitt:
\[
s(\gamma t) = s(\gamma t, f(\gamma t)) \stackrel{!}{=} \gamma s(t) = (\gamma t, j_\gamma(t) s(t)).
\]

Dermed er kravet $s(\gamma t) = j_\gamma(t) f(t)$.

La nå $H= \Hh$ og la  $L_k$ betegne linjebunten som korresponderer til $j_\gamma^{-k}$ der $j_\gamma$ er den kanoniske automorfiske faktoren for $\Gamma$ på $\Hh$. Da blir betingelsen
\[
f(\gamma t) = (ct+d)^{2k} f(t).
\]
Så seksjoner av $L_k$ er i 1-1 korrespondanse med modulære former av vekt $2k$ på $\Hh$. 

Nå sier Milne at linjebunten $L_k$ på $\Gamma \bs \Hh$ utvides til en linjebunt på $\Gamma \bs \Hh^\ast$ (kompaktifiseringen) (hvorfor???). Og dermed er seksjoner av $L_k^\ast$ i 1-1 korrespondanse med modulære former av vekt $2k$. 

Så disse formene har en viss geometrisk betydning.

\subsection{Poincaré-rekker}

Vi ønsker å konstruere modulære former.

Dette er litt samme regla som før. Først lager en en uendelig rekke som ser ut til å ha de riktige egenskapene. Så ser vi at konvergens er et problem. Så må vi massere litt, og så funker det.

Vi gjør dette mer generelt for automorfiske faktorer. Vi ønsker funksjoner slik at $f(\gamma z) = j_\gamma (z) f(z)$. 

Prøv med (la $\Gamma' = \Gamma/\pm I$)
\[
f(z) \stackrel{??}{=} \sum_{\gamma \in \Gamma'} \frac{h(\gamma z)}{j_\gamma(z)}.
\]

Her er nemlig
 \[
 f(\gamma' z) = \sum_{\gamma \in \Gamma'} \frac{h(\gamma \gamma' z)}{j_\gamma (\gamma' z)} = \sum_{\gamma \in \Gamma'} \frac{h(\gamma \gamma' z)}{j_{\gamma \gamma'}(t)} \cdot j_{\gamma'} (z)  = j_{\gamma'} (t) f(z).
\] 
Så denne ville vært en fin kandidat! Men det er liten sjangs for konvergens siden vi kan ha $j_\gamma(t) \equiv 1$ for uendelig mange $\gamma$. Nemlig fikspunkter! 

La så 
\[
\Gamma_0 = \{ \gamma \in \Gamma' \mid j_\gamma(z) \equiv 1 \}. 
\]

For $\Gamma(1)$ er dette gruppen generert av $\begin{pmatrix} 1 & h \\ 0 & 1 \end{pmatrix}$.

$\Gamma_0$ er en undergruppe av $\Gamma$.

Anta nå $h:\Hh \to \C$ er invariant under $\Gamma_0$. La $\gamma \in \Gamma'$ og $\gamma_0 \in \Gamma_0$. Da er 
\[
\frac{h(\gamma_0 \gamma z)}{j_{\gamma_0 \gamma}(z)} = \frac{h(\gamma z)}{j_{\gamma_0}(\gamma z) j_\gamma(z)} = \frac{h(\gamma z)}{j_\gamma(z)}.
\]

Så uttrykket er konstant på kosetter. Dermed ser vi heller på summen
\[
f(z) = \sum_{\gamma \in \Gamma'/\Gamma_0} \frac{h(\gamma z)}{j_\gamma(z)}
\]
hvor summen går over et sett med representanter fra hvert kosett i kvotiengruppa. Denne har ihvertfall større håp om å konvergere.


Bruk dette på $j_\gamma(z) = (cz+d)^{2k}$ og $\Gamma$ en undergruppe av endelig indeks i $\Gamma(1)$. Da er $\Gamma_0$ generert av $z \mapsto z+h$ for en $ h \in \Z$. Én slik funksjon er $\exp(2 \pi i nz/h)$. Dermed:

\begin{defi}
Vi definerer \textbf{Poincaré-rekka} av vekt 2k og karakter $n$ for $\Gamma$ til å være 
\[
\varphi_n(z) = \sum_{\gamma \in \Gamma_0 \bs \Gamma'} \frac{\exp(\frac{2 \pi i \gamma (z)}{h})}{(cz+d)^{2k}}.
\]
\end{defi}
*puste lettet ut*

Da er det et teorem (uten bevis  \smiley ) som sier at $\phi_n$ konvergerer absolutt uniformt på kompakter på $\Hh$. Så vi har en hel rekke med modulære former av vekt $2k$ for $\Gamma$. 

I tillegg:

\begin{enumerate}
\item $\varphi_0(z) = \sum \frac{1}{(cz+d)^k}$ er null på alle endelige køsper og $\phi_0(\infty)=1$. [[er dette sant? Regne på tavla??]]
\item For alle $n \geq 1$, så er $\varphi_n(z)$ en køspform. (null på køspene)
\end{enumerate}

\subsection{Litt om geometrien til $\Hh$ }

\begin{enumerate}
\item $\Hh$ kan være modelle for ikke-euklidsk hyperbolsk geometri.
\item For hver linje og punkt utenfor denne finnes det minst to linjer parallell til denne.
\item Linjer er halvsirkler som treffer den reelle aksen.
\item Lengden mellom to punkter er gitt ved 
\[
\delta(z_1,z_2) = \log D(z_1,z_2, \infty_1,\infty_2)
\]
hvor $D$ er kryssforholdet og $\infty_1,\infty_2$ er punktene på reelle aksene på linja mellom $z_1,z_2$.
\item Isometrigruppa er $\mathrm{PSL}_2(\R)$.
\item Mål: $\sigma(U) \iint_U \frac{dx dy}{y^2}$. 
\item Vi kan se på $\int_D \frac{dx dy}{y^2}$ for ethvert fundamentalområde for enhver $\Gamma$. Da er volumet gitt ved
\[
\int_D dx dy/y^2 = 2 \pi (2g-2+\nu_\infty + \sum (1-1/e_P)).
\]
\end{enumerate}
\subsection{Indreprodukt + utspenning}

La $f,g$ være modulære former av vekt $2k$ for $\Gamma$.

\begin{lemma}
  Differensialen $f(z) \overline{g(z)} y^{2k-2} dx dy$ er invariant under virkningen av $\SL_2(\R)$. 
\end{lemma}
\begin{proof}
 Utregning.
\end{proof}

Om $f,g$ er køspformer så vil integralet
\[
\iint_D f(z) \overline{g(z)} y^{2k-2} dx dy
\]
konvergere.

Dette lar oss definere \textbf{Petersson-indreproduktet} på rommet av køspformer:
\[
\langle f, g \rangle \stackrel{\Delta}{=} \iint_D f(z) \overline{g(z)} y^{2k-2} dx dy.
\]
Siden differensialen er invariant, så er dette uavhengig av $D$. (fundamentalområdet)

Dette er faktisk en positiv-definitt Hermitisk form på $\mathcal S_k(\Gamma)$. (end.dim Hilbert-rom!)
\begin{enumerate}
\item Lineært i første variabel, og semi-lineært i andre.
\item $\langle f, g \rangle = \overline{\langle g, f \rangle}$.
\item $\langle f, f \rangle > 0$ for $f \neq 0$.
\end{enumerate}

\begin{thm}
  La $f$ være en køspform av vekt $2k$ og la $\varphi_n$ være Poincaré-rekka av vekt $2k$ og karakter $n$. Da er 
\[
\langle f, \varphi_n \rangle = \frac{h^{2k} (2k-2)!}{(4\pi)^{2k-1}} n^{1-2k} a_n.
\]
Her er $a_n$ den n'te koeffisienten i Fourier-utviklingen av $f$.
\end{thm}
Hopper over beviset.

\textbf{Korollar:} Poincaré-rekkene utspenner køsp-formene $\mathcal S_k(\Gamma)$.

Bevis: Hvis ikke finnes køsp-form som er ortogonal til alle Poincaré-rekkene. Men da må alle Fourier-koeffisientene være null. Funker dårlig!

[[ Milne skriver også noe om at disse utspenner $\mathcal M_k(\Gamma)$, men jeg ser ikke hvordan dette følger! ]]
\subsection{Restrikterte Eisenstein-serier}

Hopper over. (faktisk er $\mathcal M_k(\Gamma) = \mathrm{Eisenstein-series} \oplus \mathcal S_k(\Gamma)$ (ortogonalt komplement))

\end{document}