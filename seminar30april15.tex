\documentclass[11pt, english]{article}
%\usepackage[latin1]{inputenc}
\usepackage[T1]{fontenc}
\usepackage[utf8]{inputenc}
\usepackage[english]{babel}   % S P R A A K
% \usepackage{graphicx}    % postscript graphics
\usepackage{amssymb, amsmath, amsthm, amssymb} % symboler, osv
\usepackage{mathrsfs}
\usepackage{url}
\usepackage{thmtools}
\usepackage{enumerate}  % lister $  
\usepackage{float}
\usepackage{tikz}
\usepackage{tikz-cd}
\usetikzlibrary{calc}
%\usepackage{tikz-3dplot}
\usepackage{subcaption}
\usepackage[all]{xy}   % for comm.diagram
\usepackage{wrapfig} % for float right
%\usepackage{hyperref}
 \usepackage{mystyle} % stilfilen      


\begin{document}
\title{Seminarnotater}
\author{Fredrik Meyer}
\maketitle 

\section{Innledning}

Vi husker litt notasjon.
\begin{itemize}
\item $\Gamma$ er en gruppe som virker på $\Hh$, det øvre halvplanet.
\item Vanligste eksemplene er $\Gamma(1)=\SL_2(\Z)$, og kongruensundergruppene av denne: $\Gamma(N)$.
\end{itemize}


\textbf{HUSK}

\begin{enumerate}
\item \textbf{Modular functions}: modulære funksjoner er funksjoner invariant under $\Gamma$. Det er ikke nødvendigvis spesielt mange av disse. $f(\gamma z)=f(\gamma)$ for alle $\gamma \in \Gamma$. Vi krever at de er meromorfe på $\Hh$ og på "køspene".
\item \textbf{Modular forms}: modulære former er som "brøker", evt. som homogene polynomer på $\PP^N$. Så gitt en gruppe $\Gamma$, så er en modulær form for $\Gamma$ av vekt $2k$ gitt ved en funksjon på $\Hh$ slik at 1) $f(\gamma z)=(cz+d)^{2k}f(z)$ for $z \in \Hh$ og $\gamma=\begin{pmatrix} a & b \\ c & d \end{pmatrix} \in \Gamma$. Vi krever at $f$ er holomorf på $\Hh$ og på køspene. 
\item Litt notasjon: $\mathcal M_k(\Gamma)$ er vektorrommet av modulære former av vekt $2k$ for $\Gamma$. $\mathcal S_k(\Gamma)$ er underrommet av køspformer (=null på køspene). Ved multiplikasjon av modulære former ser vi at 
$$
\mathcal M( \Gamma) = \bigoplus_{k \geq 0} \mathcal M_k(\Gamma)
$$
er en gradert ring. Kristian nevnte sist at 
\[
\dim \mathcal M_k(\Gamma) = \begin{cases} 0 & k \leq -1 \\
1 & k = 0 \\
(2k-1)(g-1) + \nu_\infty k + \sum_P k\left[1-\frac{1}{e_P}\right] & k \geq 1
\end{cases}
\]
Her er $\nu_\infty$ antall ikke-ekvivalente køsper. Summen går over representanter for elliptiske punkter $P$ av $\Gamma$. $e_P$ er orden til en eller annen stabilisator... 
\item Det viste seg at
\[
\mathcal M (\Gamma(1)) \simeq \C[T^2,T^3].
\]
(hehe, køsp på nytt)
\end{enumerate}

På dette tidspunktet regner Milne ut Fourier-koffesientene for Eisenstein-serien til $\Gamma(1)$. Jeg tror vi hopper over dette.

Vi kan vel nevne resultatet og \textbf{betrakte} sammenhengen med tallteori. La 
$$
\sigma_k(n) = \sum_{d \mid n} d^k.
$$
Da er (PROPOSISJON!!)
\[
G_k(z) = 2\zeta(2k) + 2\frac{(2\pi i) ^{2k}}{(2k-1)!} \sum_{n=1}^\infty \sigma_{2k-1}(n) q^n.
\]
(trommevirvel)


\section{Modulære former som seksjoner av linjebunter}

\textbf{TERMINOLOGI:}

La $X$ være en kompleks mangfoldighet. Da er en \textbf{linjebunt} på $X$ gitt ved en avbildning $\pi:L \to X$ slik at for en overdekning $\{ U_i \}$ av $X$, har vi at $\pi^{-1}(U_i) \simeq U_i \times \C$. 

For $U \subset X$, la  $\Gamma(U,L)$ betegne mengden av seksjoner av $\pi$ over $U$. For den trivielle linjebunten er dette bare holomorfe funksjoner.

Betrakt følgende situasjon:

La $\Gamma$ være en diskret gruppe som virker fritt og "ekte diskontinuerlig" på en Riemann-flate $H$. La $X = \Gamma \bs H$.

La $\pi:L \to X$ være en linjebunt på $X$. Da er 
$$
p^\ast (L) = \{ (h,l) \subset H \times L \mid p(h) = \pi(l) \}
$$
en linjebunt på $H$ (pullback).
\[
\xymatrix{
p^\ast (L) \pullbackcorner \ar[r] \ar[d] & H \ar[d]^p \\
L \ar[r]_\pi & X
}
\]

Dette kan sjekkes lokalt på en overdekning som trivialiserer både $\pi$ og $p$ (finnes det en mer kategorisk metode???). 

Anta gitt en isomorfi $i: H \times \C \to p^\ast (L)$. Da kan vi overføre virkningen av $\Gamma$ på $p\ast (L)$ til en virkning av $\Gamma$ på $H \times \C$ over $H$.  La $(t,z) \in H \times \C$. Vi skriver:

$$
\gamma \cdot (t,z) = \left( \gamma t , j_\gamma(t)z \right)
$$
hvor $j_\gamma(t) \in \C^x$. 

Da er 
$$
\gamma \gamma' (t,z) = \gamma(\gamma' t, j_{\gamma '}(t)z ) = (\gamma \gamma' t, j_{\gamma}(\gamma' t) j_{\gamma'}(t) z).
$$

Så
$$
j_{\gamma \gamma'}(t) = j_\gamma(\gamma' t) j_{\gamma'}(t).
$$

En funksjon $j:\Gamma \times \Hh \to \C^x$ som dette som er holomorf kalles for en \textbf{"automorfisk faktor"}.

\begin{example}
Enhver åpen delmengde av $\C$ med en gruppevirkning fra $\Gamma$ kommer med en kanonisk automorfisk faktor $j_\gamma(t)$, nemlig:
\[
\Gamma \times H \to \C, (\gamma, t) \mapsto (d\gamma)_t.
\]
I ord: $\gamma$ induserer en avbildning. Tangentrommet til $\C$ er $\C$ selv, så differensialen er bare gitt ved å multiplisere med et komplekst tall. 

At dette er en automorfisk faktor følger fra kjerneregelen! Prøv selv :)

EKSEMPEL: Se på $\Gamma(1)$ som virker på $\Hh$. Om $\gamma$ sender $z$ til $\frac{az+b}{cz+d}$ følger det at 
$$
d\gamma = \frac{1}{(cz+d)^2} dz,
$$
så $j_\gamma(t)= (cz+d)^{-2}$ og $j_\gamma(t)^k= (cz+d)^{-2k}$. 
\end{example}

Vi har følgende:
\begin{prop}
Det er en 1-1-korrespondanse mellom par $(L,i)$ hvor $L$ er en linjebunt på $\Gamma \bs H$ og $i$ er en isomorfi $H \times \C \simeq p^\ast L$ og mengden av automorfiske faktorer.
\end{prop}
\begin{proof}
Vi har sett hvordan vi går fra $(L,i) \mapsto j_\gamma(t)$. 

Gitt en automorfisk faktor $j$, bruk denne til å definere en virkning av $\Gamma$ på $H \times \C$, og la $L$ være gitt ved $\Gamma \bs H \times \C$.
\end{proof}

Siden alle linjebunter på $\Hh$ er trivielle, har vi en "klassifikasjon" av linjebunter på $\Gamma \bs \Hh$. Den trivielle linjebunten svarer til $j=1$. 


[[
Korrespondanse mellom seksjoner av $L_k$ og modulære former av vekt $2k$.
]]

\subsection{Poincaré-rekker}

...

\subsection{Litt om geometrien til H}


\subsection{Indreprodukt + utspenning}




\end{document}