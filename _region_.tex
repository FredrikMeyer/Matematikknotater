\message{ !name(mangnotater.tex)}\documentclass[11pt, english]{article}
%\usepackage[latin1]{inputenc}
\usepackage[T1]{fontenc}
\usepackage[utf8]{inputenc}
\usepackage[english]{babel}   % S P R A A K
% \usepackage{graphicx}    % postscript graphics
\usepackage{amssymb, amsmath, amsthm, amssymb} % symboler, osv
\usepackage{mathrsfs}
\usepackage{url}
\usepackage{thmtools}
\usepackage{enumerate}  % lister $  
\usepackage{float}
\usepackage{tikz}
\usepackage{tikz-cd}
\usetikzlibrary{calc}
%\usepackage{tikz-3dplot}
\usepackage{subcaption}
\usepackage[all]{xy}   % for comm.diagram
\usepackage{wrapfig} % for float right
\usepackage{hyperref}
\usepackage{mystyle} % stilfilen      

%\usepackage[a5paper,margin=0.5in]{geometry}


\begin{document}

\message{ !name(mangnotater.tex) !offset(236) }
\begin{exc}
Let $G$ be a Lie group and let $\g$ be its Lie algebra at the identity.

A smooth vector field $X$ on $G$ is called \emph{left-invariant} if $(d_hl_g)(X_h)=X_{gh}$ for all $g,h \in G$. Here $l_g$ is left-multiplication by $g$. A left-invariant vector field is completely determined by its value at $e \in G$. We denote the corresponding vector field by $X^v_g := (d_el_g)(v)$. Every element $v \in T_e G=\g$ arises this way. 

The commutator of two left-invariant vector fields is again left-invariant (a short computation), and hence we get a Lie bracket on $\g$ given by $[v,w] := [X^v, X^w]_e$. 

\begin{enumerate}
\item For $v \in \g$, let $\gamma_v$ be the maximal integral curve of $X^v$ such that $\gamma_v(0)=e$. Show that for every $g \in G$ the curve $\gamma(t)=g \gamma_v(t)$ is an integral curve of $X^v$ such that $\gamma(0)=g$. Conclude that $\gamma_v(t)$ is defined for all $t \in \R$ and the flow $(\phi_t^v)_t$ defined by $X^v$ is given by $\phi_t^v(g)=g\gamma_v(t)$. 

\item Choose a coordinate chart $x:U \to \R^n$ on $G$ containing $e \in G$ such that $x(e)=0$. Let $f:V \times V \to \R^n$, where $V$ is a small neighbourhood of $0 \in \R^n$, and $f$ be the map describing the group law of $G$ in these coordinates, so $f(a,b)=x(x^{-1}(a)x^{-1}(b))$. Consider the Taylor expansion of $f$ at $0 \in \R^{2n}$. Show that
$$
a+b+B(a,b)+h.o.t.
$$
where $B:\R^n \times \R^n \to \R^n$ is a bilinear map.

\item Show that in the chosen local coordinates the Lie bracket on $\g$ is given by
\[
[v,w] = B(v,w) - B(w,v).
\]
\item Take $G=\GL_n(\R)$. Identify $\g$ with $\mathrm{Mat}_n(\R)$. Show that $[A,B] = AB-BA$.
\end{enumerate}
\end{exc}


\message{ !name(mangnotater.tex) !offset(236) }

\end{document}