\message{ !name(reptheory.tex)}\documentclass[11pt, english]{article}
%\documentclass[12pt]{article}  
%\usepackage[papersize={108mm,144mm},margin=2mm]{geometry}  
\sloppy 
%\pagestyle{empty} 
%\usepackage[scaled]{helvet}
%\renewcommand{\familydefault}{\sfdefault}

\usepackage[T1]{fontenc}
\usepackage[utf8]{inputenc}
\usepackage[english]{babel}   % S P R A A K

\usepackage{amssymb, amsmath, amsthm, amssymb} % symboler, osv
\usepackage{mathrsfs,calligra}
\usepackage{url}
\usepackage{thmtools}
\usepackage{enumerate}  % lister $
\usepackage{float}
\usepackage{tikz}

\usepackage{young} 
\usepackage{youngtab} 

\usepackage[all]{xy}   % for comm.diagram
\usepackage{wrapfig} % for float right
\usepackage[colorlinks=true]{hyperref}
\usepackage{mystyle} % stilfilen      


\title{Representation theory}
\author{Fredrik Meyer} 
\date{}
\begin{document}

\message{ !name(reptheory.tex) !offset(1665) }
 passing through a given point:
$$
N = \{ p \in U \, | \, x^{j}(p) = x^{j}(p_0) \text{ for } j=k+1,\ldots,n \}.
$$
Uniqueness means that if $N'$ is another integral manifold passing through $p$, then $N \cap N'$ is open in $N$ and $N'$. 

This implies that every point lies in a unique maximal connected integral manifoold, called \textbf{a leaf of the foliation defined by $D$}.

\subsection{Back to Lie subgroups}

\begin{thm}
Let $G$ be a Lie group with Lie algebra $\g$. Let $\mathfrak h \subset \g$ be a Lie subalgebra.

Then there exists a unique Lie subgroup $H \subset G$ with Lie algebra $\mathfrak h$.
\end{thm}
\begin{proof}
  Consider the distribution $D$ on $G$ defined by $D_g = (d_el_g)(\mathfrak h)$. As $\mathfrak h$ is a Lie subalgebra of $\g$, it is integrable.

Since left translations map $D$ into itself, they map leaves into leaves of the corresponding foliation. 

It follows that if $H$ is the leaf passing through $e$, then $H=hH$ for all $h \in H$. Therefore $H \subset G$ is a subgroup. It is a Lie subgroup because it is locally closed.

Now, if $K \subset G$ is another connected Lie subgroup with Lie algebra $\mathfrak h$, then $K$ would be an integral manifold of $D$ passing through $e$. Thus $K$ is an open subgroup. But $K$ is closed as well, since
$$
K = H \bs \left(\bigcup_{h \not \ in K} h K\right).
$$
As $H$ is connected, $H \subset K$.
\end{proof}

\begin{prop}
  Assume $\pi:G \to H$ is a Lie group homomorphism with $G$ connected. Then $\pi$ is completely determined by $\pi_\ast: \g \to \mathfrak h$. 
\end{prop}

\begin{proof}
  Consider the subgroup $K$ generated by $\exp \g \subset G$. Since $\exp \g$ contains an open neighbourhood of $e$, $K$ is an open subgroup of $G$. But this implies that $K$ is closed, hence $K=G$.

Thus, the statement is proved since  $\pi(\exp X) = \exp \pi_\ast X$, and every element of $G$ is a product of terms looking like $\exp X$.
\end{proof}


\message{ !name(reptheory.tex) !offset(2728) }

\end{document}