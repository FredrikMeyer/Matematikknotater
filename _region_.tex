\message{ !name(seminar30april15.tex)}\documentclass[11pt, english]{article}
%\usepackage[latin1]{inputenc}
\usepackage[T1]{fontenc}
\usepackage[utf8]{inputenc}
\usepackage[english]{babel}   % S P R A A K
% \usepackage{graphicx}    % postscript graphics
\usepackage{amssymb, amsmath, amsthm, amssymb} % symboler, osv
\usepackage{mathrsfs}
\usepackage{url}
\usepackage{thmtools}
\usepackage{enumerate}  % lister $  
\usepackage{float}
\usepackage{tikz}
\usepackage{tikz-cd}
\usetikzlibrary{calc}
%\usepackage{tikz-3dplot}
\usepackage{subcaption}
\usepackage[all]{xy}   % for comm.diagram
\usepackage{wrapfig} % for float right
%\usepackage{hyperref}
 \usepackage{mystyle} % stilfilen      


\begin{document}

\message{ !name(seminar30april15.tex) !offset(74) }
virkning av $\Gamma$ på $H \times \C$ over $H$.  La $(t,z) \in H \times \C$. Vi skriver:

$$
\gamma \cdot (t,z) = \left( \gamma t , j_\gamma(t)z \right)
$$
hvor $j_\gamma(t) \in \C^x$. 

Da er 
$$
\gamma \gamma' (t,z) = \gamma(\gamma' t, j_{\gamma '}(t)z ) = (\gamma \gamma' t, j_{\gamma}(\gamma' t) j_{\gamma'}(t) z).
$$

Så
$$
j_{\gamma \gamma'}(t) = j_\gamma(\gamma' t) j_{\gamma'}(t).
$$

En funksjon $j:\Gamma \times \Hh \to \C^x$ som dette som er holomorf kalles for en \textbf{"automorfisk faktor"}.

\begin{example}
Enhver åpen delmengde av $\C$ med en gruppevirkning fra $\Gamma$ kommer med en kanonisk automorfisk faktor $j_\gamma(t)$, nemlig:
\[
\Gamma \times H \to \C, (\gamma, t) \mapsto (d\gamma)_t.
\]
I ord: $\gamma$ induserer en avbildning. Tangentrommet til $\C$ er $\C$ selv, så differensialen er bare gitt ved å multiplisere med et komplekst tall. 

At dette er en automorfisk faktor følger fra kjerneregelen! Prøv selv :)

EKSEMPEL: Se på $\Gamma(1)$ som virker på $\Hh$. Om $\gamma$ sender $z$ til $\frac{az+b}{cz+d}$ følger det at 
$$
d\gamma = \frac{1}{(cz+d)^2} dz,
$$
så $j_\gamma(t)= (cz+d)^{-2}$ og $j_\gamma(t)^k= (cz+d)^{-2k}$. 
\end{example}

Vi har følgende:
\begin{prop}
Det er en 1-1-korrespondanse mellom par $(L,i)$ hvor $L$ er en linjebunt på $\Gamma \bs H$ og $i$ er en isomorfi $H \times \C \simeq p^\ast L$ og mengden av automorfiske faktorer.
\end{prop}
\begin{proof}
Vi har sett hvordan vi går fra $(L,i) \mapsto j_\gamma(t)$. 

Gitt en automorfisk faktor $j$, bruk denne til å definere en virkning av $\Gamma$ på $H \times \C$, og la $L$ være gitt ved $\Gamma \bs H \times \C$.
\end{proof}

Siden alle linjebunter på $\Hh$ er trivielle (finnes det et kort bevis for dette?), har vi en "klassifikasjon" av linjebunter på $\Gamma \bs \Hh$. Den trivielle linjebunten svarer til $j=1$. 


--

Merk at
\[
\Gamma(X,L) = \{ s \in \Gamma(H, p^\ast L) \mid \text{ s kommuterer med $\Gamma$} \}.
\]
Anta gitt en isomorfo $p^\ast L \simeq H \times \C$. Dermed identifiserer vi $p^\ast L$ med $H \times \C$. En seksjon $s:H \to H \times \C$ kan skrives $s(t)=(t,f(t))$. $\Gamma$ virker på $H \times \C$ ved $\gamma(t,z) = (\gamma t, j_\gamma(t)z)$ for en automorfisk faktor $j_\gamma$. Da er kravet om kommutativitet gitt ved $s(\gamma t)=\gamma s(t)$. Eksplisitt:
\[
s(\gamma t) = s(\gamma t, f(\gamma t)) \stackrel{!}{=} \gamma s(t) = (\gamma t, j_\gamma(t) s(t)).
\]

Dermed er kravet $s(\gamma t) = j_\gamma(t) f(t)$.

La nå $H= \Hh$ og la  $L_k$ betegne linjebunten som korresponderer til $j_\gamma^{-k}$ der $j_\gamma$ er den kanoniske automorfiske faktoren for $\Gamma$ på $\Hh$. Da blir betingelsen
\[
f(\gamma t) = (cz+d)^{2k} f(t).
\]
Så seksjoner av $L_k$ er i 1-1 korrespondanse med modulære former av vekt $2k$ på $\Hh$. 

Nå sier Milne at linjebunten $L_k$ på $\Gamma \bs \Hh$ utvides til en linjebunt på $\Gamma \bs Hh^\ast$ (kompaktifiseringen) (hvorfor???). Og dermed er seksjoner av $L_k^\ast$ i 1-1 korrespondanse med modulære former av vekt $2k$. 

(så disse formene har en viss geometrisk betydning)

\subsection{Poincaré-rekker}

Vi ønsker å konstruere modulære former.

Dette er litt samme regla som før. Først lager en en uendelig rekke som ser ut til å ha de riktige egenskapene. Så ser vi at konvergens er et problem. Så må vi massere litt, og så funker det.

Vi gjør dette mer generelt for automorfiske faktorer. Vi ønsker funksjoner slik at $f(\gamma t z) = j_\gamma (z) f(z)$. 

Prøv med (la $\Gamma' = \Gamma/\pm I$)
\[
f(z) \stackrel{??}{=} \sum_{\gamma \in \Gamma'} \frac{h(\gamma z)}{j_\gamma(z)}.
\]

Her er nemlig
 \[
 f(\gamma' z) = \sum_{\gamma \in \Gamma'} \frac{h(\gamma \gamma' z)}{j_\gamma (\gamma' z)} = \sum_{\gamma \in \Gamma'} \frac{h(\gamma \gamma' z)}{j_{\gamma \gamma'}(t)} \cdot j_{\gamma'} (t)  = j_{\gamma'} (t).
\] 
Så denne ville vært en fin kandidat! Men det er liten sjangs for konvergens siden vi kan ha $j_\gamma(t) \equiv 1$ for uendelig mange $\gamma$. Nemlig fikspunkter! 

La så 
\[
\Gamma_0 = \{ \gamma \in \Gamma' \mid j_\gamma(z) \equiv 1 \}. 
\]

For $\Gamma(1)$ er dette gruppen generert av $\begin{pmatrix} 1 & h \\ 0 & 1 \end{pmatrix}$.

$\Gamma_0$ er en undergruppe av $\Gamma$.

Anta nå $h:\Hh \to \C$ er invariant under $\Gamma_0$. La $\gamma \in \Gamma'$ og $\gamma_0 \in \Gamma_0$. Da er 
\[
\frac{h(\gamma_0 \gamma z)}{j_{\gamma_0 \gamma}(z)} = \frac{h(\gamma z)}{j_{\gamma_0}(\gamma z) j_\gamma(z)} = \frac{h(\gamma z)}{j_\gamma(z)}.
\]

Dermed ser vi heller på summen
\[
f(z) = \sum_{\gamma \in \Gamma'/\Gamma_0} \frac{h(\gamma z)}{j_\gamma(z)}
\]
hvor summen går over et sett med representanter fra hvert kosett i kvotiengruppa. Denne har ihvertfall større håp om å konvergere.


Bruk dette på $j_\gamma(z) = (cz+d)^{2k}$ og $\Gamma$ en undergruppe av endelig indeks i $\Gamma(1)$. Da er $\Gamma_0$ generert av $z \mapsto z+h$ for en $ h \in \Z$. Én slik funksjon er $\exp(2 \pi i nz/h)$. Dermed:

\begin{defi}
Vi definerer \textbf{Poincaré-rekka} av vekt 2k og karakter $n$ for $\Gamma$ til å være 
\[
\varphi_n(z) = \sum_{\Gamma_0 \bs \Gamma'} \frac{\exp(\frac{2 \pi i \gamma (z)}{h})}{(cz+d)^{2k}}.
\]
\end{defi}
*puste lettet ut*

Da er det et teorem (uten bevis :) som sier at $\phi_n$ konvergerer absolutt uniformt på kompakter på $\Hh$. Så vi har en hel rekke med modulære former av vekt $2k$ for $\Gamma$. 

I tillegg:

\begin{enumerate}
\item $\varphi_0(z) = \sum \frac{1}{cz+d}^k$ er null på alle endelige køsper og $\phi_0(\infty)=1$. [[er dette sant? Regne på tavla??]]
\item For alle $n \geq 1$, så er $\varphi_n(z)$ en køspform. (null på køspene)
\end{enumerate}

\subsection{Litt om geometrien til $\Hh$ }


\subsection{Indreprodukt + utspenning}

La $f,g$ være modulære former av vekt $2k$ for $\Gamma$.

\begin{lemma}
  Differensialen $f(z) \overline{g(z)} y^{2k-2} dx dy$ er invariant under virkningen av $\SL_2(\R)$. 
\end{lemma}

\message{ !name(seminar30april15.tex) !offset(75) }

\end{document}