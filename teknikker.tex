\documentclass[11pt, english]{article}
%\usepackage[latin1]{inputenc}
\usepackage[T1]{fontenc}
\usepackage[utf8]{inputenc}
\usepackage[english]{babel}   % S P R A A K
%%
%% husk 
%% git pull origin 
%% git add *
%% git commit -m "..."k}    % postscript graphics
\usepackage{amssymb, amsmath, amsthm, amssymb} % symboler, \usepackage{mathrsfs} %,calligra}
\usepackage{url}
\usepackage{thmtools}
\usepackage{enumerate}  % lister $
\usepackage{float}
\usepackage{tikz}
\usepackage[all]{xy}   % for comm.diagram
\usepackage{wrapfig} % for float right
\usepackage[colorlinks=true]{hyperref}
\usepackage{mystyle} % stilfilen      
 
\title{Toolbox}
\author{Fredrik Meyer}
\begin{document}
\maketitle 

\section{Techniques}
\subsection{Compute the ideal of an affine toric variety}

Suppose given an affine toric $X_\sigma$ defined by a full-dimensional rational polyhedral convex cone in $N_\R \simeq \R^d$. Then the coordinate ring of $X_\sigma$ is given by the semigroup algebra $k[S_\sigma]$, where $S_\sigma = \sigma^\vee \cap M$.

Here $\sigma^\vee$ is the dual cone and $M$ is the dual lattice $N^\vee$. There is a canonical $k$-algebra basis for $k[S_\sigma]$ given by the \emph{Hilbert basis} of the semigroup $\sigma^\vee \cap M$. This gives us a presentation $k[\mathbf x] \to k[S_\sigma]$.

Thus there are three steps in computing the toric ideal:
\begin{enumerate}
\item First compute the dual cone $\sigma^\vee$.
\item Compute a Hilbert basis $\{m_1,\cdots,m_r\}$ of $\sigma^\vee \cap M$.
\item Compute the kernel of the map 
  \begin{align*}
    k[x_1,\cdots, x_r] &\to k[S_\sigma]  \\
 x_i &\mapsto m_i.
  \end{align*}
\end{enumerate}

Here is a \verb|Macaulay2| session that starts with a cone $\sigma \subseteq \N_\R$, and prints the corresponding toric ideal.

\begin{verbatim}
M = matrix{{3,1},{1,2}}
C = posHull M
Cd = dualCone C
hB = transpose matrix apply(hilbertBasis Cd, a -> entries a_0)
I  =  toricGroebner(hB, QQ[vars(0..#hB-1)])
\end{verbatim}


\subsection{Computing ideals of rational secant varieties}

Let $X \subset \PP^n$ be a projective variety. Then consider
$$
\Sec(X) = \ol{\bigcup_{p,q \in X} \ol{pq}}.
$$
Here $\ol{pq}$ denotes the line through $p$ and $q$. The overline indicates closure in the Zariski topology.

We do as an example the rational normal curve $C$ of degree $10 \in \PP^{10}$. It has a parametrization given by
\[
\PP^1 \ni (a:b) \mapsto (a^{10}:a^9b: \ldots: ab^9 : b^{10}) \in \PP^{10}.
\]

Then we have a parametrization of $\Sec(C)$ given by
\[
(s:t) \times (a:b) \times (a':b') \mapsto (\ldots: sa^ib^{10-i}+ta^ib^{10-i}: \ldots).
\]

On the ring level, we get the dual map
\begin{align*}
\varphi: k[x_0..x_{10}] &\to k[s,t,a,b,a',b'] \\
x_i &\mapsto sa^{i}b^{10-i} + ta^{i}b^{10-i}.
\end{align*}

The kernel of this map is the ideal of $\Sec(C)$. This can be computed in \verb|Macaulay2| as follows:

\begin{verbatim}
R2 = QQ[a,b,a1,b1,s,t]
S  = QQ[x_0..x_10]
aa = matrix{toList apply(0..10, i -> a^(10-i)*b^i)}
a2 = matrix{toList apply(0..10, i -> a1^(10-i)*b1^i)}
phi= map(R2, S', s*(aa) + t*(a2))
I = ker phi
\end{verbatim}

The ideal have $84$ cubic generators, and the Betti table looks like
\begin{verbatim}
             0  1   2   3   4   5   6  7
o15 = total: 1 84 378 756 840 540 189 28
          0: 1  .   .   .   .   .   .  .
          1: .  .   .   .   .   .   .  .
          2: . 84 378 756 840 540 189 28
\end{verbatim}
This means that all relations in the ideal are linear. It has dimension $4$, so that $\Sec(C)$ have dimension $3$ as a subset of $\PP^n$.
\end{document}
