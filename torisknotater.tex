\documentclass[11pt, english]{article}
%\usepackage[latin1]{inputenc}
\usepackage[T1]{fontenc}
\usepackage[utf8]{inputenc}
\usepackage[english]{babel}   % S P R A A K
%%
%% husk 
%% git pull origin 
%% git add *
%% git commit -m "..."
%% git push
% \usepackage{graphicx}    % postscript graphics
\usepackage{amssymb, amsmath, amsthm, amssymb} % symboler, osv
\usepackage{mathrsfs,calligra}
\usepackage{url}
\usepackage{thmtools}
\usepackage{enumerate}  % lister $
\usepackage{float}
\usepackage{tikz}
\usepackage[all]{xy}   % for comm.diagram
\usepackage{wrapfig} % for flo<at right
\usepackage[colorlinks=true]{hyperref}
\usepackage{mystyle} % stilfilen      
 
\title{Notes toric}
\author{FM}
\date{}
\begin{document}
\maketitle

\section{The torus}

Toric geometry is about the \emph{algebraic torus}, and not the
\emph{topological torus}. The latter is a product of copies of $S^1$,
and the latter is a product of copies of $\mathbb{C}^\ast$. They are
not completely dissimilar: the algebraic torus deformation retracts
onto the topological torus, so they do have the same topological
cohomology groups.

From now on, the word ``torus'' will always mean the algebraic torus.

What are the functions on the torus? Since we have the inclusion
$(K^\ast)^n \subseteq K^n$, we have an inclusion $K[\mathbb A^n]
\subseteq K[(K^\ast)^n]$. A short moment's thought tells that the
remaining functions are precisely the monomials. Thus we can say that
$K[(K^\ast)^n]=K[\Z^n]$.

The invertible functions are the monomials. Those that map $1_T
\mapsto 1_{K^\ast}$ are the monomials with coefficent $1$, the
\emph{characters}: $\{ \chi^u \mid u \in \Z^n \}$. 

What are maps between tori? That is, we want to describe
$\Hom(T,T^\prime)$.  We want the maps to preserve units. We can
compose in any coordinate direction $T \to T^\prime \to K^\ast$. A
moment's thought tells that the invertibility preserving maps are
precisely the maps between the character lattices of the tori (in the
opposite direction).

This is contravariant, and we would want to have a covariant
correspondence. Introduce the \emph{dual lattice}, namely
$N=\Hom(M,\Z)$. Thus we see that maps between tori $T \to T^\prime$ correspond
bijectively naturally to maps between the dual lattices $N \to N^\prime$.

The lattice is $N$ is often called the lattice of $1$-parameter
subgrous $K^\ast \to T$.

\begin{defi}
A \emph{toric variety} is a normal equivariant partial
compactification of the torus $T$.
\end{defi}

Normality means that the coordinate ring of an affine patch is
integrally closed in its fraction field. This is equivalent to Serre's
famous ``R1+S2'' condition. 

\textbf{Why normal?}: We have \emph{Sumihiro's theorem} (\cite{sumihiro_torus}), which says that if $X$ is a normal having the torus as a dense open subset on which the action is biregular, then $X$ as a cover by $T$-invariant affine opens.

\begin{example}
$\Aa^n$ and $\PP^n$ are obvious toric varieties with the action being coordiante-wise multiplication.
\end{example}

\begin{example}
  A nodal cubic has an action by a dense torus since it is rational, but since it is not normal (normal $\Rightarrow$ smooth for curves), it is not a toric variety. In fact, it is not covered by $T$-invariant affine opens: [[forgot how to see this]] [[picture of a sphere with two points identified]]
\end{example}

\textbf{How to understand toric varieties?}

Need to understand affine toric varieties + glueing. 

\begin{exc}
 Suppose $T \curvearrowright U$, where $U$ is affine toric and suppose $f \in K[U]$. Then $f$ is contained in a finite dimensional $T$-invariant $K$-linear subspace of $K[U]$.
\end{exc}
\begin{sol}
  ?? Something about splittings.
\end{sol}


\bibliographystyle{plain}
\bibliography{bibliografi}

\end{document}
