\documentclass[11pt, english]{article}
%\usepackage[latin1]{inputenc}
\usepackage[T1]{fontenc}
\usepackage[utf8x]{inputenc}
\usepackage[english]{babel}   % S P R A A K
% \usepackage{graphicx}    % postscript graphics
\usepackage{amssymb, amsmath, amsthm, amssymb} % symboler, osv
\usepackage{mathrsfs}
\usepackage{url}
\usepackage{thmtools}
\usepackage{enumerate}  % lister $  
\usepackage{float}
\usepackage{tikz}
\usepackage{tikz-cd}
\usetikzlibrary{calc}
%\usepackage{tikz-3dplot}
\usepackage{subcaption}
\usepackage[all]{xy}   % for comm.diagram
\usepackage{wrapfig} % for float right
\usepackage{hyperref}
\usepackage{mystyle} % stilfilen      

%\usepackage[a5paper,margin=0.5in]{geometry}

\begin{document}
\title{List of objects}
\author{Fredrik Meyer}
\maketitle

\tableofcontents 

\section{Notation}

We will use the following notation.

\begin{itemize}
	\item If $X_0$ is a degeneration of $X$, we write $X \rightsquigarrow X_0$.
\end{itemize}

\section{List of objects}

\subsection{Stanley-Reisner sphere $X_{00}$}
\label{sec:x00}

Let $E_6$ be a hexagon, and let $S$ be the join of two hexagons. This is a simplicial 3-sphere with $12$ vertices. The variables are named $x_1,\ldots,x_6$ and $z_1,\ldots,z_6$.

\subsection{Stanley-Reisner ball $Y_{00}$}
\label{sec:y00}

Let $S$ be the Stanley-Reisner sphere from Section \ref{sec:x00}. Add two variables $y_0$ and $y_1$ to get a $5$-dimensional variety. This is the join of two (filled) hexagons.

\subsection{Toric variety $Y_0$}
\label{sec:y0}

The Stanley-Reisner scheme $Y_{00}$ deforms to the toric variety with polytope $P$, where $P$ is the polytope with vertices 
\[
\setcounter{MaxMatrixCols}{20}
\begin{pmatrix}
2 & 2 & 0 & -2 & -2 & 0 & 0 & 0 & 0 & 0 & 0 & 0 \\
0 & 2 & 2 & 0  & -2 & -2 & 0 & 0 & 0 & 0 & 0 & 0 \\
0 & 0 & 0 & 0  &  0 &  0 & 2 & 2 & 0 &-2 & -2 & 0 \\
0 & 0 & 0 & 0  &  0 &  0 & 0 & 2 & 2 & 0 & -2 & -2 \\
1 & 1 & 1 & 1  & 1 &   1 & -1 & -1 & -1 & -1 & -1 & -1
\end{pmatrix}.
\]

The toric variety is a deformation of $Y_{00}$ in Section \ref{sec:y00}, by a result of Sturmfels, since the simplicial complex associated to it is a triangulation of the corresponding polytope. 

The polytope $P$ is reflexive.

\subsection{Singular Calabi-Yau $X_0$}
\label{sec:x0}

This is a complete intersection of two anticanonical hypersurfaces in $Y_0$. By general results, it is a Calabi-Yau. It has 12 singular points, each looking like $C(dP_6)$.

\subsection{Cone over del Pezzo}
\label{sec:cdp6}

The cone over the del Pezzo surface of degree $6$ has two smoothings.

\subsection{Smoother toric $Y$}
\label{sec:y}

The toric variety \ref{sec:y0} deforms to a variety with one-dimensional singularities.

\begin{remark}
I suspect that it actually smooths, but I haven't been able to prove it yet. Should be feasible.
\end{remark}

\subsection{Smooth Calabi-Yau $X$}

Since $Y$ have low-dimensional singualarities, two applications of Bertini shows that $X_0$ smooths. I have computed it to have Euler characteristic $-72$.

Problem: find which deformation is induced on the singularities of $X_0$. 

\end{document}