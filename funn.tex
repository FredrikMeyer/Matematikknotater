\documentclass[11pt, english]{article}
%\usepackage[latin1]{inputenc}
\usepackage[T1]{fontenc}
\usepackage[utf8]{inputenc}
\usepackage[english]{babel}   % S P R A A K


% \usepackage{graphicx}    % postscript graphics
\usepackage{amssymb, amsmath, amsthm, amssymb} % symboler, osv
\usepackage{mathrsfs}
\usepackage{url}
\usepackage{thmtools}
\usepackage{enumerate}  % lister $  
\usepackage{float}
\usepackage{tikz}
\usepackage[all]{xy}   % for comm.diagram
\usepackage{wrapfig} % for float right
\usepackage{hyperref}
\usepackage{mystyle} % stilfilen      


\title{Findings}
\author{FM}
\date{}
\begin{document}
\maketitle
%\tableofcontents 
\section{Smoothing}

Let $\K=D_6 \ast D_6$ be the join of two $6$-gons, and let $A_\K$ be its Stanley-Reisner ring, and $\PP(\K)$ its associated projective scheme. 

\begin{lemma}
The $f$-vector of $\K$ is $f=(1,12,48,72,36,1)$. 
\end{lemma}

The tangent and obstruction modules $T^i(\PP(\K))$ can be described as follows: 

\begin{prop}
\[
\dim_\C T^1 = 72+12 = 84
\]
\[
\dim_\C T^2 = ?
\]
\end{prop}

Here's an observation: Let $\mathcal G = D_6 \ast \{ v \}$. Then $\PP(\mathcal G)$ can be smoothed to a del Pezzo surface of degree $6$. This follows because $\mathbb G$ triangulates the associated polytope (the Minkowski sum of three line segments). It follows that $D_6$ can be smoothed as well, because it is embedded as a complete intersection in $\mathcal G$.

\begin{prop}
Also $\PP(\K)$ can be smoothed.
\end{prop}
\begin{proof}
Now let $\mathcal G=(D_6 \ast \{v \}) \ast (D_6 \ast \{w \})$ for two vertices $v,w$. Then using the above trick, $\mathcal G$ can be deformed to the projective join $T$ of two del Pezzo surfaces of degree $6$. The ideal is just given by the sum of the ideal of each del Pezzo surface, in disjoint variables. Since $\K$ is a complete intersection in $\mathcal G$, it follows that $X_0=\PP(\K)$ deforms as well, say to $X_t$.. However, $T$ has a singular locus of dimension $2$. By Bertini, it follows that $X_t$ have isolated singularities.

However: it is possible to computationally, by brute force, futher deforma $T$ to a variety having singular locus of dimension $1$. This implies that $X_t$ deforms to something smooth.
\end{proof}

It would be nice to find a non-computational argument for the existence of the smoothing. Maybe a toric deformation? It would also be nice to know if the generic fiber of $X_0$ over $Def(X_0)$ is smooth, and if not, what is the smoothing component? 

\section{Mirror symmetry}

Recall the Batyrev-Borisov construction for mirrors of complete intersections in toric varieties. 

...

\begin{lemma}
The Hodge numbers are $h^{12}=h^{22}=19$. 
\end{lemma}





\end{document}
