\documentclass[11pt, english]{article}
%\usepackage[latin1]{inputenc}
\usepackage[T1]{fontenc}
\usepackage[utf8x]{inputenc}
\usepackage[english]{babel}   % S P R A A K
% \usepackage{graphicx}    % postscript graphics
\usepackage{amssymb, amsmath, amsthm, amssymb} % symboler, osv
\usepackage{mathrsfs}
\usepackage{url}
\usepackage{thmtools}
\usepackage{enumerate}  % lister $  
\usepackage{float}
\usepackage{tikz}
\usepackage{tikz-cd}
\usetikzlibrary{calc}
%\usepackage{tikz-3dplot}
\usepackage{subcaption}
\usepackage[all]{xy}   % for comm.diagram
\usepackage{wrapfig} % for float right
\usepackage{hyperref}
\usepackage{../mystyle} % stilfilen      
\usepackage{booktabs}
\usepackage{resizegather}
\setcounter{MaxMatrixCols}{48}

\begin{document}
\title{Group action on $X$}
\author{Fredrik Meyer}
\maketitle

Let $E=k^3$. Form $V = (E \otimes E)^{\oplus 2}$. The group $S_3$ act on $E$ by $e_i \mapsto e_{\sigma(i)}$. We get an induced action on $E \otimes E$. We have a $\Z_2$-action on $V$ given by switching factors. This gives us a $D_6 \simeq S_3 \times \Z_2$-action on $V$.

Inside $\PP(V)$ we have the Calabi-Yau variety $X$, given by $M \cap H$, where $H$ is a generic codimension $6$ linear subspace, and $M$ is the zero set of the $2 \times 2$-minors of two matrices with entries in $E \otimes E$.

If we choose $H$ such that $G=\dP6$ act on $H$, we get an action on $X$. Our goal is to choose $H$ such that $X$ is nonsingular and have a $\dP6$-action.

\begin{lemma}
One such invariant hyperplane is given by the span of
$$
f_{ij}^\alpha = e_i^\alpha \otimes e_j^\alpha + t e_{-i-j}^{\alpha+1} \otimes e_{-i-j}^{\alpha+1},
$$
where $i \neq j \in \Z_3$ and $\alpha \in \Z_2$. 
\end{lemma}
\end{document}