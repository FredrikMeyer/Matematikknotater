\documentclass[11pt, english]{article}
%\usepackage[latin1]{inputenc}
\usepackage[T1]{fontenc}
\usepackage[utf8x]{inputenc}
\usepackage[english]{babel}   % S P R A A K
% \usepackage{graphicx}    % postscript graphics
\usepackage{amssymb, amsmath, amsthm, amssymb} % symboler, osv
\usepackage{mathrsfs}
\usepackage{url}
\usepackage{thmtools}
\usepackage{enumerate}  % lister $  
\usepackage{float}
\usepackage{tikz}
\usepackage{tikz-cd}
\usetikzlibrary{calc}
%\usepackage{tikz-3dplot}
\usepackage{subcaption}
\usepackage[all]{xy}   % for comm.diagram
\usepackage{wrapfig} % for float right
\usepackage{hyperref}
\usepackage{../mystyle} % stilfilen      
\usepackage{booktabs}
\usepackage{resizegather}
\setcounter{MaxMatrixCols}{48}

\begin{document}
\title{Embedding of \lh{$X$}{X}}
\author{Fredrik Meyer}
\maketitle


\section{Embedding in \lh{$\left(\PP^2\right)^4$}{p2}}

Let $H=\PP^{11}$ be spanned by twelve generic block matrices of the form
\begin{equation}
\label{eq:block}
\begin{pmatrix}
A & 0 \\
0 & B
\end{pmatrix},
\end{equation}
where $A,B$ er $3 \times 3$-matrices. This is a linear subspace in $\PP^{17}=\PP(E \otimes E \oplus E \otimes E)$, where $E$ is a 3-dimensional vector space.


Let $X$ be as before [[add explanation]]

We have a map $\pi_1: X \to \PP^2 \times \PP^2$ defined by sending a block matrix to its first block.

\begin{prop}
The map $\pi_1:X \to \PP^2 \times \PP^2$ is a morphism.
\end{prop}
\begin{proof}
$X$ can be desribed as $H \cap M$, where $M$ is the join of two disjoint copies of $\PP^2 \times \PP^2$. The singular locus is the set of block matrices where $A$ or $B$ is zero in \eqref{eq:block}. This is a set of dimension $4$. Intersecting with a general $\PP^{11}$ (which is of codimension $6$) kills off the singular locus, and also the indeterminacy locus of the map $(A,B) \mapsto A$, so that we indeed get a morphism.
\end{proof}


By eliminating variables, the matrices $A,B$ can be written as
\[
A =
\begin{pmatrix}
l_1 & x_1 & x_2 \\
x_3 & l_2 & x_5 \\
x_6 & x_7 & l_3
\end{pmatrix} \text{ and }
B = 
\begin{pmatrix}
l_4 & x_{10} & x_{11} \\
x_{12} & l_5 & x_{14} \\
x_{15} & x_{16} & l_6
\end{pmatrix},
\]
where the $l_i$ are general linear forms in the $x_i$ appearing in the matrices. Thus the fiber $\pi^{-1}(A)$ is



\end{document}