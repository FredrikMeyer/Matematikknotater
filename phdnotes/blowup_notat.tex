\documentclass[11pt, english]{article}
%\usepackage[latin1]{inputenc}
\usepackage[T1]{fontenc}
\usepackage[utf8x]{inputenc}
\usepackage[english]{babel}   % S P R A A K
% \usepackage{graphicx}    % postscript graphics
\usepackage{amssymb, amsmath, amsthm, amssymb} % symboler, osv
\usepackage{mathrsfs}
\usepackage{url}
\usepackage{thmtools}
\usepackage{enumerate}  % lister $  
\usepackage{float}
\usepackage{tikz}
\usepackage{tikz-cd}
\usetikzlibrary{calc}
%\usepackage{tikz-3dplot}
\usepackage{subcaption}
\usepackage[all]{xy}   % for comm.diagram
\usepackage{wrapfig} % for float right
\usepackage{hyperref}
\usepackage{mystyle} % stilfilen      
\usepackage{booktabs}
\usepackage{resizegather}
\setcounter{MaxMatrixCols}{48}

\begin{document}
\title{Resolution of joins}
\author{Fredrik Meyer}
\maketitle

\section{Blowing up \texorpdfstring{$\PP^{17}$}{P17} in two disjoint linear subspaces}

Let $\PP^{17}$ have coordinates $x_0,\ldots,x_{17}$. Let $H_1$ be the $\PP^8$ defined by $x_9=x_{10}=\ldots=x_{17}=0$. Let $H_2$ be the complementary linear subspace defined by $x_0=x_1=\ldots=x_8=0$.

Let $N$ be the blowup of $\PP^{17}$ in the union of $H_1$ and $H_2$. Then $N$ is the subset of $\PP^{17} \times \PP^8_{y_i} \times \PP^8_{z_i}$ defined by the $2 \times 2$-minors of
\[
\begin{pmatrix}
x_0 & x_1 & x_2 & x_3 & x_4 & x_5 & x_6 & x_7 & x_8 \\
y_0 & y_1 & y_2 & y_3 & y_4 & y_5 & y_6 & y_7 & y_8
\end{pmatrix}
\]
and of 
\[
\begin{pmatrix}
x_0 & x_1 & x_2 & x_3 & x_4 & x_5 & x_6 & x_7 & x_8 \\
z_0 & z_1 & z_2 & z_3 & z_4 & z_5 & z_6 & z_7 & z_8
\end{pmatrix}.
\]
We have a projection $Y \to \PP^8 \times \PP^8$. Let $\PP^{8,8}$ denote $\PP^8 \times \PP^8$. Then $Y$ is a  $\PP^1$-bundle over $\PP^{8,8}$. It is not hard to see that the corresponding locally free rank $2$ vector bundle is $\mathscr E = \OO_{\PP^{8,8}}(1,0) \oplus \OO_{\PP^{8,8}}(0,1)$. Hence $Y = \PP(\mathscr E)$. 




\end{document}
