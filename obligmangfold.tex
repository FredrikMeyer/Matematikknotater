\documentclass[11pt, english]{article}
%\usepackage[latin1]{inputenc}
\usepackage[T1]{fontenc}
\usepackage[utf8]{inputenc}
\usepackage[english]{babel}   % S P R A A K
% \usepackage{graphicx}    % postscript graphics
\usepackage{amssymb, amsmath, amsthm, amssymb} % symboler, osv
\usepackage{mathrsfs}
\usepackage{url}
\usepackage{thmtools}
\usepackage{enumerate}  % lister $  
\usepackage{float}
\usepackage{tikz}
\usepackage{tikz-cd}
\usetikzlibrary{calc}
%\usepackage{tikz-3dplot}
\usepackage{subcaption}
\usepackage[all]{xy}   % for comm.diagram
\usepackage{wrapfig} % for float right
\usepackage{hyperref}
\usepackage{mystyle} % stilfilen      

%\usepackage[a5paper,margin=0.5in]{geometry}


\begin{document}
\title{Mandatory assignment}
\author{Fredrik Meyer}
\maketitle 


\begin{exc}
  For $c \in \R \bs \{ 0 \}$, consider the set $C \subset \R^2$ defined by
\[
C = \{ (x,y) \mid x^3+xy+y^3=c \}
\]
\begin{enumerate}
\item Show that for $c \neq \frac 1{27}$, the set $C$ is a closed one-dimensional submanifold of $\R^2$.
\item Prove or disprove that for $c=\frac 1{27}$, the set $C$ is an embedded submanifold of $\R^2$. 
\end{enumerate}
\end{exc}

\begin{sol}
  \begin{enumerate}
  \item Let $f:\R^2 \to \R$ be given by $f(x,y)=x^3+xy+y^3$. Then $C=f^{-1}(c)$. By the inverse function theorem, we are interested in for which values of $c$ the Jacobian of $f$ have maximal rank for all points of $C$. In this case, the Jacobian is 
\[
\nabla(f) = \left( 3x^2+y , 3y^2+x \right).
\]
This has maximal rank if and only if not both of the components are zero. So suppose they are. Then $y=-3x^2$, and hence $0=3y^2+x=27x^4+x$ by the second component. Thus $x=0$ or $x^3=-\frac{1}{27}$. If $x=0$, then also $y=0$, but since $(x,y) \in C$, this forces $c=0$, which was not an option. So $x \neq 0$, so $x=-\frac 13$. Then $y=-\frac 13$ also because of the symmetric form the equation.

Hence
$$
c = -\frac 1{27} + \frac 19 - \frac 1{27} = \frac{-1+3-1}{27}=\frac{1}{27}.
$$
Thus for $c \neq \frac 1{27}$, the inverse function theorem holds, and $C$ is a one-dimensional submanifold of $\R^2$. It is closed because $f$ is continous.

\item Now suppose $c=\frac 1 {27}$. Then one can check that 
\[
x^3+y^3+xy-\frac{1}{27}=(x+y-\frac 13)(x^2-xy+y^2+\frac 13 x + \frac 13 y + \frac 19).
\]
Thus $C$ is the union of the two zero sets defined by the two components. Thus we must check two things: the two components must not intersect, and they must both be smooth. Any line is smooth, so the first component is okay.

Solving the quadratic in terms of $x$ (treating the $y$ as a constant), we see that the discriminant is $-3(y+\frac 13)^2$. Thus if we want real solutions, the only hope is $y=-\frac 13$. By symmetry, we must have $x=-\frac 13$ as well. Thus $C$ is the disjoint union of a line and a point, and thus $C$ is a non-equidimensional manifold.
  \end{enumerate}
\end{sol}

\begin{exc}
Let $X$ be a smooth vector field on a manifold $M$. Suppose there exists an $\epsilon > 0$ such that for every $p \in M$ there is an integral curve $\gamma:(-\epsilon, \epsilon) \to M$ of $X$ such that $\gamma(0)=p$. Then the maximal integral curves of $X$ are defined on the whole line $\R$.
\end{exc}

\begin{sol}
The condition says that there is a single $\epsilon$ that works for every point $p \in M$.

Let $p \in M$ and let $\gamma:(-\epsilon,\epsilon) \to M$ be an integral curve starting at $p$. Then let $q=\gamma(\frac \epsilon 2)$. Then there is an integral curve $\tilde \gamma:(-\epsilon, \epsilon) \to M$. 

Then by uniqueness of integral curves we must have $\tilde \gamma(s)=\gamma(t+\epsilon/2)$ (they both satisfy the same differential equation). In this way we can extend the domain of $\gamma$ by $\frac \epsilon 2$, and we continue this process indefinitely. Since $\R$ is the increasing union of the intervals $(-\epsilon-n \frac \epsilon 2, \epsilon + \frac \epsilon 2)$, this will define $\gamma$ on all of $\R$.
\end{sol}

\end{document}
