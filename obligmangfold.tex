\documentclass[11pt, english]{article}
%\usepackage[latin1]{inputenc}
\usepackage[T1]{fontenc}
\usepackage[utf8]{inputenc}
\usepackage[english]{babel}   % S P R A A K
% \usepackage{graphicx}    % postscript graphics
\usepackage{amssymb, amsmath, amsthm, amssymb} % symboler, osv
\usepackage{mathrsfs}
\usepackage{url}
\usepackage{thmtools}
\usepackage{enumerate}  % lister $  
\usepackage{float}
\usepackage{tikz}
\usepackage{tikz-cd}
\usetikzlibrary{calc}
%\usepackage{tikz-3dplot}
\usepackage{subcaption}
\usepackage[all]{xy}   % for comm.diagram
\usepackage{wrapfig} % for float right
\usepackage{hyperref}
\usepackage{mystyle} % stilfilen      

%\usepackage[a5paper,margin=0.5in]{geometry}


\begin{document}
\title{Mandatory assignment}
\author{Fredrik Meyer}
\maketitle 


\begin{exc}
  For $c \in \R \bs \{ 0 \}$, consider the set $C \subset \R^2$ defined by
\[
C = \{ (x,y) \mid x^3+xy+y^3=c \}
\]
\begin{enumerate}
\item Show that for $c \neq \frac 1{27}$, the set $C$ is a closed one-dimensional submanifold of $\R^2$.
\item Prove or disprove that for $c=\frac 1{27}$, the set $C$ is an embedded submanifold of $\R^2$. 
\end{enumerate}
\end{exc}

\begin{sol}
  \begin{enumerate}
  \item Let $f:\R^2 \to \R$ be given by $f(x,y)=x^3+xy+y^3$. Then $C=f^{-1}(c)$. By the inverse function theorem, we are interested in for which values of $c$ the Jacobian of $f$ have maximal rank for all points of $C$. In this case, the Jacobian is 
\[
\nabla(f) = \left( 3x^2+y , 3y^2+x \right).
\]
This has maximal rank if and only if not both of the components are zero. So suppose they are. Then $y=-3x^2$, and hence $0=3y^2+x=27x^4+x$ by the second component. Thus $x=0$ or $x^3=-\frac{1}{27}$. If $x=0$, then also $y=0$, but since $(x,y) \in C$, this forces $c=0$, which was not an option. So $x \neq 0$, so $x=-\frac 13$. Then $y=-\frac 13$ also because of the symmetric form the equation.

Hence
$$
c = -\frac 1{27} + \frac 19 - \frac 1{27} = \frac{-1+3-1}{27}=\frac{1}{27}.
$$
Thus for $c \neq \frac 1{27}$, the inverse function theorem holds, and $C$ is a one-dimensional submanifold of $\R^2$. It is closed because $f$ is continous.

\item Now suppose $c=\frac 1 {27}$. Then one can check that 
\[
x^3+y^3+xy-\frac{1}{27}=(x+y-\frac 13)(x^2-xy+y^2+\frac 13 x + \frac 13 y + \frac 19).
\]
Thus $C$ is the union of the two zero sets defined by the two components. Thus we must check two things: the two components must not intersect, and they must both be smooth. Any line is smooth, so the first component is okay.

Solving the quadratic in terms of $x$ (treating the $y$ as a constant), we see that the discriminant is $-3(y+\frac 13)^2$. Thus if we want real solutions, the only hope is $y=-\frac 13$. By symmetry, we must have $x=-\frac 13$ as well. Thus $C$ is the disjoint union of a line and a point, and thus $C$ is a non-equidimensional manifold.
  \end{enumerate}
\end{sol}

\begin{exc}
Let $X$ be a smooth vector field on a manifold $M$. Suppose there exists an $\epsilon > 0$ such that for every $p \in M$ there is an integral curve $\gamma:(-\epsilon, \epsilon) \to M$ of $X$ such that $\gamma(0)=p$. Then the maximal integral curves of $X$ are defined on the whole line $\R$.
\end{exc}

\begin{sol}
The condition says that there is a single $\epsilon$ that works for every point $p \in M$.

Let $p \in M$ and let $\gamma:(-\epsilon,\epsilon) \to M$ be an integral curve starting at $p$. Then let $q=\gamma(\frac \epsilon 2)$. Then there is an integral curve $\tilde \gamma:(-\epsilon, \epsilon) \to M$. 

Then by uniqueness of integral curves we must have $\tilde \gamma(s)=\gamma(t+\epsilon/2)$ (they both satisfy the same differential equation). In this way we can extend the domain of $\gamma$ by $\frac \epsilon 2$, and we continue this process indefinitely. Since $\R$ is the increasing union of the intervals $(-\epsilon-n \frac \epsilon 2, \epsilon + \frac \epsilon 2)$, this will define $\gamma$ on all of $\R$.
\end{sol}

\begin{exc}
Let $G$ be a Lie group and let $\g$ be its Lie algebra at the identity.

A smooth vector field $X$ on $G$ is called \emph{left-invariant} if $(d_hl_g)(X_h)=X_{gh}$ for all $g,h \in G$. Here $l_g$ is left-multiplication by $g$. A left-invariant vector field is completely determined by its value at $e \in G$. We denote the corresponding vector field by $X^v_g := (d_el_g)(v)$. Every element $v \in T_e G=\g$ arises this way. 

The commutator of two left-invariant vector fields is again left-invariant (a short computation), and hence we get a Lie bracket on $\g$ given by $[v,w] := [X^v, X^w]_e$. 

\begin{enumerate}
\item For $v \in \g$, let $\gamma_v$ be the maximal integral curve of $X^v$ such that $\gamma_v(0)=e$. Show that for every $g \in G$ the curve $\gamma(t)=g \gamma_v(t)$ is an integral curve of $X^v$ such that $\gamma(0)=g$. Conclude that $\gamma_v(t)$ is defined for all $t \in \R$ and the flow $(\phi_t^v)_t$ defined by $X^v$ is given by $\phi_t^v(g)=g\gamma_v(t)$. 

\item Choose a coordinate chart $x:U \to \R^n$ on $G$ containing $e \in G$ such that $x(e)=0$. Let $f:V \times V \to \R^n$, where $V$ is a small neighbourhood of $0 \in \R^n$, and $f$ be the map describing the group law of $G$ in these coordinates, so $f(a,b)=x(x^{-1}(a)x^{-1}(b))$. Consider the Taylor expansion of $f$ at $0 \in \R^{2n}$. Show that
$$
a+b+B(a,b)+h.o.t.
$$
where $B:\R^n \times \R^n \to \R^n$ is a bilinear map.

\item Show that in the chosen local coordinates the Lie bracket on $\g$ is given by
\[
[v,w] = B(v,w) - B(w,v).
\]
\item Take $G=\GL_n(\R)$. Identify $\g$ with $\mathrm{Mat}_n(\R)$. Show that $[A,B] = AB-BA$.
\end{enumerate}
\end{exc}

\begin{sol}
\begin{enumerate}
\item
First we need to show two things. The first is that $\gamma(0)=g$. But this is trivial: $\gamma(0)=g \gamma_v(0)=ge=g$. The second is that $(d\gamma)_t(1) = X_{\gamma(t)}^v$. By definition we have that $X_{\gamma(t)}^v=(d_e l_{\gamma(t)})(v)$. So we will check that this agrees with the left-hand-side. They key observation is that left-multiplication satisfies $l_{gh}=l_g \circ l_h$. Thus
\begin{align*}
d(\gamma)_t(1) &= d(l_g \circ \gamma_v)_t(1)=d(l_g)_{\gamma_v(t)} \circ d(\gamma_v)_t(1) \\
&= d(l_g)_{\gamma_v(t)} (X_{\gamma_v(t)}^v) = X_{g\gamma_v(t)}^v = X_{\gamma(t)}.
\end{align*}
Where we in the last equalities used that $X$ was left-invariant.

This says that all the integral curves are just translates of integral curves through the identity, and thus every integral curve can be defined on the same interval. Thus by Exercise 2, every integral curve can be defined on all of $\R$. It follows that the flow is given by traversing an integral curve through the identity and then multiplying by $g$, that is $\phi_t^v(g)=g\gamma_v(t)$. 

\item The Taylor expansion in several variables is given by the sum of all partial derivatives of all orders with an $x^i$ in front. By holding the first $n$ coordinates fixed (and vice versa), we see that the linear term has to be $a+b$, since the derivative of $ab$ with respect to the last $n$ variables is $a$.

I'm not sure how to see that the quadratic part is bilinear. 

\item 
\item This is probably not the intended solution, but well: Any $A \in \g$ can be written as $A=\sum A_{ij} \restr{\dd{}{x^{ij}}}{e}$. The product of $AB$ is defined to be the double derivative. But actually doing the composition, one sees that it is done exactly the same way as matrix multiplication.

Hence $[A,B]=AB-BA$.
 
\end{enumerate}
\end{sol}

\begin{exc}
To show existence of compactly supported commuting vector fields linearly independent at a given point. 

Fix $\epsilon > 0$ and choose a smooth function $h$ on $[0,\infty)$ such that $h'(t) > 0$ for all $t\geq 0$, $h(t)=t$ for $t \in [0,\epsilon]$, and $h(t)=1-\frac{1}{\ln t}$ for large $t$. 

Consider the map $f:\R^n \to \R^n$ defined by $f(a)= \frac{h(\lvert a \rvert)}{\lvert a \rvert} a$. 
\begin{enumerate}
\item Show that $f$ is a diffeomorphism of $\R^n$ onto the open unit ball $B \subset \R^n$.

\item Define smooth vector fields $X_i$ on $B$ by $X_i := f_\ast(\dd{}{x^i})$. Or explicitly:
$$
X_i(b) = (d_{f^{-1}(b)}f)\left( \restr{\dd{}{x^i}}{f^{-1}(b)}\right).
$$
Put $X_i(p)=0$ for $p \not \in B$. Show that the vector fields $X_i$ on $\R^n$ are smooth and $X_i(p)=\restr{\dd{}{x^i}}{p}$ for $\lvert p \rvert \leq \epsilon$.

\item Explain why the vector fields $X_i$ pairwise commute.

\item Let $(\phi_t^i)_t$ be the flow on $\R^n$ defined by $X_i$. For all $p \in \R^n$, find the limit
\[
\lim_{t \to -\infty} \phi_t^i(p) \qquad \text{ and } \qquad \lim_{t \to +\infty} \phi_t^i(p).
\]

\end{enumerate}
\end{exc}

\begin{sol}
  \begin{enumerate}
  \item The function $f$ is clearly smooth. Since $h$ is strictly increasing, it has a smooth inverse. One can check that then the inverse of $f$ is given by
$$
g(a) = \frac{h^{-1}(\lvert a \rvert)}{\lvert a \rvert} a.
$$
Thus $f$ is a diffeomorphism. To see that it is onto the open unit ball, we observe first that $f$ only scales the coordinates of $a \in \R^n$, so we want to look at the norm of $f(a)$. We have that $\rvert f(a) \lvert = h(\lvert a \rvert)$. Thus we want to see that $h([0,\infty))= [0,1)$. Clearly $h(0)=0$. We also have 
\[
\lim_{t \to \infty} h(t) = \lim_{t \to \infty} \frac{\ln t -1}{\ln t} = 1,
\]
by L'Hôpital's rule. This implies that the image of $h$ get arbitrarily close to $1$, hence $h([0,\infty))=[0,1)$ (it cannot include $1$, because then $h$ would be a diffeomorphism of a half-open interval onto a closed interval).

\item The vector fields $X_i$ are cleary smooth on $B$, since it is defined by smooth functions. To see that it is smooth on all of $\R^n$, note that $d_{f^{-1}(b)}f \to 0$ as $b \to \infty$, and this happens very quickly, thus $X_i$ is at least continous on all of $\R^n$. But by differentiating $f$, we see that all the derivatives approach zero as well, so the extension must be smooth.

Now, since $h(t)=t$ for $t \leq \epsilon$, it follows that $d_{f^{-1}(b)}f = \id$ on $T\R^n_{f^{-1}(b)}$ for small $b$. Thus $X_i(b)=\restr{\dd{}{x^i}}{p}$.

\item Since $f$ is just scaling, the vector fields $X_i$ are just scalings of the $\restr{\dd{}{x^i}}{p}$, in particular, they are still orthogonal at every point, and hence $[X_i,X_j]_p=0$ for $i \neq j$. 

\item The first limit is obtained by traversing the integral curve backwards in time. Since each $X_i$ is ``pointing outward'', the intragral curves all go through $0$ and go radially outwards to the boundary of $B$. Thus the first limit is $0$ if $p \in B$, and it is $p$ if $p \not \in B$. 

The second limit is found by traversing time forwards, and by the same explanation, it is $0$ if $p=0$, it is $p/\lvert p \rvert$ if $ p\in B\bs \{0 \}$, and it is $p$ if $ p \not \in B$.


\end{enumerate}
\end{sol}


\end{document}
