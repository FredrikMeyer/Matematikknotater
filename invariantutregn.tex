\documentclass[11pt, english]{article}
%\usepackage[latin1]{inputenc}
\usepackage[T1]{fontenc}
\usepackage[utf8]{inputenc}
\usepackage[english]{babel}   % S P R A A K
% \usepackage{graphicx}    % postscript graphics
\usepackage{amssymb, amsmath, amsthm, amssymb} % symboler, osv
\usepackage{mathrsfs}
\usepackage{url}
\usepackage{thmtools}
\usepackage{enumerate}  % lister $  
\usepackage{float}
\usepackage{tikz}
\usepackage{tikz-cd}
\usetikzlibrary{calc}
%\usepackage{tikz-3dplot}
\usepackage{subcaption}
\usepackage[all]{xy}   % for comm.diagram
\usepackage{wrapfig} % for float right
%\usepackage{hyperref}
 \usepackage{mystyle} % stilfilen      

%\usepackage[a5paper,margin=0.5in]{geometry}


\begin{document}
\title{Compute invariants using Macaulay2}
\author{Fredrik Meyer}
\maketitle 

We describe a procedure for finding invariant rings in Macaulay2, using results from ``Ideals, varieties and algorithms''.

\section{Preliminaries}

Suppose a matrix group acts on a affine space $\Aa^n$, by actions fixing the origin. If $M \in \GL_n(k)$ acts on $P \in \Aa^n$ by $P \mapsto MP$, the corresponding action on the coordinate rings are given by $x_i \mapsto A^T x_i$. 

\end{document}