\documentclass[11pt, english]{article}
%\usepackage[latin1]{inputenc}
\usepackage[T1]{fontenc}
\usepackage[utf8]{inputenc}
\usepackage[english]{babel}   % S P R A A K
% \usepackage{graphicx}    % postscript graphics
\usepackage{amssymb, amsmath, amsthm, amssymb} % symboler, osv
\usepackage{mathrsfs}
\usepackage{url}
\usepackage{thmtools}
\usepackage{enumerate}  % lister $  
\usepackage{float}
\usepackage{tikz}
\usepackage{tikz-cd}
\usetikzlibrary{calc}
%\usepackage{tikz-3dplot}
\usepackage{subcaption}
\usepackage[all]{xy}   % for comm.diagram
\usepackage{wrapfig} % for float right
%\usepackage{hyperref}
 \usepackage{mystyle} % stilfilen      

%\usepackage[a5paper,margin=0.5in]{geometry}


\usepackage{listings}
\usepackage{color}
\definecolor{dkgreen}{rgb}{0,0.6,0}
\definecolor{gray}{rgb}{0.5,0.5,0.5}
\definecolor{mauve}{rgb}{0.58,0,0.82}

\lstset{frame=tb,
  language=Java,
  aboveskip=3mm,
  belowskip=3mm,
  showstringspaces=false,
  columns=flexible,
  basicstyle={\small\ttfamily},
  numbers=none,
  numberstyle=\tiny\color{gray},
  keywordstyle=\color{blue},
  commentstyle=\color{dkgreen},
  stringstyle=\color{mauve},
  breaklines=true,
  breakatwhitespace=true,
  tabsize=3
}



\begin{document}
\title{Compute invariants using Macaulay2}
\author{Fredrik Meyer}
\maketitle 

We describe a recipe for finding invariant rings in Macaulay2 \cite{M2}, using results from ``Ideals, varieties and algorithms'' \cite{cox_ideals}.

\section{Preliminaries}

Suppose a matrix group $G$ acts on a affine space $\Aa^n$, by actions fixing the origin. If $M \in \GL_n(k)$ acts on $P \in \Aa^n$ by $P \mapsto MP$, the corresponding action on the coordinate rings are given by $x_i \mapsto A^T x_i$. 

Thus for example, if we look at $C_4$, rotating the plane by 90 degrees counterclockwise, that is
$$
P = (p_1,p_2)^t \mapsto \begin{pmatrix}
0 & -1 \\
1 & 0 
\end{pmatrix} (p_1,p_2)^t,
$$
the corresponding $k$-algebra morphism is given by $x \mapsto -y$ and $y \mapsto x$.

We will return to this example in the next section.

Now, by definition, the invariants of $k[x_1,\ldots,x_n]$ are all the polynomials left unchanged by the action of $G$. It isn't even clear from the outset that the ring $k[x_1,\ldots,x_n]^G$ is finitely generated! Luckily this is true if $G$ is a finite group, for example (or more generally, a linearly reductive group).

The crucial thing is the existence of the \emph{Reynolds operator}. This is a linear map $k[x_i] \to k[k_i]$ which is a projection onto the $G$-invariants. It is given by "averaging":
\begin{equation}
R_G( f(x)) = \frac{1}{\lvert G\rvert} \sum_{A \in G} f(A \cdot x) = \frac{1}{\lvert G\rvert} \sum_{A \in G} A \cdot f (x).
\end{equation}

In fact, we have the following theorem of Emmy Noether:

\begin{thm}
Let $G \subset \GL(n)$ be a finite matrix group. Let $R_G$ be the Reynolds operator. Then we have
$$
k[x_1,\ldots,x_n]^G = k[R_G(x^\beta) \mid \lvert \beta \rvert \leq \lvert G\rvert ].
$$
\end{thm}

This is \emph{the} theorem needed to compute the ring of invariants. Note however that \emph{many} monomials need to be computed. If $G$ has order $m$ then we need to compute the Reynolds operator $\binom{n+m}{m}$ times \footnote{This follows from the identity $\sum_{d=0}^m \binom{n+d}{d} = \binom{n+m+1}{m}$.}. However, with computer, redundancy is never a problem in small examples.


\section{Doing this}

Since we have chosen to use \verb'Macaulay2', the first problem is how to represent a group. I've chosen to represent it as a list of ring maps $k[x_i] \to k[x_i]$. 

Thus for the example of $C_4$ acting on $k[x,y]$ can be represented in the following way:

\begin{lstlisting}
r1 = map(R,R,{-y,x})
r2 = r1 * r1
r3 = r2 * r1
r4 = r3 * r1 
G = {r1,r2,r3,r4}
\end{lstlisting}

The Reynolds operator can be coded as follows:
\begin{lstlisting}
reynolds  = method()
reynolds(RingElement, List) := (f,G) (
    card := #G;
    r = (sum apply(G, g -> g f))/card;
    r
    )
\end{lstlisting}

All monomials of degree less than $\lvert G \rvert$ can be found by $basis(1,m,R)$. We compute Reynolds on all monomials:
\begin{lstlisting}
monomials = flatten entries basis(1,4,R)
rlist = unique apply(monomials, m -> reynolds(m,G))
\end{lstlisting}

We get four invariant polynomials:
\begin{verbatim}


          1 2   1 2  1 4   1 4  1 3    1   3   2 2    1 3    1   3
o33 = {0, -x  + -y , -x  + -y , -x y - -x*y , x y , - -x y + -x*y }
          2     2    2     2    2      2              2      2

o33 : List
\end{verbatim}

Not all of them are necessary however. We can get a minimal presentation  by the wonderful command \verb' minimalPresentation' in \verb' Macaulay2'. 

We write this as follows:
\begin{lstlisting}
S = QQ[z_0..z_(#rlist-1)]
phi = map(R,S, rlist)
A = S/ker phi
minimalPresentation A
\end{lstlisting}

The outut is the following:
\begin{verbatim}
i65 : minimalPresentation A

        QQ[z , z , z ]
            1   4   5
o65 = -----------------
        2       2     2
      2z z  - 2z  - 2z
        1 4     4     5

o65 : QuotientRing
\end{verbatim}

This means that the invariant ring is generated by the second, fifth and sixth (counting starts at zero) element of the Reynolds list. These are:

\begin{verbatim}
i68 : {rlist#1,rlist#4,rlist#5}

       1 2   1 2   2 2    1 3    1   3
o68 = {-x  + -y , x y , - -x y + -x*y }
       2     2            2      2
\end{verbatim}

Replacing with scalar multiples, we conclude that 
$$
k[x,y]^G = k[x^2+y^2, x^2 y^2, x^3y - xy^3].
$$


\bibliographystyle{plain}
\bibliography{bibliografi}

\end{document}