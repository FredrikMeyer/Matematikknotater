\documentclass[11pt, english]{article}
%\usepackage[latin1]{inputenc}
\usepackage[T1]{fontenc}
\usepackage[utf8]{inputenc}
\usepackage[english]{babel}   % S P R A A K


% \usepackage{graphicx}    % postscript graphics
\usepackage{amssymb, amsmath, amsthm, amssymb} % symboler, osv
\usepackage{mathrsfs}
\usepackage{url}
\usepackage{thmtools}
\usepackage{enumerate}  % lister $  
\usepackage{float}
\usepackage{tikz}
\usetikzlibrary{calc}
\usepackage[all]{xy}   % for comm.diagram
\usepackage{wrapfig} % for float right
\usepackage{hyperref}
\usepackage{mystyle} % stilfilen      


\title{Exercises deformation theory}
\author{FM}
\date{}
\begin{document} 
\maketitle
%\tableofcontents 
\section{Chapter 1.1}

\begin{exc}
Existence of the Hilbert scheme for curves in $\PP^2$. Here a \emph{curve} is a closed subscheme of $\PP^2_k$ defined by any homogeneous polynomial of degree $d$ in $S=k[x,y,z]$ (so curves are in 1-1 correspondence with points in $\PP(S^dV)$ where $V$ is a $3$-dim $k$-vector space).

Write $f$ has $a_0x_0^d+\dotsc+a_nz^d$ with $a_i \in k$ and $n=\binom{d+2}{2}-1$. Consider $(a_0,\cdots,a_n)$ as a point in $\PP_k^n$. 
\begin{enumerate}
\item Curves in $\PP^2$ of degree $d$ are in 1-1 correspondence with points in $\PP^n$ by this correspondence.
\item Define $\mathcal C \subseteq \PP^2 \times \PP^n$ by $a_0x^d+\cdots+a_n+x^d=0$. Show that the correspondence in a) is given by $ a \in \PP^n$ goes to the fiber $\mathcal C_a \subseteq \PP^2$ over the point $a$. We call $\CC$ the \emph{tautological family}.
\item For any finitely generated $k$-algebra $A$, we define a \emph{family of curves of degree $d$ in $\PP^2$ over $A$} to be a closed subscheme $X \subseteq \PP_A^2$, flat over $A$, such that the fibers over closed points of $\Spec A$ are curves of degree $d$ in $\PP^2$. Show that the ideal $I_X \subseteq A[x,y,z]$ is generated by a single homogeneous polynomial $f$ of degree $d$ in $A[x,y,z]$.
\item Conversely, if $f \in A[x,y,z]_d$, what is the condition on $f$ for the zero-scheme $X$ defined by $f$ to be flat over $A$? 
\end{enumerate}
\end{exc}
\begin{sol}
  \begin{enumerate}
  \item Obvious.
\item Let $a \in \PP^n$. Then $\CC_a$ is precisely the subscheme $\subseteq \PP^2$ cut out by $f=0$. 
\[
\xymatrix{
\CC_a \ar[r] \ar[d] & \CC \subseteq \PP^2 \times \PP^n \ar[d] \\
\{a \} \ar[r] & \PP^2
}
\]
\item By lifting, we can assume that $A=k[b_1,\cdots,b_l]$ for some $l$. Then the question is equivalent to: Suppose $I \subseteq k[b_1,\cdots,b_l] \otimes k[x,y,z]$ is such that $I \otimes_k A/\mm = \langle f \rangle $ for some $f \in k[x,y,z]$ for all $\mm \in \Spec A$. Suppose in addition that $A[x,y,z]/I$ is a flat $A$-module. Then $I = \langle \tilde f \rangle$ for some $\tilde f \in A[x,y,z]$ such that $\tilde f  \otimes A/\mm = f$.

This should follow from the equational criterium for  flatness. In particular: in each fibre, $I \otimes A/\mm$ is generated by a single polynomial, and this lifts to a generator of $I$, together with the trivial relation. If $I$ had more than one generator, there would be a relation that is trivial in all fibers. But then it must be trivial everywhere. 
\item NO IDEA
  \end{enumerate}
\end{sol}


\end{document}
