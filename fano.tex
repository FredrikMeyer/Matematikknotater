\documentclass[11pt, english]{article} 
\usepackage[latin1]{inputenc}
\usepackage[T1]{fontenc}
\usepackage[english]{babel}   % S P R A A K


% \usepackage{graphicx}    % postscript graphics
\usepackage{amssymb, amsmath, amsthm, amssymb} % symboler, osv
\usepackage[poly,matrix]{xy}   % for comm.diagram
\usepackage{url}
\usepackage{thmtools}
\usepackage{mystyle} % stilfilen
\usepackage{enumerate}  % lister $
\usepackage{float}
\usepackage{tikz}

\title{A study of a variety}
\author{Fredrik Meyer}
\date{}
\begin{document}  
\maketitle
\section{The variety $X$}

Let $X \subseteq \mathbb{P}^5$ be defined by the ideal $I=(x_0x_2x_4-x_1x_3x_5)$. It is a singular 4-fold, and the singular locus is a union of $9$ lines. One sees easily that the ideal of the singular locus of $X$ is 
\[
I(\Sing X)=  (x_2x_4,x_0x_4,x_0x_2,x_3x_5,x_1x_5,x_1x_3)
\]
If one draws the the $2$-skeleton of the associated simplicial complex, one gets the complete bipartite graph $K_{3,3}$:

\[
\xymatrix{
0\bullet \ar@{-}[rr] \ar@{-}[rrd] \ar@{-}[rrdd] & & \bullet 1 \\
2\bullet \ar@{-}[rr] \ar@{-}[rru] \ar@{-}[rrd]& & \bullet 3  \\
4\bullet \ar@{-}[rr] \ar@{-}[rru] \ar@{-}[rruu]& & \bullet 5
}
\]
There are no $3$-cycles, so the associated simplicial complex cannot be $2$-dimensional. In particular, we can identify the singular locus from this drawing: The numbers $0-5$ correspond to the ``standard points'' of $\PP^5$, i.e. the points $(1:0:0:0:0:0),\dotsc,(0:0:0:0:0:1)$. So one of the lines in $\Sing X$ is the line $[01]$, that is, all points of the form $(\lambda:\mu:0:0:0:0)$. One sees that each line in $\Sing X$ meet intersect four other lines, in two points.

Note that the generators of the ideal of the singular locus give rise to a birational map $\PP^5 \rmap X$, and we could blow up $X$ on the singular locus, to try to desingularize $X$. However, there may be other ways to remove the singularities.

In particular, consider the graph below:
\[
\xymatrix{
0\bullet \ar@{-}[rr]  \ar@{-}[rrdd] & & \bullet 1 \\
2\bullet \ar@{-}[rr] \ar@{-}[rru] & & \bullet 3  \\
4\bullet \ar@{-}[rr] \ar@{-}[rru]& & \bullet 5
}
\]

This is a cycle, so it may better be drawn as
\[
\xymatrix{
 & 0\bullet \ar@{-}[r] & \bullet 1 \ar@{-}[dr]  & \\
5\ar@{-}[ur]\bullet  & & & \bullet 2 \ar@{-}[dl] \\
 & 4\bullet \ar@{-}[ul]  &\bullet 3 \ar@{-}[l] & 
}
\]
This give rise to a rational map $\varphi:\PP^5 \rmap X$ given by
\[
(z_{01}:z_{12}:\cdots:z_{50}) \mapsto (x_0x_1,x_1x_2,x_2x_3,x_3x_4,x_4x_5,x_0x_5)
\]

Its base locus, $B$, lies inside $\Sing X$. The inverse images of the lines $[01],[12],\dotsc, [50]$, are $\PP^3$'s. In particular, they have codimension $2$ in $\PP^5$. However, this map is dominant, and we want a map from a projective space of the same dimension. So the first drawing is actually better: It gives rive to a map $\psi: \PP^2 \times \PP^2 \rmap X$ defined by
\[
(x_0:x_2:x_4,x_1:x_3:x_5) \mapsto (x_0x_1,x_1x_2,x_2x_3,x_3x_4,x_4x_5,x_0x_5)
\]
the same formula. Now the inverse images are better: Let $[ij]$ be the line between the exceptional points $P_i,P_j$. Then its inverse image is .........

\begin{center}
\begin{tabular}{l | l | c}
Line & Inverse image & $\dim \psi^{-1}([\text{line}])$\\
\hline
$[01]$ & $(\lambda x_0^2 : \lambda ^2: \mu x_0^2, 0 : 1:0)$ & 2 \\


\end{tabular}
\end{center}

\begin{prop}
The variety $X$ is naturally isomorphic to the projective spectrum of the semi group ring
\[
S = k[x_0x_1,x_1x_2,x_2x_3,x_3x_4,x_4x_5,x_5x_0].
\]
That is, $X= \Proj(S)$. This gives a coordinate-independent description of $X$.
\end{prop}
\begin{proof}
The kernel of the homomorphism
\[
k[y_1,\cdots,y_6] \to k[x_0,\cdots,x_6]
\]
sending $y_i$ to $x_ix_{i+1}$ (modulo $6$), is precisely $I=(x_0x_2x_4-x_1x_3x_5)$. 
\end{proof}

Observation: In the same manner, we can identify $\PP^2 \times \PP^2$ with
\[
\Proj(k[x_0x_1,x_0x_3,x_0x_5,\cdots,x_4x_1,x_4x_3,x_4x_5])
\]
This mean that we have a rational morphism $\psi:\PP^2 \to \PP^2 \rmap X$.

\begin{prop}
The variety $X$ is a Pfaffian hypersurface. In particular, it is given as det radical of the determinant of the $6 \times 6$-matrix
\[
\begin{pmatrix}0&
     {x}_{0}&    0&    0&     0&     {-{x}_{5}}\\
     {-{x}_{0}}&     0&     {x}_{3}&     0&     0&     0\\
     0&     {-{x}_{3}}&     0&     {x}_{4}&     0&     0\\
     0&     0&     {-{x}_{4}}&     0&     {x}_{1}&     0\\
     0&     0&     0&     {-{x}_{1}}&     0&     {x}_{2}\\
     {x}_{5}&     0&     0&     0&     {-{x}_{2}}&     0\\
     \end{pmatrix}
\]
\end{prop}

\section{The Fano variety $F_1(X)$}

The variety of lines contained in $X$ is the Fano variety $F_1(X)$, lying inside the Grassmannian $\Gr(2,6)$. It is reducible and consists of $15$ irreducible components. Nine of them are copies of $\Gr(2,4)$, and the six others are varieties defined by binomials, i.e. toric varieties. 

Using the map from the previous section and noting that it is linear, we get a map into the Fano variety:
\begin{align*}
\Xi:(\PP^2)^\vee \times \PP^2 &\rmap F_1(X) \subseteq \Gr(2,6) \\
\ell \times p &\mapsto \ell_p := [\psi(x,p),\psi(y,p)]_{x,y \in \ell}
\end{align*}

Here $\psi$ is the map below:
\begin{align*}
\psi: \PP^2 \times \PP^2 &\rmap X \\
(x,y,z) \times(a,b,c) &\mapsto (ay,by,bz,cz,cx,ax)
\end{align*}

The map can be visualized with the following figure:
\[
\xymatrix{
x\bullet \ar@{-}[rr]^5 \ar@{-}[rrdd]^4 & & \bullet a \\
y\bullet \ar@{-}[rr]_0 \ar@{-}[rru]^1  & & \bullet b  \\
z\bullet \ar@{-}[rr]_2 \ar@{-}[rru]_3  & & \bullet c
}
\]

To get the Pl�cker coordinates of the map, one forms the $2 \times 6$-matrix having as rows the value of $\psi$ at two generic points, and then one takes maximal minors. These are the Pl�cker coordinates. Since the image of $\psi$ lies inside $X$, the image of $\Xi$ lies inside $F_1(X)$.



The nine other components are isomorphic to $\Gr(2,4)$. They occur when one odd and one even coordinate is zero. 

\section{Connection to algebraic statistics}

The ideal $I$ is an example of an ideal arising from permutation matrices. For details, see \cite[page 56]{cox_toric} and \cite[page 148]{sturmfels}.

\bibliographystyle{plain}
\bibliography{bibliografi}

\end{document}
