\documentclass[11pt, english]{article}
%\usepackage[latin1]{inputenc}
\usepackage[T1]{fontenc}
\usepackage[utf8]{inputenc}
\usepackage[english]{babel}   % S P R A A K


% \usepackage{graphicx}    % postscript graphics
\usepackage{amssymb, amsmath, amsthm, amssymb} % symboler, osv
\usepackage{mathrsfs}
\usepackage{url}
\usepackage{thmtools}
\usepackage{enumerate}  % lister $  
\usepackage{float}
\usepackage{tikz}
\usetikzlibrary{calc}
\usepackage{tikz-3dplot}
\usepackage{subcaption}
\usepackage[all]{xy}   % for comm.diagram
\usepackage{wrapfig} % for float right
\usepackage{hyperref}
\usepackage{mystyle} % stilfilen      

\usepackage[a5paper,margin=0.5in]{geometry}

\begin{document}
\title{Calabi-Yau hypersurface and mirror symmetry}
\author{Fredrik Meyer}
\maketitle 

\section{Preliminaries}

\subsection{Mirror constructions}

We first review the Batyrev-Borisov construction for hypersurfaces in toric varieties.

Let $\Delta$ be a reflexive polytope of dimension $n$. Let $\PP_\Delta$ be the associated toric variety.

\section{The deformation}

Let $\K = D_6 \ast D_6 \ast \{ x_0 \}$, and let $\PP(\K)$ be the associated Stanley-Reisner scheme.

Let $dP$ be the polytope associated to the del Pezzo surface of degree $6$.

\begin{prop}
There is a flat deformation of $\PP(\K)$ to the toric variety associated to the polar dual $(dP \times dP)^\circ$.
\end{prop}

\begin{corr}
There is a flat deformation of $D_6 \ast D_6$ to a singular Calabi-Yau threefold $X_t$. It has $48$ singularities.
\end{corr}


\end{document}
