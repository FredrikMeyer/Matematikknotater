\documentclass[11pt, norsk]{article}

\usepackage[T1]{fontenc}
\usepackage[utf8]{inputenc}
\usepackage[norsk]{babel}   % S P R A A K

\usepackage{amssymb, amsmath, amsthm, amssymb} % symboler, osv
\usepackage{mathrsfs,calligra}
\usepackage{url}
\usepackage{thmtools}
\usepackage{enumerate}  % lister $
\usepackage{float}
\usepackage{tikz}

\usepackage{young} 
\usepackage{youngtab} 

\usepackage[all]{xy}   % for comm.diagram
\usepackage{wrapfig} % for float right
\usepackage[colorlinks=true]{hyperref}
\usepackage{mystyle} % stilfilen      


\title{Noen løsninger}
\author{Fredrik Meyer} 
\date{}
\begin{document}

\maketitle

Her er noen løsningsforslag til oppgaver jeg gjorde dårlig i plenumen.

\begin{oppg}[Oppgave 5.7]
La $p$ være et odde primtall. Vi har sett i kapittel 5 at dersom $x^2 \equiv -1 \pmod p$ er løsbar, så er $p \equiv 1 \pmod 4$. Her er noen generaliseringer:
\begin{enumerate}[a)]
\item Vis at dersom $x^4 \equiv -1 \pmod p$ er løsbar så er $p \equiv 1 \pmod 8$. 
\item Vis at dersom $x^8 \equiv -1 \pmod p$ er løsbar, så er $p \equiv 1 \pmod {16}$. 
\item Formuler og bevis en naturlig generalisering av $a)$ og $b)$.
\end{enumerate}
\end{oppg}

\begin{enumerate}[a)]
\item La $x$ være en løsning av $x^4 \equiv -1 \pmod p$. La først $u=x^2$. Da ser vi at ligningen $u^2 \equiv -1 \pmod 4$ har en løsning. Dette impliserer at $p -1 \equiv 1 \pmod 4$, så $4 \mid p-1$. 

Vi har da følgende kjede av likheter:
$$
1 \equiv x^{p-1} \equiv x^{\frac 44 (p-1)} \equiv (x^4)^{\frac{p-1}{4}} \equiv (-1)^{\frac{p-1}{4}} \pmod p.
$$

Merk at ingen av eksponentene er ekte brøker, så alt er veldefinert.

For at dette skal være mulig, må $\frac 14 (p-1)$ være partall. Men dette gir at $\frac 14 (p-1)=2k$, altså $p-1 = 8k$, altså $p \equiv 1 \pmod 8$.

\item Igjen, la $u=x^2$. Dermed har $u^4 \equiv -1 \pmod p$ en løsning. Fra a) impliserer det at $p-1 \equiv 0 \pmod 8$. Dermed kan vi skrive
$$
1 \equiv x^{p-1} \equiv x^{\frac 88 (p-1)} \equiv (x^8)^{\frac{p-1}{8}} \equiv (-1)^{\frac{p-1}{8}} \pmod p.
$$
Dermed må $\frac 18 (p-1)$ være partall, så $p-1 = 16k$, så $p \equiv 1 \mod {16}$. 
\item Den naturlige generaliseringen er selvsagt: om $x^{2^{k-1}} \equiv -1 \pmod p$, så er $p \equiv 1 \pmod {2^k}$ for $k \geq 2$. 

Dette kan vi bevise med induksjon. Vi har allerede sett at dette er sant for $k=2$. Anta nå at påstanden stemmer for $k=l$. Vi skal vise at den holder for $k=l+1$.

Anta at $x^{2^k} \equiv -1 \pmod p$ har en løsning $x$. Sett først $u=x^2$. Da er $x^{2^{k}}=u^{2^{k-1}} \equiv -1 \pmod p$, så ved induksjonshypotesen har vi at $p-1 \equiv 0 \pmod {2^k}$. Dermed kan vi skrive (som over)
$$
1 \equiv x^{p-1} \equiv x^{\frac{2^k}{2^k}(p-1)} \equiv {(x^{2^k})}^{\frac{p-1}{2^k}} \equiv (-1)^{\frac{p-1}{2^k}} \pmod {p}.
$$

Dermed må $(p-1)/2^k$ være partall, så vi kan skrive $(p-1)/2^k = 2l$ for en $l \in \Z$. Dermed er $p-1 = 2^{k+1}l$, så $p \equiv 1 \pmod {2^{k+1}}$, som var det vi skulle vise.

Det følger ved induksjonsprinsippet at påstanden holder for alle $k$.
\end{enumerate}

\begin{oppg}[Oppgave 5.8]
La $p$ være et primtall med $p \equiv 3 \pmod 4$. Vis at
$$
\left( \frac{p-1}{2} \right)! \equiv \pm 1 \pmod p.
$$
\end{oppg}

Merk at venstresiden er produktet av den første halvparten tall i $\Z/(p)$.  De resterende tallene i $\Z/(p)$ er $-1,-2,-3...$. Dermed kan vi skrive
$$
(p-1)! \equiv \left( \frac{p-1}{2} \right)! \left( \frac{p-1}{2} \right)! (-1)^{\frac{p-1}{2}} \equiv -1 \pmod p.
$$
Men siden $p \equiv 3 \pmod 4$, er $(-1)^{\frac{p-1}{2}}=-1$. Sett $x= \left(\frac{p-1}{2} \right)!$. Da har vi at
$$
x^2 \equiv 1 \pmod p,
$$
så $x \equiv \pm 1 \pmod p$ (ved oppgave 2a) i obligen).

Her er et eksempel på hva vi gjør for $p = 7$:
$$
-1 \equiv 1 \cdot 2 \cdot 3 \cdot 4 \cdot 5 \cdot 6 \equiv 1 \cdot 2 \cdot 3 \cdot (-3) \cdot (-2) \cdot (-1) \equiv {\left( \frac 62 \right)!}^2 (-1)^3 \pmod 7.
$$
Så ${3!}^2 \equiv 1 \pmod 7$, så $3! \equiv \pm 1 \pmod 7$ (som stemmer).


\begin{oppg}[Oppgave 6.3]
  \begin{enumerate}[a)]
  \item Vis at dersom $k \equiv 7 \pmod 8$, så kan ikke $k$ skrives som en sum av tre kvadrater.
\item Vis at dersom $4a$ kan skrives som en sum av tre kvadrater, så kan også $a$ det.
\item Vis at ingen tall på formen $4^m(8k+7)$ aldri kan skrives som en sum av tre kvadrater. Dette er den enkle delen av Gauss' resultat.
  \end{enumerate}
\end{oppg}

\begin{enumerate}[a)]
\item Vi skriver opp kvadratene modulo 8. Disse er $1^2 \equiv 1$, $2^2 \equiv 4$, $3^2 \equiv 1$, $4^2 \equiv 0$, $5^2 \equiv 1$, $6^2 \equiv 4$, $7^2 \equiv 1$ og $0^2 \equiv 0$. Så kvadratene modulo 8 er $0,1,4$. Vi sjekker at det ikke er mulig å legge sammen noen tre av disse tallene og få 7. Dette betyr at om $a=k_1^2+k_2^2+k_3^2$, så kan ikke $a \equiv 7 \pmod 8$.

\item 

Vi ser på ligningen $4a=k_1^2+k_2^2+k_3^2$ modulo $4$. Modulo $4$ er kvadratene enten $0$ eller $1$. Men vi har at $k_1^2+k_2^2+k_3^2=4a \equiv 0 \pmod 4$. For at dette skal gå, må hvert ledd være lik $0$ modulo $4$. Men dette skjer kun hvis alle $k_i$ er partall.

Dermed, om $4a=k_1^2+k_2^2+k_3^2$, kan vi skrive $k_i=2m_i$ for $m_i \in \N$ og få
$$
4a=(2m_1)^2+(2m_2)^2+(2m_3)^2 =4m_1^2+4m_2^2+4m_3^2.
$$
Vi kan dermed dele på $4$ på begge sider, og ser at $a$ også kan skrives som en sum av tre kvadrater.
\item Anta gitt et tall $r=4^m(8k+7)$ og anta for motsigelse at det kan skrives som en sum av tre kvadrater.

Da følger det fra $b)$ at $r' = 4^{m-1}(8k+7)$ også kan skrives som en sum av tre kvadrater. Vi kan fortsette, og ser at da må også $r''=4^{m-2}(8k+7)$ kunne skrives som en sum av tre kvadrater.

Til slutt ender vi opp med at $8k+7$ kan skrives som en sum av tre kvadrater. Men dette motsier a), da $8k+7 \equiv 7 \pmod 8$.
\end{enumerate}

\end{document}