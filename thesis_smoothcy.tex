%!TEX root = thesis.tex

%%%%%%%%%%%%%%%%%%%%%%
\chapter{A smooth Calabi-Yau}

Consider the hexagon $E_6$. The join $E_6 \ast E_6$ is a $3$-dimensional sphere, and so a smoothing of the corresponding Stanley-Reisner scheme would correspond to a smooth Calabi-Yau manifold. In this chapter I prove that there does indeed exist a smoothing, and I describe some of its properties.

Description, singularities, etc.

\section{Isolated singularities}
\todo{find better section title}

The smoothing process is done by deforming the ambient space. First, note that $\K= E_6 \ast \E_6 \ast \Delta^0 \ast \Delta^0$, is a $5$-dimensional simplicial sphere, which is the join of two hexagons with interior. The Stanley-Reisner ring of $\K$ is the tensor product $k[E_6 \ast \Delta^0] \otimes_k k[E_6 \ast \Delta^0]$. Each of the factors deform to the affine cone over a del Pezzo surface of degree $6$, the same on as in Chapter 2.

It follows that $\PP(\K)$ deforms to a toric variety $Y_0$, whose ideal in $\PP^{13}$ is the sum of the ideals of the del Pezzo surface in $\PP^6$ in a disjoint set of variables. Then it is not hard to see that the singular locus of $Y_0$ consists of two disjoint copies of $\dP6$. 

\begin{lemma}
Let $X_0$ be the intersection of $Y_0$ with two general hyperplanes in $\PP^{13}$. Then $X_0$ is a singular Calabi-Yau variety with $12$ isolated singularities.
\end{lemma}
\begin{proof}
Away from the singular locus of $Y_0$, the intersection is smooth by Bertini.

The singular locus of $Y_0$ is equal to $\dP6 \sqcup \dP6$, hence is of dimension $2$ and degree $12$. Two general hyperplanes will intersect the singular locus in $12$ points. 

Then one form of Bertini's theorem \cite[page 216]{harris_alggeo}, says that all singular points on $X$ comes from those of $Y$. 
\end{proof}

We can determine the types of the singularities of $X_0$.

\begin{lemma}
Let $(U,p_i)$ be the germ of $X_0$ at $p_i$. Then $(U,p_i) \simeq (C(\dP6),0)$.
\end{lemma}
\begin{proof}
In each chart, $X_0$ looks like $\Aa^2 \times C(dP_6)$. Let $\Aa^2$ have coordinates $x_2,x_6$ and $C(dP_6)$ have coordinates $z_1,\ldots,z_6,y_1$. Then $X_0$ is the zero set of $I(f,g)$, where $f,g$ are polynomials that are linear in the $z_i,y_1$ and degree $4$ in $x_2$ and $x_6$ (in fact, the Newton polyhedron of $f(x_2,x_6,0,\ldots,0)$ is a hexagon).

Let $p_i=(a_1,a_2,0,\ldots,0)$ be a singular point. By a change of variables, we can translate $p_i$ to the origin. Then $f=x_2u_1+l(z_1,\ldots,z_6,y_1)$ and $g=x_6u_2 + l'(z_1,\ldots,z_6,y_1)$, where $u_1,u_2$ are units around the origin, and $l,l'$ are linear forms in $k[z_1,\ldots,z_6,y_1]$.

Hence for a small enough $U$ (such that $u_1,u_2$ are units restricted to $U$), we can do another change of variables, letting $x_2'=x_2u_1$, and $x_6'=x_6u_2$. This allows us to eliminate $x_2,x_6$ locally from the equations, and we are left with $C(\dP6)$.
\end{proof}


\section{Smoothing $X_0$}

There is a smoothing of $X_0$ to a smooth Calabi-Yau $X$ with Euler characteristic $-72$.

\begin{lemma}
The toric variety $Y_0$ deforms to a variety $Y$ with singularities of dimension $\leq 1$.
\end{lemma}
\begin{proof}
Consider the family $Y_t$ defined by
\begin{align*}
\begin{vmatrix}
z_1 & z_2 & y_1 - t(2x_6-x_2) \\
y_1-(x_6-x_2) & z_3 & z_4 \\
z_6 & y_1 & z_5
\end{vmatrix} \leq 1.
&&
\begin{vmatrix}
x_1 & x_2 & y_0 - t(2z_6-z_2) \\
y_0-t(z_6-z_2) & x_3 & x_4 \\
x_6 & y_0 & x_5
\end{vmatrix} \leq 1.
\end{align*}
Then $Y_0$ is the central fiber, and one can check with computer help that the singular locus is a chain of $\PP^1$'s.
\end{proof}

Since $X_0$ is a complete intersection in $Y_0$, it deforms as well. By Bertini's theorem $X = X_1$ is smooth. In fact, with the help of a computer we can say a little more:

\begin{prop}
The smoothing $X$ of $X_0$ has topological Euler characteristic $-72$.
\end{prop}

\begin{proof}
Since $\chi(X) = 2\chi(\mathcal T_X) = -2\chi(\Omega_X^1)$, we will show that $\chi(\Omega_X^1)=36$. This is done by computations in \texttt{Macaulay2}, together with a good choice of Gröbner basis.

There is a choice of weight vector of the polynomial ring $\C[x_1,\ldots,x_6,z_1,\ldots,z_6,y_0,y_1]$ such that $Y$ has a Gröbner basis with $18$ elements (recall that $Y_0$ is generated by $18$ binomials). One such weight vector is $(7, 7, 8, 9, 8, 7, 1, 5, 14, 19, 14, 5, 5, 7)$. This choice of weight vector speeds up the computations enormously. 

The cotangent sequence is not exact since $Y$ is not smooth, but the sequence 
\[
0 \to J \to \restr{\Omega_{\PP^{13}}}{Y} \to \Omega_Y^1 \to 0
\]
is smooth, where $J$ is the image of Jacobian $I \to \Omega_{\PP^{13}}^1$ restricted to $Y$. \texttt{Macaulay2} is able to compute the cohomology of $J$, and it is $H^i(Y,J)=0$ for $i \neq 3$ and $10$ for $i=3$. In particular, the Euler characteristic of $J$ is $-10$.

The pulled back Euler sequence
\[
0 \to \restr{\Omega_{\PP^{13}}^1}{Y} \to \OO_Y(-1)^{14} \to \OO_Y \to 0
\]

implies that $\chi(\restr{\Omega_{\PP^{13}}^1}{Y}=-1$, by additivity of Euler characterstics. Hence $\chi(\Omega_Y^1)=9$. Similarly, we find that $\chi(\Omega_Y^1(-1))=16$ and $\chi(\Omega_Y^1(-2))=35$. 

Since $X$ is a complete intersection in $Y$, the conormal sheaf $\mathcal I_{X/Y}/\mathcal I_{X/Y}^2$ is just $\OO_X(-1)^2$. Since $X$ is smooth, the conormal sequence is exact:
\begin{equation}
\label{eq:omegax}
0 \to \OO_X(-1)^2 \to \restr{\Omega_Y^1}{X} \to \Omega_X^1 \to 0.
\end{equation}

Hence if we find the Euler characteristic of the middle term, we will have found the Euler characteristic of $X$. Consider the ideal sequence of $X$:
\[
0 \to \mathcal I_{X/Y} \to \OO_Y \to \OO_X \to 0.
\]
Tensoring with $\Omega_Y^1$ is exact because the singularities of $Y$ lie outside $X$: by the long exact Tor sequence, we want $\mathscr Tor_1^{\OO_Y}(\OO_X,\Omega_Y^1)=0$. Since this sheaf is supported on $X$, and $Y$ is smooth in a neighbourhood around $X$, the sheaf $\Omega_Y^1$ is locally free around $X$. Hence the $\mathscr Tor$ sheaf is zero. 

Again, since $X$ is a complete intersection in $Y$, we have a resolution by the Koszul complex twisted by $\Omega_Y^1$:
\[
0 \to \Omega_Y^1(-2) \to \Omega_Y^1(-1)^{\oplus 2} \to \Omega_Y^1 \to \restr{\Omega_Y^1}{X} \to 0.
\]
By the above computations, we find that $\chi(\restr{\Omega_Y^1}{X})=35-32+9=12$. From \eqref{eq:omegax} we conclude that $\chi(\Omega_X^1)=36$.
\end{proof}