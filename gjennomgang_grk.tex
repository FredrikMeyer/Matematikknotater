\documentclass[11pt, english]{article}
%\usepackage[latin1]{inputenc}
\usepackage[T1]{fontenc}
\usepackage[utf8x]{inputenc}
\usepackage[english]{babel}   % S P R A A K
% \usepackage{graphicx}    % postscript graphics
\usepackage{amssymb, amsmath, amsthm, amssymb} % symboler, osv
\usepackage{mathrsfs}
\usepackage{url}
\usepackage{thmtools}
\usepackage{enumerate}  % lister $  
\usepackage{float}
\usepackage{tikz}
\usepackage{tikz-cd}
\usetikzlibrary{calc}
%\usepackage{tikz-3dplot}
\usepackage{subcaption}
\usepackage[all]{xy}   % for comm.diagram
\usepackage{wrapfig} % for float right
\usepackage{hyperref}
\usepackage{mystyle} % stilfilen      

%\usepackage[a5paper,margin=0.5in]{geometry}


\begin{document}
\title{Exam solutions GRK}
\author{Fredrik Meyer}
\maketitle 

\begin{exc}[Eksamen 2008, 1]
\begin{enumerate}
	\item Find three different abelian groups of order $8$. Explain why all abelian groups of order $8$ are isomorphic to one of these.
	\item Find a non-abelian group of order $8$.
	\item How many subgroups of order $8$ does $S_4$ have? How many of these are abelian?
\end{enumerate}
\end{exc}
\begin{sol}
\begin{enumerate}
	\item All finite abelian groups are products of cyclic groups. If the order of the group is $8$, then the order of each factor must be $2$. Hence there are three possibilities: $\Z_8, \Z_2 \times \Z_4$ or $\Z_2 \times \Z_2 \times \Z_2$, and these are all of them.

	Note that we used the theorem about finitely generated abelian groups in the first sentence.
	\item Let $D_4$ be the group of symmetries of the square. It is generated by a rotation of order $4$, together with a reflection through one of the diagonals. Explicitly, call these two generators for $\rho$ and $\sigma$. They satisify the three relations $\rho^4=e$, $\sigma^2=e$, and $\sigma \rho \sigma = \rho^{-1}$. The last one implies that the $\rho \sigma \neq \sigma \rho$, hence the groups is not abelian.

	To see this more explicitly, make a paper square, and mark its corners with the letters $A-D$ on both sides of the paper. Then ``compute'' what the results of $\rho \sigma$ and $\sigma \rho$ are.
	\item $S_4$ is the symmetric group on $4$ letters, hence it has order $24$. A subgroup of order $8$ is a Sylow $2$-subgroup. 

	By the Sylow theorem, the number of Sylow $2$-subgroups $n_2$ is divisible by $3$ and satisfies $n_2 \equiv 1 \pmod 2$. Hence there are either $1$ or $3$ such subgroups.

	I claim that there cannot be just $1$. Here's why. Note that $D_4 \subset S_4$ as a subgroup. It is generated by the permutation $\rho=(1234)$ and $\sigma=(24)$. I claim that $D_4$ is \emph{not} a normal subgroup of $S_4$.

	Namely, let $\gamma$ be the permutation $(14)$. Then compute $\gamma^{-1}\rho \gamma$. It turns out to be equal to $(1423)$, which is not in $D_4$ (it is not a symmetry of the square!). Hence $D_4$ is not a normal subgroup of $S_4$, but it is a Sylow $2$-subgroup. Since it is not normal, there must be $3$ such groups.

	Note that all conjugate groups are isomorphic, hence there are no abelian subgroups of order $8$.
\end{enumerate}
\end{sol}

\begin{exc}[Eksamen 2008, 3]
Let $f(x)= x^4-2x^2-3$. 
\begin{enumerate}
	\item Find the splitting field $E$ of $f(x)$ over $\Q$. What is $[E:\Q]$ and $\Gal(E/\Q)$?
	\item Find an element $\alpha \in E$ such that $E=\Q(\alpha)$. What is its minimal polynomial over $\Q$?
\end{enumerate}
\end{exc}
\begin{sol}
\begin{enumerate}
	\item The splitting field of a polynomial is the smallest extension of $\Q$ that contains all the roots of $f(x)$. By putting $u=x^2$, we can use the abc-formula to find the roots. One fins that $f(x)=(x^2+1)(x^2-3)$. Hence the roots of $f(x)$ are $\pm i$ and $\pm \sqrt{3}$. Then clearly the splitting field must contain $\Q(i)$ and $\Q(\sqrt 3)$. The smallest field containing both is $\Q(i, \sqrt 3)$, which then must be the splitting field of $f(x)$. 

	Note that we have $[E:\Q)]=[\Q(\sqrt 3)(i):\Q(\sqrt 3)][\Q(\sqrt 3):\Q]$. The second factor is $2$, and since $i \not \in \Q(\sqrt 3)$, the first factor is $2$ as well. Hence the extension is of degree $4$.

	To find the Galois group, one looks at how the roots of $f(x)$ can be permuted. There are only two groups of order $4$. Either they are $\Z_4$ or $\Z_2 \times \Z_2$. The Galois groups is generated by the permutations of $E/\Q$ sending $\sqrt 3$ and $i$ to its conjugates. These two permutations are independent (they commute), hence the group must be $\Z_2 \times \Z_2$.
	\item By an educated guess, let $\alpha=\sqrt 3 + i$. Then clearly $\alpha \in E$. We have:
	\begin{align*}
	\alpha^2 &= 3 + 2 i \sqrt 3  -1 = 2 + 2i  \sqrt 3 \\
	&\Rightarrow  (\alpha^2-2)^2 = -12.
	\end{align*}
	Multiplying out gives $\alpha^4-4\alpha^2+16=0$. Then let $g(x)=x^4-4x^2+16$. I claim that $g(x)$ is irreducible. It cannot have any linear factors, because its roots are $\pm \sqrt 3 \pm i$. What could potentially happen, is that $g(x)=h_1(x)h_2(x)$, for two quadratic polynomials $h_1,h_2$. Let $h_1(x)=x^2+bx+c$ and $h_2(x)=x^2+dx+e$. Then $h_1(x)h_2(x) = x^4+(d+b)x^3+(e+bd+c)x^2+(be+cd)x+ce$. Then $b+d=0$, så $d=-b$. Putting this into the equation $e+bd+c=4$, we get $e+c=b^2-4=(b-2)(b+2)$. But we also have that $ec=16$. Hence $(e,c)$ must be one of $(1,16)$, $(2,8)$ or $(4,4)$. This gives us three possibilities: the first one gives $b^2-4=17$, the second one gives $b^2-4=10$, and the third one gives $b^2-4=8$. None of these equations have solutions, and we have reached a contradiction. Hence $g(x)$ is irreducible.

	This means that $\alpha$ is an element contained in $E$ of degree $4$ over $\Q$. But $E$ also had degree $4$, so $E= \Q(\alpha)$.
\end{enumerate}
\end{sol}

\begin{exc}[Eksamen 2010, 3]
Let $f(x)=x^3-3$ and $g(x)=x^4-2x^2-3$ be polynomials in $\Q[x]$.
\begin{enumerate}
	\item Show that $f(x)$ is irreducible over $\Q$ and that $g(x)$ is reducible.
	\item Find the splitting field of $g(x)$. Compute $[E:\Q]$ and find the Galois group. 
	\item Find the splitting field $K$ of the family $\{ f(x), g(x) \} $. Show that $[K:\Q]=12$.
\end{enumerate}
\end{exc}
\begin{sol}
\begin{enumerate}
	\item That $f(x)$ is irreducible follows for example by Eisenstein's criterium by setting $p=3$. Another way to see this is the following: if $f(x)$ was reducible, it would have a linear factor. But none of the roots of $f(x)$ are integers, hence there can be no linear factor. The other polynomial was computed with in the previous exercise.
	\item See the above.
	\item $K$ is the smallest field containing all the roots of $f(x)$ and $g(x)$. Hence it is equal to $\Q(i,\sqrt 3, \sqrt[3]{3}e^{2 \pi i/3})$.

	Let $\zeta = e^{2 \pi i/3}$. This is a solution of $z^3-1=0$, but this polynomial is reducible. The minimal polynomial of $\zeta$ is $x^2+x+1$, which have solutions $(-1 \pm \sqrt{-3})/2$. But these elements are already in $E$ (since $\sqrt 3$ and $\sqrt {-1}$ are)! Hence
	$$
	K = \Q(i, \sqrt 3, \sqrt[3]{3}e^{2\pi i/3}) = \Q(i, \sqrt 3, \sqrt[3]{3}).
	$$
	I claim that this extension has degree $12$. To see this, consider the tower of extensions
	$$
\xymatrix{
K \ar@{-}[d] \\
\Q(\sqrt{3},\sqrt[3]{3})\ar@{-}[d] \\
\Q(\sqrt{3})\ar@{-}[d] \\
\Q
}
	$$

	Then the order of $K$ over $\Q$ is the product of the orders of each intermediate extension. The first extension has order $2$, since $\sqrt{3} \not \in \Q$. I claim that the second extension has order $3$.

	First we make the following observation. We have that 
	$$
	[\Q(\sqrt 3, \sqrt[3]3):\Q]=[\Q(\sqrt 3,\sqrt[3]3):\Q(\sqrt 3)][\Q(\sqrt 3):\Q]=2a,
	$$
	 but also
	 $$
	 	[\Q(\sqrt 3, \sqrt[3]3):\Q]=[\Q(\sqrt 3,\sqrt[3]3):\Q(\sqrt[3] 3)][\Q(\sqrt[3] 3):\Q]=3b.
	 	$$
	 Hence the degree of the first two extensions must be divisible by $6$, since $2$ and $3$ are coprime.

	The element $\sqrt[3]{3}$ has order $3$ over $\Q$. Hence its degree over $\Q(\sqrt 3)$ is either $1,2$ or $3$. It cannot be $1$, because that would imply that $\sqrt[3]3 \in \Q(\sqrt 3)$, which is not the case since its minimal polynomial is irreducible over $\Q$. If the degree were $2$, that would imply that the degree of the first two extensions were $4$, which is not divisible by $6$! Hence the degree must be $3$, so the degree of the first two extensions is $6$.

	Finally, since $i$ is complex, and the two other generators are real, the last extension must be of degree $2$. In total, the degree is $2 \times 3 \times 2 = 12$.
\end{enumerate}

\end{sol}
\end{document}