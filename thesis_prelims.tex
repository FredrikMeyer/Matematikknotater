%!TEX root = thesis.tex

\chapter{Preliminary definitions}

We work over $\C$, but some theorems may be stated over a field $k$.

\section{Stanley-Reisner basics} 

Given a simplicial complex $\K$, one can associate to it a projective scheme $\PP(\K)$ defined as follows. Let $P$ be the polynomial ring with one variable for each vertex of $\K$. Then the \emph{Stanley-Reisner ideal $I_\K$} corresponding to $\K$ is generated by the monomials corresponding to \emph{non-faces} of $\K$. Then we define the \emph{Stanley-Reisner scheme} to be $\Proj P/I_\K$. 

\begin{example}
Let $\K$ be the square, with vertices $v_0,v_1,v_2,v_3$. Then the Stanley-Reisner ideal is generated by $v_0v_2$ and $v_1v_3$.
\end{example}

Some of the topology of the simplicial complex is encoded in the scheme structure of $\PP(\K)$. In particular, the simplicial (co)homology groups of $\K$ can be computed as the sheaf cohomology of $\PP(\K)$

\begin{lemma}
Let $(K;\C)$ denote the singular cohomology groups of $\K$. Then there are isomorphisms $H^i(K;\C)=H^i(\PP(\K),\OO_{\PP(\K)})$ for all $i$.
\end{lemma}
\begin{proof}
---------to come-----------
\end{proof}

\begin{corr}
We have isomorphisms $H^i(K,\C) \simeq H^{2i}(\PP(\K);\C)$ of singular cohomology groups. <- WRONG \todo{Find correct statement: $H^{2n}$ should be number of facets.}
\end{corr}
\begin{proof}
Something about $i$-cells in even dimensions
\end{proof}

\section{Calabi-Yau basics}

\begin{defi}
A \emph{Calabi-Yau variety} is a smooth projective variety satisfying the following two conditions:
\begin{enumerate}
	\item $H^i(X,\OO_X)=0$ for $0 < i < \dim X$.
	\item The canonical sheaf is trivial: $\omega_X \simeq \OO_X$. 
\end{enumerate}
\end{defi}

The classical example of a Calabi-Yau manifold is the quintic threefold in $\PP^5$. Another example is the following:

\begin{example}
Let $X$ be the double cover of $\PP^3$ ramified along a smooth octic. The projection map is affine, so the conditions on $H^i(X,\OO_X)$ are fulfilled. To see that the canonical sheaf is trivial, we use the adjunction formula, which says that $K_X= 2 \restr{K_{\PP^3}}{X} + R$, where $R$ is the ramification divisor. In this case $R$ $8H$, where $H$ is a hyperplane in $\PP^3$. Then, since $K_{\PP^3}=-4H$, it follows that $K_X=0$.
\end{example}

If $\K$ is a simplicial sphere, then a smoothing of $\PP(\K)$ will give a Calabi-Yau manifold.


---- ref: bayer-eisenbud graph curves.

The most basic invariants of Calabi-Yau manifolds are their \emph{Hodge numbers $h^{pq}$}. In algebraic geometry these can be defined as the dimensions of the cohomology groups $H^q(X,\Omega^p_X)$. This definition is however not so transparent. On a complex manifold, it is true that $h^{pq}=h^{qp}$, but this is not obvious from our definition. Instead, let us define these groups in complex algebraic geometry terms.

The de Rham complex $(\Omega^\bullet,d)$ refines to a bigraded complex $(\Omega^{\bullet,\bullet},d)$, where a differential form of bidegree $(p,q)$ can be written as 
$$
\omega = \sum {f_{IJ}} dz_{i_1} \wedge \ldots \wedge dz_{i_p} \wedge d\overline{z_1}\ldots \wedge d\overline{z_q}.
$$

The differential $d$ splits as $\partial + \overline \partial$, where $\partial: \Omega^{\bullet,\bullet} \to \Omega^{\bullet+1,\bullet}$, and $\overline \partial: \Omega^{\bullet,\bullet} \to \Omega^{\bullet,\bullet+1}$. The decomposition passes respects cohomology, so we can form the \emph{Dolbeault cohomology groups} $H^{p,q}(X)$. 

With this definition, applying complex conjugation shows that $H^{p,q}=\overline{H^{q,p}}$.

\begin{lemma}
We have natural isomorphisms $H^{p,q}(X) \simeq H^q(X,\Omega_X^p)$. 
\end{lemma}
\begin{proof}
Use that the de Rham complex is flabby
\end{proof}

For more details on this and other details from complex geometry, see  \cite{voisin_complexalg}.

The ``Hodge diamond'' is 
...

\begin{example}
Let $X$ be a smooth quintic in $\PP^4$. We will compute its Hodge numbers. Let us first compute $H^{1,1}(X)$. We have the following exact sequence
$$
0 \to \mathcal I/\mathcal I^2 \to \restr{\Omega_{\PP^4}}{X} \to \Omega_X^1 \to 0
$$
Since $\mathcal I/\mathcal I^2 \simeq \OO_X(-5)$, it follows from the long exact sequence of cohomology that $H^1(X,\Omega^1_X) \simeq H^1(X, \restr{\Omega^1_{PP^4}}{X})$. 
....
\end{example}

\section{Deformation theory}

Deformation theory is the study how varieties (or other algebraic structures like line bundles, vector bundles, ...) vary in families. 

There is a lot of technical machinery available for the deformation theorist, but for us just a few vector spaces will be of importance.

\begin{defi}
Let $X$ be a scheme over $k$. Then a \emph{deformation of $X$ over $S$} is a flat morphism $\mathfrak X \to S$ together with an isomorphism $X \simeq \mathfrak X \times_S 0$ for a closed point $0 \in S$:
$$
\xymatrix{
X \simeq X_0 \ar@{^{(}->}[r] \ar[d] &  \mathfrak X \ar[d] \\
0 \ar[r] & S
}
$$
\end{defi}

Recall that a morphism $f:X \to Y$ is \emph{flat} if the associated morphism $f^{\$}:\OO_Y \to f_\ast \OO_X$ of $\OO_Y$-modules is a flat morphism.

%%% Motivate t1 functors

%% lifting of first order deformations

%% unobstructed problems vs t2 = 0