\documentclass[11pt, norsk]{article}
%\usepackage[latin1]{inputenc}
\usepackage[T1]{fontenc}
\usepackage[utf8]{inputenc}
\usepackage[norsk]{babel}   % S P R A A K


% \usepackage{graphicx}    % postscript graphics
\usepackage{amssymb, amsmath, amsthm, amssymb} % symboler, osv
\usepackage{mathrsfs}
\usepackage{url}
\usepackage{thmtools}
\usepackage{enumerate}  % lister $  
\usepackage{float}
\usepackage{tikz}
\usetikzlibrary{calc}
\usepackage{tikz-3dplot}
\usepackage{subcaption}
\usepackage[all]{xy}   % for comm.diagram
\usepackage{wrapfig} % for float right
\usepackage{hyperref}
\usepackage{mystyle} % stilfilen      


\begin{document}
\title{Oppgaver MAT2500}
\author{Fredrik Meyer}
\maketitle 

\begin{oppg}
Vis at om $A,B,C,D$ er kollineære, så er 
\[
\overline{AD} \cdot \ol{BC}  + \ol{BD} \cdot \ol{CA} + \ol{CD} \cdot \ol{AB} = 0.
\]
\end{oppg}
\begin{losn}
 Her mener vi altså med $\ol{AB}$ \emph{fortegnslengden} til segmentet $AB$. For å gi en verdi til $\ol{AB}$ (etc.) må man velge en \emph{retning} langs linja (for hvordan skal en ellers si om $\ol{AB}$ er positiv eller negativ?). Det som holder uansett er nemlig at $\ol{AB} + \ol{BC} = \ol{AC}$ om $A,B,C$ er kollineære.

Trikset her er dermed å bruke egenskapen over gjentatte ganger, og omskrive hver av lengdene over slik at de involverer $A$. For eksempel: $\ol{BC}$ kan skrive som $\ol{AC}-\ol{AB}$. $\ol{BD} = \ol{AD}-\ol{AB}$ og $\ol{CA} = -\ol{AC}$. Dermed er
\begin{align*}
& \overline{AD} \cdot \ol{BC}  + \ol{BD} \cdot \ol{CA} + \ol{CD} \cdot \ol{AB}. \\
&= \overline{AD}\cdot (\ol{AC}-\ol{AB}) - (\ol{AD}-\ol{AB})\cdot \ol{AC} + (\ol{AD-AC}) \cdot \ol{AB} \\
&= 0
\end{align*}
På grunn av masse kansellering.
\end{losn}

\begin{oppg}
Vis følgende generalisering av Menelaos' setning: for en firkant $ABCD$ vil punktene $A',B',C',D'$ på linjene gjennom $AB, BC, CD$ og $DA$ være kollineære bare hvis 
$$
\frac{\ol{AA'}}{\ol{A'B}}\cdot \frac{\ol{BB'}}{\ol{B'C}} \cdot \frac{\ol{CC'}}{\ol{C'D}} \cdot \frac{\ol{DD'}}{\ol{D'A}} = 1
$$ 
\end{oppg}

\begin{losn}
Se på Figur 1.

Trikset er å merke at enhver firkant er satt sammen av to trekanter. Vi skal bruke Menelaos' setning på trekantene $BCA$ og $CDA$.

Fra Menelaos' setning har vi følgende likhet:
$$
\frac{\ol{BB'}}{\ol{B'C}} \cdot \frac{\ol{CE}}{\ol{EA}} \cdot
\frac{\ol{AA'}}{\ol{A'B}} = -1,
$$
brukt på den nederste av de to trekantene. Vi har også likheten
$$
\frac{\ol{CC'}}{\ol{C'D}} \cdot
\frac{\ol{DD'}}{\ol{D'A}} \cdot
\frac{\ol{AE}}{\ol{EC}}
= -1
$$
Ganger vi disse to uttrykkene sammen, får vi (etter litt omstokking):
$$
\frac{\ol{BB'}}{\ol{B'C}} \cdot
\frac{\ol{AA'}}{\ol{A'B}} \cdot 
\frac{\ol{CC'}}{\ol{C'D}} \cdot
\frac{\ol{DD'}}{\ol{D'A}} \cdot
\frac{-\ol{EC}}{-\ol{AE}} \cdot
\frac{\ol{AE}}{\ol{EC}} = 1
$$
De to siste leddene kansellerer, og vi får uttrykket vi har lyst på.
\begin{figure}
\centering
\definecolor{uuuuuu}{rgb}{0.266666666667,0.266666666667,0.266666666667}
\definecolor{xdxdff}{rgb}{0.490196078431,0.490196078431,1.}
\definecolor{zzttqq}{rgb}{0.6,0.2,0.}
\definecolor{qqqqff}{rgb}{0.,0.,1.}
\begin{tikzpicture}[line cap=round,line join=round,>=triangle 45,x=1.2cm,y=1.2cm]
\clip(0.684738292011,-1.30457300275) rectangle (6.49190082645,4.11691460055);
\fill[color=zzttqq,fill=zzttqq,fill opacity=0.1] (1.56628099174,2.73950413223) -- (1.37592286501,0.73305785124) -- (5.63239669421,0.844187327824) -- (4.62964187328,2.55217630854) -- cycle;
\draw [color=zzttqq] (1.56628099174,2.73950413223)-- (1.37592286501,0.73305785124);
\draw [color=zzttqq] (1.37592286501,0.73305785124)-- (5.63239669421,0.844187327824);
\draw [color=zzttqq] (5.63239669421,0.844187327824)-- (4.62964187328,2.55217630854);
\draw [color=zzttqq] (4.62964187328,2.55217630854)-- (1.56628099174,2.73950413223);
\draw [dotted,domain=0.684738292011:6.49190082645] plot(\x,{(--8.68549780297-0.187327823691*\x)/3.06336088154});
\draw [dotted,domain=0.684738292011:6.49190082645] plot(\x,{(--10.4665844015-1.70798898072*\x)/1.00275482094});
\draw [dotted,domain=0.684738292011:6.49190082645] plot(\x,{(-2.96733597128-0.111129476584*\x)/-4.2564738292});
\draw [dotted,domain=0.684738292011:6.49190082645] plot(\x,{(-2.6211717961--2.00644628099*\x)/0.190358126722});
\draw [domain=0.684738292011:6.49190082645] plot(\x,{(-1.29484788515--0.903991457561*\x)/0.838941209326});
\draw [line width=1.2pt] (1.56628099174,2.73950413223)-- (5.63239669421,0.844187327824);
\begin{scriptsize}
\draw [fill=qqqqff] (1.56628099174,2.73950413223) circle (1.5pt);
\draw[color=qqqqff] (1.64341597796,2.89377410468) node {$A$};
\draw [fill=qqqqff] (1.37592286501,0.73305785124) circle (1.5pt);
\draw[color=qqqqff] (1.45608815427,0.88826446281) node {$B$};
\draw [fill=qqqqff] (5.63239669421,0.844187327824) circle (1.5pt);
\draw[color=qqqqff] (5.70953168044,0.998457300275) node {$C$};
\draw [fill=qqqqff] (4.62964187328,2.55217630854) circle (1.5pt);
\draw[color=qqqqff] (4.7067768595,2.70644628099) node {$D$};
\draw[color=black] (-0.703691460055,2.80561983471) node {$e$};
\draw[color=black] (3.18611570248,5.2408815427) node {$f$};
\draw[color=black] (-0.703691460055,0.85520661157) node {$g$};
\draw[color=black] (1.91889807163,5.2408815427) node {$h$};
\draw [fill=xdxdff] (1.29202847507,-0.151220585355) circle (1.5pt);
\draw[color=xdxdff] (1.40099173554,0.0067217630854) node {$A'$};
\draw [fill=xdxdff] (2.1309696844,0.752770872205) circle (1.5pt);
\draw[color=xdxdff] (2.23845730028,0.910303030303) node {$B'$};
\draw[color=black] (6.11724517906,5.2408815427) node {$i$};
\draw [fill=uuuuuu] (4.30851171113,3.09915625506) circle (1.5pt);
\draw[color=uuuuuu] (4.42027548209,3.25741046832) node {$C'$};
\draw [fill=uuuuuu] (3.84539823638,2.60013365324) circle (1.5pt);
\draw[color=uuuuuu] (3.95746556474,2.75052341598) node {$D'$};
\draw [fill=uuuuuu] (3.24748135009,1.95585518471) circle (1.5pt);
\draw[color=uuuuuu] (3.32936639118,2.11140495868) node {$E$};
\end{scriptsize}
\end{tikzpicture}
\caption{Oppgave 2.}
\end{figure}
\end{losn}

\begin{oppg}
Vis at halveringslinjene til to vinkler i en trekant og halveringslinja til den utvendige vinkelen i det tredje hjørnet skjærer sine respektive motsatte linjer i trekanten i kollineære punkter.
\end{oppg}
\begin{losn}
Strategien er å kombinere Menelaos' setning med setningen om halveringslinjer.

Se på Figur 2. Legg først merke til at punktene $D,E,F$ er Menelaos-punktet for henholdsvis sidene $BC,CE$ og $AB$. Så ved Menelaos' setning holder det å vise at 
$$
\frac{\ol{BD}}{\ol{DC}} \cdot
\frac{\ol{CE}}{\ol{EA}} \cdot
\frac{\ol{AF}}{\ol{FB}}  = -1
$$
Nå skal vi bruke setningen om halveringslinjer. Den sier at om en linje er en halveringslinje for en vinkel, så deler den den motsatte siden i et bestemt forhold: nemlig hvis linjen $AD$ er en halveringslinje for vinkelen $\angle BAC$, så deler skjæringspunktet $D$ med $BC$ linjen $BC$ i forholdet $\frac{\ol{BD}}{\ol{DC}}=\frac{\ol{BA}}{\ol{AC}}$. Det samme gjelder om en linje deler den utvendige vinkelen, men da må vi sette et minustegn foran forholdet.

Bruk dette resultatet på alle vinklene i trekanten. Da får vi i tillegg at
$$
\frac{\ol{CE}}{\ol{EA}} = \frac{\ol{CB}}{\ol{BA}}
$$
og
$$
\frac{\ol{AF}}{\ol{FB}} = -\frac{\ol{AC}}{\ol{CB}}.
$$
Dermed er:
$$
\frac{\ol{BD}}{\ol{DC}} \cdot
\frac{\ol{CE}}{\ol{EA}} \cdot
\frac{\ol{AF}}{\ol{FB}}  = 
\frac{\ol{BA}}{\ol{AC}} \cdot
\frac{\ol{CB}}{\ol{BA}} \cdot 
-\frac{\ol{AC}}{\ol{CB}}  = -1.
$$
Dermed følger det fra Menelaos setning at $D,E,F$ er kollineære.
\begin{figure}
\centering
\definecolor{uuuuuu}{rgb}{0.266666666667,0.266666666667,0.266666666667}
\definecolor{zzttqq}{rgb}{0.6,0.2,0.}
\definecolor{qqqqff}{rgb}{0.,0.,1.}
\begin{tikzpicture}[line cap=round,line join=round,>=triangle 45,x=1.0cm,y=1.0cm]
\clip(-0.277341393951,-2.48160676533) rectangle (10.8770625339,3.52891164188);
\fill[color=zzttqq,fill=zzttqq,fill opacity=0.1] (4.46,1.78) -- (4.42,-0.12) -- (8.42,-0.2) -- cycle;
\draw [color=zzttqq] (4.46,1.78)-- (4.42,-0.12);
\draw [color=zzttqq] (4.42,-0.12)-- (8.42,-0.2);
\draw [color=zzttqq] (8.42,-0.2)-- (4.46,1.78);
\draw [domain=-0.277341393951:10.8770625339] plot(\x,{(-3.14719039428--0.692446629416*\x)/0.721469102187});
\draw [domain=-0.277341393951:10.8770625339] plot(\x,{(-1.82217753643--0.239472466458*\x)/-0.970903155731});
\draw [domain=-0.277341393951:10.8770625339] plot(\x,{(--0.780778233091-0.516749263388*\x)/-0.856136787428});
\draw (1.41180715944,-0.0598361431889)-- (4.43898765112,0.781913428293);
\draw (4.43898765112,0.781913428293)-- (5.73527291014,1.14236354493);
\begin{scriptsize}
\draw [fill=qqqqff] (4.42,-0.12) circle (1.5pt);
\draw[color=qqqqff] (4.51709538203,0.0763580451837) node {$A$};
\draw [fill=qqqqff] (8.42,-0.2) circle (1.5pt);
\draw[color=qqqqff] (8.51478902032,-0.00750965352157) node {$B$};
\draw [fill=qqqqff] (4.46,1.78) circle (1.5pt);
\draw[color=qqqqff] (4.55902923139,1.97735921584) node {$C$};
\draw[color=black] (9.66098090263,5.13637586707) node {$h$};
\draw[color=black] (-1.39557737669,2.11713871368) node {$j$};
\draw[color=black] (-1.39557737669,-1.50315028043) node {$k$};
\draw [fill=uuuuuu] (5.73527291014,1.14236354493) circle (1.5pt);
\draw[color=uuuuuu] (5.83102266175,1.33437352576) node {$D$};
\draw [fill=uuuuuu] (4.43898765112,0.781913428293) circle (1.5pt);
\draw[color=uuuuuu] (4.53107333182,0.970946831374) node {$E$};
\draw [fill=uuuuuu] (1.41180715944,-0.0598361431889) circle (1.5pt);
\draw[color=uuuuuu] (1.51183617843,0.132269844321) node {$F$};
\end{scriptsize}
\end{tikzpicture}
\caption{Oppgave 3.}
\end{figure}
\end{losn}

\begin{oppg}
  Vis at halveringslinjene til de utvendige vinklene i en trekant skjærer sine respektive motsatte linjer i trekanten i kollineære punkter.
\end{oppg}

\begin{losn}
Dette er bare nok en anvendelse av Menelaos' setning og setningen om halveringslinjer. Faktisk er løsningen helt lik forrige oppgave, så jeg gidder ikke skrive ned detaljene.

Trikset er at $(-1) \cdot (-1) \cdot (-1) = -1$. Se Figur 3, og legg merke til at $D,E,F$ er Menelaos-punkter her også.

\begin{figure}
\centering
\definecolor{uuuuuu}{rgb}{0.266666666667,0.266666666667,0.266666666667}
\definecolor{sqsqsq}{rgb}{0.125490196078,0.125490196078,0.125490196078}
\definecolor{zzttqq}{rgb}{0.6,0.2,0.}
\definecolor{qqqqff}{rgb}{0.,0.,1.}
\begin{tikzpicture}[line cap=round,line join=round,>=triangle 45,x=1.0cm,y=1.0cm]
\clip(-4.46333333333,-0.683333333333) rectangle (9.68333333333,7.75666666667);
\fill[color=zzttqq,fill=zzttqq,fill opacity=0.1] (1.9,2.92) -- (0.92,1.7) -- (5.6,2.9) -- cycle;
\draw [color=zzttqq] (1.9,2.92)-- (0.92,1.7);
\draw [color=zzttqq] (0.92,1.7)-- (5.6,2.9);
\draw [color=zzttqq] (5.6,2.9)-- (1.9,2.92);
\draw [dotted,color=sqsqsq,domain=-4.46333333333:9.68333333333] plot(\x,{(--0.5436--1.22*\x)/0.98});
\draw [dotted,color=sqsqsq,domain=-4.46333333333:9.68333333333] plot(\x,{(--10.842-0.02*\x)/3.7});
\draw [dotted,color=sqsqsq,domain=-4.46333333333:9.68333333333] plot(\x,{(-6.852-1.2*\x)/-4.68});
\draw [domain=-4.46333333333:9.68333333333] plot(\x,{(--1.81975280612--0.429850337839*\x)/0.902900153428});
\draw [domain=-4.46333333333:9.68333333333] plot(\x,{(-1.69428389223--0.840532453468*\x)/-0.541761197085});
\draw [domain=-4.46333333333:9.68333333333] plot(\x,{(--5.91305219519-0.992469713592*\x)/0.122490275545});
\draw (-2.50993600105,0.8205292305)-- (0.127479287318,2.92958119304);
\draw (0.127479287318,2.92958119304)-- (5.10508646809,6.91000560314);
\begin{scriptsize}
\draw [fill=qqqqff] (1.9,2.92) circle (1.5pt);
\draw[color=qqqqff] (1.99,3.10333333333) node {$A$};
\draw [fill=qqqqff] (0.92,1.7) circle (1.5pt);
\draw[color=qqqqff] (1.01666666667,1.89) node {$B$};
\draw [fill=qqqqff] (5.6,2.9) circle (1.5pt);
\draw[color=qqqqff] (5.69666666667,3.09) node {$C$};
\draw[color=black] (-2.75666666667,7.66333333333) node {$i$};
\draw [fill=uuuuuu] (-2.50993600105,0.8205292305) circle (1.5pt);
\draw[color=uuuuuu] (-2.41,1.01) node {$D$};
\draw [fill=uuuuuu] (0.127479287318,2.92958119304) circle (1.5pt);
\draw[color=uuuuuu] (0.216666666667,3.11666666667) node {$E$};
\draw [fill=uuuuuu] (5.10508646809,6.91000560314) circle (1.5pt);
\draw[color=uuuuuu] (5.20333333333,7.10333333333) node {$F$};
\end{scriptsize}
\end{tikzpicture}
\caption{Oppgave 4.}
\end{figure}
\end{losn}

\begin{oppg}
Vis at tangentlinjene til den omskrevne sirkelen til en trekant i hjørnene skjærer sine respektive motsatte linjer i trekanten i kollineære punkter.
\end{oppg}


\begin{losn}
 Igjen skal vi bruke Menelaos' setning. Vi ønsker å vise at det vanlige uttrykket blir $1$ (eller $-1$...).

Se på Figur 4. Vi skal bruke det vi vet om periferivinkler. Nemlig at to periferivinkler over samme bue er like store.

For det første: vinklene $\angle ABF$ og $\angle ACB$ spenner over samme bue. Dermed følger det at trekantene $BCF$ og $BFA$ er formlike. Dermed er (her bruker vi vanlig lengdemål, og ikke fortegnsmål)
$$
\frac{BC}{AB}=\frac{CF}{BF} = \frac{BF}{AF}.
$$
Tilsvarende får vi for de andre to vinklene at:
$$
\frac{AB}{CA} = \frac{BE}{AE} = \frac{AE}{CE}
$$
og 
$$
\frac{AC}{BC} = \frac{AD}{CD} = \frac{CD}{BD}.
$$
Ganger vi de to siste likhetene i hver linje sammen, får vi dermed at
\begin{align*}
\frac{BC^2}{AB^2} = \frac{CF}{AF} && \frac{AB^2}{CA^2} = \frac{BE}{CE} &&  \frac{AC^2}{BC^2} = \frac{AD}{BD}.
\end{align*}
Dette er akkurat leddene i Menelaos-ligningen, så når vi ganger dem sammen, får vi $1$, akkurat som vi ønsket (som betyr at punktene ligger på linje).

Grunnen til at vi dropper fortegnsmål her, er at vi vet at med fortegnsmål kommer svarer til å bli negativt, siden alle punktene $D,E,F$ ligger utenfor trekanten $ABC$. Dermed holder det å sjekke at vi får riktig absoluttverdi.


\begin{figure}
\centering  
\definecolor{uuuuuu}{rgb}{0.266666666667,0.266666666667,0.266666666667}
\definecolor{zzttqq}{rgb}{0.6,0.2,0.}
\definecolor{qqqqff}{rgb}{0.,0.,1.}
\begin{tikzpicture}[line cap=round,line join=round,>=triangle 45,x=0.6929824561403509cm,y=0.68298cm]
\clip(-4.3,-5.86) rectangle (18.5,6.3);
\fill[color=zzttqq,fill=zzttqq,fill opacity=0.1] (5.46,2.18) -- (3.34,-0.22) -- (10.48,0.68) -- cycle;
\draw [color=zzttqq] (5.46,2.18)-- (3.34,-0.22);
\draw [color=zzttqq] (3.34,-0.22)-- (10.48,0.68);
\draw [color=zzttqq] (10.48,0.68)-- (5.46,2.18);
\draw(7.11810874704,-1.42099605989) ellipse (2.74726226069cm and 2.70760848587cm);
\draw [domain=-4.3:18.5] plot(\x,{(-1.20310234831--1.65810874704*\x)/3.60099605989});
\draw [domain=-4.3:18.5] plot(\x,{(-12.8831023483--3.77810874704*\x)/1.20099605989});
\draw [domain=-4.3:18.5] plot(\x,{(--36.6612976517-3.36189125296*\x)/2.10099605989});
\draw [dotted,domain=-4.3:18.5] plot(\x,{(-8.4824--2.4*\x)/2.12});
\draw [dotted,domain=-4.3:18.5] plot(\x,{(--19.1336-1.5*\x)/5.02});
\draw [dotted,domain=-4.3:18.5] plot(\x,{(--4.5768-0.9*\x)/-7.14});
\draw (-0.9177585485,-0.756692254013)-- (4.22063945762,2.55032685529);
\draw (4.22063945762,2.55032685529)-- (7.85099263739,4.8867841178);
\begin{scriptsize}
\draw [fill=qqqqff] (5.46,2.18) circle (1.5pt);
\draw[color=qqqqff] (5.6,2.46) node {$A$};
\draw [fill=qqqqff] (3.34,-0.22) circle (1.5pt);
\draw[color=qqqqff] (3.48,0.06) node {$B$};
\draw [fill=qqqqff] (10.48,0.68) circle (1.5pt);
\draw[color=qqqqff] (10.62,0.96) node {$C$};
\draw [fill=uuuuuu] (7.85099263739,4.8867841178) circle (1.5pt);
\draw[color=uuuuuu] (8.,5.16) node {$D$};
\draw [fill=uuuuuu] (-0.9177585485,-0.756692254013) circle (1.5pt);
\draw[color=uuuuuu] (-0.78,-0.48) node {$E$};
\draw [fill=uuuuuu] (4.22063945762,2.55032685529) circle (1.5pt);
\draw[color=uuuuuu] (4.36,2.84) node {$F$};
\end{scriptsize}
\end{tikzpicture}
\caption{Oppgave 5.}
\end{figure}
\end{losn}

Til slutt en liten kommentar om periferivinkler. I beviset over brukte vi at $\angle ABF = \angle ACB$ fordi begge er periferivinkler med samme bue. Det er ikke helt åpenbart at dette faktisk er lov å konkludere med, siden setningen om periferivinkler ikke nødvendigvis gjelder når en av kordene er en tangent.

Se på Figur 5. Ved setningen om periferivinkler er vinklene $\angle BAC$ og $\beta = \CDB$ like siden de spenner over samme bue. Med andre er funksjonen $\beta(D)$ konstant når vi beveger $D$ på sirkelen. Men når $D \to C$, vil linjen $DC$ gå mot tangentlinjen i punktet $C$. Men siden $\beta(D)$ er konstant som funksjon av $D$, er også $\lim_{D \to C} \beta(D)$ samme verdi. Dermed kan vi snakke om vinkelen mellom en korde og en tangentlinje. 

Dette er ikke et hundre prosent rigorøst bevis. Det gir først mening når vi har innført en metrikk på $E^2$. Det kan antakelig bevises på en lignende måte som setningen om periferivinkler.

\begin{figure}
\centering
\definecolor{xdxdff}{rgb}{0.490196078431,0.490196078431,1.}
\definecolor{qqqqff}{rgb}{0.,0.,1.}
\begin{tikzpicture}[line cap=round,line join=round,>=triangle 45,x=1.0cm,y=1.0cm]
\clip(0.307635644359,-4.30828377277) rectangle (15.2758321506,3.72101369082);
\draw(6.09059529098,-0.219222567748) circle (2.59522457753cm);
\draw (3.54,0.26)-- (8.62,-0.8);
\draw (3.54,0.26)-- (7.08,2.18);
\draw (3.54,0.26)-- (8.26020192559,-1.64330039528);
\draw (8.26020192559,-1.64330039528)-- (7.08,2.18);
\draw [domain=0.307635644359:15.2758321506] plot(\x,{(-7.55708786687--0.843300395284*\x)/0.35979807441});
\begin{scriptsize}
\draw [fill=qqqqff] (3.54,0.26) circle (1.5pt);
\draw[color=qqqqff] (3.6118732837,0.41677605148) node {$A$};
\draw [fill=qqqqff] (7.08,2.18) circle (1.5pt);
\draw[color=qqqqff] (7.15842168326,2.3332338823) node {$B$};
\draw [fill=qqqqff] (8.62,-0.8) circle (1.5pt);
\draw[color=qqqqff] (8.70039924828,-0.640579993108) node[below] {$C$};
\draw [fill=xdxdff] (8.26020192559,-1.64330039528) circle (1.5pt);
\draw[color=xdxdff] (8.33693310795,-1.48866765387) node[below] {$D$};
\draw[color=black] (10.352518068,3.64391481257);
\end{scriptsize}
\end{tikzpicture}
\caption{Begrunnelse for periferivinkler og tangenter.}
\end{figure}
\end{document}

