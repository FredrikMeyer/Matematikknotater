\documentclass[11pt, norsk]{article}
%\usepackage[latin1]{inputenc}
\usepackage[T1]{fontenc}
\usepackage[utf8]{inputenc}
\usepackage[norsk]{babel}   % S P R A A K


% \usepackage{graphicx}    % postscript graphics
\usepackage{amssymb, amsmath, amsthm, amssymb} % symboler, osv
\usepackage{mathrsfs}
\usepackage{url}
\usepackage{thmtools}
\usepackage{enumerate}  % lister $  
\usepackage{float}
\usepackage{tikz}
\usetikzlibrary{calc}
\usepackage{tikz-3dplot}
\usepackage{subcaption}
\usepackage[all]{xy}   % for comm.diagram
\usepackage{wrapfig} % for float right
\usepackage{hyperref}
\usepackage{mystyle} % stilfilen      


\begin{document}
\title{Oppgaver MAT2500}
\author{Fredrik Meyer}
\maketitle 

\begin{oppg}
Vis at om $A,B,C,D$ er kollineære, så er 
\[
\overline{AB} \cdot \ol{BC}  + \ol{BD} \cdot \ol{CA} + \ol{CD} \cdot \ol{AB} = 0.
\]
\end{oppg}
\begin{losn}
 Her mener vi altså med $\ol{AB}$ \emph{fortegnslengden} til segmentet $AB$. For å gi en verdi til $\ol{AB}$ (etc.) må man velge en \emph{retning} langs linja (for hvordan skal en ellers si om $\ol{AB}$ er positiv eller negativ?). Det som holder uansett er nemlig at $\ol{AB} + \ol{BC} = \ol{AC}$ om $A,B,C$ er kollineære.

Trikset her er dermed å bruke egenskapen over gjentatte ganger, og omskrive hver av lengdene over slik at de involverer $A$. For eksempel: $\ol{BC}$ kan skrive som $\ol{AC}-\ol{AB}$. $\ol{BD} = \ol{AD}-\ol{AB}$ og $\ol{CA} = -\ol{AC}$. Dermed er
\begin{align*}
& \overline{AB} \cdot \ol{BC}  + \ol{BD} \cdot \ol{CA} + \ol{CD} \cdot \ol{AB}. \\
&= \overline{AD}\cdot (\ol{AC}-\ol{AB}) - (\ol{AD}-\ol{AB})\cdot \ol{AC} + (\ol{AD-AC}) \cdot \ol{AB} \\
&= 0
\end{align*}
På grunn av masse kansellering.
\end{losn}

\begin{oppg}
Vis følgende generalisering av Menelaos' setning: for en firkant $ABCD$ vil punktene $A',B',C',D'$ på linjene gjennom $AB, BC, CD$ og $DA$ være kollineære bare hvis 
$$
\frac{\ol{AA'}}{\ol{A'B}}\cdot \frac{\ol{BB'}}{\ol{B'C}} \cdot \frac{\ol{CC'}}{\ol{C'D}} \cdot \frac{\ol{DD'}}{\ol{D'A}} = 1
$$ 
\end{oppg}

\begin{losn}
Se på Figur 1.

Trikset er å merke at enhver firkant er satt sammen av to trekanter. Vi skal bruke Menelaos' setning på trekantene $BCA$ og $CDA$.

Fra Menelaos' setning har vi følgende likhet:
$$
\frac{\ol{BB'}}{\ol{B'C}} \cdot \frac{\ol{CE}}{\ol{EA}} \cdot
\frac{\ol{AA'}}{\ol{A'B}} = -1,
$$
brukt på den nederste av de to trekantene. Vi har også likheten
$$
\frac{\ol{CC'}}{\ol{C'D}} \cdot
\frac{\ol{DD'}}{\ol{D'A}} \cdot
\frac{\ol{AE}}{\ol{EC}}
= -1
$$
Ganger vi disse to uttrykkene sammen, får vi (etter litt omstokking):
$$
\frac{\ol{BB'}}{\ol{B'C}} \cdot
\frac{\ol{AA'}}{\ol{A'B}} \cdot 
\frac{\ol{CC'}}{\ol{C'D}} \cdot
\frac{\ol{DD'}}{\ol{D'A}} \cdot
\frac{-\ol{EC}}{-\ol{AE}} \cdot
\frac{\ol{AE}}{\ol{EC}} = 1
$$
De to siste leddene kansellerer, og vi får uttrykket vi har lyst på.
\begin{figure}
\centering
\definecolor{uuuuuu}{rgb}{0.266666666667,0.266666666667,0.266666666667}
\definecolor{xdxdff}{rgb}{0.490196078431,0.490196078431,1.}
\definecolor{zzttqq}{rgb}{0.6,0.2,0.}
\definecolor{qqqqff}{rgb}{0.,0.,1.}
\begin{tikzpicture}[line cap=round,line join=round,>=triangle 45,x=2.0cm,y=2.0cm]
\clip(0.684738292011,-1.30457300275) rectangle (6.49190082645,4.11691460055);
\fill[color=zzttqq,fill=zzttqq,fill opacity=0.1] (1.56628099174,2.73950413223) -- (1.37592286501,0.73305785124) -- (5.63239669421,0.844187327824) -- (4.62964187328,2.55217630854) -- cycle;
\draw [color=zzttqq] (1.56628099174,2.73950413223)-- (1.37592286501,0.73305785124);
\draw [color=zzttqq] (1.37592286501,0.73305785124)-- (5.63239669421,0.844187327824);
\draw [color=zzttqq] (5.63239669421,0.844187327824)-- (4.62964187328,2.55217630854);
\draw [color=zzttqq] (4.62964187328,2.55217630854)-- (1.56628099174,2.73950413223);
\draw [dotted,domain=0.684738292011:6.49190082645] plot(\x,{(--8.68549780297-0.187327823691*\x)/3.06336088154});
\draw [dotted,domain=0.684738292011:6.49190082645] plot(\x,{(--10.4665844015-1.70798898072*\x)/1.00275482094});
\draw [dotted,domain=0.684738292011:6.49190082645] plot(\x,{(-2.96733597128-0.111129476584*\x)/-4.2564738292});
\draw [dotted,domain=0.684738292011:6.49190082645] plot(\x,{(-2.6211717961--2.00644628099*\x)/0.190358126722});
\draw [domain=0.684738292011:6.49190082645] plot(\x,{(-1.29484788515--0.903991457561*\x)/0.838941209326});
\draw [line width=1.2pt] (1.56628099174,2.73950413223)-- (5.63239669421,0.844187327824);
\begin{scriptsize}
\draw [fill=qqqqff] (1.56628099174,2.73950413223) circle (1.5pt);
\draw[color=qqqqff] (1.64341597796,2.89377410468) node {$A$};
\draw [fill=qqqqff] (1.37592286501,0.73305785124) circle (1.5pt);
\draw[color=qqqqff] (1.45608815427,0.88826446281) node {$B$};
\draw [fill=qqqqff] (5.63239669421,0.844187327824) circle (1.5pt);
\draw[color=qqqqff] (5.70953168044,0.998457300275) node {$C$};
\draw [fill=qqqqff] (4.62964187328,2.55217630854) circle (1.5pt);
\draw[color=qqqqff] (4.7067768595,2.70644628099) node {$D$};
\draw[color=black] (-0.703691460055,2.80561983471) node {$e$};
\draw[color=black] (3.18611570248,5.2408815427) node {$f$};
\draw[color=black] (-0.703691460055,0.85520661157) node {$g$};
\draw[color=black] (1.91889807163,5.2408815427) node {$h$};
\draw [fill=xdxdff] (1.29202847507,-0.151220585355) circle (1.5pt);
\draw[color=xdxdff] (1.40099173554,0.0067217630854) node {$A'$};
\draw [fill=xdxdff] (2.1309696844,0.752770872205) circle (1.5pt);
\draw[color=xdxdff] (2.23845730028,0.910303030303) node {$B'$};
\draw[color=black] (6.11724517906,5.2408815427) node {$i$};
\draw [fill=uuuuuu] (4.30851171113,3.09915625506) circle (1.5pt);
\draw[color=uuuuuu] (4.42027548209,3.25741046832) node {$C'$};
\draw [fill=uuuuuu] (3.84539823638,2.60013365324) circle (1.5pt);
\draw[color=uuuuuu] (3.95746556474,2.75052341598) node {$D'$};
\draw [fill=uuuuuu] (3.24748135009,1.95585518471) circle (1.5pt);
\draw[color=uuuuuu] (3.32936639118,2.11140495868) node {$E$};
\end{scriptsize}
\end{tikzpicture}
\caption{Oppgave 2.}
\end{figure}
\end{losn}



\end{document}
