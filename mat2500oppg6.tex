\documentclass[11pt, norsk]{article}
%\usepackage[latin1]{inputenc}
\usepackage[T1]{fontenc}
\usepackage[utf8]{inputenc}
\usepackage[norsk]{babel}   % S P R A A K


% \usepackage{graphicx}    % postscript graphics
\usepackage{amssymb, amsmath, amsthm, amssymb} % symboler, osv
\usepackage{mathrsfs}
\usepackage{url}
\usepackage{thmtools}
\usepackage{enumerate}  % lister $  
\usepackage{float}
\usepackage{tikz}
\usetikzlibrary{calc}
\usepackage{tikz-3dplot}
\usepackage{subcaption}
\usepackage[all]{xy}   % for comm.diagram
\usepackage{wrapfig} % for float right
\usepackage{hyperref}
\usepackage{mystyle} % stilfilen      


\begin{document}
\title{Oppgaver MAT2500}
\author{Fredrik Meyer}
\maketitle 

\begin{oppg}
Vis at om $A,B,C,D$ er kollineære, så er 
\[
\overline{AB} \cdot \ol{BC}  + \ol{BD} \cdot \ol{CA} + \ol{CD} \cdot \ol{AB} = 0.
\]
\end{oppg}
\begin{losn}
 Her mener vi altså med $\ol{AB}$ \emph{fortegnslengden} til segmentet $AB$. For å gi en verdi til $\ol{AB}$ (etc.) må man velge en \emph{retning} langs linja (for hvordan skal en ellers si om $\ol{AB}$ er positiv eller negativ?). Det som holder uansett er nemlig at $\ol{AB} + \ol{BC} = \ol{AC}$ om $A,B,C$ er kollineære.

Trikset her er dermed å bruke egenskapen over gjentatte ganger, og omskrive hver av lengdene over slik at de involverer $A$. For eksempel: $\ol{BC}$ kan skrive som $\ol{AC}-\ol{AB}$. $\ol{BD} = \ol{AD}-\ol{AB}$ og $\ol{CA} = -\ol{AC}$. Dermed er
\begin{align*}
& \overline{AB} \cdot \ol{BC}  + \ol{BD} \cdot \ol{CA} + \ol{CD} \cdot \ol{AB}. \\
&= \overline{AD}\cdot (\ol{AC}-\ol{AB}) - (\ol{AD}-\ol{AB})\cdot \ol{AC} + (\ol{AD-AC}) \cdot \ol{AB} \\
&= 0
\end{align*}
På grunn av masse kansellering.
\end{losn}

\end{document}
