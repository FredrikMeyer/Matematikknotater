\documentclass[11pt, english]{article}
%\usepackage[latin1]{inputenc}
\usepackage[T1]{fontenc}
\usepackage[utf8]{inputenc}
\usepackage[english]{babel}   % S P R A A K
% \usepackage{graphicx}    % postscript graphics
\usepackage{amssymb, amsmath, amsthm} % symboler, osv
\usepackage{mathrsfs}
\usepackage{url}
\usepackage{thmtools}
\usepackage{enumerate}  % lister $  
\usepackage{float}
\usepackage{tikz}
\usepackage{tikz-cd}
\usetikzlibrary{calc}
%\usepackage{tikz-3dplot}
\usepackage{subcaption}
\usepackage[all]{xy}   % for comm.diagram
\usepackage{wrapfig} % for float right
\usepackage{hyperref}
\usepackage{mystyle} % stilfilen      

%\usepackage[a5paper,margin=0.5in]{geometry}


\begin{document}
\title{Hyper-Kähler manifolds}
\author{Fredrik Meyer}
\maketitle 

\begin{abstract}
These are notes from the ``summer school'' at IMPA, Warzaw, held by Kieran O'Grady.
\end{abstract}

\section{Lecture 1 - Introduction}

We will first motivate the definition of hyper-Kähler by looking at K3 surfaces. 

By definition, a K3 surface is compact Kähler 2-dimensional complex manifold that is simply connected and has trivial canonical bundle ($K_X \simeq \OO_X$).

\begin{example}
Let $X \subset \PP^3$ be a smooth quartic surface. Then by Lefschetz:
$$
\pi_(X,\ast) \xrightarrow{\sim} \pi_1(\PP^3,\ast) = \{1\},
$$
so $X$ is simply-connected. By adjunction, we have:
$$
K_x \simeq (\restr{K_{\PP^3}}{\X}) \otimes \mathcal N_{X/\PP^3} = \OO_X(-4) \otimes \OO_X(4) = \OO_X.
$$
\end{example}

We list some of the know results about K3's:

\begin{enumerate}
\item Any two K3's are deformation equivalent (Kodaira).
\item There is a Hodge-theoretic description of the Kähler cone of a K3 (after having chosen one Kähler class).
\item There is a \emph{Global Torelli Theorem} (Shafarevich and Piateski-Shapiro). Namely, a Hodge structure on $H^2(K3; \C)$ and a lattice structure on $H^2(K3,\Z)$ determines $X$ up to isomorphism.
\end{enumerate}

We now state the definition of a hyper-Kähler manifold:

A hyper-Kähler manifold $X$ is a compact Kähler manifold $X$, simply connected, such that $H^0(X,\Omega^2,X) = \C \sigma$, where $\sigma$ is \emph{symplectic}, meanig that $T_xX \times T_xX \xrightarrow{\sigma(x)} \C$ is non-degenerate for all $x \in X$.

\begin{remark}
A HK must have even dimension: the skew-symmetric form $\sigma$ can be represented by a $n\times n$-matrix $A$ that is skew-symmetric with non-zero determinant. Skew-symmety of $A$ means that $A^T = -A$. Hence $\det A = \det A^T = (-1)^n \det A$. This forces $n$ to be even if we want $\det A \neq 0$.
\end{remark}

In dimension $2$, K3's are hyper-Kähler.

To motivate hyper-Käher manifolds, we state the Beaville-Bogomolov decomposition theorem:

\begin{thm}
Let $Z$ be a compact Kähler manifold with $c_1(Z)=0$ (what does this mean?) in $H^2(X, \Q)$. Then there exists a finite étale cover $\widetilde Z \to Z$ such that 
$$
\widetilde Z = \C^d / \Lambda \times \prod_i X_i  \times \prod_j Y_i
$$
where the first factor is a compact torus. The second factor is a product of hyper-Kähler manifolds, and the second is a product of Calabi-Yau manifolds (that is, manifolds with $K_{Y_i} \simeq \OO_{Y_i}$ and $h^0(\Omega_{Y_i}^p)=0$ for all $0 < p < \dim Y_i$).
\end{thm}

\subsection{First example of a HK}

Now we define some higher-dimensional examples of HK's. Terminology: when we say ``HK variety'', we shall mean a \emph{projective} HK manifold.

Let $S$ be a K3 surface. Then let $S^{(2)}$ be the symmetric square of $S$ (that is, $S \times S/(p,q) \sim (q,p)$). This space comes equipped with two projections, $\pi_1, \pi_2$, and the form $\pi_1^\ast \sigma + \pi_2 ^\ast \sigma \in H^0(\Omega_{S \times S}^2)$ is $\tau$-invariant, hence it descends to a holomorphic 2-form on $S^{(2)}$. 

But the symmetric square is singular along the (image of the) diagonal, where two points come together. In fact, one can see that it locally looks like $\C \times \C /x \sim -x$. The last factor is a quadric cone, hence a single blowup along the diagonal will resolve the singularities in this case.

Let $D$ be the diagonal. We have a diagram
$$
\xymatrix{
	Bl_D(S^2) \ar[d] \ar[r]^{\widetilde \rho} & S^{[2]} \ar[d]^\gamma  \\
	S^2 \ar[r]^\rho & S^{(2)}
}
$$

The top right space $S^{[2]}$ can be thought of in two ways: first, notice that the involution on $S^2$ act on the blowup as well. Hence we can take the quotient of the blowup. This is one description of $S^{[2]}$. Secondly, one can think of $S^{[2]}$ as the blowup of $S^{(2)}$ along the points $2P$ (the image of the diagonal). 

Now $S^{[2]}$ is smooth projective and ``$\pi_1^\ast \sigma + \pi_2^\ast \sigma$'' is symplectic.

Then $H^2(S^{[2]})$ is spanned by this symplectic 2-form. We can show that $S^{[2]}$ is simply-connected as well:
$$
\pi_1(S^{[2]} \bs D,\ast) \twoheadrightarrow \pi_1(S^{[2]},\ast).
$$
The first group is $\Z/2$, generated by a loop around $D$ (we abuse notation: $D$ denotes the image of the diagonal in $S^{[2]}$), and it is $\Z/2$ because $S^2  \bs D$ is a double cover of $S^{[2]} \bs D$.

We call $S^{[2]}$ the \emph{Hilbert square} of $S$. It is a HK variety of dimension $4$. It can also be realized as the Hilbert scheme parametrizing length 2 subschemes of $S$.

In general,
$$
Y^{[n]} := \{  Z \subset Y  \mid l(Z) = n \},
$$
is smooth, irreducible of dimension $2n$ (if $Y$ is a surface).

Beauville shows that if $S$ is a K3, then $S^{[n]}$ is always a HK variety of dimension $2n$. Hence we have examples of hyper-Kählers in each even dimension!

If $n\geq 2$, then $b_2(S^{[2]})= 23$, which is $22+1$, the last divisor coming from the blowup. 

\subsection{Second example of a HK}

The analogue of $S^{[2]}$ with $S$ replaced by $A$, an abelian surface.

The space $A^{[2]}$ has a holomorphic symplectic form, but is far from being simply connected. But consider the maps
$$
A^{[2]} \xrightarrow{\gamma_2} A^{(2)} \xrightarrow{s_2} A .
$$
The first map sends a points $z$ to the sum $\sum_{p \in A} l(\OO_{Z,P} P $. The second map sends $P+Q$ to $(p) + (q) \in A$, where the parenthesis means that we actually consider the sum in the \emph{group} $A$.

Then the composition $s_2 \circ \gamma_2$ is a locally trivial fibration in the étale/analytic topology. It follows that the cohomology group $H^1(A)=\C^4 \hookrightarrow H^1(A^{[2]})$. 

Now look at the fiber $s_2 \circ \gamma_2^{-1}(0)$. This is a smooth Kummer surface $\approx A/ \langle -1 \rangle$. These are K3 surfaces! 

In general we look at the sequence 
$$
A^{[n+1]} \xrightarrow{\gamma_{n+1}} A^{(n+1)} \xrightarrow{s_{n+1}} A
$$
defined analogously, and we define the \emph{generalized Kummer} to be $Kum^{[n]}(A) := (s_{n+1} \circ \gamma_{n+1})^{-1}(0)$.

Beauville proved that $K^{[n]}$ is a HK variety of dimension $2n$. For $n\geq 2$, we have $b_2(K^{[n]}(A)) = 7 = b_2(A) + 1$, so we do actually have two topologically distinct families.

\subsection{Third example, lines on a cubic 4-fould}

Let $Y \subset \PP^n$ be some algebraic variety. Let $X = F(Y)$ be the set of lines contained in $Y$. It is a closed subset of the Grassmannian $\mathbb G(1,\PP^n)$, which we can think of as embedded via Plücker in $\PP^{\binom{n-1}{2}-1}$.

\begin{thm}
Let $Y \subset \PP^n$ be a smooth cubic hypersurface. Then $X=F(Y)$ is a smooth connected variety of dimension $2n-6$ and $K_X \simeq \OO_X(5-n)$. 
\end{thm}
Since we are interested in HK varieties, we put $n=5$. 

\begin{remark}
All HK's have trivial canonical bundle: the $n/2$th power of the symplectic form gives a trivialization of $K_X = \Omega^n_X$.
\end{remark}

We want to look at a incidence correspondence $\mathcal I \subset Y \times X$:
$$
\xymatrix{
&\mathcal I = \{ (y,L) \in Y \times X \mid y \in L \} \ar[dl]^\rho \ar[dr]^\pi \\
Y && X
}
$$
The fibers of $\rho$ are $\PP^1$s (how to see this?). In general we get a map
$$
H^{n-1}(Y) \xrightarrow{c} H^{n-3}(X)
$$
given by $\alpha \mapsto \pi_\ast(\rho^\ast \alpha)$. So for $n=5$, we get a map $H^4(Y) \to H^2(X)$. 

Beauville and Donagi showed that if $Y\subset \PP^5$ is a smooth cubic hypersurface, then $X=F(Y)$ is a HK variety of type $K3^{[2]}$. Moreover, the restriction of $c$ to the primitive cohomology
$$
H^4(Y)_0 := \{ \alpha \in H^4(Y) \mid \alpha - c_1(\OO_Y(1)) = 0 \}
$$
is an isomorphism to $H^2(X)_0$ of Hodge structures.

We get an isomorphism $H^{p,q}(Y)_0 \simeq H^{p-1,q-1}(X)_0)$, and there exists a bilinear symmetric form $\langle, \rangle$ on $H^2(X)_0$ such that
$$
\langle c(\alpha), c(\beta) \rangle = - \int_Y \alpha \wedge \beta
$$
for all $\alpha, \beta \in H^4(Y)_0$. 

One consequence: if $Y \subset \PP^5$ is very general, then $H^2(X)_0$ has no nonzero integral $(1,1)$-classes. [explanation follows]

Main point: if $Y$ is very general, then $F(Y) \not \simeq S^{[2]}$ (not isomorphic) ($S$ is K3).

Beauville proved however that if $Y$ is a pfaffian cubic
$$
\{ (t_0: \cdots : t_5) \in \PP^5 \mid Pf(t_0A_0+\ldots t_5A_5)\},
$$
where the $A_i$ are skewsymmetric matrices, then $F(Y) \simeq S^{[2]}$. 



\section{Lecture 2 - some of the main general results}

The general theory is developed by Bogomolov, Fujiki, Beauville, Verlotsky, Huybrechts and others. Today's lecture have three main ingredients:

\begin{enumerate}
	\item The deformation theory if HK's are unobstructed.
	\item BBF (Bogomolov-Beauville-Fujiki) quadratic form on $H^2(HK)$.
	\item Twistor families of HK's. This leads to the deepest results.
\end{enumerate}


\subsection{Deformations of a HK X}

Let $\sigma$ be the given holomorphic symplectic form on $X$. Contraction with $\sigma$ defines an isomorphism $L_\sigma: \Theta_X \to \Omega_X^1$.  Hence we get isomorphism
$$
H^i(L_\sigma):H^i(X, \Theta_X) \xrightarrow{\sim} H^i(X,\Omega_X^1).
$$
In particular, $H^0(X, \Theta_X) = H^0(X,\Omega_X^1) = 0$, since $X$ is simply-connected (to see this: by the Hodge decomposition we have $0 = H^1(X; \C) = H^0(X, \Omega_X^1) \oplus H^1(\OO_X)$).

This implies (by general results?) that there exists a universal deformation space of $X$:

$$
\xymatrix{
	X \simeq X_0 \ar@{^(->}[r]  \ar[d] & X \ar[d]^\pi \\
	0 \ar[r] & B
}
$$

Then we have $T_0B = H^1(X, \Theta_X) = H^1(X,\Omega_X^1)$. 

\begin{thm}[Bogomolov]
If $X$ is HK, then $X$ is unobstructed. That is, there exists a universal deformation space $B$ of $X$ with $B$ smooth.
\end{thm}

\begin{corr}
$$
\dim Def(X) = dim B = b_2(X)-2.
$$
\end{corr}
\begin{proof}
Since $$Def(X)$$ is smooth, we have
$$
\dim Def(X) = h^1(X,\Theta_X) = h^1(X, \Omega_X^1) = b_2(X) - h^{2,0}-h^{0,2} = b_2(X) - 2.
$$
\end{proof}

\begin{remark}
This is because the Kodaira-Spencer map $T_0B \xrightarrow{\kappa} H^1(X, \Theta_X)$ is an isomorphism if $B= Def(X)$.
\end{remark}

\begin{example}
Let $X = S^{[n]}$ and $n \geq 2$. Then $b_2(X)=23$, hence $\dim Def(X) = 21$. If $X$ is K3, then $dim Def(X) = 20$. 
\end{example}

\subsection{The BBF quadratic form}

We will study the \emph{local period map}. Again, let $\pi:\mathscr X \to B$ be a family of HK's, with $\pi$ a proper submersion. If $b \in B$, we write $X_b := \pi^{-1}(b)$. 

By choosing $B$ small enough, we can assume that the local system $R^2\pi_\ast \Z$ is trivial: so we can identify $H^2(X_0,\Z) \simeq H^2(X_{b_2},\Z)= \Lambda$ for all $b_2 \in B$ and for a fixed finitely generated torsion free abelian group $\Lambda$ of rank $r$ (i.e $R^2 \pi_\ast \Z \simeq B \times \Lambda$). 

For $b \in B$, we get isomorphisms $p_b:H^2(X_0, \Z) \to \Lambda$. Extension of scalars gives a map $p_b:H^2(X_b,\C) \to \Lambda_\C = \Lambda \otimes_\Z \C$. Note that $H^{2,0}(X_b)$ is contained in the source. 

We are now ready to define the period map:
$$
\xymatrix{
	B \ar[r]^{\mathcal P_\pi} & \PP(\Lambda_\C) \\
	b \ar@{|->}[r] & p_b(H^{2,0}(X))
}
$$

This makes sense, because $H^{2,0}(X)$ is one-dimensional, hence spans a line in $\Lambda_C$. We want to compute $d \mathcal P_\pi(0) = d \mathcal P_\pi$. \footnote{The rest of this section will be quite sketchy, mostly because I didn't understand so much.}

Let $v \in T_0B$. A priori $d\mathcal P_\pi(v) \in \Hom(H^{2,0}(X), F^1 H^2(X)/H^{2,0}(X))$. This follows from Griffiths transversality. Recall $F^1 H^2(X)= H^{2,0}(X) \oplus H^{1,1}(X)$. This last $Hom$ group is equal to $\Hom(H^{2,0}(X), H^{1,1}(X))$. Griffiths in this case tells us that $\langle d \mathcal P_\pi(v), \sigma \rangle = L_sigma(\kappa(v))$. 

We see that (?) the differential of the period map is injective with image $\Hom(H^{2,0}(X), H^{2,1}(X))$. Hence we conclude that the image of the period map is a local analytic hypersurface in $\PP(\Lambda_\C)$. 

We now state the theorem of the existence of the \emph{Bogomolov-Beauville-Fujiki quadratic form}:

\begin{thm}
There exists a quadratic form (with an associated bilinear form $(,)_X)$)
$$
q_X : H^2(X) \to \C
$$
which is integral, indivisible\footnote{What does this mean?}, and a $c_X \in \Q_+$ such that
\begin{equation}
\label{eq:bbf}
\int_X \alpha^{2n} = c_X \frac{(2n)!}{n! 2^n} q_X \alpha^n 
\end{equation}
for all $\alpha \in H^2(X)$.
\end{thm}

Why the factor $\frac{(2n)!}{n! 2^n}$? It has to with \eqref{eq:bbf} is equivalent to the assertion
$$
\int_X \alpha_1 \wedge \cdots \wedge \alpha_{2n} = c_X \sum_{\tau \in S_n} (\alpha_{\tau(1)}, \alpha_{\tau(2)})_X \cdot \ldots \cdot (\alpha_{\tau(2n-1)}, \alpha_{\tau(2n)})_X,
$$
where we sum over permutations ``without stupid repetitions'' (interpret this). 

Now let $X$ be of type $K3^{[n]}$. If $S$ is K3, then $H^2(S^{[n]},\Z) = im \mu_n \oplus \Z \zeta_n$, where $\mu_n$ is the composition
$$
H^2(S,\Z) \to H^2(S^{(n)},\Z) \to H^2(S^{[n]},\Z).
$$
The first map sends a curve $C$ to the $\{ Z \in S^{[n]} \mid Z \cap C \neq \emptyset \}$. The $\zeta_n$ is the reduced image of the diagonal: one can see that the class of the diagonal in $H^2(S^{[n]},\Z)$ is divisible by two. This is analogous to the fact that the map $Bl_D S^2 \to S^{[2]}$ is $2:1$-ramified along $D$. However, for general $n$, the construction of $S^{[n]}$ by a sequence of blowups and blowdowns is quite complicated. But for dimension reasons, we only need to look at points of $S^{(n)}$ where two points come together.

In this case the BBF quadratic form takes the form (no pun...): On $im \mu_n$ it is defined by $(\mu_n(\alpha),\mu_n(\beta)) = \int_S \alpha \wedge \beta$, and on the other summand we have $q(\zeta_n) = -2(n-1)$. And $c_X = 1$. 

This is useful, for it lets us compute for example the self-intersection of the diagonal:

$$
\Delta_n \cdot \ldots \cdot \Delta_n = \frac{(2n)!}{n! 2^n} (-8(n-1))^n.
$$

We sum up some of the things we know in a table:
\begin{center}
\begin{tabular}{c | c | c | c | c}
$X$ & $\dim X$ & $b_2(X)$ & $c_X$ & $H^2(X; \Z)$ \\
\hline 
$K3^{[n]}$ & 2n & 23 & 1 & $U^3 \oplus E_8^2 \oplus (-2(n-1))$ \\
$Kum^{[n]}$ & 2n & 7 & n+1 & $U^3 \oplus (-2(n-1))$ 
\end{tabular}
\end{center}

(he goes on to prove the theorem, but I drew a pig instead)

\section{Lecture 3 - more results}

\end{document}

