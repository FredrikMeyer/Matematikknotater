\documentclass[11pt, norsk]{article}
%\usepackage[latin1]{inputenc}
\usepackage[T1]{fontenc}
\usepackage[utf8]{inputenc}
\usepackage[norsk]{babel}   % S P R A A K
% \usepackage{graphicx}    % postscript graphics
\usepackage{amssymb, amsmath, amsthm, amssymb} % symboler, osv
\usepackage{mathrsfs}
\usepackage{url}
\usepackage{thmtools}
\usepackage{enumerate}  % lister $  
\usepackage{float}
\usepackage{tikz}
\usepackage{tikz-cd}
\usetikzlibrary{calc}
%\usepackage{tikz-3dplot}
\usepackage{subcaption}
\usepackage[all]{xy}   % for comm.diagram
\usepackage{wrapfig} % for float right
\usepackage{hyperref}
\usepackage{mystyle} % stilfilen      

%\usepackage[a5paper,margin=0.5in]{geometry}


\begin{document}
\title{Plenum Kalkulus}
\author{Fredrik Meyer}
\maketitle 

\section{7.1}


\begin{oppg}[7.1]
Du skal lage en rektangulær innehengning til hesten din. Den ene siden dekkes av låven og på de tre andre sidene skal du bygge gjerde. Hva er det største arealet innhengningen kan ha dersom du har materialer til 50 m gjerde?
\end{oppg}
\begin{losn}
La lengden til den delen av gjerdet vinkelrett på veggen ha lengde $x$. Da må den delen av gjerdet som er parallellt med veggen ha lengde $50-2x$ siden gjerdet til sammen har lengde $50$.

Arealet er dermed gitt ved $A(x)=x(50-2x)=-2x^2+50x$ for $x \in (0,25)$. Vi deriverer og får $A'(x) = -4x+50$. Denne har nullpunkt når $-4x+50=0$, altså når $x=12.5$. 

Dermed må vi for å få størst mulig areael, ha $x=12.5m$. Merk at dette gjør at det inngjerdede området blir et kvadrat.
\end{losn}

\begin{oppg}[7.5]
En sirkulær kjegle har en sidekant med lengde $L=9$. Hva er det maksimale volumet av en slik kjegle?\footnote{Sidekantlengden er altså lengden fra spissen av kjeglen til bunnen}
\end{oppg}
\begin{losn}
Vi minner oss først på at volumet til en kjegle med høyde $h$ og baseradius $r$ er gitt ved
$$
V = \frac 13 \pi hr^2.
$$	
Vi har at $L^2=h^2+r^2$ (tegn en tegning!). Dermed er $r$ en funksjon av $h$, og vi har $r=\sqrt{L^2-h^2}$. Dermed er volumet gitt ved
$$
V(h) = \frac \pi3 h(L^2-h^2) = \frac \pi 3 \left( L^2h-h^3 \right).
$$
Vi ønsker å finne maks-verdien til $V(h)$. Vi deriverer:
$$
V'(h) = \frac \pi 3 \left( L^2-3h^2 \right).
$$
Denne har (positivt) nullpunkt når $h=L/\sqrt{3}$, som er et toppunkt for $V(h)$. Dermed blir det største mulige volumet 
$$
V(L/\sqrt{3}) = \frac {\pi} {3}
 \left(
L^2 \frac{L}{\sqrt{3}} - \frac{L^3}{3\sqrt{3}} 
 \right) = 
 \frac {\pi} {3}
 \left(
\frac{3L^3}{3\sqrt{3}} - \frac{L^3}{3\sqrt{3}} 
 \right) = 
\frac {\pi} {3}
 \left(
\frac{2L^3}{3\sqrt{3}}
 \right)  = \frac{2\pi L^3}{9 \sqrt{3}}.
$$
\end{losn}

\begin{oppg}[7.7]
En renne skal lages av et rektangulært stykke blikk som er 60cm bredt, ved at man bøyer opp en vinkel $\theta$ på hver side. Tverrsnittet av rennen skal være et trapes der tre av sidene er like lange. Hvilken verdi av $\theta$ vil maksimere arealet og dermed volumet av rennen?
\end{oppg}

\begin{losn}
Først minner vi om at formelen for arealet av et trapes er gitt ved
$$
A = \frac{(a+b)h}{2},
$$
hvor $a,b$ er lengdene på topp og bunn, og $h$ er høyden. Vi må uttrykke dette ved hjelp av vinkelen $\theta$ og det faktum at tre av sidene skal være like lange.

La lengden til de tre nederste sidene være $a$. Da ser vi (ved hjelp av tegning) at $h=a \sin \theta$. 

Vi ser også at toppen må ha lengde
$$
2a\cos \theta + a.
$$
Dermed er arealet (som en funksjon av $\theta$) gitt ved
$$
A(\theta) = \frac{a(2a+2 a\cos \theta)\sin \theta}{2}=a^2 \sin \theta(1+\cos \theta).
$$
Vi deriverer og får (etter litt forenkling):
$$
A'(\theta) = a^2(2 \cos^2 \theta + \cos \theta -1).
$$
Sett $u=\cos \theta$. Vi vil da løse andregradslikningen $2u^2+u-1=0$. Denne har løsninger $u=\frac 12$ eller $u=-1$. At $\cos \theta = -1$ gir $A(\theta)=0$, som ihvertfall ikke er et maksimum. Dermed må $\cos \theta= \frac 12$ gi oss en maksimumsverdi. Dette betyr at $\theta= 60^\circ$.

Vi får
$$
A(60 ^\circ) = a^2 \frac{3}{\sqrt{2}}\frac{3}{2} = \frac{9a^2}{2\sqrt{2}}.
$$
\end{losn}

\begin{oppg}[7.1.8]
Vi har et rektrangel innskrevet som under.
\begin{verbatim}
|
|/\
|\ \
| \/
|----
\end{verbatim}
Det nedre hjørnet er $x$ enheter fra y-aksen og linjen den ytre siden danner møter x-aksen etter $a$ enheter og møter $y$-aksen etter $b$ enheter (tegn en tegning eller se i boken!!)

Finn lengden av $x$ og maksimer arealet til rektanglet.
\end{oppg}
\begin{losn}
Her må vi bruke kunnskap om formlike trekanter. Vi starter med å observere at den store trekanten laget av aksene og den ytterste linjen er formlik med den innerste trekanten.

Kall lengden til rektanglet for $h_1$. Da er
$$
\frac{h_1}{\sqrt{a^2+b^2}} = \frac{x}{a}.
$$
Så rektanglet har lengde
$$
h_1 = \frac{a}{\sqrt{a^2+b^2}}x.
$$
På samme måte finner vi lengden på rektanglets andre side. Her bruker vi at ene siden i den ytterste trekanten har lengde $a-x$. Ved formlikhet har vi at 
$$
\frac{h_2}{b} = \frac{a-x}{\sqrt{a^2+b^2}}
$$
som gir at
$$
h_2 = \frac{b(a-x)}{\sqrt{a^2+b^2}}.
$$
Dermed er arealet som funksjon av $x$ gitt ved
$$
A(x) = \frac{ab(a-x)x}{a^2+b^2}.
$$
Vi deriverer og får at $x=\frac a2$ gir maksimalt areal.
$$
A\left(\frac a2 \right) = \frac{a^3b}{4(a^2+b^2)}.
$$
\end{losn}


\begin{oppg}[7.1.15]
Vi har et trapes innskrevet i en sirkel slik at nederste av trapeset er diameteren i sirkelen. De to andre hjørnene ligger på sirkelomkretsen. Finn det største arealet et slikt trapes kan ha.
\end{oppg}
\begin{losn}
Tegn tegning.

Tegn en loddrett strek fra sentrum av sirkelen og opp, og tegn en strek $S$ fra sentrum til et av hjørnene i trapeset. Denne streken må ha lengde $r$ siden hjørnet ligger på sirkelen. La $\theta$ være vinkelen mellom den loddrette streken og $S$. (tegn dette!)

Da er høyden i trapeset gitt ved $h=r \cos \theta$ og den øvre lengden i trapeset blir gitt ved $b=2r \sin \theta$. Dermed blir arealet
$$
A(\theta) = \frac{2r+2r\sin \theta}{2} r \cos \theta = r^2(1+ \sin \theta) \cos \theta.
$$

Vi deriverer og får
$$
A'(\theta) = r^2(-2\sin^2 \theta - \sin \theta + 1)
$$

Dette gir at ekstremalpunktene er når $\sin \theta = -1$ eller når $\sin \theta = \frac 12$. Førstnevnte er ikke en løsning siden vinkelen $\theta$ må være i intervallet $[0,90^\circ]$. Dermed må vi ha $\sin \theta = \frac 12$, og dette skjer når $\theta = 30^\circ$. Dermed blir det største muige arealet gitt ved 
$$
A(30^\circ) = r^2(1+\frac 12) \frac{\sqrt 3}{2} = \frac{3 \sqrt 3}{4} r^2.
$$
\end{losn}


\section{7.2}

\begin{oppg}[7.2.1]
En $4$ meter høy stige står opptil en vegg på et flatt underlag. Foten av stigen sklir bort fra veggen med konstant hastighet $0.5 m/s$. Hvor fort beveger toppen av stigen seg når den er $2$ meter over bakken?
\end{oppg}

\begin{losn}
Kall posisjonen til nedre del av stigen for $x(t)$ og høyden for $y(t)$. Ved Pythagoras er $x(t)^2+y(t)^2=16$. Vi deriverer og får
$$
2x'(t)x(t) +2y(t)y'(t)=0.
$$
Siden $x'(t)$ er konstant lik $0.5$ sier dette at 
$$
y'(t) = -\frac 12 \frac{x(t)}{y(t)}.
$$
Når høyden er $2$ meter må vi ha (ved Pythagoras) at $x(t) = 2 \sqrt{3}$. Dermed blir
$$
y'(t) = -\frac 12 \frac{2 \sqrt{3}}{2}= - \frac{\sqrt{3}}{2}.
$$

Så toppen av stigen beveger seg mot bakken med en fart på $\sqrt 3 / 2 m/s$. 
\end{losn}

\begin{oppg}[7.2.3]
Et fyrtårn sender ut en lysstråle som roterer med konstant fart $2$ omdreininger i minuttet. Fyrtårnet ligger $0.5$ km fra en rettlinjet strandlinje. Finn strålens fart langs strandkanten i et punkt på stranden $1$ km fra fyrtårnet.
\end{oppg}

\begin{losn}
Vi tegner en rettvinklet trekant, hvor hypotenusen er stranden. Den ene siden er $0.5 km$ lang, den andre er $x(t)$ kilometer lang. Vi har at $\theta'(t)=4 \pi$. Ved trigonometri er
$$
\tan \theta(t) = \frac{x(t)}{0.5} = 2x(t).
$$
Vi deriverer med hensyn på $t$ på begge sider:
$$
\frac{\theta'(t)}{\cos^2 \theta(t)} = 2 x'(t).
$$
Dermed er 
$$
x'(t) = \frac{\theta'(t)}{2 \cos^2 \theta(t)}.
$$
Når $x(t)=1 km$, får vi, igjen med trigonometri, at $\cos \theta(t)=0.5$. Dermed er 
$$
x'(t) = \frac{4 \pi}{2 \cdot 0.25} = 8 \pi.
$$
Så fyrtårnet lyser med en fart på circa $25.13 km/min \approx 1500 km/t$ 1km fra tårnet.
\end{losn}

\begin{oppg}[7.2.5]
En jente er ute og fisker. En fisk biter på kroken og svømmer rett fra båten med en konstant fart på $4 m/s$ og konstant dybde $8 m$. Hvor mange meter lne løper ut av snellen per sekund i det øyeblikket 10 meter line er ute?
\end{oppg}

\begin{losn}
La $x(t)$ være posisjonen til fisken etter $t$ sekunder. Da er $x'(t)=4$. La $s(t)$ være lengden på snella etter $t$ sekunder. Ved Pythagoras er $x(t)^2 +8^2 = s(t)^2$. Vi deriverer på begge sider med hensyn på $t$ og får
$$
2x(t)x'(t) = 2s(t) s'(t).
$$
Dette gir at
$$
s'(t) = \frac{4x(t)}{s(t)}.
$$
Når $s(t)=10$, må $x(t)=6$ ved Pythagoras. Dermed er 
$$
s'(t) = \frac{4 \cdot 6}{10} = 2.4.
$$
\end{losn}

\begin{oppg}[7.2.7]
Du er ute og går med en lommelykt som lyser opp $60^\circ$ rundt seg. Du går mot et gjerde med en fart på $1 m/s$. Hvor fort minker den opplyste delen av gjerdet?
\end{oppg}

\begin{losn}
Ved enkel trigonometri er lengden av gjerdet gitt ved $l(t) = 2h(t)/\sqrt{3}$. Så $l'(t)= 2/\sqrt{3}$.
\end{losn}

\begin{oppg}[7.2.9]
Et svømmebasseng er 25 m langt, 10 m bredt og 1 m dypt i den grunne enden og 6 m dypt på den dype enden. Bunnen skrår jevnt. Bassenget fylles med vann, 2000 liter per minutt. Hvor fort stiger vannet i bassenget ved det tidspunktet da vanndybden i den dype enden er 3 meter?
\end{oppg}

\begin{losn}
Vi må ha enhetene riktig. 2000 liter er 2 kubikkmeter. 

Før vannstanden har nådd 5 meter må vi regne ut volumet til et prisme (tegn tegning!). Dette har volum 
$$
V(t) = \frac 12 b(t)h(t)l,
$$
hvor $b$ er lengden til vannstanden, $l$ er bredden til bassenget og $h$ er høyden. Lengden er konstant og $b$ og $h$ avhenger av $t$. Siden vi får 2 kubikkmeter i minuttet, er $V'(t)=2$.

Merk at vi har en formlik trekant her. Dermed finner vi at $b=5h$, så
$$
V(t) =  \frac 5h(t)^2 \cdot 10 = 25h(t)^2.
$$
Deriverer vi på begge sider får vi
$$
2 = V'(t) = 50h'(t)h(t).
$$
Om $h(t)=3$, får vi dermed
$$
h'(t) = \frac{2}{150} m/min.
$$
Vi gjør om til $cm$ per $min$ og får $4/3$ cm per minutt.
\end{losn}


\begin{oppg}[7.2.13]
En radar er plassert på en stolpe 7 meter over en vei. En bil nærmer seg stolpen. I det øyeblikket avstanden fra bilen til stolpen er 24 meter, viser radaren at avstanden fra bilen til radaren avtar med 30 meter i sekundet. Hvor fort kjører bilen?
\end{oppg}
\begin{losn}
Tegn tegning.

La $h(t)$ være avstanden fra radaren etter $t$ sekunder og la $x(t)$ være posisjonen til bilen. Vi får en rettvinklet trekant, og ved Pythagoras er 
$$
x(t)^2 + 7^2 = h(t)^2.
$$
Som gir
$$
x(t)x'(t) = h(t)h'(t).
$$
Om $x(t)=24$ får vi ved Pythagoras at $h(t)=25$. Dermed er
$$
x'(t) = \frac{25 \cdot (-30)}{24} = -31.25 m/s.
$$
\end{losn}


\section{7.4 - Omvendte funksjoner}

\begin{oppg}[7.4.1]
Vis at funksjonen er injektiv og finn den omvendte funksjonen. Angi definisjonsområdet til den omvendte funksjonen.
\end{oppg}
\begin{losn}
\begin{enumerate}[a)]
\item La $f(x)=x^3$ med $D_f=\R$. Vi har at $f'(x)=3x^2$. Dette er en strengt positiv funksjon (for $x\neq 0$), så $f$ må være injektiv. Inversen er gitt ved $g(x)=\sqrt[3]{x}$ med definisjonsområde $\R$.
\item La $f(x)=x^2$ for $D_f=[0, \infty)$. Igjen er $f$ injektiv fordi den er strengt voksende på definisjonsområdet. Inversen er gitt ved $g(x) = \sqrt{x}$ med definisjonsområde $[0,\infty)$.
\item La $f(x)=x^2$ igjen, denne gangen med definisjonsområde $(-\infty,0]$. Nå er $f$ strengt synkende, så igjen må den være injektiv. Inversen her er $g(x)=-\sqrt{x}$ og $D_f=[0, \infty)$. 
\end{enumerate}
\end{losn}

\begin{oppg}[7.4.3]
Vis at funksjonen
$$
f(x) = 2xe^x + 1,
$$
definert for $x \geq 1$ er injektiv. La $g$ være den omvendte funksjonen og beregn $g'(1)$.
\end{oppg}
\begin{losn}
Vi deriverer og får $f'(x) = 2e^x+2xe^x=2e^x(1+x)$. Vi ser at om $x \geq -1$, så er den deriverte positiv, så $f$ må være strengt voksende. Dermed er den injektiv. 

Vi har at 
$$
x=g(f(x))
$$
så vi deriverer på begge sider og får 
$$
1 = g'(f(x))f'(x).
$$
Vi har at $f(0)=1$, så vi setter $x=0$ over, og får
$$
g'(1)=\frac{1}{f'(0)}=\frac{1}{2}.
$$
\end{losn}

\begin{oppg}[7.4.5]
La $f(x)=\tan(2x)$ og vis at $f$ er injektiv på intervallet $(-\pi/4,\pi/4)$. Finn den deriverte til den omvendte funksjonen i punktet $x=1$.
\end{oppg}
\begin{losn}
Vi har at $f'(x) = \frac{2}{\cos^2(2x)}$. Dermed er $f(x)$ strengt voksende så lenge $\cos(2x) \neq 0$ (for da er den ikke definert). Dette skjer presist når $x \in (-\pi/4,\pi/4)$.

La $g(x)$ være den omvendte funksjonen. Igjen vet vi at
$$
g'(f(x)) = \frac{1}{f'(x)}.
$$
Så vi må finne når $f(x)=1$. Men dette skjer når $x=\pi/8$, for da er $\tan(2x)=\tan(\pi/4)=1$. Så
$$
g'(1)=\frac{1}{f'(\pi/u)} = \frac 14.
$$
\end{losn}

\begin{oppg}[7.4.8]
Anta at $g$ er den omvendte funksjonen til en kontinuerlig, strengt monoton funksjon $f$ og at $f$ er to ganger deriverbar i punktet $y=g(x)$. Vis at $g$ er to ganger deriverbar i $x$ og at 
$$
g''(x) = -\frac{f'(g(x))g'(x)}{f'(g(x))^2}.
$$
\end{oppg}
\begin{losn}
Start med
$$
x=f(g(x)).
$$
Da er 
$$
1 = f'(g(x))g'(x).
$$
Dermed er
$$
g'(x) = \frac{1}{f'(g(x))}.
$$
Så 
$$
g''(x) = \frac{-f''(g(x))g'(x)}{f'(g(x))^2}.
$$
La $f(x)=\sin x$. Vi har at $\sin(\pi/6)=\frac 12$. Dermed er
$$
g'\left(\frac 12\right) = \frac{1}{\sin'(\pi/6)}=\frac{1}{\cos(\pi/6)}= \frac{1}{\sqrt{3}/2}=\frac{2}{\sqrt{3}}.
$$
Dermed er
$$
g''\left( \frac 12 \right) = \frac{-(-\sin(g(1/2)))(2/\sqrt{3})}{\cos(g(1/2))^2}=\frac{1/\sqrt{3}}{1-sin(g(1/2))^2}=\frac{4}{3\sqrt 3}.
$$
Sjekk selv at dette gir samme resultat som å dobbelderive $\arcsin x$.
\end{losn}

\begin{oppg}[7.4.10]
Vis at funksjonen $f(x)=xe^{\frac 12(1-x^2)}$ er injektiv på intervallet $[-1,1]$. Finn definisjonsområdet til den omvendte funksjonen $g$ og beregn
$$
\lim_{y \to 1^-} (1-y)[g'(y)]^2.
$$
\end{oppg}
\begin{losn}
Ved å derivere ser vi at den deriverte er strengt positiv på det indre av intervallet $[-1,1]$, så funksjonen må være injektiv ved tidligere resultater. Vi har også at $f(-1)=-1$ og $f(1)=1$, så definisjonsområdet til $g$ er $D_g=[-1,1]$. 

For å regne ut grenseverdien trenger vi et par deriverte. Vi regner og finner at
$$
f'(x) = e^{\frac 12 (1-x^2)}(1-x^2)
$$
og 
$$
f''(x) = e^{\frac 12 (1-x^2)}(x^3-3x).
$$
Merk at $f'(1)=0$ og $f''(1)=-2$.

Det første vi gjør for å finne grenseverdien er å bruke at
$$
g'(y) = \frac{1}{f'(g(y))}.
$$
Dermed er
$$
\lim_{y \to 1^-} (1-y)[g'(y)]^2 = \lim_{y \to 1^-} \frac{1-y}{f'(g(y))^2}
$$
Men siden $g(1)=1$ og $f'(1)=0$, er dette et $0/0$-uttrykk. Vi kan derfor bruke L'Hôpital's regel, og vi får
$$
\lim_{y \to 1^-} (1-y)[g'(y)]^2 = \lim_{y \to 1^-} \frac{-1}{2f'(g(y))f''(g(y))g'(y)}.
$$

Her var vi nødt til å bruke kjerneregelen to ganger. Nå bruker vi igjen at $g'(y)=1/f'(g(y))$, og får at grenseverdien er
$$
\lim_{y \to 1^-} (1-y)[g'(y)]^2 = \lim_{y \to 1^-} \frac{-1}{2f''(g(y))}=-\frac{1}{4}.
$$
\end{losn}

\section{7.5}
\begin{oppg}[7.5.3ab]
Finn grenseverdier.
\end{oppg}

\begin{losn}
  \begin{enumerate}[a)]
  \item Vi skal finne
$$
\lim_{x \to 0} x \cot x.
$$
Husk at $\cot x = \cos x /\sin x = 1/\tan x$. Dermed er
$$
\lim_{x \to 0} x \cot x = \lim_{x \to 0} \frac{x}{\tan x} = \lim_{x \to 0} \frac{1}{1/\cos^2(x)} = \lim_{x \to 0} \cos^2 x = 1.
$$
\item Vi skal finne
$$
\lim_{x \to \pi/2} \frac{\cot x}{\pi/2 - x}.
$$
Dette er ren L'Hôpital:
$$
\lim_{x \to \pi/2} \frac{\cot x}{\pi/2 - x} = \lim_{x \to \pi/2} \frac{\frac{-1}{\sin^2(x)}}{-1} = 1.
$$
  \end{enumerate}
\end{losn}

\section{7.6}
\begin{oppg}
  Finn eksakte verdier.
\end{oppg}
\begin{losn}
  \begin{enumerate}[a)]
  \item Siden $\sin(\pi/6)=\frac 12$ er $\arcsin \frac 12 = \pi/6$. 
\item Siden $\cos(\pi/3)= \frac 12$ er $\arccos \frac 12 = \pi/3$. 
\item Siden $\tan(\pi/3) = sqrt{3}$, er $\arctan(\sqrt 3) = \pi/3$.
  \end{enumerate}
\end{losn}

\begin{oppg}
  Finn de deriverte.
\end{oppg}
\begin{losn}
  Den deriverte av $f(x) = \arcsin \sqrt {x}$. Her må vi bruke mye kjerneregel.
$$
f'(x) = \frac{1}{\sqrt{1-\sqrt{x}^2}} \cdot \frac{1}{2\sqrt{x}} = \frac{1}{2\sqrt{x-x^2}}.
$$

Og også finne den deriverte til 
$$
g(x) = \frac{\arcsin(x)}{\sin (3x)}.
$$
Dette er ren formelmanipulasjon. Svaret er
$$
g'(x) = \frac{\frac{\sin (3x)}{\sqrt{1-x^2}} - 3\cos(3x) \arcsin x}{\sin^2 (3x)}.
$$
\end{losn}

\begin{oppg}
  Finn grenseverdier.
\end{oppg}

\begin{losn}
  \begin{enumerate}
  \item Først skal vi finne
$$
\lim_{x \to 0} \frac{\arctan 2x}{x}.
$$
Vi har at $\tan 0 = 0$, så $\arctan 0=0$ også. Dermed er dette et $0/0$-uttrykk og vi bruker L'Hôpital:
$$
\lim_{x \to 0} \frac{\arctan 2x}{x} = \lim_{x \to 0} \frac{\frac{2}{\cos^2(2x)}}{1} = \lim_{x \to 0} \frac{2}{\cos^2(2x)} = 2.
$$
\item

Nå skal vi finne 
$$
\lim_{x \to 0} \frac{\arcsin x}{\sin 3x} = \lim_{x \to 0} \frac{\sqrt{1-x^2}^{-1}}{3\cos (3x)} = \frac{1}{3}.
$$
\item Og til slutt:
$$
\lim_{x \to 0} \frac{\arctan x - x}{x^3} = \lim_{x \to 0} \frac{\frac{1}{\cos^2 x} - 1}{3x^2}
$$
Vi ganger med $\cos^2 x$ både oppe og nede på brøken og bruker at grensen av et produkt er produktet av grensene:
$$
\lim_{x \to 0} \frac{\frac{1}{\cos^2 x} - 1}{3x^2} = \lim_{x \to 0} \frac{1-\cos^2 x}{3x^2 \cos^2 x} = \lim_{x \to 0} \frac{1}{\cos^2 x} \frac{1- \cos^2 x}{3x^2} = \lim_{x \to 0} \frac{1}{\cos^2 x} \lim_{x \to 0} \frac{1-\cos^2 x}{3x^2}.
$$
Første leddet er lik $1$. Så vi står igjen med 
$$
\lim_{x \to 0} \frac{1-\cos^2 x}{3x^2}.
$$
Igjen bruker vi L'Hôpital:
$$
\lim_{x \to 0} \frac{2 \cos x  \sin x}{6x} = \lim_{x \to 0} \frac{\sin x}{3 x} = \frac 13.
$$
  \end{enumerate}
\end{losn}

\begin{oppg}
  La $f(x) = x \arctan x$. 
  \begin{enumerate}[a)]
  \item Avgjør hvor $f$ er voksende og hvor den er avtagende. 
\item Hvor er $f$ konveks og hvor er den konkav?
\item Finn asymptotene til $f$ og skisser grafen. 
  \end{enumerate}
\end{oppg}

\begin{sol}
Vi deriverer og får:
$$
f'(x) = \arctan x + \frac{x}{1+x^2}.
$$
Vi deriverer enda en gang og får
$$
f''(x) = \frac{1}{(1+x^2)^2}.
$$
Dermed ser vi at $f'(0)=0$, og siden $f''(x) > 0$ må dette være det eneste nullpunktet. Dermed er $f(x)=x \arctan x$ synkende for $x < 0$ og stigende for $x > 0$.  Så a) er ferdig.

For b). Siden $f''(x) > 0$ hele tiden er $f$ overalt konveks.

Asympotentene er gitt ved $y=\pi/2 x - 1$ og $y=-\pi / 2 x -1$.
\end{sol}

\begin{oppg}
 Se boka.
\end{oppg}
\end{document}