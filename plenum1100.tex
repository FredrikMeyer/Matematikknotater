\documentclass[11pt, norsk]{article}
%\usepackage[latin1]{inputenc}
\usepackage[T1]{fontenc}
\usepackage[utf8]{inputenc}
\usepackage[norsk]{babel}   % S P R A A K
% \usepackage{graphicx}    % postscript graphics
\usepackage{amssymb, amsmath, amsthm, amssymb} % symboler, osv
\usepackage{mathrsfs}
\usepackage{url}
\usepackage{thmtools}
\usepackage{enumerate}  % lister $  
\usepackage{float}
\usepackage{tikz}
\usepackage{tikz-cd}
\usetikzlibrary{calc}
%\usepackage{tikz-3dplot}
\usepackage{subcaption}
\usepackage[all]{xy}   % for comm.diagram
\usepackage{wrapfig} % for float right
\usepackage{hyperref}
\usepackage{mystyle} % stilfilen      

%\usepackage[a5paper,margin=0.5in]{geometry}


\begin{document}
\title{Plenum Kalkulus}
\author{Fredrik Meyer}
\maketitle 

\section{7.1}


\begin{oppg}[7.1]
Du skal lage en rektangulær innehengning til hesten din. Den ene siden dekkes av låven og på de tre andre sidene skal du bygge gjerde. Hva er det største arealet innhengningen kan ha dersom du har materialer til 50 m gjerde?
\end{oppg}
\begin{losn}
La lengden til den delen av gjerdet vinkelrett på veggen ha lengde $x$. Da må den delen av gjerdet som er parallellt med veggen ha lengde $50-2x$ siden gjerdet til sammen har lengde $50$.

Arealet er dermed gitt ved $A(x)=x(50-2x)=-2x^2+50x$ for $x \in (0,25)$. Vi deriverer og får $A'(x) = -4x+50$. Denne har nullpunkt når $-4x+50=0$, altså når $x=12.5$. 

Dermed må vi for å få størst mulig areael, ha $x=12.5m$. Merk at dette gjør at det inngjerdede området blir et kvadrat.
\end{losn}

\begin{oppg}[7.5]
En sirkulær kjegle har en sidekant med lengde $L=9$. Hva er det maksimale volumet av en slik kjegle?\footnote{Sidekantlengden er altså lengden fra spissen av kjeglen til bunnen}
\end{oppg}
\begin{losn}
Vi minner oss først på at volumet til en kjegle med høyde $h$ og baseradius $r$ er gitt ved
$$
V = \frac 13 \pi hr^2.
$$	
Vi har at $L^2=h^2+r^2$ (tegn en tegning!). Dermed er $r$ en funksjon av $h$, og vi har $r=\sqrt{L^2-h^2}$. Dermed er volumet gitt ved
$$
V(h) = \frac \pi3 h(L^2-h^2) = \frac \pi 3 \left( L^2h-h^3 \right).
$$
Vi ønsker å finne maks-verdien til $V(h)$. Vi deriverer:
$$
V'(h) = \frac \pi 3 \left( L^2-3h^2 \right).
$$
Denne har (positivt) nullpunkt når $h=L/\sqrt{3}$, som er et toppunkt for $V(h)$. Dermed blir det største mulige volumet 
$$
V(L/\sqrt{3}) = \frac {\pi} {3}
 \left(
L^2 \frac{L}{\sqrt{3}} - \frac{L^3}{3\sqrt{3}} 
 \right) = 
 \frac {\pi} {3}
 \left(
\frac{3L^3}{3\sqrt{3}} - \frac{L^3}{3\sqrt{3}} 
 \right) = 
\frac {\pi} {3}
 \left(
\frac{2L^3}{3\sqrt{3}}
 \right)  = \frac{2\pi L^3}{9 \sqrt{3}}.
$$
\end{losn}

\begin{oppg}[7.7]
En renne skal lages av et rektangulært stykke blikk som er 60cm bredt, ved at man bøyer opp en vinkel $\theta$ på hver side. Tverrsnittet av rennen skal være et trapes der tre av sidene er like lange. Hvilken verdi av $\theta$ vil maksimere arealet og dermed volumet av rennen?
\end{oppg}

\begin{losn}
Først minner vi om at formelen for arealet av et trapes er gitt ved
$$
A = \frac{(a+b)h}{2},
$$
hvor $a,b$ er lengdene på topp og bunn, og $h$ er høyden. Vi må uttrykke dette ved hjelp av vinkelen $\theta$ og det faktum at tre av sidene skal være like lange.

La lengden til de tre nederste sidene være $a$. Da ser vi (ved hjelp av tegning) at $h=a \sin \theta$. 

Vi ser også at toppen må ha lengde
$$
2a\cos \theta + a.
$$
Dermed er arealet (som en funksjon av $\theta$) gitt ved
$$
A(\theta) = \frac{a(2a+2 a\cos \theta)\sin \theta}{2}=a^2 \sin \theta(1+\cos \theta).
$$
Vi deriverer og får (etter litt forenkling):
$$
A'(\theta) = a^2(2 \cos^2 \theta + \cos \theta -1).
$$
Sett $u=\cos \theta$. Vi vil da løse andregradslikningen $2u^2+u-1=0$. Denne har løsninger $u=\frac 12$ eller $u=-1$. At $\cos \theta = -1$ gir $A(\theta)=0$, som ihvertfall ikke er et maksimum. Dermed må $\cos \theta= \frac 12$ gi oss en maksimumsverdi. Dette betyr at $\theta= 60^\circ$.

Vi får
$$
A(60 ^\circ) = a^2 \frac{3}{\sqrt{2}}\frac{3}{2} = \frac{9a^2}{2\sqrt{2}}.
$$
\end{losn}

\begin{oppg}
Vi har et rektrangel innskrevet som under.
\begin{verbatim}
|
|/\
|\ \
| \/
|----
\end{verbatim}
Det nedre hjørnet er $x$ enheter fra y-aksen og linjen den ytre siden danner møter x-aksen etter $a$ enheter og møter $y$-aksen etter $b$ enheter (tegn en tegning eller se i boken!!)

Finn lengden av $x$ og maksimer arealet til rektanglet.
\end{oppg}
\begin{losn}
Her må vi bruke kunnskap om formlike trekanter. Vi starter med å observere at den store trekanten laget av aksene og den ytterste linjen er formlik med den innerste trekanten.

Kall lengden til rektanglet for $h_1$. Da er
$$
\frac{h_1}{\sqrt{a^2+b^2}} = \frac{x}{a}.
$$
Så rektanglet har lengde
$$
h_1 = \frac{a}{\sqrt{a^2+b^2}}x.
$$
På samme måte finner vi lengden på rektanglets andre side. Her bruker vi at ene siden i den ytterste trekanten har lengde $a-x$. Ved formlikhet har vi at 
$$
\frac{h_2}{b} = \frac{a-x}{\sqrt{a^2+b^2}}
$$
som gir at
$$
h_2 = \frac{b(a-x)}{\sqrt{a^2+b^2}}.
$$
Dermed er arealet som funksjon av $x$ gitt ved
$$
A(x) = \frac{ab(a-x)x}{a^2+b^2}.
$$
Vi deriverer og får at $x=\frac a2$ gir maksimalt areal.
$$
A\left(\frac a2 \right) = \frac{a^3b}{4(a^2+b^2)}.
$$
\end{losn}


\begin{oppg}[7.15]
Vi har et trapes innskrevet i en sirkel slik at nederste av trapeset er diameteren i sirkelen. De to andre hjørnene ligger på sirkelomkretsen. Finn det største arealet et slikt trapes kan ha.
\end{oppg}
\begin{losn}
Tegn tegning.

Tegn en loddrett strek fra sentrum av sirkelen og opp, og tegn en strek $S$ fra sentrum til et av hjørnene i trapeset. Denne streken må ha lengde $r$ siden hjørnet ligger på sirkelen. La $\theta$ være vinkelen mellom den loddrette streken og $S$. (tegn dette!)

Da er høyden i trapeset gitt ved $h=r \cos \theta$ og den øvre lengden i trapeset blir gitt ved $b=2r \sin \theta$. Dermed blir arealet
$$
A(\theta) = \frac{2r+2r\sin \theta}{2} r \cos \theta = r^2(1+ \sin \theta) \cos \theta.
$$

Vi deriverer og får
$$
A'(\theta) = r^2(-2\sin^2 \theta - \sin \theta + 1)
$$

Dette gir at ekstremalpunktene er når $\sin \theta = -1$ eller når $\sin \theta = \frac 12$. Førstnevnte er ikke en løsning siden vinkelen $\theta$ må være i intervallet $[0,90^\circ]$. Dermed må vi ha $\sin \theta = \frac 12$, og dette skjer når $\theta = 30^\circ$. Dermed blir det største muige arealet gitt ved 
$$
A(30^\circ) = r^2(1+\frac 12) \frac{\sqrt 3}{2} = \frac{3 \sqrt 3}{4} r^2.
$$
\end{losn}


\section{7.2}

\begin{oppg}
En $4$ meter høy stige står opptil en vegg på et flatt underlag. Foten av stigen sklir bort fra veggen med konstant hastighet $0.5 m/s$. Hvor fort beveger toppen av stigen seg når den er $2$ meter over bakken?
\end{oppg}

\begin{losn}
Kall posisjonen til nedre del av stigen for $x(t)$ og høyden for $y(t)$. Ved Pythagoras er $x(t)^2+y(t)^2=16$. Vi deriverer og får
$$
2x'(t)x(t) +2y(t)y'(t)=0.
$$
Siden $x'(t)$ er konstant lik $0.5$ sier dette at 
$$
y'(t) = -\frac 12 \frac{x(t)}{y(t)}.
$$
Når høyden er $2$ meter må vi ha (ved Pythagoras) at $x(t) = 2 \sqrt{3}$. Dermed blir
$$
y'(t) = -\frac 12 \frac{2 \sqrt{3}}{2}= - \frac{\sqrt{3}}{2}.
$$

Så toppen av stigen beveger seg mot bakken med en fart på $\sqrt 3 / 2 m/s$. 
\end{losn}

\begin{oppg}[7.3]
Et fyrtårn sender ut en lysstråle som roterer med konstant fart $2$ omdreininger i minuttet. Fyrtårnet ligger $0.5$ km fra en rettlinjet strandlinje. Finn strålens fart langs strandkanten i et punkt på stranden $1$ km fra fyrtårnet.
\end{oppg}

\begin{losn}
Vi tegner en rettvinklet trekant, hvor hypotenusen er stranden. Den ene siden er $0.5 km$ lang, den andre er $x(t)$ kilometer lang. Vi har at $\theta'(t)=4 \pi$. Ved trigonometri er
$$
\tan \theta(t) = \frac{x(t)}{0.5} = 2x(t).
$$
Vi deriverer med hensyn på $t$ på begge sider:
$$
\frac{\theta'(t)}{\cos^2 \theta(t)} = 2 x'(t).
$$
Dermed er 
$$
x'(t) = \frac{\theta'(t)}{2 \cos^2 \theta(t)}.
$$
Når $x(t)=1 km$, får vi, igjen med trigonometri, at $\cos \theta(t)=0.5$. Dermed er 
$$
x'(t) = \frac{4 \pi}{2 \cdot 0.25} = 8 \pi.
$$
Så fyrtårnet lyser med en fart på circa $25.13 km/min \approx 1500 km/t$ 1km fra tårnet.
\end{losn}

\begin{oppg}[7.5]
En jente er ute og fisker. En fisk biter på kroken og svømmer rett fra båten med en konstant fart på $4 m/s$ og konstant dybde $8 m$. Hvor mange meter lne løper ut av snellen per sekund i det øyeblikket 10 meter line er ute?
\end{oppg}

\begin{losn}
La $x(t)$ være posisjonen til fisken etter $t$ sekunder. Da er $x'(t)=4$. La $s(t)$ være lengden på snella etter $t$ sekunder. Ved Pythagoras er $x(t)^2 +8^2 = s(t)^2$. Vi deriverer på begge sider med hensyn på $t$ og får
$$
2x(t)x'(t) = 2s(t) s'(t).
$$
Dette gir at
$$
s'(t) = \frac{4x(t)}{s(t)}.
$$
Når $s(t)=10$, må $x(t)=6$ ved Pythagoras. Dermed er 
$$
s'(t) = \frac{4 \cdot 6}{10} = 2.4.
$$
\end{losn}

\begin{oppg}[7.7]
Du er ute og går med en lommelykt som lyser opp $60^\circ$ rundt seg. Du går mot et gjerde med en fart på $1 m/s$. Hvor fort minker den opplyste delen av gjerdet?
\end{oppg}

\begin{losn}
Ved enkel trigonometri er lengden av gjerdet gitt ved $l(t) = 2h(t)/\sqrt{3}$. Så $l'(t)= 2/\sqrt{3}$.
\end{losn}






\end{document}