\documentclass[11pt, norsk]{article}
%\usepackage[latin1]{inputenc}
\usepackage[T1]{fontenc}
\usepackage[utf8]{inputenc}
\usepackage[norsk]{babel}   % S P R A A K


% \usepackage{graphicx}    % postscript graphics
\usepackage{amssymb, amsmath, amsthm, amssymb} % symboler, osv
\usepackage{mathrsfs}
\usepackage{url}
\usepackage{thmtools}
\usepackage{enumerate}  % lister $  
\usepackage{float}
\usepackage{tikz}
\usetikzlibrary{calc}
\usetikzlibrary{intersections}
\usepackage{tikz-3dplot}
\usepackage{subcaption}
\usepackage[all]{xy}   % for comm.diagram
\usepackage{wrapfig} % for float right
\usepackage{hyperref}
\usepackage{mystyle} % stilfilen      


\begin{document}
\title{Oppgaver MAT2500}
\author{Fredrik Meyer}
\maketitle 

\begin{oppg}
  Det geometriske stedet for midtpunktet til parallelle koorder i en parabel er ei linje som er parallell med symmetriaksen. Vis dette.

\textbf{MERK:} Her må det menes ``det geometriske stedet for midtpunkter til alle korder parallell til en gitt korde''.
\end{oppg}
\begin{losn}

Her kan vi med en gang gjøre en forenkling. En generell parabel gjennom origo er gitt ved $y=cx^2$. Men vi kan gjøre enda en forenkling: siden vi har lyst til å finne ligningen for en linje, og linjer forblir linjer under skalering av koordinataksene, kan vi sette $x = \frac{x'}{\sqrt{c}}$. Dermed blir den nye ligningen $y=x^2$, som er den enkleste parabelen av alle.

En generell korde er gitt ved linjen $y=ax+b$. Ved å sette inn i ligningen $y=x^2$, får vi at midtpunktet har koordinater $(\frac a2,\frac{a^2}{2} +b)$.

Lar vi $b$ variere (og holder $a$ fast, siden vi ser på parallelle korder), ser vi at midtpunktet beveger seg langs linjen $x=\frac a2$, som er parallell med symmetriaksen $x=0$.

MEN! Legg merke til at om $b$ er for liten, så får vi ingen korde. Dermed får vi at det geometriske stedet faktisk er \textbf{linjestykket} fra $(a/2,a^2/4)$ parallell med symmetriaksen.
\end{losn}

\begin{oppg}
Stigningstallet til en rekke parallelle korder til parabelen $y^2=2px$ er $k$. Finn ligningen til diameteren gjennom midtpunktet av disse kordene.
\end{oppg}

\begin{losn}
Dette er bare en kopi av forrige oppgave. En diameter i en parabel er en linje gjennom sentrum av parabelen. En parabel har sentrum ``i det uendelige'', og et linjestykke er derfor en diameter i parabelen om det er parallelt med symmetriaksen. 

Svaret er derfor det samme som i forrige oppgave, bortsett fra at nå er ligningen $y^2=2px$ i stedet for $y=x^2$. Vi kan regne på akkurat samme måte å få at ligningen for linjestykket er gitt ved $y = p/k$.
\end{losn}

\begin{oppg}
La være gitt parabelen $y=4x$ med brennpunkt $F$ og et punkt $Q=(-2,4)$ utenfor parabelen. La $l_Q$ være ei linje gjennom $Q$ som skjærer parabelen i to punkter $A$ og $B$ og la $l_d$ være parabelen gjennom midtpunktet til korden $AB$. Diameteren $l_d$ og styrelinja til parabelen skjærer hverandre i punktet $C$. La $P$ være skjæringspunktet mellom linjene $CF$ og $QA$. Finn det geometriske stedet for $P$ når $l_Q$ varierer, og vis at både $F$ og $Q$ ligger på dette geometriske stedet.
\end{oppg}

\begin{losn}
Linja $l_Q$ kan vi skrive som
$$
y = a(x-2)+4.
$$
Fra forrige oppgave har $l_d$ ligning $y=2/a$. Styrelinja er $x=-1$, så $C=(-1,2/a)$. Linja $CF$ er gitt ved ligningen
$$
y = -\frac 1a x - \frac 1a + \frac 2a = -\frac 1a x +\frac 1a.
$$
Linjen $QA$ er samme som $l_Q$. Vi kan regne ut skjæringspunktet mellom disse to linjene til å være
$$
P = \left( \frac{1-2a^2-4a}{a^2+1} , \frac{3a+4}{a^2+1} \right).
$$

Vi kan plotte dette i Geogebra med kommandoen 
\begin{verbatim}
Curve((1-2a^2-4a)/(1+a^2),(3a+4)/(1+a^2),a,-10,10)
\end{verbatim}
Og vi ser at vi får noe som minner svært meget om en sirkel. Problemet er å finne hvilken sirkel dette er. Men vi kan gjette oss til hva det er ved å studere sirkelen i Geogebra. Ved inspeksjon ser det ut til at sirkelen har sentrum $x=-\frac 12$ og $y=2$. Dermed burde ligningen være $(x+\frac 12)^2+(y-2)^2=r^2$, hvor $r$ er radiusen.

Vi kan regne ut at
$$
x^2+x = \frac{2a^4+12a^3+11a^2-12a+2}{(1+a^2)^2}
$$
og at
$$
y^2-4y = \frac{-12a^3-7a^2+12a}{(1+a^2)^2}.
$$
Dermed blir
$$
x^2+x+y^2-4y = \frac{2a^4+4a^2+2}{(1+a^2)^2} = 2.
$$
Fullfører vi kvadratene får vi dermed at radiusen er $r= 5/2$.

Dermed er det geometriske stedet en sirkel, og vi kan sjekke at både $F$ og $Q$ ligger på sirkelen ved å regne at du har avstand $5/2$ fra $(-\frac 12,2)$.
\end{losn}

\begin{oppg}
Finn ligningen til en generell sirkel i planet. Hvor mange punkter trengs for å beskrive en sirkel?
\end{oppg}

\begin{losn}
En sirkel trenger et sentrum og en radius. Radiusen er ett tall og sentrum er to ekstra tall.

I symboler ser derfor en generell sirkel ut som
$$
(x-a)^2+(x-b)^2=r^2
$$

Det trengs derfor \emph{to} punkter, eller på et vis ``ett og et halvt'' punkt, siden om vi har oppgitt sentrum (ett punkt), trenger vi bare vite lengden til punktet på randa, og ikke selve punktet.
\end{losn}

\begin{oppg}
  Finn ligningen til en hyperbel med gitte asymptoter symmetrisk om aksene. Hvor mange punkter trengs for å bestemme en hyperbel med disse asymptotene?
\end{oppg}
\begin{losn}
Gitt to linjer med ligning $ax+by=0$ og $a'x+b'y=0$, så kan vi lage en hyperbel med disse som asymptoter ved ligningen $(ax+by)(a'x+b'y)=c$, for en konstant $c$. For å bestemme $c$ trengs det et punkt på hyperbelen, så svaret er at gitt asymptotene, så trengs det ett punkt for å bestemme hele hyperbelen.
\end{losn}

\begin{oppg}
 Dersom alle sidene i en trekant tangerer en parabel, så går den omskrevne sirkelen til trekanten gjennom brennpunktet til parabelen. Vis dette.
\end{oppg}
\begin{losn}

\end{losn}

\begin{oppg}
 Vis at polaren til et brennpunkt er den tilhørende styrelinja til et kjeglesnitt.
\end{oppg}

\begin{losn}
 Dette er trivielt såsnart en har gått gjennom definisjonene.

Gitt et kjeglesnitt, kan en til hvert punkt $P$i planet assosiere en linje $l(P)$ som kalles \emph{polaren} til $P$ (med hensyn på det gitte kjeglesnittet). 

Vi har tre typer kjeglesnitt, og dermed tre typer polarer. I tilfellet parabel med ligning $y^2=4cx$, vet vi at polaren er gitt ved $y_0y=2c(x+x_0)$ der $(x_0,y_0)$ er polen til den polare. Brennpunktet til kjeglesnittet er $(c,0)$, så bare ved å sette inn for $x_0,y_0$, får vi at polaren er linjen $2c(x+c)=0$, eller med andre ord $x=-c$. Men dette er, som vi vet, styrelinja til parabelen.

Vi tar ett eksempel til, og lar den ivrige oppgaveløser gjøre tilfellet hyperbel. Så vi ser på ellipsen. Gitt en ellipse med standardligning $\frac{x^2}{a^2} + \frac{y^2}{b^2}=1$, er den polare gitt ved $$\frac{xx_0}{a^2}+\frac{yy_0}{b^2}=1$. Brennpunktene er gitt ved $(\pm c,0)$, og ved innsetting gir dette $\frac \pm xc}{a^2}= 1$. Men dette er
$$
x = \frac {\pm a^2}{c} = \pm a (\frac ca)^{-1} = \pm \frac ae,
$$
hvor $e=c/a$. Men dette er, fra Tabell 3 i heftet, nøyaktig lik styrelinjene til en ellipse.
\end{losn}

\end{document}
