\documentclass[11pt, norsk]{article}
%\usepackage[latin1]{inputenc}
\usepackage[T1]{fontenc}
\usepackage[utf8x]{inputenc}
\usepackage[norsk]{babel}   % S P R A A K
% \usepackage{graphicx}    % postscript graphics
\usepackage{amssymb, amsmath, amsthm, amssymb} % symboler, osv
\usepackage{mathrsfs}
\usepackage{url}
\usepackage{thmtools}
\usepackage{enumerate}  % lister $  
\usepackage{float}
\usepackage{tikz}
\usepackage{tikz-cd}
\usetikzlibrary{calc}
%\usepackage{tikz-3dplot}
\usepackage{subcaption}
\usepackage[all]{xy}   % for comm.diagram
\usepackage{wrapfig} % for float right
\usepackage{hyperref}
\usepackage{mystyle} % stilfilen      

%\usepackage[a5paper,margin=0.5in]{geometry}

\begin{document}
\title{Løsningsforslag noen oppgaver}
\author{Fredrik Meyer}
\maketitle

\begin{oppg}[3.3.7]
Bruk de Moivre's formel til å uttryke $\sin 4\theta$ og $\cos 4 \theta$ ved hjelp av $\sin \theta$ og $\cos \theta$.
\end{oppg}
\begin{losn}
Husk at de Moivre's formel sier at
\[
(\cos \theta + i \sin \theta)^n = \cos (n \theta) + i \sin (n \theta).
\]
Vi setter $n=4$, og får 
\[
(\cos \theta + i \sin \theta)^4 = \cos (4 \theta) + i \sin (4 \theta).
\]
Strategien er nå å gange ut parentesene og deretter sammenligne realdeler og imaginærdeler. En grei strategi er å først regne ut $(\cos \theta + i \sin \theta)^2$, og deretter gange dette med seg selv.

Vi får
\begin{align}
\label{eq:coscos}
(\cos \theta + i \sin \theta)^2 &= (\cos^2 \theta - \sin^2 \theta) +  i( 2\cos \theta \sin \theta).
\end{align}
Vi ganger dette med seg selv og får:

\begin{align*}
((\cos^2 \theta - \sin^2 \theta) +  i( 2\cos \theta \sin \theta))^2 &= (\cos^2 \theta - \sin^2 \theta)^2 -4 \cos \theta^2 \sin^2\theta + 2i(4 \cos^3 \theta \sin \theta - 4 \cos \theta \sin^2 \theta) \\
&= \cos^4 \theta - 6 \cos^2 \theta \sin^2 \theta + \sin^4 \theta + i (4\cos^3 \theta \sin \theta - 4 \cos \theta \sin^2 \theta)
\end{align*}

Sammenligner vi realdeler og imaginærdeler i \eqref{eq:coscos}, ser vi at
\[
\cos 4 \theta = \cos^4 \theta - 6 \cos^2 \theta \sin^2 \theta + \sin ^4 \theta,
\]
og 
\[
\sin 4 \theta = 4\cos^3 \theta \sin \theta - 4 \cos \theta \sin^2 \theta.
\]

\end{losn}


\begin{oppg}[3.3.8]
Regn ut $(1+i)^{804}$ og $(\sqrt 3 - 1)^{173}$.
\end{oppg}
\begin{losn}
Her lønner det seg å skrive om på polarform. Vi har at 
\[
1+i = \sqrt 2 e^{i \pi/4}
\]
og 
\[
-1+\sqrt{3} = 2 e^{-\pi i/6}.
\]
Dermed er
\[
(1+i)^{804} = \sqrt{2}^{804} \left(e^{i \pi/4} \right)^{804} = 2^{402} e^{201 \pi i} = -2^{402},
\]
siden $e^{\pi n}$ er $-1$ om $n$ er odde.

Den andre er hakket verre. Her har vi at
\[
(\sqrt{3}-i)^{173} = 2^{173} e^{\pi 173 i /6}.
\]

Her er trikset å bruke delealgoritmen til å skrive $173=28 \cdot 6 + 5$. Dermed er
\[
2^{173} e^{\pi 173 i /6} = 2^{173} e^{\pi(28 +5/6)i} = 2^{173} e^{\pi 5i/6}.
\]
Dette er lik
\[
2^{173}\left( - \frac{sqrt 3}{2} + i \frac 12 \right) = 2^{172}(-\sqrt 3 +i).
\]

\end{losn}

\end{document}