\documentclass[11pt, norsk]{article}
%\usepackage[latin1]{inputenc}
\usepackage[T1]{fontenc}
\usepackage[utf8]{inputenc}
\usepackage[norsk]{babel}   % S P R A A K
% \usepackage{graphicx}    % postscript graphics
\usepackage{amssymb, amsmath, amsthm, amssymb} % symboler, osv
\usepackage{mathrsfs}
\usepackage{url}
\usepackage{thmtools}
\usepackage{enumerate}  % lister $  
\usepackage{float}
\usepackage{tikz}
\usepackage{tikz-cd}
\usetikzlibrary{calc}
%\usepackage{tikz-3dplot}
\usepackage{subcaption}
\usepackage[all]{xy}   % for comm.diagram
\usepackage{wrapfig} % for float right
\usepackage{hyperref}
\usepackage{mystyle} % stilfilen      

%\usepackage[a5paper,margin=0.5in]{geometry}


\begin{document}
\title{Notater}
\author{Fredrik Meyer}
\maketitle 

\subsection{Tautologisk linjebunt på $\PP^n$}

Vi kan definere en tautologisk linjebunt på $\PP^n$. Dette er en bunt hvis fiber over et punkt $p$ er linjen i $\Aa^{n+1}$ utspent av (de homogene koordinatene til) $p$. La $q \in \langle p \rangle$ bety at $q$ er med i spennet av $p$. Da er
$$
\mathscr T := \{ (q,p) \in \Aa^{n+1} \times \PP^n \mid q \in \langle p \rangle \}.
$$
Ved å regne overgangsfunksjoner kan en se at $\mathscr T \simeq \OO_{\PP^n}(-1)$.

Det finnes også andre måter å se dette på, eksempelvis slik Mike Eastwood forklarte det på siste forelesning, men det har jeg glemt av nå (!!). 

\subsection{Embedding Grassmannian}

Grassmannian har en tautologisk linjebunt $\mathscr E$, hvis seksjoner kan skrives som matriser. Fiberen over et punkt $[V]$ i Grassmannian er nettopp det lineære underrommet punktet representerer.

Da vil $\wedge^k \mathscr E$ være en linjebunt på Grassmannian, og seksjonene vil være utspent av alle minorene. Så dette er linjebunten Plücker-embeddingen svarer til.

\end{document}