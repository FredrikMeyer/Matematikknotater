\documentclass[11pt, norsk]{article}
%\usepackage[latin1]{inputenc}
\usepackage[T1]{fontenc}
\usepackage[utf8]{inputenc}
\usepackage[norsk]{babel}   % S P R A A K


% \usepackage{graphicx}    % postscript graphics
\usepackage{amssymb, amsmath, amsthm, amssymb} % symboler, osv
\usepackage{mathrsfs}
\usepackage{url}
\usepackage{thmtools}
\usepackage{enumerate}  % lister $  
\usepackage{float}
\usepackage{tikz}
\usetikzlibrary{calc}
\usepackage[all]{xy}   % for comm.diagram
\usepackage{wrapfig} % for float right
\usepackage{hyperref}
\usepackage{mystyle} % stilfilen      

\begin{document}
\title{Oppgaver MAT2500}
\author{Fredrik Meyer}
\maketitle 

\begin{oppg}
Bruk forrige oppgave til å vise at hvis $m$ er orienteringsreverserende, så er $m^2$ en translasjon. (merk: forrige oppgave sa at alle isometrier er på formen $t_{\vec a} \rho_\theta$ eller $t_{\vec a} \rho_\theta s$.)
\end{oppg}
\begin{losn}
Først, legg merke til at isometrier på den første formen er orienteringsbevarende (verken rotasjoner eller translasjoner endrer orientering). Så $m$ må kunne skrives som $m=t_{\vec a} \rho_\theta s$. Vi viste også at 

Dermed er
\begin{align*}
m^2 &= (t_{\vec a} \rho_\theta s)(t_{\vec a} \rho_\theta s) \\
&= t_{\vec a} \rho_\theta s t_{\vec a} \rho_\theta s \\
&\stackrel{*}{=} t_{\vec a} s \rho_{-\theta} t_{\vec a} \rho_\theta s \\
&\stackrel{**}{=} t_{\vec a} s \rho_{-\theta} \rho_{\theta}t_{\rho_\theta(t_{\vec a})} s \\
&\stackrel{***}{=} 
\end{align*}
Litt forklaring. Første likhet er bare å fjerne parenteser. I (*) brukte vi at $\rho_\theta t_{\vec a} = t_{\rho_\theta \vec a}\rho_\theta$, som vi viste i forrige oppgave. I (**) brukte vi at to translasjoner satt sammen er en translasjon ($t_{\vec a}t_{\vec b} = t_{\vec a + \vec b}$). I (***) brukte vi at $s\rho_\theta = \rho_{-\theta s}$, og i (****) brukte vi at $\rho_\theta \rho_{-\theta}=id$ og $s^2=id$ (å rotere først $-\theta$ og så $\theta$ er det samme som å rotere ingenting, og å speile to ganger er det samme som å gjøre ingenting). 

Vi ender opp med $t_{\vec b}$ hvor $\vec b=\vec a + \rho_\theta \vec a$, så $m^2$ er en translasjon.
\end{losn}

\begin{oppg}
Gi en begrunnelse for hver likhet i utregningen til slutt i beviset for Setning 2.4.
\end{oppg}
\begin{losn}
Det vi har lyst å vise er at $t_{\vec a} s_l = t_{\vec{w_1}} s_{l^\prime}$. La oss huske hva alle disse bokstavene betyr. Tidligere i beviset ble det vist at en orienterings\textbf{reverserende} isometri kan skrives på formen $m=t_{\vec a} s_l$, hvor $s_l$ er en speiling om en linje $l$.

Her hjelper det veldig å tegne en tegning.

La $l$ være linjen utspent av vektoren $\vec v$. Siden $\R^2$ er $2$-dimensjonal, er $\{ \vec v, \vec v^\perp \}$ en basis for $\R^2$. Så vi kan skrive $\vec a = (\vec a \cdot \vec v) \vec v + (\vec a \cdot \vec v^\perp) \vec v^\perp$. La $\vec w_1 \stackrel{def}{=} (\vec a \cdot \vec v) \vec v$ og $\vec w_2 \stackrel{def}{=} (\vec a \cdot \vec v^\perp) \vec v^\perp$. La $l^\prime$ være linja $\{ \frac 12 \vec w_2 + \lambda \vec v \mid \lambda \in \R\}$. 

Da påstår vi først at $s_{l^\prime} = t_{\frac 12 \vec w_2} s_l t_{-\frac 12 \vec w_2}$. Skriv $\vec x \in \R^2$ som $\vec x = c_1 \vec w_1 + c_2 \vec w_2$. Da kan vi anvende isometriene over på $\vec x$ og håpe vi får det samme:
\begin{align*}
  s_{l'}(\vec x) &= s_{l^\prime}(c_1 \vec w_1 + c_2 \vec w_2) \\
&= c_1 \vec w_1 + (1-c_2) \vec w_2 \\
&= c_1 \vec w_1 + \vec w_2 -c_2 \vec w_2
\end{align*}
Også:
\begin{align*}
  t_{\frac 12 \vec w_2} s_l t_{-\frac 12 \vec w_2}(\vec x) &= t_{\frac 12 \vec w_2} s_l \left(c_1 \vec w_1+c_2 \vec w_2 -\frac 12 \vec w_2\right) \\
&= t_{\frac 12 \vec w_2} \left(c_1 \vec w_1 -c_2 \vec w_2+\frac 12 \vec w_2\right) \\
&= c_1 \vec w_1 -c_2 \vec w_2 + \vec w_2.
\end{align*}
Så høyresiden er lik venstresiden og alt er fint. 

Neste steg er å se at $s_l t_{-\frac 12 \vec w_2} = t_{\frac 12 \vec w_2} s_l$. Dette er ikke så vanskelig å se geometrisk. Her er en algebraisk måte å se det på:
\begin{align*}
  s_l t_{-\frac 12 \vec w_2}(\vec x) &=   s_l t_{-\frac 12 \vec w_2}(c_1 \vec w_1 + c_2 \vec w_2) \\
&= s_l \left(c_1 \vec w_1 + c_2 \vec w_2 - \frac 12 \vec w_2 \right)\\
&= c_1 \vec w_1 - c_2 \vec w_2 + \frac 12 \vec w_2.
\end{align*}
Dette var altså venstresiden (merk at å speile i $l$ er det samme som å sette alle koeffisienter av $\vec w_2$ til det negative).

Så for høyresiden:
\begin{align*}
  t_{\frac 12 \vec w_2}s_l(c_1 \vec w_1 + c_2 \vec w_2) &= t_{\frac 12 \vec w_2}(c_1 \vec w_1 - c_2\vec w_2) \\
&= c_1 \vec w_1 -c_2 \vec w_2 + \frac 12 \vec w_2.
\end{align*}
Så venstresiden er lik høyresiden og alt er fint.

Siste likhet som ble brukt i beviset var at $t_{\vec w_1}t_{\frac 12 \vec w_2}t_{\frac 12 \vec w_2}=t_{\vec a}$, men dette er åpenbart når vi husker at $\vec a = \vec w_1 + \vec w_2$.
\end{losn}

\begin{oppg}
 Anta $m$ er en isometri av planet som tar en linje $l$ på seg selv, $m(l)=l$, og at $\restr{m}{l}$ er en translasjon med en vektor $\vec a$. Gi et geometrisk argument for at $m$ enten er en speiling, en glidespeiling eller translasjonen $t_{\vec a}$.
\end{oppg}

\begin{losn}
For det første: vi kan bytte ut $m$ med $mt_{-\vec a}$. Da blir $\restr{m}{l}=id$, siden det ikke er noen translasjon langs linjen lenger. Så problemet blir nå: gitt at vi skal holde en hel linje fast, hvordan kan vi da flytte rundt på planet? Det er da ``klart'' at eneste mulighet er å speile om linja $l$ (rotasjon og translasjon går ihvertfall ikke).

Konklusjon: om $\vec a=0$, var dette allerede en speiling eller bare identitet. Om $\vec a \neq 0$, var dette enten en glidespeiling eller translasjon, avhengig om vi valgte å speile.
\end{losn}

\begin{oppg}
  
\end{oppg}

\end{document}
