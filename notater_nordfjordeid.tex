\documentclass[11pt, english]{article}
%\usepackage[latin1]{inputenc}
\usepackage[T1]{fontenc}
\usepackage[utf8]{inputenc}
\usepackage[english]{babel}   % S P R A A K
% \usepackage{graphicx}    % postscript graphics
\usepackage{amssymb, amsmath, amsthm} % symboler, osv
\usepackage{mathrsfs}
\usepackage{url}
\usepackage{thmtools}
\usepackage{enumerate}  % lister $  
\usepackage{float}
\usepackage{tikz}
\usepackage{tikz-cd}
\usetikzlibrary{calc}
%\usepackage{tikz-3dplot}
\usepackage{subcaption}
\usepackage[all]{xy}   % for comm.diagram
\usepackage{wrapfig} % for float right
\usepackage{hyperref}
\usepackage{mystyle} % stilfilen      

%\usepackage[a5paper,margin=0.5in]{geometry}


\begin{document}
\title{Nordfjordeid 2016}
\author{Fredrik Meyer}
\maketitle 

\section{Derived categories and derived functors}

We first start with notation. Let $\mathbb A$ be an abelian category. Let $C(\mathbb A)$ be the category of complexes with objects and maps from $\mathbb A$. We denote by $C^b(\mathbb A)$ the category of bounded complexes. $C^+(\Aa)$ is the category of bounded below complexes, and $C^-(\Aa)$ is the category of bounded above complexes.

\begin{defi}
Two morphisms $f,g:A^\bullet \to B^\bullet$ are \emph{homotopy equivalent} if there exists a set of maps $h^i:A^i \to B^{i-1}$ such that
\[
f^i-g^i = h^{i+1} \circ d^i_A + d^{i-1}_B \circ h^i.
\]
\end{defi}

The homotopy category $K(\Aa)$ is the category with the same objects as $C(\Aa)$, but where homotopic morphisms are identified.

This is a triangulated category. For this we need a shift operator: $A^\bullet \mapsto A^\bullet[1]$ (topologists like to write $A^\bullet[1]=\Sigma A^\bullet$). The differential is $d_{A[1]}^i=-d_A^{i+1}$. 

The \emph{cone of a morphisms} $j:A ^\bullet \to B^\bullet$ is denoted by $C(j)$ and defined as follows: each term is $C(j)^i=A^{i+1} \oplus  B^i$, and the maps are
\[
d_{C(j)} = \begin{pmatrix}
-d_A^{i+1} & 0 \\
j^{i+1} & d_B^i
\end{pmatrix}.
\]

Then a \emph{distinguised triangle} is a sequence of objects in $K(\Aa)$ and morphisms isomorphic to the sequence
\[
A^\bullet \xrightarrow{j} B^\bullet \to C(j) \to A[1].
\]

The derived category $D(\Aa)$ is the category with objects the same as $C(\Aa)$, and morphisms obtained by localizing $K(\Aa)$ with respect to quasiisomorphisms (that is, we require quasiisomorphisms to be invertible). Formally, a morphisms $A^\bullet \to B^\bullet$i n the derived category is defined as a \emph{roof}:
\[
\xymatrix{
& C^\bullet \ar[dl] \ar[dr]\\
A^\bullet & & B^\bullet
}
\]
The left arrow is a quasiisomorphism.

Knowing that $K(\Aa)$ is triangulated, we deduce (somehow) that $D(\Aa)$ is triangulated as well.

\subsection{Two examples}

\begin{enumerate}
	\item Let $X$ be a smooth point. Then $D(X)$ is easy to describe. All complexes of vector spaces are quasiisomorphic to the direct sum of their cohomologies. (more explicit??)
	\item Let $X$ be a smooth projective curve and $\mathcal E \in \Coh(X)$. Then there is an exact sequence
\[
0 \to \mathcal T \to \mathcal E \to \mathcal E_{tf} \to 0,
\]
where the last term is torsionfree and the first term is a torsion sheaf. This sequence splits (why?). And one can show that
\[
D^b(X) \ni \mathcal E = \oplus_{i \in I} \mathscr H^i(\mathcal E)[-i]  = \oplus_i (\mathscr H^i(\mathcal E_{tors}) \oplus \mathscr H^i(\mathcal E)_{tf})
\]
(I don't see why?)
\end{enumerate}

\subsection{Semiorthogonal decomposition for $D^b(X)$}

Suppose we have a sequence of full triangulated subcategories of $D^b(X)$.

\begin{defi}
A colleection $T_1,\ldots,T_n$  defines a \emph{semiorthogonal decomposition} for $D^b(X)$ if
\begin{enumerate}
	\item $\Hom_{D^b(X)}(T_i,T_j)=0$ for $i > j$.
	\item For all $F \in D^b(X)$, there exists a chain of morphisms
	$$
0 \to E_n \to E_{n-1} \to \ldots \to E_0=F,
	$$
	such that $\Cone(E_i,E_{i-1}) \in T_i$.
\end{enumerate}
\end{defi}
We write $D^b(X) = \langle T_1, \ldots, T_n \rangle$.

\begin{example}
Beilinson proved that $D^b(\PP^n)=\langle \OO_{\PP^n} ,\ldots \OO_{\PP^n}(n) \rangle$.
\end{example}

\begin{example}
If $X$ is a curve (smooth projective over $\C$), then $D^b(X)$ has a nontrivial semiorthogonal decomposition if and only if the genus is zero.
\end{example}

\begin{example}
If $X$ is smooth projective with $\omega_X=\OO_X$, then $D^b(X)$ has no semi-orthogonal decomposition.
\end{example}

\subsection{Derived functors}

\begin{itemize}
\item Derived pushforward
\item Derived tor
\item Fourier Mukai
\end{itemize}


\end{document}