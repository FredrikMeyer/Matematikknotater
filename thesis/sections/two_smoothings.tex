\chapter{The two smoothings of \texorpdfstring{$C(\dP6)$}{C(dP6)}}
%\chapter[The two smoothings of $\C(\dP6)$][The two smoothings of C(dP6)]{The two smoothings of $\boldsymbol{C(\dP6)}$}

\section{The del Pezzo surface \texorpdfstring{$\dP6$}{dP6}}
\label{sec:twosmoothings}

Denote by $\dP6$ the blow-up of $\P^2$ in three generic points.  These points can be chosen to be the coordinate points $(1:0:0),(0:1:0)$ and $(0:0:1)$. The torus action on $\P^2$ extends to an action on $\dP6$, so it is a toric variety.

\todo{picture of its fan/polytope}

There are several ways to describe the equations of $\dP6$, and we describe them here. Since $\dP6$ is the blowup of $\P^2$ in three points, we can blow them up separately. Let $x_0,x_1,x_2$ be coordinates of $\P^2$. Then the blowup of $\P^2$ in the point $(1:0:0)$ can be realized as the closed subscheme of $\P^2 \times \P^1$ given by the equation $r_0x_1-r_1x_2=0$, where $r_0,r_1$ are coordinates on $\P^1$. We can repeat this procedure on the two other points $(0:1:0)$ and $(0:0:1)$ to obtain similar equations. Collecting these, we see that $\dP6$ is given by the matrix equation
\[
M\vec x = 
\begin{pmatrix}
0 & r_0 & -r_1 \\
s_1 & 0 & -s_0 \\
-t_0 & t_1 & 0
\end{pmatrix}
\begin{pmatrix}
x_0 \\ y_0 \\ z_0
\end{pmatrix}= 0.
\]
in $\P^2 \times \P^1 \times \P^1 \times \P^1$. Note now that the matrix cannot have rank $1$ or lower. Consider the projection onto forgetting the $\P^2$-factor:
$$
\pi:\P^2 \times \P^1 \times \P^1 \times \P^1 \to \P^1 \times \P^1 \times \P^1.
$$

This means that the restriction of $\pi$ to $\dP6$ is an isomorphism onto the hypersurface given by $\det M=0$ in $\P^1 \times \P^1 \times \P^1$.

On the other hand, blowups can also be realized as closures of graphs of rational maps. Let $\varphi: \P^2 \rmap \P^2$ be the Cremona transformation given by $(x_0:x_1:x_2) \mapsto \left( \frac 1{x_0}: \frac 1{x_1}:\frac 1{x_2} \right)$. Then, in coordinates $(a_i,b_i)$ on $\P^2 \times \P^2$, the equations $a_0b_0=a_1b_1=a_2b_2$ hold. Hence $\dP6$ can also be realized as the intersection of two $(1,1)$-divisors in $\P^2 \times \P^2$. 

Hence, using the Segre embedding, $\dP6$ lives naturally in both ${\left( \P^1 \right)}^3 \hookrightarrow \P^7$ and $\P^2 \times \P^2 \hookrightarrow \P^8$. 

\section{The cone over \texorpdfstring{$\dP6$}{dP6} and its two smoothings}

The singularity $Z=C(\dP6)$ is one of the most studies singularities with an obstructed deformation space, see for example \cite{altmann_versaldeformation}.

\todo{Compute its $T^1$ from scratch?}

For well-behaved singularities, often one can describe all of its deformations by writing up a ``format'' of the equations. For example, for codimension three Gorenstein projective schemes, there is a structure theorem for the whole resolution, involving pfaffians. For codimension $4$, there is no such result, though there have been some research in this direction \todo{cite Reid}.

It is worthwhile to note that both smoothings of $Z$ arise by ``sweeping out the cone'' \todo{Cite Stevens}. 

There are two ways of writing up the homogeneous equations for $C(\dP6)$ as a subvariety of $\P^6$. The first way, which give rise to one of the smoothing components, comes from thinking of $\dP6$ as the graph of the Cremona transformation $\tau: \P^2 \rmap \P^2$. Then $\dP6$ is described as a subvariety of $\P^2 \times \P^2$ intersected with two hyperplanes. In fact, by choosing the blown up points appropriately, the equations take the form
\begin{equation}
\begin{vmatrix}
y & x_1 & x_2 \\
x_3 & y & x_4 \\
x_5 & x_6 & y
\end{vmatrix} \leq 1,
\end{equation}
where $\leq 1$, means taking all $2 \times 2$-minors.

On the other hand, $\dP6$ can be realized as a subvariety of $\P^1 \times \P^1 \times \P^1$ as well. The equations can be described as follows: draw a cube, and let each vertex correspond to a variable. Then the equations of $\P^1 \times \P^1 \times \P^1$ in its Segre embedding are given by taking all ``minors'' along all sides of the cube together with the three long diagonals. To get $\dP6$, one identifies two opposite corners. Thus in total there are $8-1=7$ variables, just as above. 

The two smootings are obtained by varying the defining hyperplane in each of the embeddings. 

Let us go into more detail.

The first smoothing is obtained by deforming the equations of $\dP6$ as a subvariety of $\P^2 \times \P^2$. Consider the following matrix:

\begin{equation}
\label{eq:def2}
\begin{vmatrix}
x_1 & y_0 & x_6 \\
x_2 & x_3 & y_0-t_1 \\
y_0-t_2 & x_4 & x_5
\end{vmatrix} \leq 1.
\end{equation}


For $t_1=t_2=0$, we get the cone over $\dP6$, while for generic $t_i$, we get a smooth variety. In fact, we can compute that the discrimant locus (the set of points in $\A^2_{t_1,t_2}$ with singular fiber) are the $t_1$-axis, the $t_2$-axis and the line $t_1=t_2$. 

Call (any) smooth fiber $X_2$. 

\begin{lemma}
Let $M=\P(\mathcal T_ {\P^2})$ be the projective bundle associated to the tangent sheaf on $\P^2$. Then the smoothing $X_2$ is isomorphic to $M \bs \dP6$. 
\end{lemma}
\begin{proof}
The technique is the same as in the previous proof. First homogenize the equations \eqref{eq:def2} with respect to $y_1$. Call the homogenized variety $M$. Put $y_0'=y_0$, $y_1' = y_0-ty_1$ and $y_2'=y_0-t_2y_1$. Then we have the relation
\[
h = t_2y_1'-t_1y_2' - (t_1-t_2)y_0' = 0.
\]
Hence we see that $M=\P^2 \times \P^2 \cap (h = 0)$. We can pull back the coordinates $y_i'$ to $\P^2 \times \P^2$. Let $\P^2 \times \P^2$ have coordinates $x_0,x_1,x_2$ and $y_0,y_1,y_2$. Then $h$ pulls back to the equation
\[
(x_0,x_1,x_2) \cdot (-t_1y_2, (t_1-t_2)y_0,t_2y_1) = 0
\]
in $\P^2 \times \P^2$. As long as $t_1 \neq t_2$ and $t_1,t_2 \neq 0$, we can do a change of coordinates in $\P^2_{y_0y_1y_2}$, so that $h$ transforms to
\[
(x_0,x_1,x_2) \cdot(y_0,y_1,y_2) = 0.
\]
Hence we see that $M$ is isomorphic to the total space of the Grassmannian of lines in $\P^2$ (each point in one of the $\P^2$'s give a line in the other $\P^2$). This is in turn isomorphic to $\P(\mathcal T_{\P^2})$, since each tangent vector through a point determines a line through it.

Now, what have we gained by homogenizing? The divisor at infinity is $y_1=0$, which is a $dP_6$ again. In our new coordinates this is equivalent to $y_1'=y_2'=y_0'$. Hence in the coordinates of $\P^2 \times \P^2$, the $dP_6$ is given by the two equations $x_1y_0-x_2y_1=x_1y_0-x_0y_2=0$. 
\end{proof}

The other smoothing is the obtained by replacing one of the $y$'s in the defining cube equations \todo{reformulate} with $y'=y+t$, obtained a one-parameter smoothing. Note that it is obtained by ``sweeping out the cone over over $(\P^1)^3$''.

Call this smoothing $X_1.$

\begin{lemma}
The smoothing $X_1$ is isomorpic to $\P^1 \times \P^1 \times \P^1 \bs \dP6$.
\end{lemma}
\begin{proof}
Homogenize, notice what is gained, then subtract.
\end{proof}

Observe that $\mathcal T(\P^2)$ is homotopy equivalent to $\P^1 \times \P^2$ \todo{source?}. It follows that its Euler characteristic, which is invariant under homotopy, is equal to $2 \times 3=6$.

This information let us calculate the Euler characteristics of the smoothings. Note that $\chi(\P^1)=2$ and $\chi(\mathcal T(\P^2))=6$. By additivity of the Euler characteristics we have $\chi(X_1)=2$ and $\chi(X_2)=0$, since $\chi(\dP6)=6$.

It follows that the two smoothing components correspond to topologically different smoothings. This can explain the obstructedness of the deformations of $X_0$ in \cref{sec:constructions}.


\begin{enumerate}
	\item Introduce $\dP6$
	\item Talk about 9-16-resolutions
	\item Its two smoothings
	\item They are topologically different
	\item Their cohomology groups
\end{enumerate}

