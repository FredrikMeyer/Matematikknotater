\chapter{Introduction}
\label{sec:intro}

General theme: smoothings of SR-schemes in order to construct new manifolds (or to put known manifolds in nice families)



\begin{enumerate}
    \item Smoothings of Stanley-Reisner schemes
    \item Paths on the Hilbert scheme
    \item Relation to triangulation of $\C \P^2$.
\end{enumerate}



\listoftodos[Notes]

\section{Notation}

If $V$ is a vector space, we denote by $\P(V)$ its projectivisation. We write $k$ for a field, which is almost always assumed to be $\C$. If $X$ is a projective variety, we write $S(X)$ for its homogeneous coordinate ring (if the embedding is implicit). All schemes are noetherian. We will often write $\stackrel \Delta = $ for definitions (instead of  ``$:=$'', common in computer science literature).