\chapter{Introduction}
\label{sec:intro}

The work leading up to this thesis started with a naïve idea.

 concerning smoothings of certain Stanley--Reisner schemes. Stanley--Reisner schemes are certain combinatorially defined highly singular projective schemes. The input data is a finite simplicial complex, and the output is a scheme defined by square-free monomials. If the simplicial complex is homeomorphic to $S^1$, then a smoothing of the Stanley--Reisner scheme yields an elliptic curve. Similarly, if the simplicial complex is a sphere, a smoothing of the Stanley--Reisner scheme will give a K3 surface.

The mentioned naïve idea was this: what if the simplicial complex is a triangulated $\C \P^2$? A smoothing of the associated Stanley--Reisner scheme would then give us a hyper-Kähler variety, as we explain in Chapter 2. This would be interesting, since there are very few known families of hyper-Kähler varieties.

Unfortunaly, given a triangulation of $\C \P^2$ with relatively few vertices, finding a smoothing of the Stanley--Reisner scheme turned out to be too difficult. However, one particular triangulation of $\C \P^2$ led us to study the problems in the next two chapters. This triangulation, found by Gaifullin \cite{cp2_15_chess}, is the union over three $4$-balls, all of which are suspensions over joins of hexagons. Leaving the idea of studying triangulations of $\C \P^2$, we began studying the triangulations of the $3$-sphere of which it was a union.

The join of two hexagons is a triangulated $3$-sphere. A smoothing of the associated Stanley--Reisner scheme $X_0$ is a Calabi--Yau variety. Finding new Calabi--Yau varieties has become a small industry, which we did not hesitate to join. This decision turned out to be profitable. The scheme $X_0$ deforms to several interesting varieties, and several of them are smooth. There are many connections to the physics literature, and to works by other mathematicians.

I found that the Stanley--Reisner scheme $X_0$ smooths to \emph{three} topologically different Calabi--Yau manifolds, which to my knowledge has not been described in the literature. Furthermore, by using the technique of orbifolding, I produce mirror candidates for two of them.

The organization of the thesis is as follows:

\begin{itemize}
	\item  In the first chapter, I gather background material which is relevant for the next chapters. I have erred on the side of \emph{too much} background information rather than too little. 

\item In the second chapter I motivate the original naïve idea about smoothing triangulations of $\C \P^2$'s to find new hyper-Kähler varieties.

I describe some triangulations of $\C \P^2$, and describe some of the obstacles encountered in trying to smooth them, together with the information about their $T^i$'s.

\item The third chapter is devoted to a special toric singularity, namely the affine cone $C(\dP6)$ over the del Pezzo surface $\dP6$. This singularity has two topologically different smoothings, and I compute their singular homology groups using techniques from toric geometry.

I start the chapter by discussing $\dP6$ in some generality. I discuss its Picard group and two natural embeddings in $\P^1 \times \P^1 \times \P^1$ and $\P^2 \times \P^2$, respectively.

It is well known that $C(\dP6)$ has two smoothings components. I identify them as hyperplane complements, and use this to compute their singular homology groups.

\item The final chapter is devoted to the construction of the new Calabi--Yau varieties and their mirrors.

I start the chapter by discussing the Stanley--Reisner scheme $X_0$, which comes from the simplicial complex that is the join of two hexagons. I compute its Hilbert polynomial, and explain how it deforms to a special singular Calabi--Yau variety $X_Y$.

Then I explain the construction of three topologically different smoothings $X_i$ ($i=1,2,3$) of $X_Y$ (and hence of $X_0$). They are topologically different because a \MM computation shows that their topological Euler characteristics are different. The construction is very similar to that of Rødland \cite{rodland_pfaffian}.

Then we explain the existence of special singular subfamilies of $X_1$ and $X_2$ which are invariant under a finite group. Using orbifolding and a formula by Roan, we propose conjectural mirror candidates for $X_1$ and $X_2$.

We end with many open questions, which we hope to see answered in the future.


\end{itemize}

The source code of the thesis and all computer computations are available on GitHub at \url{https://github.com/FredrikMeyer/Matematikknotater/tree/master/thesis} and \url{https://github.com/FredrikMeyer/m2files}, respectively.


%\listoftodos[Notes]

\section{Notation}

If $V$ is a vector space, we denote by $\P(V)$ its projectivisation. We write $k$ for a field, which is almost always assumed to be $\C$. If $X$ is a projective variety, we write $S(X)$ for its homogeneous coordinate ring (if the embedding is implicit). If $X$ is a scheme over $k$, we write $X/k$. We will write $h^i(X,\mathscr F)$ for $\dim_k H^i(X,\mathscr F)$. All schemes are noetherian. We will often write $\stackrel \Delta = $ for definitions (instead of  ``$:=$'', common in computer science literature).