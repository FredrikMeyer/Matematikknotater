\chapter{Computer code}
\label{sec:computercode}

I have used extensive use of computer software such as \MM and SAGE. In this Appendix I reproduce some of the code used to experiment and prove some of the results.

\section{Computing the singular locus}

In some cases, equations simplify significantly in affine charts. Therefore, using the naive command \texttt{singularLocus} in \MM often takes unnecessarily long time (and sometimes the computations never finish). The command \texttt{minimalPresentation} eliminates variables to produce a new ring isomorphic to the first one, but with fewer equations.

The following code produces a list of the components of the singular locus of the projective scheme with ideal sheaf \texttt{IX}. 

\begin{verbatim}
singlist = {}
for i from 0 to 11 do {
    affineChart    = sub(IX, x_i => 1);
    sings 	       = radical ideal mingens ideal
    					singularLocus minimalPresentation affineChart;
    sings          = decompose sings;
    invz = affineChart.cache.minimalPresentationMap;
    singlist = singlist | apply(sings, I -> saturate(homogenize(preimage(invz, sings),x_i)));
    }
\end{verbatim}

\section{Torus action}

The following lines checks if a projective scheme with ideal sheaf \texttt{IX} admits an action of a subtorus of $G=(\C^\ast)^n \subset \P^n$. To check this, we check if the equations are still valid after a torus action. Since $G$ is abelian, it act on functions by $\lambda \cdot f(x_1,\ldots,x_n)=f(\lambda_1 x_1, \ldots, \lambda_n x_n)$. 


\begin{lemma}
Suppose $\{ f_1,\ldots, f_r \}$ is a homogeneous generating set for $I_X=\text{\texttt{IX}}$. Then subgroup of $G$ acting on $X \subset \P^n$ is generated by those $\lambda \in G$ such that $\lambda \cdot f_i  = c f_i$ for some $c \in \C^\ast$.
\end{lemma}
\begin{proof}
Let $H$ be the subgroup of $G$ fixing the ideal $I_X$. Let $H'$ be the subgroup of $g \in G$ acting on the $f_i$ by scalar multiplication: $g \cdot f_i =c f_i$. Clearly $H' \subseteq H$.  Now suppose $g \in H$. Then
$$
g \cdot f_1 = \sum_j a_j f_j
$$
for some constants $a_j$. Now $g \cdot f_1 = f_1(\lambda_1 x _1 ,\ldots, \lambda_n x_n)$. Suppose the leading term of $f_1$ is $x_1^{a_1}\cdots x_n^{a_n}$. Then comparing leading terms in the left hand side and the right hand side, we see that $a_1 = \lambda_1^{a_1}\cdots \lambda_n^{a_n} := \lambda^m$. Hence the right hand side is $\lambda^m f_1 + \text{other terms}$. But now there are the same number of terms on each side of the equation, so there are no other terms. Hence $H=H'$. 
\end{proof}

It follows that to find the subgroup of $G$ acting on $X$, we have to find the $\lambda \in G$ such that the $f_i$ are simultaneous eigenvectors for them.

\begin{example}
Let  $X$ be defined by $f = x_0x_1x_2x_3x_4+\sum_{i=0}^5 x_i^5$ in $\P^4$. Then for $\C^4$ to act on it, we must have $\lambda_0\lambda_1\lambda_2\lambda_3\lambda_4=\lambda_0^5=\ldots=\lambda_4^5$. By stting $\lambda_0=1$, we see that all the $\lambda_i$ are the fifth roots of unity. Hence the subgroup acting on $H$ is the subgroup of $\Z/5^5/Z_5$ given by $\{ (a_0,\ldots,a_5) \mid \sum a_i = 0 \}$.
\end{example}

The following code find the subtoruses of $G$ acting on $X$ in this way, by equating terms in the polynomials defining $X$.

\begin{lstlisting}[language=Macaulay2]
loadPackage "Binomials"
torus = ideal apply(flatten apply(apply(apply(flatten entries gens IX, monomials), v ->  flatten entries v), j -> subsets(j,2)),    s -> s_0-s_1)
toruskomps = BPD torus
toruskomps = select(toruskomps, I -> dim I == 1)
\end{lstlisting}
\begin{proof}[Explanation]
The ideal \texttt{torus} is the ideal generated by the differences of terms in the polynomials defining $X$. The \MM package \texttt{Binomials} can decompose binomials over cyclic extensions of $\Q$ with the command \texttt{BPD}. Finally, we select the components corresponding to finite subgroups of the torus.

Then we check manually if these actually correspond to non-trivial actions.
\end{proof}


