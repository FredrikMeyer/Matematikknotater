\chapter{Relation to triangulations of \texorpdfstring{$\C\P^2$}{CP2}}
\label{sec:cp2triangs}

This chapter will not contain any new results of any significance, but is rather a report on an idea which branched off to the explorations in the next two chapters.

We explain an interesting connection between the topological space $\C \P^2$ and degenerations of hyper-Kähler manifolds.

%%%%%%%%%%%
%%%%%%%%%%%
\section{Hyper-Kähler manifolds} % (fold)
\label{sec:hyper_kähler_manifolds}

One often divides varieties into three types: those with positive, negative or trivial canonical class. Of those with trivial canonical class, two prominent types stand out: Calab--Yau-manifolds and hyper-Kähler manifolds.

\begin{definition}
A \emph{hyper-Kähler manifold} $X$ is a simply connected compact Kähler\footnote{Recall that a complex manifold is \emph{Kähler} if it is equipped with a Hermitian metric $h$ whose associated two-form $\sigma$ is closed. The two-form $\sigma$ is defined by $\sigma(u,v) = \Re h(iu,v)$.} complex manifold such that $H^0(X, \Omega_X^2)$ is generated by a non-degenerate $2$-form $\sigma: TX \times TX \to \C$.
\end{definition}

\begin{remark}
Since the two-form $\sigma$ is non-degenerate, it follows that the canonical sheaf $\omega_X=\Omega_{X/\C}^n$ is trivial. The map $1 \mapsto \sigma^{n/2}$ gives an isomorphism $\OO_X \to \omega_X$. 
\end{remark}

\begin{remark}
In dimension $2$, there is no difference between Calabi--Yau varieties and hyper-Kähler manifolds. These are the K3 surfaces\footnote{K3 surfaces are named after Kummer, Kähler and Kodaira.}. 
\end{remark}

Because of the non-degeneracy of the symplectic form $\sigma \in H^0(X, \Omega_X^2)$, hyper-Kähler manifolds only occur in even dimensions: the determinant of the skew-symmetric form is equal to $\det \sigma = (-1)^n \det \sigma$, implying $(-1)^n=1$, so that $n$ has to be even.

Only a few explicit families of hyper-Kähler manifolds are known. Below we sketch the construction of two such families.

\subsection{The Hilbert square $S^{[2]}$}

Let $S$ be a K3-surface with symplectic form $\sigma$, and let $S^{(2)}$ be its symmetric square: $S \times S / \{ (p,q) \sim (q,p) \}$. Let $\pi_i:S \times S \to S$ be the two projections ($i=1,2$). Then the 2-form $\pi_1^\ast \sigma + \pi_2^\ast \sigma$ is $\Z/2$-invariant, hence it descends to a 2-form  $\tau$ on $S^{(2)}$.

The space $S^{(2)}$ is singular along the diagonal: locally it is isomorphic to $\C^2 \times \C^2 /(x \sim -x)$. The last factor is a quadric cone, so a single blowup along the diagonal will resolve the singularities. The form $\tau$ lifts to a non-degenerate form on the blowup $\Bl_\Delta S^{(2)}$, which we denote by $S^{[2]}$. It can be shown that it is in fact a hyper-Kähler variety of dimension $4$. The resulting space is denoted by $S^{[2]}$, and is called the \emph{Hilbert square of $S$}, or the \emph{Hilbert scheme of two points on $S$}. It parametrizes length two subschemes of $S$.

For more details on this construction, see Beauville's original paper \cite{beauville_hyperkahler}.

\subsection{Lines on hypersurfaces}

There is another construction of hyper-Kähler varieties that is interesting to us. Let $X$ be a smooth cubic fourfold in $\P^5$. Let $F(X)$ denote the set of lines contained in $X$. It is the \emph{Fano variety of lines on $X$}, and is a closed subset of the Grassmannian $\mathbb G(1,\P^5)$. One can can show that $F(X)$ is a hyper-Kähler variety of dimension $4$.

In the article \cite{beauville_donagi_fano}, Beauville and Donagi shows that $F(X)$ is deformation equivalent to $S^{[2]}$ for some K3 surface $S$. They also show that if $X$ is a \emph{pfaffian} hypersurface, then $F(X)$ is actually \emph{isomorphic} to $S^{[2]}$ for some K3 surface $S$. Furthermore, the family $\{ F(X) \}$  obtained this way is 19-dimensional, and is a hypersurface in the deformation space of $S^{[2]}$.

For more details on hyper-Kähler manifolds and their constructions, we recommend the lecture notes \cite{lehn_symplectic}.


%%%%%%%%%%
\section{Connection to the complex projective plane}

Let $X$ be a topological space. Recall that the symmetric product $X^{(2)}$ is defined as follows:
\[
X^{(2)} = X \ast X = X \times X / \{ (x,y) \sim (y,x) \}.
\]

If $X=S^2$, we have that $X^{(2)}$ is naturally isomorphic to $\C \P^2$, which can be seen as follows: $S^2$ can be identified with $\P^1_\C$. Unordered pairs of points in $\P^1$ correspond to degree $2$ polynomials up to scalar multiplication. Hence we have identifications
$$
(S^2)^{(2)} = (\P^1)^{(2)} = \{ (P,Q) \in \P^1 \times \P^1 \} / \Z_2 = \P\left( H^0(\OO_{\P^1}(2)) \right)=\C \P^2.
$$

Here is an observation.

\begin{lemma}
If $\mathcal K$ is a simplicial complex that is a \emph{manifold}, isomorphic to $S^2$, then a smoothing of $\mathcal K$ is K3 surface.
\end{lemma}
\begin{proof}
See \cite{eisenbud_graphcurves}.
\end{proof}

Stanley--Reisner degenerations of K3 surfaces correspond to triangulated $2$-spheres. Since the symmetric square of a sphere is $\C \P ^2$, a Stanley--Reisner degeneration of the symmetric square of a K3 surface should correspond to a triangulated $\C \P^2$. 

\begin{proposition}
Suppose $\K$ is a triangulation of $\C \P^2$ and $X_0=\P(\K)$ is its associated Stanley--Reisner-scheme. Then a smoothing $X$ of $X_0$ will be a hyper-Kähler manifold.
\end{proposition}
\begin{proof}
\todo{hva med simply-connected?}
The dimensions of the groups $H^i(X,\OO_X)$ are constant in flat families. Since $H^0(X,\Omega^2_X)=H^2(X,\OO_X)=H^2(\K;\C)=\C$ (the first equality is complex conjugation), we have that $H^0(X,\Omega^2_X)$ is generated by a single 2-form. It is non-degenerate since $\omega_X \simeq \OO_X$. It follows that $X$ is hyper-Kähler.
\end{proof}

\section{Attempt to smooth triangulations}

If $\K$ is a triangulation of $\C \P^2$ and $\P(\K)$ is the associated Stanley--Reisner-scheme, a smoothing of $\P(\K)$ will give a hyper-Kähler variety. Using this idea, and the \MM package \texttt{VersalDeformations} (by Nathan Ilten, see \cite{ilten_versaldeformations}), we tried to find potentially new hyper-Kähler varieties. Unfortunaly, it looks like all the triangulations we experimented with were not smoothable.

In the next four subsections we describe four different triangulations of $\C \P^2$, their ideal structure, and compute some of their deformation theoretic invariants. In all cases we conclude that the corresponding Stanley--Reisner scheme is probably not smoothable.

Before we go on to describe the triangulations, we recall some basic facts about combinatorial manifolds.

We can decompose $\C \P^2$ into three four-dimensional closed balls $B_j$, whose pairwise intersections are solid tori $\Pi_{ij}$, and whose triple intersection is a two-dimensional torus $T$. The closed balls $B_0$ is defined as 

$$
B_0 = \left\{ [x_0:x_1:x_2] \in \C \P ^2 \mid x_0\overline{x_0} \geq x_1 \overline{x_1}, \,  x_0\overline{x_0} \geq x_2\overline{x_2} \right\},
$$

and similarly for $B_1$ and $B_2$. This is sometimes called the \emph{equilibrium decomposition} of the complex projective plane.

A triangulation of $\C \P^2$ is \emph{equilibrium} if the closed balls, the solid tori, and the torus $T$ are subcomplexes of the triangulation. Several of the triangulations below are equilibrium.

\subsection{The 15-vertex triangulation}

A very interesting triangulation $\mathcal T$ of $\C\P^2$ is found in \cite{cp2_15_chess}. The author describes a triangulation of $\C \P^2$ using $15$ vertices. One reason it is interesting is that the corresponding Stanley--Reisner scheme $\P(\mathcal T)$ has the same Hilbert-polynomial as $F_1(X)$, the Fano variety of lines on a cubic hypersurface. This means that they live in the \emph{same} Hilbert scheme, and one could naively hope that they live in the same component as well, meaning that there exists a degeneration of $F_1(X)$ to $\P(\mathcal T)$.

 We will spend some time describing this tringulation, since parts of it inspired our the construction of the Calabi--Yau's in the last chapter. We cite the definition ad verbatim from \cite{cp2_15_chess}. 

 \begin{definition}
 Let $V_4 \subset S_4$ be the Klein four group. The vertex set of $\mathcal T$ is defined as
 \begin{equation}
 V = (V_4 \backslash \{e \}) \sqcup \left( \{1,2,3,4 \} \times \{ 1,2,3 \} \right).
 \end{equation}
 Thus the vertices of $\mathcal T$ are the permutations $(12)(34)$,$(13)(24)$ and $(14)(23)$ and the pairs of integers $(a,b)$ with $1 \leq a \leq 4$ and $1 \leq b \leq 3$. The maximal faces are spanned by the sets 
 \begin{equation}
 \nu, (1,b_1), (2,b_2), (3,b_3), (4,b_4)
 \end{equation}
 with $\nu \in V_4 \backslash \{ e \}$ and $1 \leq b_a \leq 3$ ($a=1,2,3,4$) such that $b_{\nu(a)} \neq b_a$ for $a=1,2,3,4$.
 \end{definition}

See \cref{sec:compute_gaifullin} for a SAGE \cite{sagemath} script for computing the maximal facets of $\mathcal T$. The f-vector is $(15,90,240,270,108)$.

The triangulation $\mathcal T$ is the union over the cones over three $3$-spheres $S_j$, so that $\mathcal T$ is an equilibrium triangulation. Each $S_j$ is a very simple $3$-sphere. It is the join of two hexagons (recall that $S^1 \ast S^1 \approx S^3$).

It is the Stanley--Reisner-scheme of $S_j$ and some if its deformations that is studied in Chapter 4, leading to constructions of some new Calabi--Yau manifolds. 

We compute some deformation-theoretic invariants of $\P(\mathcal T)$, the Stanley--Reisner scheme associated to $\mathcal T$.

\begin{proposition}
We have that $\dim_\C  {T^1(S_{\P(\mathcal T)}/k, S_{\P(\mathcal T)})}_0=90$ and $\dim_\C {T^2(S_{\P(\mathcal T)}/k, S_{\P(\mathcal T)})}_0=306$. The normal sheaf $\mathcal N_{\P(\mathcal T)/\P^{14}}$ has $300$ global sections.
\end{proposition}

The proof is a computation in \MM. We remark that since $\P(\mathcal T)$ is not Cohen--Macaulay, some standard comparison theorems does not hold. In our case we only have an inclusion $T^1_{A(\mathcal T)} \hookrightarrow T^1_{\P(\mathcal T)}$ (see the article of Kleppe \cite{kleppe_deformations} and Theorem 3.9). This means that there might be deformations of $\P(\mathcal T)$ that are not induced from the ambient projective space. \todo{usikker på hva dette betyr!!}

Because of the high number of parameters, we have not been able to say anything meaningful regarding the deformations of $\P(\mathcal T)$. However, it is possible to deform $\P(\mathcal T)$ into the union of three toric varieties, each being deformations of the Stanley--Reisner scheme $\P(B_j)$. This is not surprising, since $B_j$ is a triangulation of the normal polyhedron of the corresponding toric variety. This deformation reduces the number of components of $\P(\mathcal T)$ from $108$ to $3$.

It is not clear however if this union of toric varieties can be further deformed.

%%%%%%%%%%
\subsection{Kühnel's 9-vertex triangulation}

The minimal triangulation $\mathcal T_9$ of $\C \P^2$ is a 9-vertex triangulation with f-vector $(9,36,84,90,36)$. This means that the associated Stanley--Reisner scheme $\P(\mathcal T_9)$ lives in $\P^8$ and is of degree $36$. The automorphism group of $\mathcal T_9$ is a group of order $54$, and it can be realized as a semidirect product $(\Z_3 \times \Z_3) \ltimes \Z_3 \ltimes \Z_2$. For a very readable account of the construction and motivation of this triangulation, consult \cite{kuhnel_9vertex}.

The ideal has a resolution of the form (in \MM format):

\begin{verbatim}
             0  1  2  3  4 5 6
      total: 1 36 90 84 37 9 1
          0: 1  .  .  .  . . .
          1: .  .  .  .  . . .
          2: .  .  .  .  . . .
          3: . 36 90 84 36 9 1
          4: .  .  .  .  . . .
          5: .  .  .  .  1 . .
\end{verbatim}

This means that the ideal of $\P(\mathcal T_9)$ is generated by $36$ cubic monomials, and there are $90$ relations between them, lying in $\OO_{\P(\mathcal T_9)}(-5)$, et cetera. Since the resolution is not symmetric, we see immediately that $\P(\mathcal T_9)$ is not Gorenstein.

\begin{proposition}
We have that $\dim_\C  {T^1(S_{\P(\mathcal T)}/k, S_{\P(\mathcal T)})}_0=21$ and $\dim_\C = {T^2(S_{\P(\mathcal T)}/k, S_{\P(\mathcal T)})}_0=126$. The normal sheaf $\mathcal N_{\P(\mathcal T_9)/\P^{8}}$ has $93$ global sections.
\end{proposition}

We can compute the action of the automorphism group on $T^1$. Using SAGE, we find that the $21$ deformation parameters split in two orbits, one of size $3$ and one of size $18$.

I have not been able to lift any first-order deformation of $\P(\mathcal T_9)$ to a family over $\Spec \C[t]$. 

%%%%%%%%%
%%%%%%%%%
\subsection{The minimal equilibrium triangulation}

In \cite{banchoff_equilibrium}, the authors construct a $10$ vertex equilibrium triangulation $\mathcal T_{10}$  of $\C \P^2$. They start with the minimal $7$-vertex triangulation of the torus, and then they construct $\mathcal T_{10}$ by taking cones over unions of three tori.

The automorphism group is order $42$, and comes from the symmetries of the torus. 

The Betti table of the resolution of the ideal of $\P(\mathcal T_{10})$ is the following:
\begin{verbatim}
             0  1   2   3   4  5  6 7
      total: 1 38 128 177 123 46 10 1
          0: 1  .   .   .   .  .  . .
          1: .  3   2   .   .  .  . .
          2: .  .   .   .   .  .  . .
          3: . 35 126 175 120 45 10 1
          4: .  .   .   2   3  .  . .
          5: .  .   .   .   .  1  . .
\end{verbatim}

Again we see that the ideal is not Gorenstein.

\begin{proposition}
We have that $\dim_\C  {T^1(S_{\P(\mathcal T)}/k, S_{\P(\mathcal T)})}_0=42$ and $\dim_\C = {T^2(S_{\P(\mathcal T)}/k, S_{\P(\mathcal T)})}_0=105$. The normal sheaf $\mathcal N_{\P(\mathcal T_{10})/\P^{8}}$ has $132$ global sections.
\end{proposition}

In fact, it is possible to lift the versal family of deformation parameters to an honest family over $\Spec \C[t_1,\cdots,t_{42}]$, using the \texttt{VersalDeformations} package. Surprisingly, even though the $T^2$ module is big, there are no obstructions in the family (in the sense that the base space is $\A^{42}$). However, the generic member of this family is reducible (verified in \MM for ``random'' values of the deformation parameters), implying that $\P(\mathcal T_{10})$ is not smoothable.

The automorphism group act transitively on the natural basis of $T^1$, so that $\dim_\C {{T^1(S_{\P(\mathcal T)}/k, S_{\P(\mathcal T)})}_0}^G=1$. 

%%%%%%%%%%%%%%%%%%%
\subsection{The Bagchi--Datta triangulation}

There is another $10$-vertex triangulation $\mathcal T_{BD}$ of $\C \P^2$, which is obtained as a $\Z/2$-quotient of a triangulation of $S^2 \times S^2$. It is described in \cite{bagchi_datta}. The automorphism group is the alternating group $A_4$. The f-vector is $(10,45,110,120,48)$.

The triangulation is bistellarly equivalent to both the 9-vertex triangulation and the 10-vertex triangulation above.

\begin{proposition}
We have that $\dim_\C  {T^1(S_{\P(\mathcal T)}/k, S_{\P(\mathcal T)})}_0=41$ and $\dim_\C = {T^2(S_{\P(\mathcal T)}/k, S_{\P(\mathcal T)})}_0=180$. The normal sheaf $\mathcal N_{\P(\mathcal T_{BD})/\P^{8}}$ has $131$ global sections.
\end{proposition}

We have not been able to find any meaningful lifting of the first-order deformations here either.

\section{Naïve attempt to degenerate}

Degenerating the ideal of $F_1(X) \subset \P^N$ to a square-free monomial ideal should give a triangulation of $\C \P^2$. Since $F_1(X)$ sits inside $\Gr(1,5)$, and there are many known degenerations of $\Gr(1,5)$, we hoped that maybe $F_1(X)$ would degenerate inside $\Gr(1,5)$. Unfortunaly, I did not succed, mainly because we could not see any structure in the ideal of $F_1(X)$.

It was possible to explicitly compute $F_1(X)$ for some hypersurfaces, both pfaffian and non-pfaffian. However, the ideals were too complicated and the Gröbner bases too big to find any initial ideals with only squarefree generators (and even their existence is unclear).

%%%%%%%%%%%%%%%%%%%%
\section{Conclusion}

It would be interesting to study other triangulations of $\C \P^2$. One way to proceed would be to start with existing triangulations, and analyze which parts of it correspond to non-zero elements of the $T^2$ module, perhaps using the results from \cite{deforming_christophersen}. Then one can do bistellar flips away from these combinations, ideally obtaining triangulations corresponding to unobstructed Stanley--Reisner schemes.

This is an interesting and very hard question. Even with an unobstructed triangulation, it is not clear how to proceed to smooth it in a computationally feasible way. Already with Gröbner bases with ~$50$ elements (for deformations of the 15-vertex triangulation, they had around $70$ elements), computations take far too long (and consumes too much memory) to be feasible to work with.

Without the presence of any good parallell processing Gröbner basis algorithms (which would allow the use of clustered super--computers), there is need for either more patience or smarter solutions to computational algebra problems.