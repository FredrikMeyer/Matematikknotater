\chapter{Relation to triangulations of \texorpdfstring{$\C\P^2$}{CP2}}
\label{sec:cp2triangs}

This chapter will not contain any new results of any signifance, but is rather a report on an idea which led to the deliberations in the later chapters.

We explain a connection between the topological space $\C \P^2$ and hyper-Kähler manifolds.

\section{Introductory remarks} % (fold)

\subsection{Hyper-Kähler manifolds}
\label{sec:hyper_kähler_manifolds}

Among the known families of manifolds, hyper-Kähler manifolds are among the most elusive. One often divides manifolds into three types: those with positive, negative or trivial canonical class. Of those with trivial canonical class, two prominent types stand out: Calab--Yau-manifolds and hyper-Kähler manifolds.

\begin{definition}
A \emph{hyper-Kähler manifold} $X$ is a simply connected compact Kähler manifold such that $H^0(X, \Omega_X^2)$ is generated by a non-degenerate $\sigma: TX \times TX \to \C$.
\end{definition}

\begin{remark}
Since the two-form $\sigma$ is non-degenerate, it follows that the canonical sheaf $\omega_X=\Omega_{X/\C}^n$ is trivial. The map $1 \mapsto \sigma^{n/2}$ gives an isomorphism $\OO_X \to \omega_X$. 
\end{remark}

For example, in dimension two, K3 surfaces are hyper-Kähler (\emph{and} Calab--Yau). Because of the non-degeneracy of the symplectic form $\sigma \in H^0(X, \Omega_X^2)$, hyper-Kähler manifolds only occur in even dimensions. Only a few explicit families of hyper-Kähler manifolds are known. Below we sketch the construction of one such family.

Let $S$ be a K3-surface with symplectic form $\sigma$, and let $S^{(2)}$ be its symmetric square: $S \times S / \{ (p,q) \sim (q,p) \}$. Let $\pi_i:S \times S \to S$ be the two projections ($i=1,2$). Then the 2-form $\pi_1^\ast \sigma + \pi_2^\ast \sigma$ is $\Z/2$-invariant, hence it decends to a 2-form  $\tau$ on $S^{(2)}$.

The space $S^{(2)}$ is singular along the diagonal: locally it is isomorphic to $\C \times \C /(x \sim -x)$. The last factor is a quadric cone, so a single blowup along the diagonal will resolve the singularities. The form $\tau$ lifts to a non-degenerate form on $S^{[2]}$, and it can be shown that it is in fact a hyper-Kähler variety of dimension $4$. The resulting space is denoted by $S^{[2]}$, and is called the \emph{Hilbert square of $S$}, or the \emph{Hilbert scheme of two points on $S$}. It parametrizes length two subschemes of $S$.

For more details on this construction, see Beauville's original paper \cite{beauville_hyperkahler}.


\subsection{The variety of lines on a cubic fourfold}

There is another construction of hyper-Kähler varieties that is interesting to us. Let $X$ be a smooth cubic fourfold in $\P^5$. Let $F(X)$ denote the set of lines contained in $X$. It is the \emph{Fano variety of lines on $X$}, and is a closed subset of the Grassmannian $\mathbb G(1,\P^5)$. One can can show that $F(X)$ is a hyper-Kähler variety of dimension $4$.

In the article \cite{beauville_donagi_fano}, Beauville and Donagi shows that $F(X)$ is deformation equivalent to $S^{[2]}$ for some K3 surface $S$. They also show that if $X$ is a \emph{pfaffian} hypersurface, then $F(X)$ is actually \emph{isomorphic} to $S^{[2]}$ for some K3 surface $S$. Furthermore, the family $\{ F(X) \}$  obtained this way is 19-dimensional, and is a hypersurface in the deformation space of $S^{[2]}$.


\section{Connection to the complex projective plane}

Let $X$ be a topological space. Recall that the symmetric product $X^{(2)}$ is defined as follows:
\[
X \ast Y = X \times Y / \{ (x,y) \sim (y,x) \}.
\]

If $X=S^2$, we have that $X^{(2)}$ is naturally isomorphic to $\C \P^2$, which can be seen as follows: $S^2$ can be identified with $\P^1_\C$. Unordered pairs of points in $\P^1$ correspond to degree $2$ polynomials up to scalar multiplication. Hence we have identifications
$$
(S^2)^{(2)} = (\P^1)^{(2)} = \{ (P,Q) \in \P^1 \times \P^1 \} / \Z_2 = \P\left( H^0(\OO_{\P^1}(2)) \right)=\C \P^2.
$$

Stanley--Reisner degenerations of K3 surfaces correspond to triangulated spheres. Since the symmetric square of a sphere is $\C \P ^2$, a Stanley-Reisner degeneration of the symmetric square of a K3 surface should correspond to a triangulated $\C \P^2$. 

\begin{proposition}
Suppose $\K$ is a triangulation of $\C \P^2$ and $X_0=\P(\K)$ is its associated Stanley--Reisner-scheme. Then a smoothing $X$ of $X_0$ will be a hyper-Kähler manifold.
\end{proposition}
\begin{proof}
\todo{hva med simply-connected?}
The dimensions of the groups $H^i(X,\OO_X)$ are constant in flat families. Since $H^0(X,\Omega^2_X)=H^2(X,\OO_X)=H^2(\K;\C)=\C$ (the first equality is complex conjugation), we have that $H^0(X,\Omega^2_X)$ is generated by a single 2-form. It is non-degenerate since $\omega_X \simeq \OO_X$. It follows that $X$ is hyper-Kähler.
\end{proof}

\section{Attempt to smooth triangulations}

If $\K$ is a triangulation of $\C \P^2$ and $\P(\K)$ is the associated Stanley--Reisner-scheme, a smoothing of $\P(\K)$ will give a hyper-Kähler variety. Using this idea, and the \MM package \texttt{VersalDeformations} (by Nathan Ilten, see \cite{ilten_versaldeformations}), we tried to find potentially new hyper-Kähler varieties. Unfortunaly, it looks like all the triangulations we experimented with were non-smoothable.

In the next four subsections we describe four different triangulations of $\C \P^2$, and also their deformation theory using the results of \cite{deforming_christophersen}. In all cases we conclude that the corresponding Stanley--Reisner scheme is probably not smoothable.

Before we go on to describe the triangulations, we recall some basic facts about combinatorial manifolds.

We can decompose $\C \P^2$ into three four-dimensional closed balls $B_j$, whose pairwise intersections are solid tori $\Pi_{ij}$, and whose triple intersection is a two-dimensional torus $T$. The closed balls $B_0$ is defined as 
$$
B_0 = \{ [x_0:x_1:x_2] \in \C \P ^2 \mid x_0\overline{x_0} \geq x_1 \overline{x_1}, \,  x_0\overline{x_0} \geq x_2\overline{x_2} \},
$$
and similarly for $B_1$ and $B_2$. This is sometimes called the \emph{equilibrium decomposition} of the complex projective plane.

A triangulation of $\C \P^2$ is \emph{equilibrium} if the closed balls, the solid tori, and the torus $T$ are subcomplexes of the triangulation. Several of the triangulation below are equilibrium.

\begin{enumerate}
	\item Kühnels 9-vertex
	\item The equilibrium triangulation (10 vertices)
	\item Bagci/Datta
	\item 15-vertex triang
\end{enumerate}

\subsection{The 15-vertex triangulation}

A very interesting triangulation $\mathcal T$ of $\C\P^2$ is found in \cite{cp2_15_chess}. The author describes a triangulation of $\C \P^2$ using $15$ vertices. One reason it is interesting is that it the corresponding Stanley--Reisner scheme $\P(\mathcal T)$ has the same Hilbert-polynomial as $F_1(X)$, the Fano variety of lines on a cubic hypersurface. This means that they live in the \emph{same} Hilbert scheme, and one could naively hope that they live in the same component as well.

 We will spend some time describing this tringulation, since parts of it inspired our the construction of the Calabi--Yau's in the last chapter. We cite the definition ad verbatim from \cite{cp2_15_chess}. 

 \begin{definition}
 Let $V_4 \subset S_4$ be the Klein four group. The vertex set of $\mathcal T$ is defined as
 \begin{equation}
 V = (V_4 \backslash \{e \}) \sqcup \left( \{1,2,3,4 \} \times \{ 1,2,3 \} \right).
 \end{equation}
 Thus the vertices of $\mathcal T$ are the permutations $(12)(34)$,$(13)(24)$ and $(14)(23)$ and the pairs of integers $(a,b)$ with $1 \leq a \leq 4$ and $1 \leq b \leq 3$. The maximal faces are spanned by the sets 
 \begin{equation}
 \nu, (1,b_1), (2,b_2), (3,b_3), (4,b_4)
 \end{equation}
 with $\nu \in V_4 \backslash \{ e \}$ and $1 \leq b_a \leq 3$ ($a=1,2,3,4$) such that $b_{\nu(a)} \neq b_a$ for $a=1,2,3,4$.
 \end{definition}

See \cref{sec:compute_gaifullin} for a SAGE \cite{sagemath} script for computing the maximal facets of $\mathcal T$.

The triangulation $\mathcal T$ is the union over the cones over three $3$-spheres $S_j$ (the cone over $S_j$ is the ball $B_j$ in the definition of an equlibrium triangulation). Each $S_j$ is a very simple $3$-sphere. It is the join of two hexagons (recall that $S^1 \ast S^1 \approx S^3$).

We compute some deformation-theoretic invariants of $\P(\mathcal T)$, the Stanley--Reisner scheme associated to $\mathcal T$.

\begin{lemma}
We have that $\dim_\C = T^1_{A(\mathcal T),0}=90$ and $\dim_\C = T^2_{A(\mathcal T),0}=306$. The normal sheaf has $\mathcal N_{\P(\mathcal T)/\P^{14}}$ has $300$ global sections.
\end{lemma}
The proof is a computation in \MM. We remark that since $\P(\mathcal T)$ is not Cohen--Macaulay, some standard comparison theorems does not hold. In our case we only have an inclusion $T^i_{A(\mathcal T)} \hookrightarrow T^1_{\P(\mathcal T)}$ (see the article of Kleppe \cite{kleppe_deformations} and Theorem 3.9).

\subsection{Kühnel's 9-vertex triangulation}



\section{Naïve attempt to degenerate}

On the other hand, degenerating the ideal of $F_1(X) \subset \P^N$ to a square-free monomial ideal should give a triangulation of $\C \P^2$. Since $F_1(X)$ sits inside $\Gr(1,5)$, and there are many known degenerations of $\Gr(1,5)$, we hoped that maybe $F_1(X)$ would generate inside $\Gr(1,5)$. Unfortunaly, we did not succed, mainly because we could not see any structure in the ideal of $F_1(X)$.

\begin{enumerate}
	\item Conversely, a smoothing of a triangulation of $\C \P^2$ would give a potentially new hyper-Kähler family.
	\item Results: computed $F_1(X)$ for some hypersurfaces, both pfaffian and non-pfaffian.
	\item Also looked at deformations of a specific 15-vertex triangulation of $\C \P^2$. It deforms to the union of three toric varieties (which are cones over joins of del Pezzo surfaces). Probably not smoothable. $T^2$ is very big.
\end{enumerate}