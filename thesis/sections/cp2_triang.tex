\chapter{Relation to triangulations of \texorpdfstring{$\C\P^2$}{CP2}}
\label{sec:cp2triangs}

This chapter will not contain any new results of any signifance, but is rather a report on an idea which led to the deliberations in the later chapters.

We explain a connection between the topological space $\C \P^2$ and hyper-Kähler manifolds.

\section{Introductory remarks} % (fold)

\subsection{Hyper-Kähler manifolds}
\label{sec:hyper_kähler_manifolds}

Among the known families of manifolds, hyper-Kähler manifolds are among the most elusive. One often divides manifolds into three types: those with positive, negative or trivial canonical class. Of those with trivial canonical class, two prominent types stand out: Calab--Yau-manifolds and hyper-Kähler manifolds.

\begin{definition}
A \emph{hyper-Kähler manifold} $X$ is a simply connected compact Kähler manifold such that $H^0(X, \Omega_X^2)$ is generated by a non-degenerate $\sigma: TX \times TX \to \C$.
\end{definition}

\begin{remark}
Since the two-form $\sigma$ is non-degenerate, it follows that the canonical sheaf $\omega_X=\Omega_{X/\C}^n$ is trivial. The map $1 \mapsto \sigma^{n/2}$ gives an isomorphism $\OO_X \to \omega_X$. 
\end{remark}

For example, in dimension two, K3 surfaces are hyper-Kähler (\emph{and} Calab--Yau). Because of the non-degeneracy of the symplectic form $\sigma \in H^0(X, \Omega_X^2)$, hyper-Kähler manifolds only occur in even dimensions. Only a few explicit families of hyper-Kähler manifolds are known. Below we sketch the construction of one such family.

Let $S$ be a K3-surface with symplectic form $\sigma$, and let $S^{(2)}$ be its symmetric square: $S \times S / \{ (p,q) \simeq (q,p) \}$. Let $\pi_i:S \times S \to S$ be the two projections ($i=1,2$). Then the 2-form $\pi_1^\ast \sigma + \pi_2^\ast \sigma$ is $\Z/2$-invariant, hence it decends to a 2-form  $\tau$ on $S^{(2)}$.

The space $S^{(2)}$ is singular along the diagonal: locally it is isomorphic to $\C \times \C(/ x \simeq -x)$. The last factor is a quadric cone, so a single blowup along the diagonal will resolve the singularities. The form $\tau$ lifts to a non-degenerate form on $S^{[2]}$, and it can be shown that it is in fact a hyper-Kähler variety of dimension $4$. The resulting space is denoted by $S^{[2]}$, and is called the \emph{Hilbert square of $S$}, or the \emph{Hilbert scheme of two points on $S$}. It parametrizes length two subschemes of $S$.

For more details on this construction, see Beauville's original paper \cite{beauville_hyperkahler}.


\subsection{The variety of lines on a cubic fourfold}

There is another construction of hyper-Kähler varieties that is interesting to us. Let $X$ be a smooth cubic fourfold in $\P^5$. Let $F(X)$ denote the set of lines contained in $X$. It is the \emph{Fano variety of lines on $X$}, and is a closed subset of the Grassmannian $\mathbb G(1,\P^5)$. One can can show that $F(X)$ is a hyper-Kähler variety of dimension $4$.


In the article \cite{beauville_donagi_fano}, Beauville and Donagi shows that $F(X)$ is deformation equivalent to $S^{[2]}$ for some K3 surface $S$. They also show that if $X$ is a \emph{pfaffian} hypersurface, then $F(X)$ is actually \emph{isomorphic} to $S^{[2]}$ for some K3 surface $S$.



\section{A special triangulation of complex projective space}



\begin{enumerate}
	\item The Fano variety of lines $F_1(X)$ on a cubic fourfold is a hyper-Kähler. A monomial degeneration of this one would correspond to a triangulation of $\C \P ^2$.
	\item Reason: degenerated K3s correspond to triangulations of spheres, $S^2$. For pfaffian $X$, $F_1(X)$ is isomorphic to $S^{[2]}$, where $S$ is K3. Then since
	$$
	\C \P^2 = S^2 \ast S^2,
	$$
	and that $F_1(X) = S^{[2]}$, it is \emph{plausible} that a degeneration of $F(X)$ will give a triangulation of $S^2 \ast S^2$. 


	\item Conversely, a smoothing of a triangulation of $\C \P^2$ would give a potentially new hyper-Kähler family.
\end{enumerate}