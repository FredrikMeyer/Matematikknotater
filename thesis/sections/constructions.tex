\chapter{Construction of \CY's}
\label{sec:constructions}

In this chapter I describe the construction of three topologically different smoothings of a singular Calabi-Yau manifold. They correspond to different components of the Hilbert scheme of threefolds in $\P^{11}$ with Hilbert polynomial $p(t)=6t^3+6$. 

We first describe a degenerate \CY $X_0$ in the form of a Stanley--Reisner scheme $\P(\K)$, which has a quite large symmetry group.

We show that $X_0$ has several topologically distinct smoothings $X_i$ (${i=1,2,3}$), which lie on different components of the Hilbert scheme in $\P^{11}$.

Furthermore, in the last section, we propose mirror candidates for two of the constructions, based on orbifolding.

%%%%%%
\section{The special fiber}

Let $E_6$ be the hexagon as a simplicial complex. The associated Stanley--Reisner scheme $\P(E_6)$ is a degenerate elliptic curve in $\P^5$. If $\P^5$ have coordinates $x_0,\ldots,x_5$, the equations of $\P(E_6)$ are $x_ix_{i+2}=x_ix_{i+3}=0$, where $i$ is taken modulo $6$. This gives a total of $9$ quadratic equations.

\begin{lemma}
The Hilbert polynomial of $\P(E_6)$ is $h(t)=6t$.
\end{lemma}
\begin{proof}
We want to count the dimension of $S_{\P(E_6)}$ in degree $t$. Any monomial in $S_{\P(E_6)}$ has support on the simplicial complex $E_6$, so its support is either a vertex or an edge. In the first case, the monomial has the form $x_i^t$, so there are six of these.

In the other case, it has the form $x_i^ax_{i+1}^b$, with $a+b=t$ and $a,b \neq 0$. Counting, there are $6(t-1)$ of these monomials. In total, the dimension is $6+6(t-1)=6t$.
\end{proof}
\begin{remark}
Alternatively, we could note that $\P(E_6)$ smooths to an elliptic curve of degree $6$. Since Hilbert polynomials are constant in flat families, it follows from the Riemann--Roch theorem that \[h(t)=\deg \OO_{\P(E_6)}(6t)-1+1=6t.\]
\end{remark}

Note that the Hilbert polynomial only differ from the Hilbert function for $t=0$, since $h(0)=0$, while $\dim_\C {(S_{\P(E_6)})}_0=1$. 

We now introduce the central fiber in the discussions onward. Let $\K$ be the simplicial complex $E_6 \ast E_6$. It is a triangulation of the $3$-sphere.

Denote the  vertices of the left $E_6$ by $x_1,\ldots,x_6$, and the vertices of the right $E_6$ by $z_1,\ldots,z_6$. Then the maximal faces of $\K$ are of the form $x_ix_{i+i}z_jz_{j+1}$, where $i,j \in \Z_6$. The number of $i$-faces are easy to compute:

\begin{lemma}
The $f$-vector of $\K$ is $(12,48,36)$. 
\end{lemma}
\begin{proof}
There are $12$ vertices, and $6 \times 6=36$ maximal facets. Since $\K$ is a 3-sphere, it follows that $12-f_1+36=\chi(S^3)=0$ so that $f_1=48$.\footnote{Here we used that in a cell complex, the Euler characteristic is also the alternating sum of the number of cells in each dimension. This is Theorem 2.44 in \cite{hatcher_topology}.}
\end{proof}

\begin{lemma}
The Hilbert polynomial of $\P(\K)$ is $h(t)=6t^3+6$.
\end{lemma}
\begin{proof}
The homogeneous coordinate ring $S_{\P(\K)}= \bigoplus_{t \geq 0} S_t$ of $\P(\K)$ is the graded tensor product of $S_{\P(E_6)}$ with itself. It follows from the previous lemma that
\[
\dim S_t = \sum_{i+j=k, ij \neq 0} 36ij + 12k,
\]
where the last term is a correction term because $h(t) \neq 1$. It is now a routine computation using formulas for sums of squares to verify the claim.
\end{proof}

\begin{corollary}
Any smoothing of $\P(\K)$ satisfy $|H|=12$, $c_2 \cdot H = 72$, and $H^3=36$.
\end{corollary}
\begin{proof}
All these invariants can be read off from the Hilbert polynomial.
\end{proof}

Either by using \MM or by using the more combinatorial description of the $T^i$-modules from \cite{deforming_christophersen}, we can compute: 

\begin{proposition}
We have that
\begin{align*}
\dim_k T^1(S_{\P(\K)}/k,S_{\P(\K)})_0 &= 84 \\
\dim_k T^2(S_{\P(\K)}/k,S_{\P(\K)})_0 &= 72. \\
\end{align*}
\end{proposition}
\begin{proof}
We will prove this using the techniques and notation from \cite{deforming_christophersen}. Our goal is to compute the degree zero part of $T^1_{A_\K}$. We will do this using \cref{thm:t1dims}.

First notice that all links of vertices of $\K=E_6 \ast E_6$ are double suspensions over hexagons (they are denoted by $\Sigma E_6$ in Christophersen's article). 

According to Table 1 in Christophersen's article, double suspensions over hexagons contribute with one dimension to $T^1_{A_\K}$, namely in degree $x_i^2/x_{i-1}x_{i+1}$ (if $\mathbf a =x_i^2$). In total there are $6+6=12$ contributions of this form.

Taking the link at the vertex $x_iz_j$ produces a square with vertices $x_{i+1}$,$z_{j+1}$,$x_{i-1}$,$z_{j-1}$ (in that order). According to Table 1 in Christophersen's article, these links contribute with dimension $2$ to $T^1_{A_\K}$. The contributions have degrees $x_iz_j/x_{i+1}x_{i-1}$ and $x_iz_j/z_{j+1}z_{j-1}$. There are $2 \cdot 6 \cdot 6=72$ contributions of this form.

Thus, in total, $T^1_{A_\K}$ have $\C$-dimension $84$.

We now compute $T^2_{A_\K}$. The contributions come from choosing $\mathbf a=x_i^2$ and $\mathbf a=x_ix_{i+1}$, respectively. If $|a|=1$ (as in the first case), the results from Christophersen's article imply that $L_b := \cap_{b' \subset b} \lk(b',\lk(x_i,\K))$ must have more than one connected component (the contribution comes from $\widetilde H^0(L_b,\C)$). This is the case if $b$ consist of two opposite vertices in the suspended circle. In total there are $2 \cdot 6 \cdot 3=36$ contributions of this form. 

If $|a|=2$, the contributing links are hexagons, and in this case the contributions come from $b$ such that $L_b=\emptyset$. Again choosing $b$ to consist of opposite vertices of the hexagon, we find three pairs $b$ with $L_b=\emptyset$ for each hexagon. Thus in total there are $2 \cdot 6 \cdot 3=36$ contributions of this form.

In sum, $T^2_{A_\K}$ is $36+36$-dimensional.
\end{proof}

The automorphism group of $\K$ is $D_6 \times D_6 \times \Z_2$, and of order $12 \cdot 12 \cdot 2=288$. It is not difficult to see that the induced action on the basis of $T^1(S_{X_0}/k,S_{X_0})$ have two orbits under $\Aut(\K)$, corresponding to first order deformations of the form $x_ix_{i-2}+t x_{i+1}z_j$ and $x_{i-1}x_{i+1}+tx_i^2$, respectively. 

%%%%%%%%%%%%%%%%
\section{A natural toric deformation}

\begin{figure}[b]
\centering
\includestandalone[width=0.3\textwidth]{./figures/hexagon}
\caption{A hexagon.}
\label{fig:hexagon}
\end{figure}

Consider \cref{fig:hexagon}. It is the $2$-dimensional polytope associated to the del Pezzo surface of degree $6$. The fan over this polytope correspond to a unimodular regular triangulation of the polytope, and it follows by Theorem 8.3 in \cite{sturmfels}, that $\dP6$ degenerates to the Stanley--Reisner scheme $\P(E_6 \ast \{pt\})$, where $\{ pt \}$ correspond to the origin. Concretely, the equations of $\dP6$ are given by $x_ix_{i+2}-yx_{i+1}=x_ix_{i+3}-y^2=0$ inside $\P^6$. The degeneration to $\P(E_6 \ast \{ pt \})$ is given by setting the second terms to zero.

Now form the join of two copies of $\dP6$, to get a new variety $Y \subset \P^{13}$. By \cref{lemma:join}, this is a $2+2+1=5$-dimensional toric variety with singular locus consisting of two copies of $\dP6$. Since the coordinate ring is just the tensor product of two copies of $S(\dP6)$, it follows that $Y$ degenerates to $\P(E_6 \ast \{pt\} \ast E_6 \ast \{ pt \})=\P(\K \ast \Delta^1)$.

The following holds:

\begin{proposition}
There is a deformation of the Stanley--Reisner scheme $X_0$ to an irreducible \CY variety $X_Y \subset Y$ with isolated singularities. There are twelve of them, and they are locally isomorphic to cones over del Pezzo surfaces. More precisely: let $(U,p_i)$ be the germ of $X_0$ at $p_i$. Then $(U,p_i) \simeq (C(\dP6),0)$.
\end{proposition}
\begin{proof}
Since $X_0$ is a complete intersection side $\P(\K \ast \Delta^1)$, it follows that $X_0$ deforms to a complete intersection inside any deformation of $\P(\K \ast \Delta^1)$. We explained above that $\P(\K \ast \Delta^1)$ deforms to the join $Y$ of two del Pezzo surfaces, and it follows that $X_0$ deforms to $Y$ intersected with two generic hyperplanes.

Since $Y$ has singular locus of dimension $2$ and degree $6+6=12$, it follows by Bertini's theorem \cite[Chapter II, Theorem 8.18]{hartshorne} that $X_0$ has twelve isolated singularities $p_i$.

To see how the singularities look locally, we argue as follows. Locally, $Y$ looks like $\A^2_{a_1,a_2} \times C(\dP6)_{x_i}$, where the subscripts refer to the coordinates. This is the ideal of $Y$ consists of two sets of equations, each defining a smooth toric variety, and smooth toric varieties are isomorphic to $\A^d$ in affine  charts.

The claim now follows from two applications of Theorem 3.1.5 in \cite{batyrev_mirrorsymmetry}, which says that the singularities on $\Sigma$-regular toric hypersurfaces are inherited from the ambient toric variety.
\end{proof}

Since the cone over $\dP6$ deforms in two topologically different ways, we might expect that $X_Y$ does this too. This is indeed true.

%%%%%%%%%%
\section[Smoothings of XY]{Smoothings of $X_Y$}

By embedding $\dP6$ in different spaces, we obtain different smoothings of subvarieties of the join of these spaces.

\subsection{The block matrix construction}

We are inspired by the construction in Rødland's thesis \cite{rodland_pfaffian}\footnote{Rødland's construction is a linear subvariety of $\P(E \wedge E)$, where $E$ is $7$-dimensional.}.

Let $E$ be a 3-dimensional vector space. Let $\{e_1,e_2,e_3\}$ be a basis for $E$. Then we can form the vector space $V=(E \otimes E) \oplus (E \otimes E)$, which has dimension $18$. Let $\P^{17}=\P(V)$. Choose coordinates $x_1,\ldots,x_{18}$ on $\P^{17}$. 

Thinking of $E \otimes E$ as $3 \times 3$-matrices, we can think of the elements of $\P^{17}$ as pairs of $3 \times 3$-matrices up to scalar, not both zero. Concretely, two pairs of matrices $(A',B')$ and $(A,B)$ are equivalent if $(A',B') = (\lambda A, \lambda B)$ for some $\lambda \in \C^\ast$.

We can also interpret $\P^{17}$ as the geometric join of $\P(E \otimes E)$ with itself. This is the set of all lines connecting pairs of $3 \times 3$-matrices.

There is a natural rational map $\pi:\P^{17} \rmap \P^8 \times \P^8$, which is the identity on coordinates, given by dividing out by the antidiagonal $\C^\ast$-action: $\lambda' \cdot (A,B) = (\lambda',{\lambda'}^{-1} B)$.

Denote by $V_1$ and $V_2$ the subspaces $x_1=\ldots=x_9=0$ and $x_{10}=\ldots=x_{18}=0$, respectively. Blow up $\P^{17}$ in $V_1 \cup V_2$, to get $\widetilde{\P^{17}}$. The spaces $V_i$ are exactly the indeterminacy locus of $\pi$, so $\pi$ extends to a map $\pi:\widetilde{\P^{17}} \to \P^8 \times \P^8$. Denote by $\pi_1$ and $\pi_2$ the two natural projections to $\P^8$. Then it is true that $\widetilde{\P^{17}}= \P_{\P^8 \times \P^8}(\pi_1^\ast \OO_{\P^8}(1) \oplus \pi_2^\ast \OO_{\P^8}(1))=\P(\OO_{\P^8\times \P^8} \oplus \OO_{\P^8 \times \P^8}(1,-1))$. This is explained further in Section C7 in \cite{altman_joins}.

Let $M$ be the closure of the set of pairs $(A,B)$ where $\rank A = \rank B = 1$.  

\begin{proposition}
\label{prop:m}
The variety $M$ is the join of two copies of $\P^2 \times \P^2 \subset \P^8$, and has singular locus $\P^2 \times \P^2 \subset V_i$ of dimension $4$.

The canonical sheaf is $\omega_M = \OO_M(-6)$, so that $M$ is a Fano toric variety.
\end{proposition}

\begin{proof}
If $\P^{17}$ have coordinates $x_1,\ldots,x_{18}$, let $M_1$ and $M_2$ be the matrices
\[
M_1 = \begin{pmatrix}
x_1 & x_2 & x_3 \\
x_4 & x_5 & x_6 \\
x_7 & x_8 & x_9 
\end{pmatrix}\,
\text{ and }
M_2 = \begin{pmatrix}
x_{10} & x_{11} & x_{12} \\
x_{13} & x_{14} & x_{15} \\
x_{16} & x_{16} & x_{17}
\end{pmatrix}.
\]

Then $M$ is defined by the zeroes of the $2 \times 2$-minors of $M_1$ and $M_2$. Then it is clear that $M$ is the projective join of two copies of $\mathbb P^2 \times \mathbb P^2 \hookrightarrow \mathbb P^8 \subset \P^{17}$, since the sets of variables are disjoint.

The variety $M$ is $9$-dimensional: the affine cone over $M$, $C(M)$, is equal to $C(\mathbb P^2 \times \P^2) \times C(\mathbb P^2 \times \P^2)$. This variety has dimension $5+5=10$, hence its projectivization $M$ is $9$-dimensional. 

The singular locus of $M$ consists of the pairs $(0,B)$, and $(A,0)$, where $\rank A= \rank B = 1$, hence $\dim \Sing M = \dim \left( \P^2 \times \P^2 \right) = 4$. See also \cref{lemma:join}.

By Remark \ref{remark:canonical}, it follows that $\omega_M = \OO_M(-6)$, since 
\[
\omega_{\P^2 \times \P^2} = \restr{\OO_{\P^8}(-3)}{\P^2 \times \P^2}.
\]
\end{proof}

Here comes our first construction. Let $X_1$ be the intersection of $M$ with a generic $\P^{11}$. Then the following is true.

\begin{proposition}
\label{prop:x1}
$X_1$ is a smooth Calabi--Yau variety with $\chi(X_1)=-72$.
\end{proposition}
\begin{proof}
The singularities of $M$ are of dimension $4$. By Bertini's theorem, intersecting $M$ with a codimension $6$ hyperplane gives a smooth variety $X_1$.

That $X_1$ is Calabi--Yau follows from \cref{prop:anticanonicalsection}.

To find the topological Euler characteristic, we compute in \MM. Computing the whole cotangent sheaf of $X_1$ is infeasible with current computer technology\footnote{An external computer has been trying to compute this sheaf for several months now without terminating.}, we make use of standard exact sequences. Let $\mathscr I$ be the ideal sheaf of $M$ in $\P^{17}$. First off, we have the exact sequence
$$
0 \to \restr{\mathscr I/\mathscr I^2}{X_1} \to \restr{\Omega_{\P^{17}}^1}{X_1} \to \restr{\Omega_M^1}{X_1} \to 0.
$$

The restriction to ${X_1}$ is exact since $\mathscr I/\mathscr I^2$ is locally free on the smooth locus.

The \MM command \texttt{eulers} computes the Euler characteristics of generic linear sections of a sheaf $\mathscr F$ (behind the scene, this is equivalent to computing the Koszul resolution of the relative ideal sheaf $\mathscr I_{X_1/M}$). Using this command, we find that $\chi(\restr{\mathscr I/\mathscr I^2}{X_1})=-180$. Using the exact sequence
$$
0 \to \restr{\Omega_{\P^{17}}^1}{X_1} \to \OO_{X_1}(-1)^{18} \to \OO_{X_1} \to 0,
$$
we find that the Euler characteristic of $\restr{\Omega_{\P^{17}}^1}{X_1}$ is $-216=12\cdot 18$. It follows from the first exact sequence that $\restr{\Omega_M^1}{X_1}$ has Euler characteristic $-36$.

Since $X_1$ is a complete intersection in $M$, the conormal sequence is
$$
0 \to \OO_{X_1}(-1)^6 \to \restr{\Omega_M}{X_1}  \to \Omega_{X_1}^1 \to 0.
$$
Hence $\chi(\Omega_X^1) = -36+72 = 36$.

It follows that the topological Euler characteristic is $\chi(X_1) \stackrel \Delta = \chi(\mathcal T_{X_1}) = -2\chi(\Omega_{X_1}^1)=-72$.
\end{proof}

\begin{remark}
\label{remark_X1}
We can give explicit equations for a flat family with special fiber $X_Y$ and general fiber $X_1$. Let $y_0=h_1(x_1,\ldots,x_{12})$ and $y_1=h_2(x_1,\ldots,x_{12})$ be the generic linear forms in $\P^{13}$ defining $X_Y$ as a subscheme of $Y$. Let $g_i$ (for $i=1,\ldots,6$) be generic linear forms in $\P^{11}$. Then such a flat family is defined by the $2 \times 2$-minors of the two matrices below:

\[
A_1 = \begin{pmatrix}
h_1 + tg_1  & x_2 & x_3 \\
x_4 & h_1+tg_2  & x_6 \\
x_7 & x_8 & h_1+tg_3 
\end{pmatrix}\,
\text{and }
A_2 = \begin{pmatrix}
h_2+tg_4 & x_{11} & x_{12} \\
x_{13} & h_2+tg_5 & x_{15} \\
x_{16} & x_{16} & h_2+tg_6
\end{pmatrix}.
\]

For $t=0$, we get $X_Y$. Note that the subscheme defined by the minors of 
\[
A_1 = \begin{pmatrix}
y_1  & x_2 & x_3 \\
x_4 & y_1  & x_6 \\
x_7 & x_8 & y_1
\end{pmatrix}\,
\text{and }
A_2 = \begin{pmatrix}
y_2 & x_{11} & x_{12} \\
x_{13} & y_2 & x_{15} \\
x_{16} & x_{16} & y_2
\end{pmatrix}
\]
is the join of $\dP6$ with itself. Since $X_Y \subset \dP6 \ast \dP6$, we see that $X_1$ lies in a deformation of $\dP6 \ast \dP6$.
\end{remark}

\begin{remark}
I have not been able to rigorously compute the Hodge nubmers of $X_1$. However, over several finite fields I have computed the dimension of the degree zero part of the $T^1$ module in \MM. By \cref{prop:t1h1}, we have that $(T^1(S_{X_1}/k,S_{X_1}))_0 = H^1(X_1,\Omega^2_X)$. This module can be computed in \MM.

After about a week of computation on a modern desktop computer, the answer turns out to be $\dim_{\mathbb F_p} (T^1(S_{X_1}/\mathbb F_p,S_{X_1}))_0 = 39$ for several large primes $p$.

This is plausible because of the following heuristic moduli count: $X_1$ is parametrized by the Grassmannian $\mathbb Gr(12,(E \otimes E)^{\oplus 2} )$, which has dimension $(18-12)\cdot 12 = 72$. Each $E$-factor is acted upon by $\GL(E)$. There are four of these factors, so we have an action of $\prod^4 \GL(E)$ on the Grassmannian. There is a torus subgroup $(\C^\ast)^4$ acting by $(v \otimes w, r \otimes s) \mapsto (t_1t_2 v \otimes w, t_3t_4 r \otimes s)$ on $\P^{17}$. Elements of $(\C^\ast)^4$ satisfying $t_1t_2=t_3t_4$ act trivially, forming a isotropy subgroup $K$.  Hence we have an action of the quotient group $G \stackrel \Delta = \left(\prod_{i=1}^4 \GL(4)\right)/K$ on the Grassmannian. This quotient group has dimension $9 \cdot 4 - 3 = 33$.

We form the quotient space $\mathbb G(12,(E \otimes E)^{\oplus 2})/G$, which have dimension $72-33=39$.

If this is true, then $X_1$ has Hodge numbers $h^{11}=3$ and $h^{12}=39$, since we have computed the Euler characteristic. It is not clear which other divisors there are besides the hyperplane divisor.
\end{remark}

\begin{remark}
Since $X_1$ avoids the fundamental subscheme $V_1 \cup V_2$, the inverse image $\pi^{-1}(X_1) \subset \widetilde{\P^{17}}$ is isomorphic to $X_1$. Thus we can realize $X_1$ as a subvariety of a \emph{smooth} variety. Unfortunaly, $X_1$ is cut out by non-ample divisors in $\widetilde{\P^{17}}$.
\end{remark}

%%%%%%%%%%%%
%%%%%%%%%%%%
\subsection{The three-tensor construction}

The construction in the previous section used the embedding of $\dP6$ in $\P^2 \times \P^2$ to deform $X_Y$. There is also the embedding of $\dP6$ in $\P^1 \times \P^1 \times \P^1$ to exploit. The construction is similar.

Let $F$ be a $2$-dimensional vector space with basis $\{f_1,f_2\}$. Then we can form the vector space $V = (F \otimes F \otimes F)^{\oplus 2}$. Let $\P^{15}=\P(V)$. Choose coordinates $a_{ijk}=(f_i \otimes f_j \otimes f_j,0)$ and $b_{ijk}=(0,f_i\otimes f_j \otimes f_k)$ ($i,j,k=0,1$) for $\P^{15}$. 

The elements of $\P^{15}$ are pairs $(A,B)$ of $2 \times 2 \times 2$ tensors, not both zero. There is also in this case a natural map $\pi:\P^{15} \to \P^7 \times \P^7$, given by dividing out by the antidiagonal $\C^\ast$-action. Just as above, let $V_1$ and $V_2$ be the subspaces $A=0$ and $B=0$, respectively. Let $\widetilde{\P^{15}}$ be the blowup of $\P^{15}$ in $V_1 \cup V_2$. The $V_i$'s are exactly the indeterminacy locus of $\pi$, so $\pi$ extends to a morphism $\pi:\widetilde{\P^{15}} \to \P^8 \times \P^8$, which is a $\P^1$-bundle. Also in this case it is true that $\widetilde{\P^{15}}=\P\left(\OO_{\P^7 \times \P^7} \oplus \OO_{\P^7 \times \P^7}\left(1,-1\right)\right)$.

Let $N$ be the closure of set of pairs $(A,B)$ where both $A$ and $B$ have tensor rank $1$\footnote{An element of $F^{\otimes 3}$ have rank $1$ if it is a pure tensor. It has rank $\leq k$ if it can be written as a sum of $k$ pure tensors.}.

\begin{proposition}
The variety $N$ is the join of two copies of $\P^1 \times \P^1 \times \P^1 \subset \P^7$, and has singular locus $\P^1 \times \P^1 \times \P^1 \subset V_i$ of dimension $3$.

The canonical sheaf is $\omega_N = \OO_N(-4)$, so that $N$ is a Fano toric variety.
\end{proposition}

\begin{proof}
A pure $2 \times 2 \times 2$-tensor can be visualized as a cube with vertices $a_{ijk}$. See the diagram in \vref{fig:222tensor}. 

\begin{figure}[t]
\centering
\includestandalone{./figures/222tensor}
\caption{A $2 \times 2 \times 2$-tensor, seen from ``above''.}
\label{fig:222tensor}
\end{figure}

The equations of the set of rank $1$ tensors in $\P(F \otimes F \otimes F)$ are obtained as the ``minors'' along the $6$ sides of the cube, together with the minors along with the $3$ long diagonals, giving a total of $9$ binomial equations. We write this symbolically as $[a_{ijk}] \leq 1$. 

Hence the equations for $N$ are given by $[a_{ijk}] \leq 1$, together with $[b_{ijk}] \leq 1$. Since these are equations in a disjoint set of variables, it is clear that $N=\left(\P^1 \times \P^1 \times \P^1\right)^{\ast 2}$.

The claim about the singular locus and the canonical sheaf follow as in the proof of \cref{prop:m}.
\end{proof}

Let $X_2$ be the intersection of $N$ with a general $\P^{11}$.

\begin{proposition}
\label{prop:x2}
$X_2$ is a smooth Calabi--Yau variety with $\chi(X_2)=-48$.
\end{proposition}
\begin{proof}
The proof is identical to the proof of \cref{prop:x1}.
\end{proof}

\begin{remark}
A heuristic moduli count works also in this case.

$X_2$ lies in a $\P^{11}$ in $\P((F \otimes F \otimes F)^{\oplus 2})$. Such planes are parametrized by $\Gr(12,16)$, the Grassmannian of $12$-planes in $k^{16}$. This space is $12 \cdot (16-12)=48$-dimensional. There is an action of the group $\prod_{i=1}^6 \GL(F)$ on $(F \otimes F \otimes F)^{\oplus 2}$. There is also in this case a torus subgroup acting trivially. Namely, the elements satisfying $t_1t_2t_3=t_4t_5t_6$. Call this subgroup $K$. Thus we really have an action of the group $\left(\prod_{i=1}^6 \GL(F)\right)/K$, which have dimension $6 \cdot 4 - 5 = 19$. Thus in total we have $48-19=29$ moduli parameters.

Since we know the Euler characteristic, we predict the Hodge numbers to be $(h^{11},h^{12})=(5,29)$.
\end{remark}

\subsection{The mixed smoothing}

In the above cases, we formed the join of equal varieties. We mix things up: let $V=(E \otimes E) \oplus (F \otimes F \otimes F)$. Then let $\P^{16}=\P(V)$.

Now let $W$ be the set of ``mixed'' rank $1$ tensors. In a way similar to above, we find that $W$ is a singular Fano toric variety of dimension $8$. The singular locus is of dimension $4$, so a $5$-fold complete intersection is again a smooth Calabi-Yau variety $X_3$.

\begin{proposition}
\label{prop:x3}
$X_3$ is a smooth Calabi--Yau variety with $\chi(X_3)=-60$.
\end{proposition}
\begin{proof}
The proof is identical to the proofs above.
\end{proof}

\begin{remark}
We again give a heuristic moduli count. The Grassmannian in this case is $60$-dimensional. The group acting on it is $\prod^2_{i=1} \GL(E) \times \prod^3_{i=1} \GL(F)$. Here the trivially acting torus subgroup will be those satisfying $t_1t_2=t_3t_4t_5$. It follows the the parameter space is $60-(18+12-4)=34$-dimensional.

Hence we predict the Hodge numbers to be $(h^{11},h^{12})=(4,34)$.
\end{remark}



%%%%%%%%%%%
\section{Degeneration of the toric join}
\label{sec:degenjoin}

Consider the construction of $X_1$ from above, and the explicit equations from \cref{remark_X1}. Putting $t=0$ and $h_1=h_2$, gives a degeneration of $X_Y$ to another, more singular, variety $X_{Y'}$. Explicitly, it is given by the $2 \times 2$-minors of the following two matrices, where $h$ is a generic hypersurface in the variables.

\[
A_1 = \begin{pmatrix}
h   & x_2 & x_3 \\
x_4 & h  & x_6 \\
x_7 & x_8 & h
\end{pmatrix}\,
\text{and }
A_2 = \begin{pmatrix}
h & x_{11} & x_{12} \\
x_{13} & h & x_{15} \\
x_{16} & x_{16} & h
\end{pmatrix}.
\]

We can realize $X_{Y'}$ as a hypersurface in a toric variety as follows. Introduce a new variable $y$, and consider the  variety defined by the $2 \times 2$ minors of
\[
A_1 = \begin{pmatrix}
y   & x_2 & x_3 \\
x_4 & y  & x_6 \\
x_7 & x_8 & y
\end{pmatrix}\,
\text{and }
A_2 = \begin{pmatrix}
y & x_{11} & x_{12} \\
x_{13} & y & x_{15} \\
x_{16} & x_{16} & y
\end{pmatrix}.
\]

This is a $4$-dimensional toric variety. It is the toric variety associated to the polytope $\Delta$ with vertices the columns of the matrix
\[
 \makeatletter\c@MaxMatrixCols=12\makeatother\begin{pmatrix}{-1}&      1&      0&      1&      0&      {-1}&      0&      0&      0&      0&      0&      0\\      0&      0&      {-1}&      {-1}&      1&      1&      0&      0&      0&      0&      0&      0\\      0&      0&      0&      0&      0&      0&      {-1}&      1&      0&      1&      0&      {-1}\\      0&      0&      0&      0&      0&      0&      0&      0&      {-1}&      {-1}&      1&      1\\      \end{pmatrix}.
\]

A computation shows that $Y'$ has $1$-dimensional singularities, and the singular locus is a graph of $\P^1$'s: take two hexagons, and join each vertex of one of them with all vertices of the other one. This makes in total $48$ $\P^1$'s.

The variety $Y'$ is a Fano toric variety, and as such, it has a anticanonical section $X_{Y'}$ which is a singular Calabi--Yau variety. A local computation shows that $X_{Y'}$ has $12$ singularities that are locally isomorphic to $C(\dP6)$, and $36$ double points. This can also be seen torically: the cones in the fan of $Y'$ corresponding to the singular locus comes in two types. The first type is a cone over a hexagon, and the other type is the cone over a square. These correspond to (algebro-geometrically) cones over $\dP6$ and double points, respectively.

Since $Y'$ is a four-fold, it follows that $X_{Y'}$ has a maximal projective crepant resolution of singularities (a \emph{MPCP-desingularization}) (see for example the comment on page 55 in \cite{mirrorsymmetry}), which we denote by $\widetilde{X_{Y'}}$.

A computation using PALP \cite{palp} shows that $\widetilde{X_{Y'}}$ has Hodge numbers $(44,8)$ and Euler-characteristic $72$. 

\begin{remark}
There is a heuristic surgical reason for the Euler characteristic being $+72$. Our $X_{Y'}$ deforms to $X_1$, which has Euler characteristic $-72$. This is obtained by smoothing $36$ double points and $12$ cones over del Pezzo surfaces. By the inclusion-exclusion principle, it follows that a small resolution of the singularities of $X_{Y'}$ have Euler characteristic $\chi(X_1)+2\cdot 36 + 6 \cdot 12=72$.
\end{remark}

\begin{remark}
The variety $X_{Y'}$ has also been described elsewhere. The polar polytope $\Delta^{\circ}$ is equal to the product of two hexagons, and it follows that $\P_{\Delta^\circ}$ is equal to the product of two del Pezzo surfaces. An anticanonical hypersurface in $\dP6 \times \dP6$ has Euler characteristic $-72$ (see Theorem 3.1 in \cite{bestiary_hubsch}).

In the article \cite{braun_smallhodgenumbers}, Braun et al. study this hypersurface and a group action on it. They also describe, in detail, a resolution of singularities of $X_{Y'}$.
\end{remark}

\begin{remark}
In \cite{candelas_newcy}, the authors study \CY complete intersections admitting free actions by finite groups, and certain transitions between them (these are similar to Morrison's extremal transitions). They find that there is a Calabi--Yau with Hodge numbers $(3,39)$ and a Calabi--Yau with Hodge numbers $(8,44)$ beloning to the same family of transitions, both admitting $\Z/3$-actions. It is not clear to us if their $(3,39)$-manifold is the same as our $X_1$.
\end{remark}

%%%%%%%%%%%%%%%
%%%%%%%%%%%%%%%
\section{Invariant \CY's and a mirror construction}

In this section, I will explain natural group actions on the $X_i$'s constructed above. Using the mirror construction Ansatz from above, we propose mirror candidates for $X_1$ and $X_2$.

\subsection{Invariant subfamily of $X_1$} 

Let us first consider $M=(\P^2 \times \P^2)^{\ast 2}$. Recall that $M$ can be thought of as pairs of rank one $3 \times 3$ matrices up to scalar. We will describe several natural finite group actions on $M$.

There is a natural $\Z/3$-action on $M$, defined as follows. If $E$ is a $3$-dimensional vector space with basis $\{e_0,e_1,e_2\}$, then we can define $e_i \mapsto \omega^i e_i$, where $\omega^i$ is a third root of unity. This action extends to an action on $E \otimes E$ by the rule $e_{ij} \mapsto \omega^{i+j} e_{ij}$\footnote{We write $e_{ij}$ for $e_i \otimes e_j$.}. Furthermore, it extends to an action on $E \otimes E \oplus E \otimes E$ by $(v,w) \mapsto (gv,gw)$. Call the generator for this group for $g$. 

There is also a non-toric permutation action defined as follows. Let $\Z/3 \subset S_3$ be the cyclic permutation action on $\{ e_0,e_1,e_2 \}$ defined by $e_i \mapsto e_{i+1}$. Again, we get an action on $E \otimes E$ by $e_{ij} \mapsto e_{i+1,j+1}$, and by extension an action on $E \otimes E \oplus E \otimes E$. Call the generator of this group for $\sigma$. 

Furthermore, there is a $\Z/2$-action switching the $E \otimes E$-factors. Call the generator for this group for $\tau$.

All these groups commute up to a scalar, so we get a $\Z/3 \times \Z/2 \times \Z/2$-action on $\P(E \otimes E \oplus E \otimes E)$. Let $G$ be the abelian group generated by $g$ and $\sigma$. Let $G'$ be the group generated by $g, \sigma$ and $\tau$. 

For the $G$-action to restrict to $X_1=M \cap H$, we must choose $H$ to be invariant under the group action. Here we describe a family of $G$-invariant $\P^{11}$'s: denote a unit matrix in the first factor of $(E \otimes E) \oplus (E \otimes E)$ by $e_{ij}^0$, and denote a unit matrix in the second factor by $e_{ij}^1$, where $0,1$ are taken modulo $2$. 

Now consider the $H_t=\P^{11}$ spanned by the following matrices:
\begin{equation}
\label{eq:fija}
f_{ij}^\alpha = e_{ij}^\alpha + t_{i-j}^\alpha e_{-i-j,-i-j}^{\alpha+1} \in E \otimes E \oplus E \otimes E,
\end{equation}

where $i \neq j \in \Z_3$ and $\alpha \in \Z_2$, and $t_{i+j}^\alpha$ is a parameter. Note that $g \cdot f_{ij}^\alpha = \omega^{i+j} f_{ij}^\alpha$, so that $H$ is spanned by eigenvectors of the $\Z/3$-action. This gives us a $4$-parameter family of $G$-invariant planes. However, multiplying all the $t_{i-j}^\alpha$ by the same number yield isomorphic families, so we really have a $3$-parameter family.

Denote the intersection between $M$ and $H$ by $X_{H_t}$. Denote by $P_i$ the coordinate points $(0:\ldots:1:\ldots:0)$. Then $\Z/3$ act without fixed points outside these points (this can be computed in \MM). A \MM computation also shows that for $t_i \neq 1,0$, the family has $48$ isolated singularities: the $P_i$, and $36$ other points, which come in two orbits under the $G$-action. These are all double points, which can be verified by local computations.

\begin{lemma}
There exists a minimal resolution of $X_{H_t}$ ($t \neq 0,1$), respecting the group action by $G$, leaving the dualizing sheaf trivial.
\end{lemma}
\begin{proof}
Analytically, a small resolution is a local operation. The singularities come in $3$ orbits under the action, so it is enough to do the resolution on one singularity in each orbit.

Since the singularity is small, the change happen in codimension $2$. The holomorphic $3$-form on $X_{H_t}$ extends holomorphically to all of the resolution by Hartog's theorem.
\end{proof}

\begin{lemma}
After resolving the double points as above, the action of $g$ has $24$ fixed points on $\widetilde{X_{H_t}}$, two on each of the $\P^1$'s of the initial fixed points.

Furhermore, the resolution has Euler characteristic $24$.
\end{lemma}
\begin{proof}
To see that the $g$-action has two fixed points on the $\P^1$s coming from the initial fix points, we find local equations of $X_{H_t}$. This is done in \MM. By writing the equations of $X_{H_t}$ as $x_iu+g=0$, where $u$ is a unit locally around each fixed point, we can eliminate the variable $x_i$ locally. Doing this repeatedly, we end up with a single local equation for $X_{H_t}$: (we're now looking in the chart where $x_1 \neq 0$)
\[
x_{10}x_{11}-x_8x_{12} + (\text{higher order terms}).
\]
The coordinates of the corresponding $\P^1$ are given by (up to flops):
\[
[z_0:z_1] = [x_{10}:x_8] = [x_{11}:x_{12}].
\]
The action of $g$ on the $x_i$ are given by $g \cdot x_8 = \omega^2 x_8$, $g \cdot x_{10} = x_{10}$, $g \cdot x_{11} = \omega^2 x_{11}$ and $g \cdot x_{12} = x_{12}$. This makes $ g \cdot [z_0:z_1] = [z_0:\omega^2 z_1]$, which has two fixed points (the points $[1:0]$ and $[0:1]$).

Similar local equations are given in the eleven other charts.

The Euler characteristic of a small resolution is given by $\chi(\widetilde{X_{H_t}}) = \chi(X_t) + 2s$, where $s$ is the number of double points, and $X_t$ is a smooth member of a smooth smoothing family of $X_{H_t}$ (which we know exists by construction, and is $X_1$ from above). There are $48$ double points, so it follows that the Euler characteristic is $-72+2 \cdot 48 = 24$.
\end{proof}

These resolutions are still Calabi--Yau manifolds. One reference for this fact is \cite{clemens_double}. 

Let $\Z/3$ denote the torus subgroup group acting on $\widetilde{X_{H_t}}$. 

\begin{theorem}
Let $X_1^\circ$ be a minimal resolution of $\widetilde{X_{H_t}}/(\Z/3)$. Such a resolution exists, and it has Euler-characteristic $+72$, making it a potential mirror for $X_1$.
\end{theorem}
\begin{proof}
The existence of this kind of quotient singularity is proved in Roan's article \cite{roan_minres}. Furthermore, in his article \cite{roan_euler}, he proves a formula for the Euler characteristic of such resolutions (let $V=\widetilde{X_{H_t}})$:

\[
\chi\left(\widetilde{V/(\Z/3)} \right) = \frac{1}{3} \sum_{g,h \in \Z/3}  \chi(V^g \cap V^h).
\]

For $(g,h) \neq (e,e)$, $\chi(V^g \cap V^h)$ is just the finite set of of fixed points. There are $24$ of these. For $(g,h)=(e,e)$, $\chi(V^e \cap V^e)=\chi(V)$ is the Euler characteristic of the resolution of $X_{H_t}$, which is $24$.

In sum, we find

$$
\chi(X_1^\circ) = \frac 13 \left(24 + 8 \cdot 24\right)=72.
$$
\end{proof}

\begin{remark}
We still have the cyclic permutation action $\sigma$. Since $\sigma$ commutes (up to scalar) with $g$, it act on the mirror as well. It can be checked that it has no fixed points on $X_{H_t}$. Thus the induced $\Z/3$-action is free, and we can form the quotiens $X_{H_t}/\langle \sigma \rangle $ and $X_{H_t}^\circ/\langle \sigma \rangle$. These will have Euler characteristics $24$ and $-24$, respectively. However, the fundamental group will be non-trivial.
\end{remark}

\subsection{Invariant subfamily of $X_2$}

Also in this case we are able to produce a mirror candidate. We start by describing natural group actions on $N$, and then describe a natural invariant subfamily. 

Recall that $F$ is a $2$-dimensional vector space with basis $f_0, f_1$. There is, as above, a natural $\Z/2$-action given by $f_i \mapsto (-1)^i f_i$. Concretely, $\Z/2$ act by sending $f_0$ to itself and multiplying $f_1$ by $-1$. This action extend in the natural way to an action on $\P(F^{\otimes 3} \oplus F^{\otimes 3})$.

Furhermore, there is another $\Z/2$-action given by $f_i \mapsto f_{i+1}$ (indices taken modulo $2$).

Using the same notation as in the previous section, define $K_t$ to be $\P^{11}$ 
spanned by the following matrices:

\begin{equation}
\label{eq:gijk}
g_{ijk}^\alpha = e_{ijk}^\alpha + t_{i,j,k} e_{i+j+k,i+j+k,i+j+k}^{\alpha+1}
\end{equation}

for $(i,j,k)\neq (0,0,0),(1,1,1)$ and $\alpha=0,1$. Thus these matrices span a $\P^{11}$. Then, as above, the $g_{ijk}^\alpha$ are eigenvectors for the $\Z/2$-action.

For $t_{i,j,k}=1$ for all $i,j,k$ the variety $X_{K_t} \stackrel \Delta =  N \cap K_t$ has $36$ double points. Using the same arguments as in the previous section, it follows that a small resolution of $X_{K_1}$ has Euler-characteristic $24$ as well. Again, using Roan's formula, we find that a small resolution of the quotient $X_{K_1}$ has Euler-characteristic $+48$. Thus we have a mirror candidate for $X_2$ as well.

\begin{proposition}
There exists a mirror candidate for $X_2$ as well. More precisely, there exists a \CY desingularization $X_2^\circ $of the quotient $\widetilde {X_{K_t}}/H$ in such a way that the Hodge numbers satisfy $\chi(X_2) = -\chi(X_2^\circ) = -48$.
\end{proposition}

%%%%%%%%%%%%%%%%%%%%
%%%%%%%%%%%%%%%%%%%%
\subsection{Comment about $X_3$}

The same mirror construction does not work for $X_3$, at least not directly. In the case of $X_1$ and $X_2$, there were natural torus actions on $E$ and $F$,  respectively. This action extendend to $\P\left(\left( E \otimes E\right)^{\oplus 2}\right)$ and $\P\left(\left(F \otimes F \otimes F\right)^{\otimes 2}\right)$, and we intersected with invariant $\P^{11}$s to get special Calabi--Yaus.

In the case of $X_3$, there is no natural torus action on the ambient projective space coming from the join factors, so the same construction does not apply.

%%%%%%%%%%%%%%%%%%%%
%%%%%%%%%%%%%%%%%%%%
\section{Conclusion and further questions}

In this final chapter we constructed several smooth Calabi--Yau manifolds. Three of them, $X_1,X_2$ and $X_3$ lie in the same flat family. They are all smoothings of $X_Y$, a complete intersection in a 5-dimensional toric variety $Y$. This $X_Y$ has a maximally crepant resolution of singularities which is a smooth Calabi--Yau. We constructed mirror candidates and finite group quotiens of $X_1$ and $X_2$.

We end with a few open questions that we would like to see answered in the future.


The Calabi--Yau with Hodge numbers $(44,8)$ in \cref{sec:degenjoin} seem to have some connection with our $X_1$. Its mirror dual $X_{8,44}$ is a complete intersection in $\P^2 \times \P^2 \times \P^2 \times \P^2$, while $X_1$ is a complete intersection in $(\P^2 \times \P^2) \ast (\P^2 \times \P^2)$ with the same Euler characteristic. There seem to be some kind of duality going on, which is unfortunaly not described (to my knowledge) in the literature.

We have a morphism $\pi:X_1 \to \P^2 \times \P^2 \times \P^2 \times \P^2$ defined by $(v \otimes w, r \otimes s) \mapsto v \otimes w \otimes r \otimes s$. The morphism is generically $1-1$. I have not been able to see what the image is (or if the morphism is an isomorphism).

The same situation occurs with $X_2$. Here there is a morphism ${\pi:X_2 \to (\P^1)^{\times 6}}$. I don't know what the image is. Also here there should be a connection with $X_{8,44}$, since $X_{8,44}$ also can be realized as a complete intersection in $(\P^1)^{\times 6}$. See the introduction of \cite{braun_smallhodgenumbers}.

We have a morphism $\pi:X_1 \to \P^2 \times \P^2 \times \P^2 \times \P^2$ defined by $(v \otimes w, r \otimes s) \mapsto v \otimes w \otimes r \otimes s$. The morphism is generically $1-1$. I have not been able to see what the image is (or if the morphism is an isomorphism).

It would also be interesting to find proofs of the Euler characteristics being $-72,-60$ and $-48$ not involving computer calculations. In all cases the Grassmannian parametrizing the $X_i$ have dimension $72$, $60$, and $48$. 

Assuming that my calculation of the Hodge numbers of the $X_i$ are correct, what are representatives of the generators of $\Pic X_i$? (being $\Z^3$, $\Z^5$ and $\Z^4$, respectively)

Can my construction via joins be generalized to produce other (potentially new) Calabi--Yau varieties?