\chapter{Construction of \CY's}
\label{sec:constructions}

In this chapter I describe the construction of three topologically different smoothings of a singular Calabi-Yau manifold. They (should) correspond to different components of the Hilbert scheme of threefolds with Hilbert polynomial $p(t)=6t^3+6$ in $\P^{11}$. 

We first describe a degenerate \CY in the form of a Stanley-Reisner scheme $\P(\K)$, which has a quite large symmetry group. There is a natural deformation to a $X_0$, which is a hypersurface inside toric variety, with isolated singularities.

We show that $X_0$ has several topologically distinct smoothings, which should lie on different components of the Hilbert scheme in $\P^{11}$.

\section{The mirror construction Ansatz} % (fold)
\label{sec:the_mirror_construction_ansatz}

In many cases of interest, given a construction of a Calabi--Yau manifold, the following Ansatz produces a mirror.

Let $\K$ be a simplicial complex, with associated Stanley--Reisner scheme $X_0$. Let $G$ be the automorphism group of $\K$. Then $G$ induces an action on $T_{X_0}^1$ in the following way: each element of $T_{X_0}^1$ can be represented by a $\phi \in \Hom(I/I^2,A)$, and then $g \cdot \phi$ is given by $(g \cdot \phi)(f) = g \cdot \phi(g^{-1} \cdot f)$, the contragredient action.

There is an action of $T_n=(\C^\ast)^{n+1}/\C^\ast$ on $\P^n$, and since $I_{X_0}$ is generated by monomials, the action restricts to an action on $X_0$ as well.

Given a smoothing family with general fiber $X$ and special fiber $X_0$, we can consider a subfamily with only isolated singularities on which $G$ act.  Let $H \subset T_n$ be the subgroup of the torus acting on this family. Then the mirror candidate to $X$ is given by a crepant resolution of $Y_t=X_t/H$.

Though it is often overlooked (or stated in another language) in the literature, even the mirror construction of the famous quintic arises this way. Briefly, the quintic Calabi--Yau is given by the zero locus of a general quintic in $H^0(\P^4,\OO_{\P^4}(5))$. The special quintic given by $f=x_0x_1x_2x_3x_4$ is the Stanley--Reisner scheme corresponding to the 3-simplex. The automorphism group is $S_5$, and an invariant $1$-parameter family is given by $f_t=\sum_{i=0}^4 x_i^5 + t x_0 x_1x_2x_3x_4$. There is a $H=(\Z/5)^5/\Z^5$-action on $X_t = Z(f_t)$ given by coordinate-wise multiplication by fifth roots of unity. The general element of the family $X_t$ is smooth, so the only singularities of the quotient $Y_t=X_t/H$ comes from points with non-trivial stabilizer. These can be resolved by methods from toric geometry. For details, see the first chapter of Ingrid Fausk's thesis \cite{fausk_thesis}.

% section the_mirror_monstruction_ansatz (end)

\section{The special fiber}

Let $E_6$ be the hexagon as a simplicial complex. The associated Stanley--Reisner scheme $\P(E_6)$ is a degenerated elliptic curve in $\P^5$. If $\P^5$ have coordinates $x_0,\ldots,x_5$, the equations of $E_6$ are $x_ix_{i+2}=x_ix_i+3=0$, where $i$ is taken modulo $6$. This gives a total of $9$ quadratic generators.

\begin{lemma}
The Hilbert polynomial of $\P(E_6)$ is $h(t)=6t$.
\end{lemma}
\begin{proof}
We want to count the dimension of $S_t=S_{E_6}(t)$. Any monomial in $S_k$ has support on the simplicial complex $E_6$, so its support is either a vertex or an edge. In the first case, the monomial has the form $x_i^t$, so there are six of these.

In the other case, it has the form $x_i^ax_{i+1}^b$, with $a+b=t$ and $a,b \neq 0$. Counting, there are $6(t-1)$ of these monomials. In total, the dimension is $6+6(t-1)=6t$.
\end{proof}
\begin{remark}
Alternatively, we could note that $\P(E_6)$ smooths to an elliptic curve of degree $6$. Since Hilbert polynomials are constant in flat families, it follows from Riemann--Roch that $h(t)=\deg \OO_{\P(E_6)}(6t)-1+1=6t$.
\end{remark}

Note that the Hilbert polynomial only differ from the Hilbert function for $t=0$, since $h(0)=0$, while $\dim_\C S_0=1$. 

We now introduce the central fiber in the discussions onward. Let $\K$ be the simplicial complex $E_6 \ast E_6$. It is a triangulation of the $3$-sphere. The maximal faces are unions of maximal faces from each factor.

Denote the  vertices of the left $E_6$ with $x_1,\ldots,x_6$, and the vertices of the right $E_6$ with $z_1,\ldots,z_6$. Then the maximal faces of $\K$ are of the form $x_ix_{i+i}z_jz_{j+1}$, where $i,j \in \Z_6$. The number of $i$-faces follows are easy to compute:

\begin{lemma}
The $f$-vector of $\K$ is $(1,12,48,36)$. 
\end{lemma}
\begin{proof}
There are $12$ vertices, and $6 \times 6=36$ maximal facets. Since $\K$ is a 3-sphere, it follows that $12-f_1+36=\chi(S^3)=0$ that $f_1=48$.\footnote{Here we used that in a cell complex, the Euler characteristic is also the alternating sum of the number of cells in each dimension. This is Theorem 2.44 in \cite{hatcher_topology}.}
\end{proof}

\begin{lemma}
The Hilbert polynomial of $\P(\K)$ is $h(t)=6t^3+6$.
\end{lemma}
\begin{proof}
The homogeneous coordinate ring $S=\oplus_{t \geq 0} S_t$ of $\P(\K)$ is the graded tensor product of $\P(E_6)$ with itself. It follows from the previous lemma that
\[
\dim S_t = \sum_{i+j=k, ij \neq 0} 36ij + 12k,
\]
where the last term is a correction term because $h(t) \neq 1$. It is now a routine computation using formulas for sums of squares to verify the claim.
\end{proof}

Either by using \MM or by using the more conceptual description of the $T^i$ modules from \cite{deforming_christophersen}, we can compute: 

\begin{lemma}
The dimensions of $T^1(\K)$ and $T^2(\K)$ are $84$ and $72$, respectively.
\end{lemma}
\begin{proof}
We will prove this using the techniques and notation from \cite{deforming_christophersen}. Our goal is to compute the degree zero part of $T^1_{A_\K}$. We will do this using \cref{thm:t1dims}.

First notice that all links of vertices of $\K=E_6 \ast E_6$ are double suspensions over hexagons (they are denoted by $\Sigma E_6$ in Christophersen's article). 

According to Table 1 in Christophersen's article, double suspensions over hexagons contribute with one dimension to $T^1_{A_\K}$, namely in degree $x_i^2/x_{i-1}x_{i+1}$ (if $\mathbf a =x_i^2$). In total there are $6+6=12$ contributions of this form.

Taking the link at the vertex $x_iz_j$ produces a square with vertices $x_{i+1}$,$z_{j+1}$,$x_{i-1}$,$z_{j-1}$ (in that order). According to Table 1 in Christophersen's article, these links contribute with dimension $2$ to $T^1_{A_\K}$. The contributions have degrees $x_iz_j/x_{i+1}x_{i-1}$ and $x_iz_j/z_{j+1}z_{j-1}$. There are $2 \cdot 6 \cdot 6=72$ contributions of this form.

Thus, in total, $T^1_{A_\K}$ have $\C$-dimension $84$.

We now compute $T^2_{A_\K}$. The contributions come from choosing $\mathbf a=x_i^2$ and $\mathbf a=x_ix_{i+1}$, respectively. If $|a|=1$ (as in the first case), the results from Christophersen's article imply that $L_b := \cap_{b' \subset b} \lk(b',\lk(x_i,\K))$ must have more than one connected component (the contribution comes from $\widetilde H^0(L_b,\C)$). This is the case if $b$ consist of two opposite vertices in the suspended circle. In total there are $2 \cdot 6 \cdot 3=36$ contributions of this form. 

If $|a|=2$, the contributing links are hexagons, and in this case the contributions come from $b$ such that $L_b=\emptyset$. Again consisthoosing $b$ to consist of opposite vertices of the hexagon, we find three pairs $b$ with $L_b=\emptyset$ for each hexagon. Thus in total there are $2 \cdot 6 \cdot 3=36$ contributions of this form.

In sum, $T^2_{A_\K}$ is $36+36$-dimensional.
\end{proof}

The automorphism group of $\K$ is $D_6 \times D_6 \times \Z_2$, and of order $12 \cdot 12 \cdot 2=288$. It is not difficult to see that the induced action on $T_{X_0}^1$ have two orbits under $\Aut(\K)$, corresponding to deformations of the form $x_ix_{i-2}+t x_{i+1}z_j$ and $x_{i-1}x_{i+1}+tx_i^2$, respectively. 

%%%%%%%%%%%%%%%%
\section{A natural toric deformation}

\begin{figure}[b]
\centering
\includestandalone[width=0.3\textwidth]{./figures/hexagon}
\caption{A hexagon.}
\label{fig:hexagon}
\end{figure}

Consider \cref{fig:hexagon}. It is the $2$-dimensional polytope associated to the del Pezzo surface of degree $6$. The fan over this polytope correspond to a unimodular regular triangulation of the polytope, and it follows by \cref{eq:unimodular_triangs}, that $\dP6$ degenerates to the Stanley--Reisner scheme $\P(E_6 \ast \{pt\})$, where $\{ pt \}$ correspond to the origin. This is an embedded deformation inside $\P^6$.

Concretely, the equations of $\dP6$ are given by $x_ix_{i+2}-yx_{i+1}=x_ix_{i+3}-y^2=0$. The degeneration to $\P(E_6 \ast \{ pt \})$ is given by setting the second terms to zero.

Now form the join of two copies of $\dP6$, to get a new variety $Y \subset \P^{13}$. By \cref{lemma:join}, this is a $2+2+1=5$-dimensional toric variety with singular locus consisting of two copies of $\dP6$. Since the coordinate ring is just the tensor product of two copies of $S(\dP6)$, it follows that $Y$ degenerates to $\P(E_6 \ast \{pt\} \ast E_6 \ast \{ pt \})=\P(\K \ast \Delta^1)$.

The following holds:

\begin{proposition}
There is a deformation of the Stanley--Reisner scheme $X_0$ to an irreducible \CY variety $X_Y \subset Y$ with isolated singularities. There are twelve of them, and they are locally isomorphic to cones over del Pezzo surfaces. More precisely: let $(U,p_i)$ be the germ of $X_0$ at $p_i$. Then $(U,p_i) \simeq (C(\dP6),0)$.
\end{proposition}
\begin{proof}
Since $X_0$ is a complete intersection side $\P(\K \ast \Delta^1)$, it follows that $X_0$ deforms to a complete intersection inside any deformation of $\P(\K \ast \Delta^1)$. We explained above that $\P(\K \ast \Delta^1)$ deforms to the join $Y$ of two del Pezzo surfaces, and it follows that $X_0$ deforms to $Y$ intersected with two generic hyperplanes.

Since $Y$ has singular locus of dimension $2$ and degree $6+6=12$, it follows by Bertini's theorem \cite[Chapter II, Theorem 8.18]{hartshorne} that $X_0$ has twelve isolated singularities $p_i$.

To see how the singularities look locally, we argue as follows. Locally, $Y$ looks like $\A^2_{a_1,a_2} \times C(\dP6)_{x_i}$, where the subscripts refer to the coordinates. This is the ideal of $Y$ consists of two sets of equations, each defining a smooth toric variety, and smooth toric varieties are isomorphic to $\A^d$ in affine  charts.

The claim now follows from two applications of Theorem 3.1.5 in \cite{batyrev_mirrorsymmetry}, which says that the singularities on $\Sigma$-regular toric hypersurfaces are inherited from the ambient toric variety.
\end{proof}

Since the cone over $\dP6$ deforms in two topologically different ways, we might expect that $X_Y$ does this too. This is indeed true.


%%%%%%%%%%
\section[Smoothings of XY]{Smoothings of $X_Y$}

By embedding $\dP6$ in different spaces, we obtain different smoothings of subvarieties of the join of these spaces.

\subsection{The block matrix construction}

We are inspired by the construction in Rødland's thesis \cite{rodland_pfaffian}. 

Let $E$ be a 3-dimensional vector space. Let $\{e_1,e_2,e_3\}$ be a basis for $E$. Then we can form the vector space $V= (E \otimes E) \oplus (E \otimes E)$, which has dimension $18$. Let $\P^{17}=\P(V)$. Choose coordinates $x_1,\ldots,x_{18}$ on $\P^{17}$. 

Thinking of $E \otimes E$ as $3 \times 3$-matrices, we can think of the elements of $\P^{17}$ as pairs of $3 \times 3$-matrices up to scalar, not both zero. Concretely, two pairs of matrices $(A',B')$ and $(A,B)$ are equivalent if $(A',B') = (\lambda A, \lambda B)$ for some $\lambda \in \C^\ast$.

There is a natural rational map $\pi:\P^{17} \rmap \P^8 \times \P^8$, which is the identity on coordinates, given by dividing out by the antidiagonal $\C^\ast$-action: $\lambda' \cdot (A,B) = (\lambda',{\lambda'}^{-1} B)$.

Denote by $V_1$ and $V_2$ the subspaces $x_1=\ldots=x_9=0$ and $x_10=\ldots=x_18=0$, respectively. Blow up $\P^{17}$ in $V_1 \cup V_2$, to get $\widetilde{\P^{17}}$. The spaces $V_i$ are exactly the indeterminacy locus of $\pi$, so $\pi$ extends to a map $\pi:\widetilde{\P^{17}} \to \P^8 \times \P^8$. Denote by $\pi_1$ and $\pi_2$ the two natural projections to $\P^8$. Then it is true that $\widetilde{\P^{17}}= \P_{\P^8 \times \P^8}(\pi_1^\ast \OO_{\P^8}(1) \oplus \pi_2^\ast \OO_{\P^8}(1))=\P(\OO_{\P^8\times \P^8} \oplus \OO_{\P^8 \times \P^8}(1,-1))$. This is explained further in Section C7 in \cite{altman_joins}.


Let $M$ be the closure of the set of pairs $(A,B)$ where $\rank A = \rank B = 1$.  

\begin{proposition}
The variety $M$ is the join of two copies of $\P^2 \times \P^2 \subset \P^8$, and has singular locus $\P^2 \times \P^2 \subset V_i$ of dimension $4$.

The canonical sheaf is $\omega_M = \OO_M(-6)$, so that $M$ is a Fano toric variety.
\end{proposition}

\begin{proof}
If $\P^{17}$ have coordinates $x_1,\ldots,x_{18}$, let $M_1$ and $M_2$ be the matrices
\[
M_1 = \begin{pmatrix}
x_1 & x_2 & x_3 \\
x_4 & x_5 & x_6 \\
x_7 & x_8 & x_9 
\end{pmatrix}\,
\text{ and }
M_2 = \begin{pmatrix}
x_{10} & x_{11} & x_{12} \\
x_{13} & x_{14} & x_{15} \\
x_{16} & x_{16} & x_{17}
\end{pmatrix}.
\]

Then $M$ is defined by the zeroes of the $2 \times 2$-minors of $M_1$ and $M_2$. Then it is clear that $M$ is the projective join of two copies of $\mathbb P^2 \times \mathbb P^2 \hookrightarrow \mathbb P^8 \subset \P^{17}$, since the variable sets are disjoint.

The variety $M$ is $9$-dimensional: the affine cone over $M$, $C(M)$, is equal to $C(\mathbb P^2 \times \P^2) \times C(\mathbb P^2 \times \P^2)$. This variety has dimension $5+5=10$, hence its projectivization $M$ is $9$-dimensional. 

The singular locus of $M$ consists of the pairs $(0,B)$, and $(A,0)$, where $\rank A= \rank B = 1$, hence $\dim \Sing M = 4$. See also \cref{lemma:join}.

By Remark \ref{remark:canonical}, it follows that $\omega_M = \OO_M(-6)$, since $\omega_{\P^2 \times \P^2} = \restr{\OO_{\P^8}(-3)}{\P^2 \times \P^2}$.
\end{proof}

By Bertini's theorem, intersecting $M$ with a codimension $6$ hyperplane gives a smooth variety $X_1$.

Note that by putting $x_1=x_5=x_6$ and $x_{10}=x_{14}=x_{17}$, we get the join of two del Pezzos, so we see that $X_1$ deforms to $X_0$, which is a singular Calabi--Yau variety. It follows that $X_1$ is a smooth Calabi--Yau. This also follows from the adjunction formula: $\omega_{X_1} = \omega_M \otimes \wedge^6 \OO_X(1) = \OO_X$.

A \MM computation give us some information about the geometry of $X_1$.

\begin{proposition}
\label{prop:x1euler} 
$X_1$ has topological Euler characteristic $-72$.
\end{proposition}
\begin{proof}
This is a computation in \MM. Since computing the whole cotangent sheaf of $X_1$ is impossible with current computer technology, we make use of standard exact sequences. Let $\mathscr I$ be the ideal sheaf of $M$ in $\P^{17}$. First off, we have the exact sequence
$$
0 \to \restr{\mathscr I/\mathscr I^2}{X} \to \restr{\Omega_\P^1}{X_1} \to \restr{\Omega_M^1}{X_1} \to 0.
$$
The \MM command \texttt{eulers} computes the Euler characteristics of generic linear sections of a sheaf $\mathscr F$. Using this command, we find that $\chi(\restr{\mathscr I/\mathscr I^2}{X_1})=-180$. Using the exact sequence
$$
0 \to \restr{\Omega_\P^1}{X_1} \to \OO_{X_1}(-1)^{18} \to \OO_{X_1} \to 0,
$$
we find that the Euler characteristic of $\restr{\Omega_\P^1}{X_1}$ is $-216=12\cdot 18$. It follows from the first exact sequence that $\restr{\Omega_M^1}{X_1}$ has Euler characteristic $-36$.

Since $X_1$ is a complete intersection in $M$, the conormal sequence looks like
$$
0 \to \OO_{X_1}(-1)^6 \to \restr{\Omega_M}{X_1}  \to \Omega_{X_1}^1 \to 0.
$$
Hence $\chi(\Omega_X^1) = -36+72 = 36$.

It follows that the topological Euler characteristic is $\chi_{X_1} = -2\chi(\Omega_{X_1}^1)=-72$.
\end{proof}

I have not been able to compute the Hodge nubmers of $X_1$. However, counting parameters, we can give a conjectural size of $H^1(X,\mathcal T_{X_1})$, which is the space of complex structures on $X_1$:

$X_1$ lies in a family parametrized by planes containing twelve $3 \times 3$ block matrices (spanning the $\P^{11}$), giving $12 \cdot 18=216$ parameters. Conjugation by invertible $3 \times 3$ block matrices produce isomorphic varieties, reducing the amount of parameters by $3^2 + 3^2=18$. Furthermore, rotation in $\P^{11}$ reduces the dimension by $H^0(\P^{11},\mathcal T)=12^2-1=143$. In total, we have $216-18-143=55$ complex parameters. Since the Euler characteristic is $-72$, heuristically, the Hodge numbers should be $(19,55)$. 

%%%%%%%%%%%%
\subsection{The three-tensor construction}

Now let $F$ be a $2$-dimensional vector space with basis $\{f_1,f_2\}$. Then we can form the vector space $V = ((F \otimes F \otimes F)^{\oplus 2})$. Let $\P^{15}=\P(V)$.

The elements of $\P$ are pairs $(A,B)$ of $2 \times 2 \times 2$-tensors, not both zero. 

Let $N$ be the closure of set of pairs $(A,B)$ where both $A$ and $B$ have tensor rank $1$\footnote{An element of $F^\otimes 3$ have rank $1$ if it is a pure tensor. It has rank $k$ if it can be written as a sum of $k$ pure tensors.}. A pure $2 \times 2 \times 2$-tensor can be visualized as a box in $\Z^3$ of unit volume. Let the variables on $\P^{15}$ be $a_{ijk}$ and $b_{ijk}$ for $i,j,k=0,1$. See the diagram in \vref{fig:222tensor}.

\begin{figure}[t]
\centering
\includestandalone{./figures/222tensor}
\caption{A $2 \times 2 \times 2$-tensor.}
\label{fig:222tensor}
\end{figure}

The equations of the set of rank $1$ tensors are obtained as the ``minors'' along the $6$ sides together with the minors along with the $3$ long diagonals, giving a total of $9$ binomial equations. 

Note that $N$ is the projective join of two copies of $\P^1 \times \P^1 \times \P^1$. As above, it follows that $N$ is a singular Fano variety with anticanonical sheaf equal to $\OO_N(4)$. 

The singular locus of $N$ consists of the pairs $(A,0)$ and $(0,B)$ where both $A,B$ have rank $1$. Hence the singular locus is of dimension $3$.

Intersecting $N$ with a codimension $4$-hyperplane gives a smooth variety $X_2$. It is Calabi-Yau and has topological Euler characteristic $-48$.

\begin{proposition}
The topological Euler characteristic of $X_2$ is $-48$.
\end{proposition}
\begin{proof}
The proof is identical to the proof of \cref{prop:x1euler}.
\end{proof}

\begin{remark}
Using the same heuristic as with $X_1$, we conjecture the Hodge numbers to be $(h^{11},h^{12})=(9,33)$. 
\end{remark}

\subsection{The mixed smoothing}

In the above cases, we formed the join of equal varieties. In the above notation, let $V=(E \otimes E) \oplus (F \otimes F \otimes F)$. Then let $\P^{16}=\P(V)$.

Now let $W$ be the set of ``mixed'' rank $1$ tensors. In a way similar to above, we find that $W$ is a singular Fano toric variety of dimension $8$. The singular locus is of dimension $4$, so a $5$-fold complete intersection is again a smooth Calabi-Yau variety $X_3$.

\begin{proposition}
The Euler characteristic of $X_3$ is $-60$.
\end{proposition}
\begin{proof}
The proof is identical to the proofs above.
\end{proof}

\begin{remark}
The heuristic above gives $(h^{11},h^{12})=(18,48)$. 
\end{remark}



%%%%%%%%%%%
\section{Degeneration of the toric join}

The coordinate ring of the toric variety $Y$ can be described as $R = S \otimes_k S$, where $S$ is the coordinate ring of the del Pezzo surface of degree $6$ in its anticanonical embedding. $S$ has a special element $y_0$ corresponding to the origin of the defining polytope. Concretely, $R$ can be described as follows: it is a quotient of the polynomial ring $\C[x_1,\ldots,x_6,y_0,z_1,\ldots,z_6,y_1]$. We then mod out by the ideal $I_1 + I_2$, where $I_i$ ($i=1,2$) is the ideal of the del Pezzo surface in the variables ($x_i,y_0$ and $z_i,y_1$, respectively). Then $Y=\Proj(\C[x_1,\ldots,x_6,y_0,z_1,\ldots,z_6,y_1]/(I_1+I_2)$.

Consider the hypersurface $Y'$ in $Y$ given by $y_0=y_1$. Then $Y'$ is a $4$-dimensional toric variety. It is the toric variety associated to the polytope $\Delta$ with vertices the columns of the matrix
\[
 \makeatletter\c@MaxMatrixCols=12\makeatother\begin{pmatrix}{-1}&
      1&
      0&
      1&
      0&
      {-1}&
      0&
      0&
      0&
      0&
      0&
      0\\
      0&
      0&
      {-1}&
      {-1}&
      1&
      1&
      0&
      0&
      0&
      0&
      0&
      0\\
      0&
      0&
      0&
      0&
      0&
      0&
      {-1}&
      1&
      0&
      1&
      0&
      {-1}\\
      0&
      0&
      0&
      0&
      0&
      0&
      0&
      0&
      {-1}&
      {-1}&
      1&
      1\\
      \end{pmatrix}.
\]

A computation shows that $Y'$ has $1$-dimensional singularities, and the singular locus is a graph of $\P^1$'s: take two hexagons, and join each vertex of one of them with all vertices of the other one. This makes in total $48$ $\P^1$'s. \todo{illustration graph}

The variety $Y'$ is a Fano toric variety, and as such, it has a anticanonical section $X_{00}$ which is a singular Calabi--Yau variety. A local computation shows that $X_{00}$ has $12$ singularities that are locally isomorphic to $C(\dP6)$, and $36$ double points.

Since $Y'$ is a four-fold, it follows that $X_{00}$ has a maximal projective crepant resolution of singularities (a \emph{MPCP-desingularization}) (see for example the comment on page 55 in \cite{mirrorsymmetry}), which we denote by $\widetilde{X_{00}}$.

A computation using PALP \cite{palp} shows that $\widetilde{X_{00}}$ has Hodge numbers $(44,8)$ and Euler-characteristic $72$. 

\begin{remark}
The variety $X_{00}$ has also been described elsewhere. The polar polytope $\Delta^{\circ}$ is equal to the product of two hexagons, and it follows that $\P_{\Delta^\circ}$ is equal to the product of two del Pezzo surfaces. An anticanonical hypersurface in $\dP6 \times \dP6$ has Euler characteristic $-72$ (see Theorem 3.1 in \cite{bestiary_hubsch}).

In the article \cite{braun_smallhodgenumbers}, Braun et al. study this hypersurface and a group action on it. They also describe a resolution of singularities of $X_{00}$.
\end{remark}

\begin{remark}
There is a heuristic surgical reason for the Euler characteristic being $+72$. Our $X_{00}$ deforms to $X_1$, which has Euler characteristic $-72$. This is obtained by smoothing $36$ double points and $12$ cones over del Pezzo surfaces. By the inclusion-exclusion principle, it follows that a resolution of the singularities of $X_{00}$ have Euler characteristic $\chi(X_1)+2\cdot 36 + 6 \cdot 12=72$.
\end{remark}

\todo{connection to physics paper, fremheve at Euler-kar utregninger er kompatible, finn noe med +48}


%%%%%%%%%%%%%%%
\section{Invariant \CY's}

One way to produce mirror candidates of \CY manifolds is the following procedure: put $X$ in a family, and consider a subfamily with an action of a finite group $G$. The generic member $X_t$ of this subfamily often has singularities. Then one forms the quotient and try to resolve the singularities, hoping to preserve the \CY propert. The mirror candidate is then $Y \stackrel{\Delta}{=}  \widetilde{X_t/G}$. This is the approach taken by Rødland in his thesis \cite{rodland_pfaffian}.

In this section, I will explain natural group actions on the $X_i$´s constructed above.

Denote by $D_6$ the dihedral group of order $12$, the symmetries of a hexagon. It is generated by a rotation $\rho$ of order $6$, together with a reflection $\sigma$, subject to $\sigma \rho \sigma = \rho^{-1}$. There is an isomorphism $D_6 \simeq S_3 \times \Z_2$: $S_3$ is identified with $\langle \rho^2, \sigma \rangle$ and $\Z_2$ is identied with $\rho^3$.

\begin{lemma}
There are $D_6$-actions on both $M$ and $N$.
\end{lemma}
\begin{proof}
Recall that $M$ is the join of two copies of $\P^2 \times \P^2$ embedded in $\P^8$. We can think of $M$ as block matrices of rank $1+1$ in $\P(E \otimes E \oplus E \otimes E)$, where $E$ is a $3$-dimensional vector space. Choosing a basis $\{e_1,e_2,e_3\}$ for $E$, we have a natural $S_3$ action on $E$ given by $e_i \mapsto e_{\sigma(i)}$. This action extends to $E \otimes E$ by $v \oplus w \mapsto g \cdot v \oplus g \cdot w$.  

Switching the direct summands of ${\left(E \otimes E\right)}^{\oplus 2}$, gives us a $\Z_2$-action. In total we now have a $S_3 \times \Z_2$-action, which by the above remark is a $D_6$-action. Note that since the action was defined on $E$, rank is preserved, so that we indeed have an action on $M$.

Similarly, $N$ is the rank $1+1$ tensors in ${\left(E \otimes E \otimes E \right)}^{\oplus 2}$, where now $E$ is a $2$-dimensional vector space. 
\end{proof}

\todo{Torus actions}

By choosing invariant hyperplanes, the group actions on the ambient spaces descend to the \CY's. We first consider the case when the ambient space was the join of two copies of $\P^2 \times \P^2$, which was denoted by $M$ above.

Denote a unit matrix in the first factor of $(E \otimes E) \oplus (E \otimes E)$ by $e_{ij}^0$, and denote a unit matrix in the second factor by $e_{ij}^1$, where $0,1$ are taken modulo $2$. 

In this case, one such invariant hyperplane is given by the span of
$$
f_{ij}^\alpha = e_{ij}^\alpha + t e_{-i-j,-i-j}^{\alpha+1} \in E\otimes E \oplus E \otimes E,
$$
where $i \neq j \in \Z_3$ and $\alpha \in \Z_2$. Denote the intersection between $M$ and $H$ by $X_{H_t}$. Then the following is true:

\begin{proposition}
Both the finite group $D_6$ and the group $\Z_3$ act on $X_{H_t}$. The symmetric variety $X_{H_t}$ have $24$ isolated singularities for $t \neq 0,1$, and they come in two orbits under the $D_6$-action.

For $t=1$, it has $36$ isolated singularities.
\todo{Check \emph{which} singularities these are, also: fix points}
\end{proposition}

There is also a torus action on $E$, defined by $e_i \mapsto \omega^i e_i$, where $\omega$ is a third root of unity. This a $\Z/3$-action, which extends to an action on $ E\otimes E \oplus E \otimes E$. Let $H=\Z/3$. Then note that $X_{H_t}$ is $\Z/3$-invariant as well. 

\todo{identify $H$-invariant singularities + fix points}