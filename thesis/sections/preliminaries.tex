\chapter{Preliminaries}
\label{sec:prelims}

\begin{enumerate}
	\item Stanley-Reisner schemes
	\item Toric geometry notation
\end{enumerate}

\todo{joins}

Given two projective varieties $X \subset \P^N$ and $Y \subset \P^N$, one can for their \emph{projective join}. It is the closure of the union of the set of lines connecting points from $X$ and $Y$. Denote the join of $X$ and $Y$ by $X \ast Y$. 

\begin{lemma}
\label{lemma:join}
Suppose $X$ and $Y$ are smooth and lie in disjoint linear subspaces, and $\dim X=a$ and $\dim Y=b$. Then their join, $X \ast Y$ have dimension $a+b+1$. The singular locus is of dimension $\max\{a,b\}$ and consist of the disjoint union of $X$ and $Y$.
\end{lemma}
\begin{proof}
Denote the homogeneous coordinate rings of $X$ and $Y$ by $S(X)$ and $S(Y)$, respectively. I first claim that the homogeneous coordinate ring of $X \ast Y$ is the graded tensor product of $S(X)$ and $S(Y)$. 

For suppose $X \subset \P^n$ and $Y \subset \P^m$ with $\P^n \cap \P^m = \emptyset$. Then $I_X=J_X+H_X$, where $H_X$ are linear equations defining $\P^n$, and similarly $I_Y = J_Y+H_Y$, with $H_X + H_Y=\mm$. Hence $S(X)$ is isomorphic to a graded ring without any variables occuring in the equations of $Y$ (and the same statement for $S(Y)$). Call these ring for $S_X$ and $S_Y$, respectively. Then the claim is that $X \ast Y$ has coordinate ring $S_X \otimes_k S_Y$. Even more concretely, if $x_0,\ldots,x_n$ are coordinates on $\P^n$, and $y_0,\ldots,y_n$ are coordinates on $\P^m$, then the claim is that $S(X \ast Y)=k[x_0,\ldots,x_n,y_0,\ldots,y_n]/(J_X+J_Y)$. For suppose $P$ is a point in $\P^{n+m+1}$ such that $\restr{J_X+J_Y}{P}=0$. Then $P$ has to be of the form $tP_1 + sP_2$ with $P_1$ and $P_2$ satisfying the equations of $X$ and $Y$. Hence $P$ lies on a line between $X$ and $Y$.

\todo{finn velformulert forklaring på sing lokus}
\todo{svæææært klønete bevis}
\end{proof}

\section{Deformation theory}

Given a scheme $X_0$ over $\C$, a \emph{family of deformations} of $X_0$ is a flat morphism $\pi:\mathscr X \to (S,0)$ with $S$ connected such that $\pi^{-1}(0)=X_0$. If $S$ is the spectrum of an artinian $\C$-algebra, then $\pi$ is an \emph{infinitesimal deformation}. If $S=\Spec \C[\epsilon]/\epsilon^2$, then  $\pi$ is a \emph{first order deformation}. A \emph{smoothing of $X_0$} is a deformation such that the general fiber is smooth.

\section{Stanley-Reisner schemes}

\subsection{Simplicial complexes and Stanley--Reisner schemes}

Denote by $[n]$ the set $\{0,1,\ldots,n\}$, and by $\Delta_n$, the set of all subsets of $[n]$. This is the \emph{$n$-dimensional simplex}. A \emph{simplicial complex} $\K$ is a subset of $\Delta_n$ that is closed under the operation of taking subsets. The subsets of $\K$ are called \emph{faces}. A good reference is Stanley's green book \cite{stanley_green}.

Let $k$ be a field, and let $P_\K$ be the polynomial ring over $k$ with variables indexed by the vertices of $\K$. Then the \emph{face ring} or \emph{Stanley--Reisner ring} of $\K$ is the quotient ring $A_\K = P_\K/I_\K$, where $I_\K$ is the ideal generated by monomials corresponding to non-faces of $\K$. 

\begin{example}
Let $\K$ be the triangle with vertices $\{ v_1,v_2,v_3\}$. Its maximal faces are $v_1v_2, v_2v_3$ and $v_1v_3$. The Stanley--Reisner ring is $k[v_1,v_2,v_3]/(v_1v_2v_3)$.
\end{example}

The ideal $I_\K$ is graded since it is defined by monomials. This leads us to define the \emph{Stanley--Reisner scheme} $\P(\K)$ as $\Proj A_\K$. 

There is a correspondence between certain degenerations of toric varieties and so-called unimodular triangulations. \todo{define these}

Let $M$ be a lattice (by which we mean a free abelian group of finite rank). Let $\nabla \subset M_\Q = M \otimes_\Z \Q$ be a lattice polytope, and let $S_\nabla$ be the semigroup in $M \times \Z$ generated by the elements $(u,1) \in \nabla \cap M$. Then we define $\P(\nabla)=\Proj \C[S_\nabla]$, and call it the \emph{toric variety associated to $\nabla$}. 

By Theorem 8.3 and Corollary 8.9 in \cite{sturmfels}, there is a one-one correspondence between unimodular regular triangulations of $\nabla$ and the square-free initial ideals of the toric ideal of $\P(\nabla)$. 

\todo{fill in as needed}

\section{Smoothings of Stanley--Reisner schemes}

Because many properties of varieties are easier read off their degenerations, it is an interesting problem to study smoothings of Stanley--Reisner-schemes, which are highly singular.

\begin{lemma}
\label{lemma:srcohom}
If $\K$ is a simplicial complex, then $H^i(\K;k) \simeq H^i(\P(\K),\OO_{\P(\K)})$.
\end{lemma}
\todo{Find a good proof for this. A reference could be \cite{eisenbud_graphcurves}}

\begin{lemma}
If $\K$ is a 3-dimensional simplicial sphere, then a smoothing of $X_0=\P(\K)$ will be Calabi--Yau.
\end{lemma}
\begin{proof}
Since $\K$ is a sphere, it follows from \ref{lemma:srcohom} that $H^i(X_0,\OO_X)=k$ for $i=0,3$, and zero for $i \neq 0,3$. It also \todo{How does triviality of the canonical sheaf follow?}
\end{proof}

\todo{something about polyhedral complexes and face rings?}

\section{Toric geometry and toric degenerations}
\label{sec:toricgeometry}

\todo{join of reflexive polytopes is reflexive!}

\begin{proposition}
\label{eq:unimodular_triangs}
There is a 1--1 correspondence between regular unimodular triangulations of polytopes and squarefree monomial ideals.
\end{proposition}

By Theorem 8.3 and Corollary 8.9 in \cite{sturmfels}......

Thus, if we can find a polytope whose boundary has a regular unimodular triangulation, we know that the associated Stanley-Reisner ring has at least one deformation. 

\todo{reflexive polytopes, mirror symmetry, anticanonical sections}