\chapter{Preliminaries}
\label{sec:prelims}

In this chapter we introduce the consistent notation and results which will be used later. Some of the material in this chapter plays the role of motivation more than preliminary results.

%%%%% join
\section{The join of projective varieties}

There are many ways to define the join of two projective varieties $X$ and $Y$. We will define it in a particularly general way, as described by Altman and Kleiman in \cite{altman_joins}. Then we will specialize to our situation.

Fix a base scheme $S$. Let $\mathscr C$ be the category of graded, quasi-coherent $\OO_S$-algebras, generated in degree $1$. The tensor product of two $\OO_S$-algebras $\mathscr R$ and $\mathscr S$  is naturally graded: the degree $d$ part is given by
\[
(\mathscr R \otimes_{\OO_S} \mathscr S)_d = \bigoplus_{p+q=d} \mathscr R_p \otimes \mathscr S_q.
\]

Let $X=\Proj \mathscr R$ and $Y = \Proj \mathscr S$. Then we define the \emph{join} of the graded $\OO_S$-algebras to be

\[
X \ast Y \stackrel \Delta =  \Proj (\mathscr R \otimes_{\OO_S} \mathscr S).
\]

If $X$ and $Y$ are projective varieties over $S$, they come with graded $\OO_S$-algebras $\mathscr R = \Sym_S \OO_X(1)$ and $\mathscr S = \Sym_S \OO_Y(1)$. Then we define the join of $X$ and $Y$ to be join of these algebras.

The join construction is a contravariant functor in two variables from the category of graded $\OO_S$-algebras and surjective maps to the category of projective varieties.

\begin{example}
Let $X=\P(E)$ and $Y=\P(F)$, where $E,F$ are quasi-coherent $\OO_S$-modules. Then we have the equality $\P(E) \ast \P(F) = \P(E \oplus F)$, because of the linear algebra fact $\Sym(E) \otimes \Sym(F) = \Sym(E \oplus F)$.
\end{example}

The algebra $\mathscr R \otimes_{\OO_S} \mathscr S$ contains the ideal $\mathscr R \otimes \mathscr S_+$. The associated subscheme is denoted by $V_X$, and is isomorphic to $X = \Proj \mathscr R$. We define $V_Y$ similarly. We call $V_X$ and $V_Y$ the \emph{fundamental subschemes} of $X \ast Y$.

There is a geometric definition of the join, as described in setion (C11) in \cite{altman_joins}. Let $E,F$ be quasi-coherent $\OO_S$-modules\footnote{In our case, $S=\Spec k$ almost always. So $E,F$ are just vector spaces.}. Suppose $X,Y$ are subschemes of $\P(E)$ and $\P(F)$. Then $X\ast Y$ is a closed subscheme of $\P(E \oplus F)$. Identify $X$ and $Y$ with their fundamental subschemes in $X \ast Y$. Then it is not difficult to see that $X \ast Y$ is the (closure of the) locus of points lying on the lines of $\P(E \oplus F)$ determined by pairs of points of $X$ and $Y$. 

For a graded $\OO_S$-algebra $\mathscr R$ there are two natural homomorphisms. We have the \emph{structure map} $\rho: \OO_S \to \mathscr R$, and the \emph{augmentation map} $\epsilon: \mathscr R \to \OO_S$, sending everything in positive degree to zero. Clearly $\epsilon \circ \rho = \id_S$.


\begin{proposition}
\label{lemma:join}
Suppose $X/k \subset \P^n$ and $Y/k \subset \P^m$ are smooth projective schemes. Then their join, $X \ast Y$ have dimension $\dim X+\dim Y+1$. The singular locus is of dimension $\max\{a,b\}$ and consist of the disjoint union of $X$ and $Y$.
\end{proposition}
\begin{proof}
Let $\mathscr R = \bigoplus_{d \geq 0} \OO_X(d)$ and $\mathscr S = \bigoplus_{d \geq 0} \OO_Y(d)$ be the homogeneous coordinate rings of $X$ and $Y$. Then $X \ast Y \subset \P^{n+m+1}$. 

Denote by $C(X \ast Y)$ the scheme $\Spec (\mathscr R \otimes \mathscr S)$, the affine cone over $X \ast Y$. 

The singular locus of $C(X \ast Y) = C(X) \times C(Y)$ is equal to $\Sing C(X) \times C(Y) \bigcup C(X) \times \Sing C(Y)$. Since $X$ and $Y$ are smooth, the only singular point on the affine cones are the origins. Hence
$$
\Sing\left(C(X \ast Y)\right) = 0 \times \Sing(C(Y)) \bigcup \Sing(C(X)) \times 0.
$$

Projectivising, we find that $\Sing(X \ast Y) = \Sing Y \sqcup \Sing X$, since $(0,\ldots,0)$ is the only common point of the affine cones. 
\end{proof}

\todo{Sjekk definisjoner!!}
Recall that a scheme $X$ is \emph{Gorenstein} if it has a dualizing sheaf. It is \emph{Cohen--Macaulay} if the dualizing sheaf is a line bundle. If $S$ is the homogeneous coordinate ring of a projective Gorenstein variety $X \subset \P^n$, the canonical sheaf can be computed as the sheaf associated to the graded module $\Ext^{\codim X}(S,S(-n-1))$. \todo{sjekk definisjoner!!}

If $X$ and $Y$ are two Gorenstein schemes, then their join is also Gorenstein. Furthermore, we can compute the canonical sheaf in terms of the canonical sheaves of $X$ and $Y$.

\begin{proposition}
Let $X=\Proj \mathscr R$ and $Y=\Proj \mathscr S$ be Gorenstein projective schemes with dualizing sheaves $\omega_X, \omega_Y$, respectively (here $\mathscr R,\mathscr S$ are sheaves of graded $\mathscr O_S$-algebras). Then $X \ast Y$ is Gorenstein with  dualizing sheaf $\omega_X \otimes_{S} \omega_Y$.
\end{proposition}
\begin{proof}
The question is local on $S$, so we may assume that $R$ and $S$ are homogeneous coordinate rings. Then the statement follows from Theorem 4.2 in \cite{herzog_canonical}, where the authors prove that the canonical module of a tensor product is the tensor product of the canonical modules.
\end{proof}

\begin{remark}
\label{remark:canonical}
If $X$ and $Y$ are Gorenstein projective schemes, the resolution of the structure sheaf is symmetrical. It follows that $\omega_X=\OO_X(-n)$ for some $n \geq 0$. If $\omega_Y=\OO_Y(-m)$, it follows from the above proposition that $\omega_{X \ast Y}(-m-n)$.
\end{remark}

%%%%%%%%%%
%%%%%%%%%%
\section{Calabi--Yau manifolds and mirror symmetry} % (fold)
\label{sec:calabi_yau}

Parts of this thesis will be concerned with the construction of new Calabi--Yau manifolds. For us, a Calabi-Yau manifold will be a irreducible complex projective variety $X$ such that $\omega_X \simeq \OO_X$ and $H^i(X,\OO_X)=0$ for $i=1,\ldots,\dim X-1$. We will always have $\dim X=3$. Be ware that the literature often requires \CY manifolds to be smooth, or to have only certain kinds of singularities.

Mathematically, \CY varieties are interesting because they are among the varities having Kodaira dimension zero. This means that they have trivial canonical models, making them harder to study. Before the 90's there were only sporadic constructions of \CY varieties, but after the advent of toric geometry and the construction of Batyrev in \cite{batyrev_mirrorsymmetry}, thousands of new examples were found, all of which was anticanonical sections of Fano toric varieties. We will explain this construction in the next section. \todo{do that}

Let $\Omega_X^1$ be the sheaf of holomorphic one-forms on $X$, and assume that $\dim X =3$. Then we can form the \emph{Hodge diamond} of $X$, which is a format of writing the dimensions $h^{ij}$ of the cohomology groups $H^j(X,\Omega_X^i)$. Here $\Omega_X^i$ is by definition the wedge product $\wedge^i \Omega_X^1$. 

\[
\begin{array}[c]{ccccccc}
&&& h^{00} \\  
 &         & h^{01} & & h^{10}  \\
&	h^{02} && h^{11}  && h^{20}  \\
h^{03} && h^{12} && h^{21} && h^{30} \\
&	h^{13} && h^{22}  && h^{31}  \\
 &         & h^{23} & & h^{32}  \\
 &&& h^{33} 
\end{array}
\]

Because of the \CY condition, we have that $h^{j0}=0$ for $0 < j < 3$, and also that $h^{00}=h^{0d}=1$. It follows by Serre duality (see \cite[Corollary 7.7, Chapter III]{hartshorne}) that $h^{ij}=h^{3-i,3-j}$. Note that this amounts to a horizontal symmetry of the Hodge diamond. Since $X$ was assumed to be a complex manifold, it follows that $h^{ij}=h^{ji}$ by complex conjugation\footnote{This follows from the $\bar \delta$-Poincaré lemma.}. This amounts to vertical symmetry of the Hodge diamond. It follows that for 3-dimensional \CY varieties, the Hodge diamond simplifies to
\[
\begin{array}[c]{ccccccc}
&&& 1 \\  
 &         & 0 & & 0  \\
&	0 && h^{11}  && 0  \\
1 && h^{12} && h^{12} && 1 \\
&	0 && h^{11}  && 0  \\
 &         & 0 & & 0  \\
 &&& 1 
\end{array}.
\]
The \emph{Hodge decomposition} theorem \cite[page 142]{voison_cg} states that the singular cohomology groups decomposes as
$$
H^k(X,\C) = \bigoplus_{i+j=k}^{\dim X} H^i(X,\Omega_X^j).
$$
The \emph{topological Euler characteristic} is defined as
$$
\chi(X) = \sum_{k=0}^{2 \dim X} \dim_\C H^k(X, \C).
$$
For 3-dimensional \CY varieties, it follows from the above discussion that $\chi(X)$ can be computed as $2(h^{11} - h^{12})$. 

\begin{example}
The canonical example of a \CY variety is the quintic in $\P^4$. Let $X=V(f)$ be defined by a general element in $H^0(\P^4,\OO_{\P^4}(5))$. Then $X$ is smooth, and by adjunction formula we have $\omega_X = \restr{\omega_{\P^4}}{X} \otimes \OO_X(5)=\OO_X(-5) \otimes \OO_X(5) = \OO_X$, so the canonical bundle is trivial. By the ideal sheaf sequence, we find that $H^i(X,\OO_X) \simeq H^i(X,\OO_{\P^4}(-5))$, which by \cite[Theorem 5.1, Chapter III]{hartshorne} implies the required vanishing of the structure sheaf cohomology groups.

The Euler characteristic can be computed as the degree of the top Chern class of $X$. If $Y$ is a degree $d$ hypersurface in $\P^n$, the following formula holds:
$$
c_{n-1}(T_X) = h^{n-1} \left( \binom{n+1}{n-1} - d \binom{n+1}{n-2} + d^2 \binom{n+1}{n-3} + \ldots\right).
$$
Putting $n=4$ and $d=5$, we find that $\chi(X)=-200$. To compute $h^{11}$, we see first from the conormal sequence that $H^1(\Omega_Y^1) \simeq H^1(\restr{\Omega_{\P^4}^1}{Y})$. Now it follows easily from the restricted Euler sequence that $h^{11}=1$, and since $\chi(X)=2(h^{11}-h^{12})$ it follows that $h^{12}=101$.
\end{example}

In general it is very hard to compute the Hodge numbers of Calabi--Yau varieties, with the exception of hypersurfaces in four-dimensional toric varieties. Often the best one can hope for is the topological Euler characteristic $\chi(X)$, which is much easier to compute.

There is a connection between \CY manifolds and Fano manifolds. In particular, the following is true:

\begin{proposition}
\label{prop:anticanonicalsection}
Let $Y \subset \P^N$ be an $n$-dimensional Fano variety with $\omega_Y=\OO_Y(-k)$. Then a general section $X$ of $\OO_Y(1)^{\oplus k}$ is an $n-k$-dimensional Calabi--Yau variety.
\end{proposition}
\begin{proof}
This follows from the adjunction formula, which says that
$$
\omega_X = \omega_Y \otimes \wedge^k (\mathcal I_X /\mathcal I_X^2)^\vee.
$$
A general section of $\OO_Y(1)^{\oplus k}$ is a complete intersection, and the normal bundle is then equal to $\OO_X(1)^{\oplus k}$. It is then true that $\wedge^k \OO_X(1)^{\oplus k} = \OO_X(k)$, from which the triviality of the canonical bundle follows.

Recall that the structure cohomology groups $H^i(Y,\OO_Y)=0$ for $ i> 0$ when $Y$ is a Fano variety. That the cohomology groups $H^i(X,\OO_X)$ are zero for $i=1,\ldots,n-k$ can be \todo{prove this (use technique of \cite{mirrorsymmetry})}
\end{proof}

The last chapter of this thesis is concerned about the construction of new examples of Calabi--Yau manifold. They are complete intersections in different (singular) toric varieties.

\subsection{Mirror symmetry}

After the invention of string theory in the late 60's, \CY varieties caught the attention of theoretical physicists. They predict that space-time is really $10$-dimensional, and locally looks like $\R^4 \times X$, where $M$ is a \CY manifold, of complex dimension $3$. They discovered a duality between different \CY 3-manifolds $X$ and $X^\circ$ such that their Hodge numbers satisfy $h^{11}(X) = h^{12}(X^\circ)$ and $h^{12}(X)=h^{11}(X^\circ)$ (in particular $\chi(X) = - \chi(X^\circ)$). This a conjectural correspondence between the complex moduli space of $X$ (which have tangent space $H^1(X,\Omega_X^2$), and the Kähler moduli space of $\omega^\circ$, the Kähler class on $X^\circ$. These two manifolds, though mathematically different, are thought to give rise to equivalent physical theories. This correspondence is called \emph{mirror symmetry}.

In the 90's, Candelas et al. constructed the mirror of the general quintic \cite{candelas_string}. They calculate certain Hodge theoretic invariants on the mirror, and use them to count \emph{rational curves} on the quintic. This greatly surprised the mathematical community, and led to the invention of Gromov--Witten-invariants and homological mirror symmetry.

Mirror symmetry is a fascinating and notoriously technical topic. There are several good introductions, depending upon taste and technical proficiency. Two of the most comprehensive introductions are \cite{mirrorsymmetry,mirrorsymmetry_vafa}.

Constructing mirrors of Calabi--Yau manifolds have become a small industry in the mathematics community. In the last chapter of this thesis, I propose mirror candidates for two of my Calabi--Yau constructions. 

\todo{flytt ansatz-avsnittet hit. Forklar med eksempler.}

% section calabi_yau (end)

%%%%%%%%%
%%%%%%%%%
\section{Toric geometry} % (fold)
\label{sec:toric_geometry}

A toric variety is an irreducible and normal variety containing the torus $T=(\C^\ast)^n$  as a dense subset, such that the action of the torus on itself extends to an action on the variety.

We fix some notation that will be used throughout. Details and proofs can be found in \cite{cox_toric,fulton_toric}. Each toric variety comes with two dual lattices. The \emph{lattice of 1-parameter subgroups $N$} and the \emph{character lattice $M$}. A one-parameter subgroup is a morphism $\lambda:\C ^\ast \to T$ that is a group homomorphism. The set of one-parameter subgroups is a lattice isomorphic to $\Z^n$, and we denote it by $N$. A \emph{character} is a morphism $\chi:T \to \C^\ast$ that is a group homomorphism. The set of characters is a lattice $M$ isomorphic to $\Z^n$ which is naturally dual to $N$.

Let $V$ be a $\R$-vector space. Let $V^\vee$ be the dual vector space. A \emph{convex polyhedral cone} is a subset $\sigma$ of $V$ of the form
$$
\sigma = \{ r_1 v_1 + \ldots + r_s v_s \mid r_i \geq 0 \text{ for all } i \},
$$
where the $v_i$ is a finite set of vectors in $V$ and the $r_i$ are real numbers. A \emph{rational polyhedral cone} is a cone such that the vectors $v_i$ can be taken to have rational coordinates.

The \emph{dual cone $\sigma^\vee$} lives in $V^\vee$, and is defined as the set of functionals that are positive on $\sigma$:
$$
\sigma^\vee = \{ u \in V^\vee \mid \langle u,v\rangle \geq 0, v \in \sigma \}.
$$
Cones have two descriptions: either as the positive hull of a finite set of vectors (as above), or implicitly, as the intersection of finitely many half-spaces. If $u_i$ generate $\sigma^\vee$, then it is true that
$$
\sigma = \sigma^{\vee \vee } = \{ v \in V \mid \langle u_i,v \rangle \geq 0 \text{ for all } i \}.
$$
The vectors $u_i$ are the inner normal vectors of the facets of $\sigma$.

A \emph{(commutative) semigroup} is a set $S$ with a commutative binary operation $S \times S \to S$, together with an identity element $0 \in S$. Given a cone $\sigma \subset N$, we can form a semigroup $S\stackrel{\Delta}{=} \sigma^\vee \cap M \subset M$. Given a semigroup $S$, we can form the \emph{semigroup algebra} $\C[S]$: it is the algebra generated by the elements of $S$, with multiplicative structure inherited from $S$. We then define $U_\sigma$ as $\Spec \C[\sigma^\vee \cap M]$, and call it the \emph{affine toric variety associated to $\sigma$}. 

\begin{example}
Let $\sigma=\langle (1,0), (1,2) \rangle \subset \R^2$. Then
\[
\sigma^\vee = \langle (2,-1),(1,0),(0,1) \rangle \subset \R^2.
\]
Then $\C[\sigma^\vee \cap M] = \C[x,y,x^2/y]$, where we have identified $x$ and $y$ with the standard basis of $\R^2$. This ring is isomorphic to $\C[a,b,c]/(a^2-bc)$, which is a quadric cone.
\end{example}

General toric varities are described using collections of cones called \emph{fans}. A set $\Sigma$ of cones is called a \emph{fan} if it closed under intersections and faces of cones: if $\sigma, \sigma' \in \Sigma$, then we also have $\sigma \cap \sigma' \in \Sigma$, and if $\sigma' \subset \sigma$ is a face with $\sigma \in \Sigma$, then $\sigma' \in \Sigma$ also. Thus, given a fan $\Sigma$, we get a collection of affine toric varieties $U_\sigma$ for each cone $\sigma \in \Sigma$. We have inclusions $U_{\sigma \cap \sigma'} \subset U_\sigma$, and using these inclusions we glue to get a separated toric variety. If the fan is \emph{complete} (meaning that the union of its cones is equal to $N$), the corresponding toric variety is complete. 

\begin{example}
Consider \cref{fig:p2fan}. This is the fan corresponding to the toric variety $\P^2$. The dual cones $\sigma_i^\vee$ give rise to algebras $\C[x,y]$, $\C[\frac 1x, \frac yx]$ and $\C[\frac xy, \frac 1y]$.
\begin{figure}
\centering 
\includestandalone{./figures/fanp2}
\caption{The fan corresponding to the toric variety $\P^2$.}
\label{fig:p2fan}
\end{figure}
These glue to form $\P^2$. More complicated fans give rise to exponents in the monomial generators.
\end{example}

Projective toric varieties can be constructed from lattice polytopes. We describe the procedure here. Let $\Delta$ be a lattice polytope in $M \simeq \Z^n$. Let $M' = M \oplus \Z$, and embed $\Delta$ in $M'$ by sending $v$ to $(v,1)$. Let $C(\Delta)$ be the cone over $\Delta$ in $M'$. Then $\C[C(\Delta) \cap M']$ is a $\Z$-graded algebra. We let $X_\Delta$ be the associated projective variety. 

\subsection{Divisors and Picard groups of toric varieties}

Recall that a \emph{Weil divisor} is a formal linear combination of codimension $1$ subvarieties of a scheme $X$ (satisfying the ``star'' condition in Hartshorne). The group of Weil divisors modulo linear equivalence is the \emph{class group} of $X$, and is denoted by $\Cl(X)$. The group of line bundles modulo isomorpism is the \emph{Picard group} of $X$, and is denoted by $\Pic(X)$. The two groups coincide for smooth varieties. They are in general very hard to compute, but for toric varieties the computation is exceptionally easy, relying only on structure of the rays in the fan $\Sigma$ defining the toric variety.

For completeness, we describe the divisors on toric varities. The description will be used in \cref{chap:smoothings}, where we work out the geometry of the two smoothings of the affine cone over the del Pezzo surface of degree $6$.

Let $X$ be a smooth toric variety, and let $\Sigma(1)$ denote the set of one-dimensional cones (called \emph{rays}) in the fan $\Sigma$ defining $X$. For each ray $\rho$, let $u_\rho \in N$ denote the primitive ray generator of $\rho$. Then one can show that the torus-invariant divisors on $X$ are in one to one correspondence with the rays $\rho \in \Sigma(1)$. Furthermore, every divisor on $X$ is linearly equivalent to a torus-invariant divisor. Using these two facts, one can prove the following:

There is an exact sequence:
\[
\begin{tikzcd}
0 \arrow{r} &  M \arrow{r}{C} & \Z^{\Sigma(1)}\arrow{r} & \Pic(X)   \arrow{r} & 0,
\end{tikzcd}
\]
where the rows of the matrix $C$ are the vectors $u_{\rho}$. See \cite{cox_toric}, Chapter 4, for a proof.

There is also a description of the Cartier divisors on $X$ in terms of \emph{support functions} on $N$: a support function is a function $\varphi:|\Sigma| \to \R$ such that the restriction $\restr{\varphi}{\sigma}$ of $\varphi$ to each cone in $\Sigma$ is linear. A support function is \emph{integral with respect to $N$} if $\varphi(|\Sigma| \cap N) \subset \Z$. This means that for each cone $\sigma$, there is an $m_\sigma \in M$, such that $\varphi(v)=\langle v, m_\sigma \rangle$ if $v \in \sigma$.

The set of support functions is an abelian group under addition, and by Theorem 4.2.12 in \cite{cox_toric}, there is an isomorphism between the group of integral support functions on $\Sigma$ and the torus invariant Cartier divisors on $X$.

\todo{describe support functions}

% section toric_geometry (end)

\section{Deformation theory and the Hilbert scheme}

Deformation theory is the infinetesimal study of algebro-geometric objects in families. Examples of such objects can be families of schemes, families of projective schemes (respecting the embedding), families of vector bundles, and so on.

In this section I will review some notation and motivation from deformation theory. Although results from deformation theory are not central in thesis, many of the methods and objects have roots from or connections with deformation theory. 
A reference for deformation theory is the book by Hartshorne \cite{hartshorne_deformations}. For a leisurely popular account, the article \cite{mazur_deformations} is a nice read.

Given a scheme $X_0$ over $\C$, a \emph{family of deformations} of $X_0$ is a flat morphism $\pi:\mathscr X \to (S,0)$ with $S$ connected such that $\pi^{-1}(0)=X_0$. If $S$ is the spectrum of an artinian $\C$-algebra, then $\pi$ is an \emph{infinitesimal deformation}. If $S=\Spec \C[\epsilon]/\epsilon^2$, then  $\pi$ is a \emph{first order deformation}. An \emph{embedded deformation} of an embedded scheme $X_0 \subset \P^n$ is a deformation $\pi:\mathscr X \to (S,0)$ with $\mathscr X \subset \P^n \times S$ such that $\pi$ is the restriction of the projection $\pi:\P^n \times S \to S$. A deformation is \emph{trivial} if it is isomorphic to the projection $X_0 \times S \to S$.

A \emph{smoothing of $X_0$} is a deformation such that the general fiber is smooth.

The set of embedded deformations and the set of abstract deformations have interpretations in terms of ``familiar'' objects (see \cite{hartshorne_deformations} for proofs):

\begin{proposition}
The set of all first order embedded deformations of a projective scheme $X$ is in one--one correspondence with the group $H^0(X,\mathcal N_{X/\P^n})$, where $\mathcal N_{X/\P^n}$ is the normal sheaf of $X$ in $\P^n$.\footnote{Recall that this is by definition $\mathcal Hom(\mathcal I/\mathcal I^2, \OO_X)$, where $\mathcal I$ is the ideal sheaf of $X$.}
\end{proposition}

\begin{proposition}
The set of all first order deformations of a smooth scheme $X$ is in one--one correspondence with the group $H^1(X,\mathcal T_X)$.
\end{proposition}

\begin{remark}
The intuition behind this result is the following. From the normal sequence
\[
0 \to \mathcal T_X \to \restr{\mathcal T_\P}{X} \to \mathcal N_{X/\P^n} \to 0,
\]
we get a surjection (for $n \geq 2$):
\[
H^0(X, \restr{\mathcal T_\P}{X}) \to H^0(X,\mathcal N_{X/\P^n}) \to H^1(X,\mathcal T_X) \to 0.
\]
The interpretation is that abstract deformations correspond to embedded deformations modulo infinetesimal automorphisms of $\P^n$.
\end{remark}

If we denote by $\Def(X)$  (resp. $\EmbDef(X)$) denote the ``space'' of all (resp. embedded) deformations of a scheme $X$, then then the above proposition tells us that $H^1(X,\mathcal T_X)$ (resp. $H^0(X,\mathcal N_{X/\P^n})$) is the tangent space of the point $[X]$ in $\Def(X)$ (resp. $\EmbDef(X)$).

There is a complex, called the \emph{cotangent complex}, associated to commutative rings $R$ and modules $M$, that measures various deformation theoretic aspects of $\Spec R$. These are modules $T^i(B/k,M)$ for $i \geq 0$. Only the first three will be relevant for us, and we will present some ad hoc definitions.

Let $B$ be an $A$-algebra, where $A$ is a commutative ring. Let $R$ be a polynomial ring surjecting onto $B$ and $I$ the kernel. Let $F$ be a free module surjecting onto $I$, and let $Q$ be its kernel. Then we have an exact sequence
\[
0 \to Q \to F \xrightarrow{j} I \to 0.
\]
There is a ``Koszul'' submodule $F_0$ of $F$ generated by the elements $aj(b)-bj(a)$. Note that $j(F_0)=0$, implying that $F_0 \subset Q$. Let $L_2 \stackrel \Delta = Q/F_0$. Let $L_1 = F \otimes_R B$, and let $L_0 = \Omega_{R/A}^1 \otimes_R B$. These are the first few terms of the cotangent complex:
\[
L_\bullet: L_2 \xrightarrow{d_2} L_1 \xrightarrow{d_1}  L_0 \xrightarrow{0} 0.
\]
The map $d_2$ is induced by the inclusion $Q \to F$. The map $d_1$ is the composition of $j:F \to I$ and the derivation $R \to \Omega^1_{R/A}$.

For any $R$-module $M$, we now define $T^i(B/A,M)$ as the homology $H^i(\Hom_B(L_\bullet,M))$. There are many things to be checked, but the details are all in \cite{hartshorne_deformations}.

We list a few of the important properties of the $T^i$-functors here:

\begin{itemize}
	\item We have an equality $T^0(B/A,M) = \Der_A(B,M)$. If $M=B$, this is the \emph{tangent module} of $B$ over $A$.
	\item If $A=k[x_1,\ldots,x_n]$ and $B=A/I$, then we have an exact sequence
	\begin{equation}
	\label{eq:t1seq}
	\Hom(\Omega_{A/k}^1,M) \to \Hom(I/I^2,M) \to T^1(B/k,M) \to 0.
	\end{equation}
	This gives us a way to compute $T^1(B/k,M)$ which is amenable to computer algebra software. Algorithms for computing $T^i(B/k,B)$ for $i=0,1,2$ are implemented in the \MM package \texttt{VersalDeformations} by Nathan Ilten \cite{ilten_versaldeformations}.
	\item The module $T^1(B/k,B)$ classifies first order deformations of the affine scheme $\Spec B$. It is a finite-dimensional $k$-vector space if $\Spec B$ has only isolated singularities. Both $T^1(B/k,B)$ and $T^2(B/k,B)$ are zero if $B$ is smooth.
	\item The module $T^2(B/k,B)$ contains ``obstructions'' for lifting deformations to larger artinian rings.
	\item If $B$ and $M$ are graded, then $T^i(B/A,M)$ are graded as well. 
\end{itemize}

If $X$ is a projective variety and $A$ its homogeneous coordinate ring, let $U_X$ denote $\Spec A$. Then the deformation theory of the affine cone and $X$ itself is closely related. This is studied for example in Schlessinger's article \cite{schlessinger_rigid}, from which the following useful result can be proved:

\begin{proposition}
\label{prop:t1h1}
Let $X/k$ be a smooth projective Calabi--Yau variety, and let $A$ be its homogeneous coordinate ring. Then we have an isomorphism
$$
T^1(A/k,A)_0  \simeq H^1(X, \mathcal T_X),
$$
where $\mathcal T_X$ is the tangent sheaf of $X$.
\end{proposition}
\begin{proof}
This is a combination of Theorem 2.5 and Corollary 2.6 in \cite{t1_ref}, using the fact that $X$ is Calabi--Yau and smooth.
\end{proof}

This result makes computing Hodge numbers of projective smooth Calabi--Yau's relatively easy.

We include a somewhat lengthy example of how to compute the $T^i$ modules for a relatively simple ring.

\begin{example}
Let $B=k[x,y]/(x^2,xy,y^2)$ be the coordinate ring of the double point in $\A^2$. We want to compute $T^i(B/k,B)$ for $i=0,1,2$. 

We have that $T^0(B/k,B) = \Der_k(B,B)$, and this can be identified with the kernel of the map (see Proposition 3.10 in \cite{hartshorne_deformations})
$$
\Hom(\Omega_{k[x,y]/k},B) \xrightarrow{\varphi} \Hom(I/I^2,B).
$$
The map $\varphi$ can be identified with the transpose of the Jacobian matrix of $I$. The module to the left is free, generated by $\frac{\partial}{\partial x}$ and $\frac{\partial}{\partial y}$. Up to scalars, $\varphi$ is given by 
\[
\varphi = \begin{pmatrix}
x & 0 \\ y & x \\ 0 & y
\end{pmatrix}.
\]
Thus $T^0(B/k,B)$ is equal to the set of $(f,g) \in R^2$ annihilated by the ideal $\mathfrak m = (x,y)$ in $B$. But since $R$ is $k[x,y]/\mathfrak m^2$, this is equal to $\mathfrak m \oplus \mathfrak m$. Thus $\dim_k \Der_k(B,B) = 4$, corresponding to the fact that a fat point can move by moving its support and also by moving its ``tangent arrow''.

We can use the exact sequence \eqref{eq:t1seq} to compute $T^1(B/k,B)$. We see that $T^1(B/k,B)$ is the cokernel of $\varphi$. We must first identify $\Hom(I/I^2,B)$.

To compute this module, we start with a free resolution of $I$ over $P=k[x,y]$:
\[
0 \to 
P^2 \xrightarrow{d_1 = 
	\begin{pmatrix}
	y & 0 \\ -x & y \\ 0  & -x
	\end{pmatrix}
} P^3 \xrightarrow{ d_0 = 
	\begin{pmatrix}
	x^2 \\ xy \\ y^2
	\end{pmatrix}
} I \to 0.
\]
It is then true that $\Hom(I/I^2,B)$ can be identified with $\ker ( d_1^\vee \otimes B)$. An easy argument shows that this is $\mathfrak m \oplus \mathfrak m  \oplus \mathfrak m$.

But the image of $\varphi$ is a two-dimensional subset of $\Hom_B(I/I^2,B)$. Hence $\dim_k T^1(B/k,B) = 6-2=4$.

The computation of $T^2(B/k,B)$ is usually the hardest. We can identify $T^2(B/k,B)$ with $\Hom_B(Q/F_0,B)/\mathrm{image} \left(d_1 \otimes B\right)^{\vee}$, where $F_0$ is the module of Koszul relations and $Q= \mathrm{image} d_1$. Let us first compute $\Hom_B(Q/F_0,B)$. 

We start with finding a presentation for $Q/F_0$. The module $F_0$ is the submodule of $F=P^3$ generated by the columns of the matrix
\[
\psi = \begin{pmatrix}
y^2 & xy & 0 \\
0 & -x^2 & y^2 \\
-x^2 & 0 & xy
\end{pmatrix}.
\]
The image of $d_1$ is isomorphic to $R^2$. Using this isomorphism, $Q/F_0$ fits into an exact sequence
\[
R^3 \xrightarrow{
	M = \begin{pmatrix}
	x & y & 0 \\ 0 & x & y
	\end{pmatrix}
} R^2 \to Q/F_0 \to 0.
\]
Applying $\Hom_B(-,B)$ is left-exact, so we get an exact sequence:
\[
0 \to \Hom_B(Q/F_0,B) \to B^2 \xrightarrow{
	\begin{pmatrix}
	x & 0 \\ y & x \\ 0 & y
	\end{pmatrix}
} B^3
\]

It follows that $\Hom_B(Q/F_0,B)= \mathfrak m \oplus \mathfrak m$.  The image of $d_1^\vee \otimes B$ kills off three of the four generators, so that $T^2(B/k,B)$ is a $4-3=1$-dimensional vector space over $k$. This reflects the fact that the fat point correspond to a singular point in its Hilbert scheme.

As we can see, already for this small example, there is a lot of computation involved. Especially the computation of a free resolution is resource demanding when the ideal have more generators. Therefore computer algebra software is essential when doing experiments in deformation theory.
\end{example}

\subsection{A few words about Hilbert schemes}

The \emph{Hilbert scheme} $\mathscr H_{P(t)}$ parametrizes projective schemes with a given Hilbert polynomial $P(t)$. The proof of its existence is non-trivial, and given first by Grothendieck in \cite{MR1611822}. The proof was later simplified by Mumford \cite{MR0209285}. However, it is often just as easy to work with the functorial description of the Hilbert scheme - namely with the functor it represents rather than the scheme itself.

The functor that the Hilbert scheme represents is the following: $h_{P(t)}(S)$ is the set set of all flat families $\mathscr X \subset S \times \P^n  \to S$ where the fibers have Hilbert polynomial $P(t)$. With this definition, it is not difficult to show for example that the tangent space of $\mathscr H_{P(t)}$ at a point corresponding to a scheme $X$ is given by $H^0(X,\mathscr N_{X/\P^n})$, where $\mathscr N_{X/\P^n}$ is the normal sheaf of $X$. Thus for a ``generic'' scheme, the dimension of the component on the Hilbert scheme on which it lies, is given by $h^0(X,\mathscr N_{X/\P^n})$.

Note that two different points on $\mathscr H_{P(t)}$ might represent isomorphic schemes. Two schemes are different if they occupy different points in $\P^n$. Allowing deformations outside $\P^n$ corresponds to applying the forgetful functor $\Def(X,S) \to \Def(X)$.





%%%%%%%%%%%%%%%%%
%%%%%%%%%%%%%%%%%
\section{Stanley--Reisner schemes}

Stanley--Reisner schemes are certain degenerate projective schemes modelled on simplicial complexes. We first recall some facts about simplicial complexes. 

Let $[n]$ denote the set of numbers $\{0,\ldots,n \}$. The power set of $[n]$ is called the \emph{$n$-simplex} and is denoted by $\Delta_n$.

\begin{definition}
A \emph{simplical complex} is a subset $\K \subseteq \Delta_n$ (for some $n$), such that if $f \in \K$ and $g \subseteq f$, then $g \in \K$. The subsets of $\K$ of cardinality one are called the \emph{vertices} of $\K$. The subsets of codimension one are called {facets} of $\K$. The subsets of $\K$ are called \emph{faces}. The \emph{dimension} of a face $f$ is equal to $|f| - 1$. 
\end{definition}


The \emph{join} of two simplicial complexes $\K$ and $\K'$ is defined as
$$
\K \ast \K' \stackrel{\Delta}{=} \{ f \sqcup g \mid f \in \K, g \in \K' \},
$$
where $\sqcup$ denotes the disjoint union. 

A good reference for this and the following is Stanley's green book \cite{stanley_green}.


If $f \subset \K$ is a face, the \emph{link of $f$ in $\K$} is the simplicial complex defined by
$$
\lk(f,\K) \stackrel{\Delta}{=} \{ g \in \K \mid f \cap g = \emptyset, f \cup g \in \K \}.
$$
If $D_+(x_f) \subset \P(\K)$ denotes the distinguished open set corresponding to the monomial $x^f$, we have that $D_+(x_f) = \A(\lk(f,\K)) \times (k^\ast)^{\dim f}$. 

Every simplicial complex has a \emph{geometric realization}, which as a set is defined as follows:
$$
|\K| \stackrel{\Delta}{=}  \{ \alpha: [n] \to [0,1] \mid \supp(\alpha) \in \K, \sum_{i=1}^n \alpha(i) = 1 \}.
$$

This is an example of a piecewise linear manifold, or a PL-manifold for short. For more on PL-manifolds or combinatorial topology, we refer the reader to \cite{glaser_combtop,spanier_topology,hudson_pl}.

Motivated by this, we single out a class of simplicial complexes:

\begin{definition}
A simplicial complex $\K$ is called a \emph{manifold} if the geometric realization of every link $\lk(\K,v)$ ($v$ is a vertex) is homeomorphic to a sphere.
\end{definition}

\begin{figure}[b]
\centering 
\hspace*{\fill}%
\subbottom[A non-manifold simplicial complex. The link at the two right vertices are not spheres.]{
\includestandalone[scale=1.5]{./figures/non_manifold}
\label{fig:nonmanifold}
}
 \hspace*{\fill}%
\subbottom[A simplicial manifold.]{
\includestandalone[scale=1.8]{./figures/cube}
\label{fig:cube}
}
 \hspace*{\fill}%
\caption{Two examples of simplicial complexes.}
\end{figure}

Let $k$ be a field, and let $P_\K$ be the polynomial ring over $k$ with variables indexed by the vertices of $\K$. Then the \emph{face ring} or \emph{Stanley--Reisner ring} of $\K$ is the quotient ring $A_\K = P_\K/I_\K$, where $I_\K$ is the ideal generated by monomials corresponding to non-faces of $\K$. 

\begin{example}
Let $\K$ be the triangle with vertices $\{ v_1,v_2,v_3\}$. Its maximal faces are $v_1v_2, v_2v_3$ and $v_1v_3$. The Stanley--Reisner ring is $k[v_1,v_2,v_3]/(v_1v_2v_3)$.
\end{example}

The ideal $I_\K$ is graded since it is defined by monomials. This leads us to define the \emph{Stanley--Reisner scheme} $\P(\K)$ as $\Proj A_\K$. 

Note that $\P(\K \ast \K') = \P(\K) \ast \P(\K')$, where the second star means the join of two projective varieties.

The following result of Christophersen (\cite[Theorem 4.6]{deforming_christophersen}) is useful. It expresses the deformation theory of Stanley--Reisner schemes purely in terms of their combinatorial data.

\begin{theorem}
\label{thm:t1dims}
If $\K$ is a simplicial manifold, and $\mathbf c = \mathbf a - \mathbf b$ (with disjoint supports $a$ and $b$), then
$$
\dim_k T_{A_\K,\mathbf c}^1 = \begin{cases}
1 & \text{ if } a  \in \K \text{ and } b \in \mathcal B(\lk(a,\K)) \\
0 & \text{ otherwise.}
\end{cases}
$$
\end{theorem}
Here $\mathcal B(\K)$ is defined as follows:
\begin{definition}
The set $\mathcal B(\K)$ is the set of $b \subset \K$ with $|b| \geq 2$ such that
\begin{enumerate}
	\item $\K = L \ast \partial b$, wehere $|L|$ is a $(n-|b|+1)$-sphere, if $b \not \in \K$.
	\item $\K = L \ast \partial b \cup \partial L \ast \bar{b}$ where $|L|$ is a $(n-|b|+1)$-ball, if $b \in \K$.
\end{enumerate}
\end{definition}

\section{Smoothings of Stanley--Reisner schemes}

Because many properties of varieties are easier read off their degenerations, it is an interesting problem to study smoothings of Stanley--Reisner-schemes, which are highly singular. Below are the two main lemmas motivating the study of triangulations in deformation theory.

\begin{lemma}
\label{lemma:srcohom}
If $\K$ is a simplicial complex, then $H^i(\K;k) \simeq H^i(\P(\K),\OO_{\P(\K)})$.
\end{lemma}
The lemma is essentially due to Hochster, and is proved (in a different form) in Stanley's book \cite{stanley_green}. This is true essentially because the \v{C}ech complex computing the simplicial cohomology and the \v{C}ech complex computing sheaf cohomology look exactly the same.

\begin{lemma}
If $\K$ is a 3-dimensional simplicial sphere, then a smoothing of $X_0=\P(\K)$ will be Calabi--Yau.
\end{lemma}
\begin{proof}
Let $\pi:\mathscr X \to S$ be a smoothing. Since $\K$ is a sphere, it follows from \ref{lemma:srcohom} that $H^i(X_0,\OO_{X_0})=k$ for $i=0,3$, and zero for $i \neq 0,3$. The triviality of the canonical bundle is proved in Theorem 6.1 in \cite{eisenbud_graphcurves}. Since $H^1(\K;K)=H^2(\K;K)=0$, it follows from the semicontinuity theorem (Theorem 12.8 in Chapter III in \cite{hartshorne}) that $H^i(X_t,\OO_{X_t})=0$ for all $t \in S$. Similarly, if $\omega_0$ all nearby fibers must have trivial canonical bundle as well.
\end{proof}

\section{Toric geometry and toric degenerations}

There is a correspondence between certain degenerations of toric varieties and so-called unimodular triangulations. \todo{define these}

Let $M$ be a lattice (by which we mean a free abelian group of finite rank). Let $\nabla \subset M_\Q = M \otimes_\Z \Q$ be a lattice polytope, and let $S_\nabla$ be the semigroup in $M \times \Z$ generated by the elements $(u,1) \in \nabla \cap M$. Then we define $\P(\nabla)=\Proj \C[S_\nabla]$, and call it the \emph{toric variety associated to $\nabla$}. 

By Theorem 8.3 and Corollary 8.9 in \cite{sturmfels}, there is a one-one correspondence between unimodular regular triangulations of $\nabla$ and the square-free initial ideals of the toric ideal of $\P(\nabla)$. \todo{check this correspondence}

\begin{proposition}
\label{eq:unimodular_triangs}
There is a 1--1 correspondence between regular unimodular triangulations of polytopes and squarefree monomial ideals.
\end{proposition}

By Theorem 8.3 and Corollary 8.9 in \cite{sturmfels}......

Thus, if we can find a polytope whose boundary has a regular unimodular triangulation, we know that the associated Stanley-Reisner ring has at least one deformation. 