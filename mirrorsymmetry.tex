\documentclass[11pt, english]{article}
%\usepackage[latin1]{inputenc}
\usepackage[T1]{fontenc}
\usepackage[utf8]{inputenc}
\usepackage[english]{babel}   % S P R A A K


% \usepackage{graphicx}    % postscript graphics
\usepackage{amssymb, amsmath, amsthm, amssymb} % symboler, osv
\usepackage{mathrsfs}
\usepackage{url}
\usepackage{thmtools}
\usepackage{enumerate}  % lister $  
\usepackage{float}
\usepackage{tikz}
\usetikzlibrary{calc}
\usepackage{tikz-3dplot}
\usepackage{subcaption}
\usepackage[all]{xy}   % for comm.diagram
\usepackage{wrapfig} % for float right
\usepackage{hyperref}
\usepackage{mystyle} % stilfilen      

\usepackage[a5paper,margin=0.5in]{geometry}

\begin{document}
\title{Notes mirror symmetry}
\author{Fredrik Meyer}
\maketitle 

\section{The quintic}

We want to completely understand the quintic in $\PP=\PP^4$ and its mirror. The quintic Calabi-Yau is defined to be the zero set of a general quintic in $H^0(\PP, \OO_X(5))$. Note that since $\omega_{\PP} \simeq \OO_{\PP}(-5)$, this is also just a general section of the anticanonical bundle on $\PP$.

Recall the defintion of Calabi-Yau:
\begin{defi}
An algebraic variety $X$ is \emph{Calabi-Yau} if $H^i(X,\OO_X)=0$ for $i\neq 0,n$ (where $n=\dim X$) and $\omega_X=\wedge^n \Omega_{X/k}$ is trivial, that is, $\omega_X \simeq \OO_X$.
\end{defi}

Denote the quintic by $Y \subset \PP$. We want to show that $Y$ is Calabi-Yau.

\begin{prop}
A general section of $H^0(\PP,\OO_\PP(5))$ is Calabi-Yau. In addition $h^{11}=1$ and $h^{12}=101$.
\end{prop}
\begin{proof}
We have the ideal sheaf sequence
$$
0 \to \mathscr I \to \OO_\PP \to i^\ast \OO_Y \to 0,
$$
where $i:Y \to \PP^4$ is the inclusion. Note that $\mathscr I = \OO_\PP(-5)$. Thus we have from the long exact sequence of cohomology that
$$
\cdots \to H^i(\PP, \mathscr I) \to H^i(\PP,\OO_\PP) \to H^i(Y,\OO_Y) \to H^{i+1}(\PP,\mathscr I) \to \cdots
$$
Note that $H^{i+1}(\PP,\mathscr I) = 0$ for $i \neq 3$ and $1$ for $i=3$. Also $H^i(\PP,\OO_\PP)=0$ unless $i=0$ in which case it is $1$. Thus we get that $H^i(Y,\OO,Y)$ is $k$ for $i=0$, for $i=1,2$ it is $0$, and for $i=3$ it is $k$. For higher $i$ it is zero by Grothendieck vanishing.

The adjunction formula relates the canonical bundles as follows: if $\omega_\PP$ is the canonical bundle on $\PP$, then $\omega_Y = i^\ast \omega_\PP \otimes_{\OO_{\PP}} \det(\mathscr I/\mathscr I^2)^\vee$. The ideal sheaf is already a line bundle, so taking the determinant does not change anything. Now
\begin{align*}
(\mathscr I/\mathscr I^2)^\vee &= \Hom_Y(\mathscr I/\mathscr I^2,\OO_Y) \\
&= \Hom_X(\mathscr I,\OO_Y)=\Hom_X(\OO_Y(-5),\OO_Y)=\OO_Y(5).
\end{align*}
It follows that $\omega_Y = \OO_Y(-5) \otimes \OO_Y(5)=\OO_Y$. Thus the canonical bundle is trivial and we conclude that $Y$ is Calabi-Yau.

It remain to compute the Hodge numbers. We start with $h^{11} = \dim_k H^1(Y,\Omega_Y)$. We have the conormal sequence of sheaves on $Y$:
$$
0 \to \mathscr I / \mathscr I^2 \to \Omega_{\PP} \otimes \OO_Y \to \Omega_Y \to 0,
$$
which gives us the long exact sequence:
$$
\cdots \to H^i(\mathscr I/\mathscr I^2) \to H^i(\Omega_\PP \otimes \OO_Y) \to H^i(\Omega_Y) \to H^{i+1}(\mathscr I/\mathscr I^2) \to \cdots 
$$
We first compute the cohomology of $\mathscr I/\mathscr I^2$. We use the short exact sequence
\begin{equation}
\label{normaleq}
0 \to \OO_\PP(-10) \to \OO_\PP(-5) \to \mathscr I / \mathscr I^2 \to 0.
\end{equation}
we have $H^i(\PP,\OO_\PP(-10))=0$ for $i=0,1,2,3$, and for $i=4$ we have $H^4(\PP,\OO_\PP(-10))=H^0(\PP,\OO_\PP(5))=k^{126}$. Similarly $H^i(\PP,\OO_\PP(-5))=0$ for $i=0,1,2,3$ and $H^4(\PP,\OO_\PP(-5))=H^0(\PP,\OO_\PP)=k$. We conclude that $h^i(Y,\mathscr I/\mathscr I^2) = 0$ for $i=0,1,2$ and $125$ for $i=3$.

In particular $H^1(\Omega_Y) \simeq H^1(\Omega_\PP \otimes \OO_Y)$. We have the Euler sequence:
$$
0 \to \Omega_\PP \to \OO_\PP(-1)^{\oplus 5} \to \OO_\PP \to 0
$$
Now $\OO_Y=\OO_\PP/\mathscr I$ is a flat $\OO_\PP$-module since $\mathscr I$ is principal and generated by a non-zero divisor. Thus we can tensor the Euler sequence with $\OO_Y$ and get
$$
0 \to \Omega_\PP \otimes \OO_Y \to \OO_Y(-1)^5 \to \OO_Y \to 0,
$$
from which it easily follows that $H^1(Y,\Omega_\PP \otimes \OO_Y) \simeq H^0(\OO_Y) = k$. We conclude that $h^{11}=1$. 

Now we compute $h^{12} = \dim_k H^1(Y,\Omega^2)$. This is equal to $H^2(Y, \Omega_Y)$ by Serre duality. Again we use the conormal sequence. From the Euler sequence we get that $H^2(Y, \Omega_\PP \otimes \OO_Y)=0$. We also get that $h^3(Y,\Omega_\PP \otimes \OO_Y)=24$. NOW $H^3(\Omega_Y)=0$ (WHY??), and it follows from the above computations that $h^{12}=125-24=101$.
\end{proof}

The moduli space of all quintics is $101$-dimensional, and we can see that as follows: the space of all quintic polynomials is $h^0(\PP^4,\OO_\PP(5))=125$-dimensional. But $\dim \PGL(n+1)=25-1=24$, so that the space of quintics up to automorpisms is $125-24=101$-dimensional. Note that this is the same as $h^{12}=\dim_k H^2(Y,\Omega_Y)$, and this is no coincidence.

The mirror quintic should have its Hodge numbers switched, so we are looking for a $1$-dimensional family. The family $\lvert H^0(\PP,\OO_\PP(5)) \rvert$ have a subfamily invariant under $S_5$, given by
\begin{equation}
\label{dwork}
x_0^5+x_1^5+x_2^5+x_3^5+x_4^5 + \psi x_0x_1x_2x_3x_4=0  
\end{equation}

for $\psi \in k$. Now let
$$
G = \{ (a_0,\cdots, a_4) \in \Z_5^5 \mid \sum a_i \equiv 0 \pmod 5 \} / \Z_5
$$
act on $\PP^4$ by multiplication:
$$
g \cdot (x_0:\cdots:x_4) = (\zeta^{a_1}x_0:\cdots : \zeta^{a_4}x_4),
$$
where $\zeta$ is a fifth root of unity. The family of \eqref{dwork} is invariant under $G$, so that we get a family of hypersurfaces in $\PP^4/G$. This can be shown to give the ``correct'' Calabi-Yau. [[see proofs]]

\begin{itemize}
\item Complete computation of the quintic and its mirror. $h^{11}$,$h^{12}$. 
\item Picard Fuchs etc?
\end{itemize}


\section{Batyrev and Batyrev-Borisov}

The Batyrev construction gives a huge number of mirror candidates using toric geometry. 


\end{document}
