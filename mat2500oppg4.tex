\documentclass[11pt, norsk]{article}
%\usepackage[latin1]{inputenc}
\usepackage[T1]{fontenc}
\usepackage[utf8]{inputenc}
\usepackage[norsk]{babel}   % S P R A A K


% \usepackage{graphicx}    % postscript graphics
\usepackage{amssymb, amsmath, amsthm, amssymb} % symboler, osv
\usepackage{mathrsfs}
\usepackage{url}
\usepackage{thmtools}
\usepackage{enumerate}  % lister $  
\usepackage{float}
\usepackage{tikz}
\usetikzlibrary{calc}
\usepackage{tikz-3dplot}
\usepackage[all]{xy}   % for comm.diagram
\usepackage{wrapfig} % for float right
\usepackage{hyperref}
\usepackage{mystyle} % stilfilen      


\begin{document}
\title{Oppgaver MAT2500}
\author{Fredrik Meyer}
\maketitle 
\begin{oppg}
Beskriv et polyeder med $5$ hjørner og $6$ sider der alle sidene er trekanter. Beskriv to polyedre med $6$ hjørner og $8$ sider der alle sidene er trekanter.
\end{oppg}
\begin{losn}
Vi vet fra Eulers formel at det alltid gjelder at
\[
v-e+f = 2
\]
for konvekse polyedre. Her er $v$ antall hjørner, $e$ er antall kanter, og $f$ er antall flater. Dermed finner vi at den første figuren må ha $9$ kanter. Dette gir oss en ide om hva det kan være, og etter litt tenking skjønner vi at et svar er den rette dobbelpyramiden på en trekant. Se Figur 1.

På samme måte ser vi at et polyeder med $6$ hjørner og $8$ sider må ha $12$ kanter. Ett slikt polyeder er oktaeder, som vi har tegnet i Figur 2. 

\begin{figure}[h]
\begin{center}

\tdplotsetmaincoords{81}{127}
  \begin{tikzpicture}[scale=5,tdplot_main_coords]

    \draw[thick,->] (0,0,0) -- (1,0,0) node[anchor=north east]{$x$};
    \draw[thick,->] (0,0,0) -- (0,1,0) node[anchor=north west]{$y$};
    \draw[thick,->] (0,0,0) -- (0,0,1) node[anchor=south]{$z$};

   \coordinate (A) at (-0.5,-{sqrt(3)/6},0);
   \coordinate (B) at (0.5,-{sqrt(3)/6},0);
   \coordinate (C) at (0,{sqrt(3)/3},0);
   \coordinate (E) at (0,0,{sqrt(2/3)});
   \coordinate (F) at (0,0,-{sqrt(2/3)});

\draw[dashed, red] (A) -- (B);
\draw[thick, red] (B) -- (E);
\draw[dashed, red] (E) -- (A);
\draw[thick, red]  (B) -- (F);
\draw[dashed, red]  (A) -- (F);
\draw[thick, red] (C) -- (F);
\draw[thick, red] (C)  -- (E);
\draw[dashed, red] (C)  -- (A);
\draw[thick, red] (C)  -- (B);

  \end{tikzpicture}
\end{center}
\caption{Den rette bipyramiden på en regulær trekant.}
\end{figure}

\begin{figure}[h]
\begin{center}

\tdplotsetmaincoords{81}{127}
  \begin{tikzpicture}[scale=3,tdplot_main_coords]

%    \draw[thick,->] (0,0,0) -- (1,0,0) node[anchor=north east]{$x$};
%    \draw[thick,->] (0,0,0) -- (0,1,0) node[anchor=north west]{$y$};
%    \draw[thick,->] (0,0,0) -- (0,0,1) node[anchor=south]{$z$}; 

   \coordinate (A) at (1,1,0);
   \coordinate (B) at (-1,1,0);
   \coordinate (C) at (-1,-1,0);
   \coordinate (D) at (1,-1,0);
   \coordinate (E) at (0,0,1);
   \coordinate (F) at (0,0,-1);

\draw[red] (D) -- (A) -- (B);
\draw[dashed, red] (B) -- (C) -- (D);
\draw[red] (E) -- (A);
\draw[red] (E) -- (B);
\draw[dotted,red] (E) -- (C);
\draw[red] (E) -- (D);
\draw[red] (F) -- (A);
\draw[red] (F) -- (B);
\draw[dotted,red] (F) -- (C);
\draw[red] (F) -- (D);

  \end{tikzpicture}
\end{center}
\caption{Den rette bipyramiden på en regulær trekant.}
\end{figure}

\begin{figure}[h]
\begin{center}

\tdplotsetmaincoords{100}{200}
  \begin{tikzpicture}[scale=3,tdplot_main_coords]

%    \draw[thick,->] (0,0,0) -- (1,0,0) node[anchor=north east]{$x$};
%    \draw[thick,->] (0,0,0) -- (0,1,0) node[anchor=north west]{$y$};
%    \draw[thick,->] (0,0,0) -- (0,0,1) node[anchor=south]{$z$}; 

   \coordinate (A) at ({(1/4)*sqrt(10+2*sqrt(5))},{0.25*(sqrt(5)-1)},0);
   \coordinate (B) at (0,1.1,-0.1);
   \coordinate (C) at (-{(1/4)*sqrt(10+2*sqrt(5))},{0.25*(sqrt(5)-1)},0);
   \coordinate (E) at (-{(1/4)*sqrt(10+2*sqrt(5))},{-0.25*(sqrt(5)-1)},0);
   \coordinate (F) at ({(1/4)*sqrt(10+2*sqrt(5))},{-0.25*(sqrt(5)-1)},0);
   \coordinate (D) at (0,0,1);

\draw[thick, red] (A) -- (B) -- (C) -- (E) -- (F) -- cycle;
\fill[green, opacity=0.2] (A) -- (B) -- (C) -- (E) -- (F) -- cycle;
\fill[blue, opacity=0.2] (D) -- (E) -- (F) -- cycle;
\fill[red, opacity=0.2] (D) -- (E) -- (C) -- cycle;
\fill[orange, opacity=0.2] (D) -- (C) -- (B) -- cycle;
\fill[yellow, opacity=0.2] (D) -- (B) -- (A) -- cycle;
\fill[pink, opacity=0.2] (D) -- (A) -- (F) -- cycle;
\fill[black, opacity=0.2] (B) -- (C) -- (E) -- cycle;
\fill[black, opacity=0.2] (B) -- (E) -- (F) -- cycle;
\fill[black, opacity=0.2] (B) -- (F) -- (A) -- cycle;
\draw (D) node[above] {T} -- (A);
\draw (D) -- (B);
\draw (D) -- (C);
\draw (D) -- (E);
\draw (D) -- (F);
\draw (B) -- (E);
\draw (B) -- (F);

  \end{tikzpicture}
\end{center}
\caption{Pentagon med én side lenger enn de andre.}
\end{figure}

\end{losn}


\end{document}
