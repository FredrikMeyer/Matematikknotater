\documentclass[11pt, english]{article}
%\usepackage[latin1]{inputenc}
\usepackage[T1]{fontenc}
\usepackage[utf8]{inputenc}
\usepackage[english]{babel}   % S P R A A K
% \usepackage{graphicx}    % postscript graphics
\usepackage{amssymb, amsmath, amsthm, amssymb} % symboler, osv
\usepackage{mathrsfs}
%\usepackage{url}
\usepackage{thmtools}
\usepackage{enumerate}  % lister $  
\usepackage{float}
%\usepackage{tikz}
%\usepackage{tikz-cd}
%\usetikzlibrary{calc}
%\usepackage{tikz-3dplot}
%\usepackage{subcaption}
%\usepackage[all]{xy}   % for comm.diagram
%\usepackage{wrapfig} % for float right
\usepackage{hyperref}
\usepackage{mystyle} % stilfilen      

%\usepackage[a5paper,margin=0.5in]{geometry}


\begin{document}
\title{Prosjektideer til MAT2000}
\author{Fredrik Meyer}
\date{}
\maketitle 

\begin{enumerate}
\item \textbf{Gröbner-baser}.

Dette forslaget krever at du har hatt Grupper og Ringer-kurset. Gröbner-baser er en polynom-ring-analog for basis for et vektorrom, og gjør at en kan regne ut mye. Spesielt er det en sikker måte på å regne ut alle løsninger til et system av polynomer. Én retning kan være å implementere dette i et dataprogram hvor inputen er noen polynomer, og outputen er løsningsmengden som en liste av punkter.

Er du for eksempel også interessert i algebraisk geometri, har disse mange anvendelser der: for eksempel kan disse brukes til å regne ut idealene til projeksjoner, osv.

\item \textbf{Innføring i} \verb|Macaulay2|. Denne krever nok at du har tatt Kommutativ Algebra-kurset. Skriv en liten innføring i \verb|Macaulay2| (eller et annet data-algebra-program) som kan dekke vanlige problemer i kommutativ algebra og algebraisk geometri (hvordan regne ut $\Tor$/$\Ext$/singulariteter, resolusjoner, osv). Du må selvsagt lære disse ordene underveis.

Det går også an å bytte ut ordet \verb|Macaulay2| med SAGE, og skrive om hvordan om for eksempel gruppeteori.

\item \textbf{Knuteteori}. Velger du dette, kan jeg hjelpe deg å finne litteratur og gi tips om hvordan skrive matematikk. Du kan skrive om invarianter av knuter og hvordan man regner ut disse (f.eks Reidermeister-bevegelsene, Jones-polynomet, etc).

Det er også en morsom kobling mellom knuteteori og algebraisk geometri: tar du den algebraiske kurven $\{ x^2+y^3 = 0 \}$ i $\C^2$\footnote{Eller den punkterte Riemann-flaten, om du vil.}, og snitter denne med en liten $3$-sfære $\lvert x \rvert ^2 + \lvert y \rvert ^2 = \epsilon$, vil man få en ikke-triviell knute.

En oppgave kan være å vise og \emph{forstå} dette.
\end{enumerate}


\end{document}
