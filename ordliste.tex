\documentclass[11pt, english]{article}
%\usepackage[latin1]{inputenc}
\usepackage[T1]{fontenc}
\usepackage[utf8]{inputenc}
\usepackage[english]{babel}   % S P R A A K

% \usepackage{graphicx}    % postscript graphics
\usepackage{amssymb, amsmath, amsthm, amssymb} % symboler, osv
\usepackage{mathrsfs,calligra}
\usepackage{url}
\usepackage{thmtools}
\usepackage{enumerate}  % lister $
\usepackage{float}
\usepackage{tikz}
\usepackage[all]{xy}   % for comm.diagram
\usepackage{wrapfig} % for float right
\usepackage{hyperref}
\usepackage{mystyle} % stilfilen      

\title{Dictionary}
\author{Fredrik Meyer}
\date{}
\begin{document}
\maketitle

\section{Algebraic Geometry}

\subsection{General properties}
\subsubsection{Fano variety}
\label{fano}

A variety $X$ is Fano if the \hyperref[anticanonical]{anticanonical sheaf} $\omega_X^{-1}$ is \hyperref[amplelinebundle]{ample}. 

\subsection{Results and theorems}
\subsubsection{Bertini's Theorem}
\label{bertini}
Let $X$ be a nonsingular closed subvariety of $\PP_k^n$, where $k=\bar k$. Then the set of of hyperplanes $H \subseteq \PP_k^n$ such that $H \cap X$ is regular at every point) and such that $H\not  \subseteq X$ is a dense open subset of the complete linear system $|H|$. See \cite[Thm II.8.18]{hartshorne}.

\subsubsection{Euler sequence}
\label{eulersequence}
If $A$ is a ring and $\PP_A^n$ is projective $n$-space over $A$, then there is an exact sequence of sheaves on $X$:
\[
0 \to \Omega_{\PP^n_A/A} \to \OO_{\PP^n_A}(-1)^{n+1} \to \OO_{\PP_A^n} \to 0.
\]
See \cite[Thm II.8.13]{hartshorne}.

\subsubsection{Kodaira vanishing}
\label{kodairavanishing}
If $k$ is a field of characteristic zero, $X$ is a smooth and projective $k$-scheme of dimension $d$, and $\LL$ is an \hyperref[amplelinebundle]{ample} invertible sheaf on $X$, then ${H^q(X,{\LL \otimes_{\OO_X} \Omega^p_{X/k}})=0}$ for $p+q > d$. In addition, $H^q(X,\LL^{-1} \otimes_{\OO_X} \Omega_{X/k}^p)=0$ for $p+q < d$. 

\subsubsection{Riemann-Roch for curves}
\label{riemannroch}

The \textbf{Riemann-Roch theorem} relates the number of sections of a line bundle with the genus of a smooth curve $C$. Let $\LL$ be a line bundle $\omega_C$ the canonical sheaf on $C$. Then
\[
h^0(C,\LL) - h^0(C, \LL^{-1} \otimes_{\OO_{C}} \omega_C) = \deg(\LL) +1 -g.
\]
This is \cite[Theorem IV.1.3]{hartshorne}.

\subsubsection{Serre vanishing}
\label{serrevanishing}

One form of Serre vanishing states that if $X$ is a proper scheme over a noetherian ring $A$, and $\LL$ is an \hyperref[amplelinebundle]{ample} sheaf, then for any coherent sheaf $\FF$ on $X$, there exists an integer $n_0$ such that for each $i > 0$ and $n \geq n_0$ the group $H^i(X, \FF \otimes_{\OO_X} \LL^n)=0$ vanishes. See \cite[Proposition III.5.3]{hartshorne}.

\subsection{Sheaves and bundles}
\subsubsection{Ample line bundle}
\label{amplelinebundle}
A line bundle $\LL$ is \textbf{ample} if for any coherent sheaf $\FF$ on $X$, there is an integer $n$ (depending on $\FF$) such that $\FF \otimes_{\OO_X} \LL^{\otimes n}$ is generated by global sections. Equivalently, a line bundle $\LL$ is ample if some tensor power of it is \hyperref[veryample]{very ample}.  

\subsubsection{Very ample line bundle}
\label{veryample}
A line bundle $\LL$ is \textbf{very ample} if there is an embedding $i:X \hookrightarrow \PP_S^n$ such that the pullback of $\OO_{\PP_S^n}(1)$ is isomorphic to $\LL$. In other words, there should be an isomorphism $i^\ast \OO_{\PP^n_S}(1) \simeq \LL$.


\subsubsection{Anticanonical sheaf}
\label{anticanonical}

The \textbf{anticanonical sheaf} $\omega_X^{-1}$ is the inverse of the \hyperref[canonicalsheaf]{canonical sheaf} $\omega_X$, that is $\omega_X^{-1} = \Shom_{\OO_X}(\omega_X,\OO_X)$.

\subsubsection{Canonical divisor}
\label{canonicaldivisor}

The \textbf{canonical divisor} $K_X$ is the class of the \hyperref[canonicalsheaf]{canonical sheaf} $\omega_X$ in the divisor class group.

\subsubsection{Canonical sheaf}
\label{canonicalsheaf} 

If $X$ is a smooth algebraic variety of dimension $n$, then the canonical sheaf is $\omega := \wedge^n \Omega^1_{X/k}$ the $n$'th exterior power of the cotangent bundle of $X$.

\subsection{Toric geometry}
\subsubsection{Polarized toric variety}
A toric variety equipped with an \hyperref[amplelinebundle]{ample} $T$-invariant divisor.

\section{Commutative algebra}
\subsection{Modules}
\subsubsection{Depth}
Let $R$ be a noetherian ring, and $M$ a finitely-generated $R$-module and $I$ an ideal of $R$ such that $IM \neq M$. Then the $I$-depth of $M$ is (see \hyperref[ext]{Ext}): \[\inf\{i \mid \Ext^i_R(R/I,M) \neq 0\}.\]
This is also the length of a maximal $M$-sequence in $I$.

\subsection{Rings}
\subsubsection{Cohen-Macaulay ring}
A local Cohen-Macaulay ring (CM-ring for short) is a commutative noetherian local ring with Krull dimension equal to its depth. A ring is Cohen-Macaulay if its localization at all prime ideals are Cohen-Macaulay.

\subsubsection{Depth of a ring}

The depth of a ring $R$ is is its depth as a module over itself.

\subsubsection{Gorenstein ring}

A commutative ring $R$ is Gorenstein if each localization at a prime ideal is a Gorenstein local ring. A Gorenstein local ring is a local ring with finite injective dimension as an $R$-module. This is equivalent to the following: $\Ext_R^i(k,R)=0$ for $i \neq n$ and $\Ext_R^n(k,R) \simeq k$ (here $k = R/\mm$ and $n$ is the Krull dimension of $R$).

\section{Convex geometry}
\subsection{Cones}
\subsubsection{Simplicial cone}
A cone $C$ generated by $\{v_1, \cdots, v_k \} \subseteq N_\R$ is simplicial if the $v_i$ are linearly independent.

\subsection{Polyhedra}

\subsubsection{Dual (polar) polyhedron}

If $\Delta$ is a polyhedron, its dual $\Delta^\circ$ is defined by
\[
\Delta^\circ = \left\{ x \in N_\R \mid \langle x,y\rangle \geq -1 \,\forall \, y \in \Delta \right\}.
\]

\subsubsection{Reflexive polytope}

A polytope $\Delta$ is reflexive if the following two conditions hold:
\begin{enumerate}
\item All facets $\Gamma$ of $\Delta$ are supported by affine hyperplanes of the form $\{ m \in M_{\R} \mid \langle m,v_\Gamma \rangle \}$ for some $v_\Gamma \in N$.
\item The only interior point of $\Delta$ is $0$, that is: $\mathrm{Int}(\Delta) \cap M = \{0\}$.
\end{enumerate}

\section{Homological algebra}
\subsection{Derived functors}
\subsubsection{Ext}
\label{ext}
Let $R$ be a ring and $M,N$ be $R$-modules. Then $\Ext_R^i(M,N)$ is the right-derived functors of the $\Hom(M,-)$-functor. In particular, $\Ext^i_R(M,N)$ can be computed as follows: choose a projective resolution $C_.$ of $N$ over $R$. Then apply the left-exact functor $\Hom_R(M,-)$ to the resolution and take homology. Then $\Ext_R^i(M,N)=h^i(C_.)$.

\subsubsection{Tor}
\label{tor}
Let $R$ be a ring and $M,N$ be $R$-modules. Then $\Tor_R^i(M,N)$ is the right-derived functors of the $- \otimes_R N$-functor. In particular $\Tor_R^i(M,N)$ can be computed by taking a projective resolution of $M$, tensoring with $N$, and then taking homology.

\bibliographystyle{plain}
\bibliography{bibliografi}

\end{document}
