\documentclass[11pt, english]{article}
%\usepackage[latin1]{inputenc}
\usepackage[T1]{fontenc}
\usepackage[utf8]{inputenc}
\usepackage[english]{babel}   % S P R A A K
%%
%% husk 
%% git pull origin 
%% git add *
%% git commit -m "..."
%% git push
% \usepackage{graphicx}    % postscript graphics
\usepackage{amssymb, amsmath, amsthm, amssymb} % symboler, osv
\usepackage{mathrsfs,calligra}
\usepackage{url}
\usepackage{thmtools}
\usepackage{enumerate}  % lister $
\usepackage{float}
\usepackage{tikz}
\usepackage[all]{xy}   % for comm.diagram
\usepackage{wrapfig} % for float right
\usepackage[colorlinks=true]{hyperref}
\usepackage{mystyle} % stilfilen      
 
\title{Algebraic Geometry Buzzlist}
\author{Fredrik Meyer}
\date{}
\begin{document}
\maketitle

\section{Algebraic Geometry}
 
\setcounter{tocdepth}{2} 
\tableofcontents


\subsection{General terms}
\subsubsection{Cartier divisor}
\label{cartierdivisor}

Let $\mathcal K_X$ be the \emph{sheaf of total quotients} on $X$, and let $\OO_X^\ast$ be the sheaf of non-zero divisors on $X$. We have an exact sequence 
\[
1 \to \OO_X^\ast \to \mathcal K_X \to \mathcal K_X/\OO_X^\ast \to 1.
\]
Then a \textbf{Cartier divisor} is a global section of the quotient sheaf at the right.

\subsubsection{Categorical quotient}
\label{categoricalquotient}

Let $X$ be a scheme and $G$ a group. A \textbf{categorical quotient} is a morphism $\pi:X \to Y$ that satisfies the following two properties:
\begin{enumerate}
\item It is invariant, in the sense that $\pi \circ \sigma = \pi \circ p_2$ where $\sigma:G \times X \to X$ is the group action, and $p_2:G \times X \to X$ is the projection. That is, the following diagram should commute:
\[
\xymatrix{
G \times X \ar[r]^\sigma \ar[d]_{p_2} \ar[r] & X \ar[d]^{\pi} \\
X \ar[r]_{\pi} & Y
} 
\]
\item The map $\pi$ should be \emph{universal}, in the following sense: If $\pi^\prime:X \to Z$ is any morphism satisfying the previous condition, it should uniquely factor through $\pi$. That is:
\[
\xymatrix{
X \ar[d]_{\pi^\prime}\ar[r]^{\pi}  & Y \\
Z \ar[ur]_{\exists ! h}
}
\]
\end{enumerate}

Note: A categorical quotient need not be surjective.

\subsubsection{Chow group}
\label{chowgroup}
Let $X$ be an algebraic variety. Let $Z_r(X)$ be the group of $r$-dimension cycles on $X$, a \emph{cycle} being a $\Z$-linear combination of $r$-dimensional subvarieties of $X$. If $V \subset X$ is a subvariety of dimension $r+1$ and $f:X \rmap \Aa^1$ is a rational function on $X$, then there is an integer $\ord_W(f)$ for each codimension one subvariety of $V$, the order of vanishing of $f$. For a given $f$, there will only be finitely many subvarieties $W$ for which this number is non-zero. Thus we can define an element $[\div(f)]$ in $Z_r(X)$ by $\sum \ord_W(f) [W]$.

We say that two $r$-cycles $U_1,U_2$  are \emph{rationally equivalent} if there exist $r+1$-dimensional subvarieties $V_1,V_2$ together with rational functions $f_1\colon V_1 \rmap \Aa^1$, $f_2 \colon V_2 \rmap \Aa^1$ such that $U_1 - U_2 = sum_i [\div(f_i)]$. The quotient group is called the \textbf{Chow group} of $r$-dimensional cycles on $X$, and denoted by $A_r(X)$. 


\subsubsection{Complete variety}
\label{completevariety}
Let $X$ be an  integral, \hyperref[separated]{separated} scheme over a field $k$. Then $X$ is  \textbf{complete} if is \hyperref[proper]{proper}.

\subsubsection{Crepant resolution}
\label{crepantresolution}
A \textbf{crepant resolution} is a resolution of singularities $f:X \to Y$ that does not change the \hyperref[canonicalsheaf]{canonical bundle}, i.e. such that $\omega_X \simeq f^\ast \omega_Y$.

\subsubsection{Dominant map}
\label{dominantmap}

A rational map $f:X \rmap Y$ is  \textbf{dominant} if its image (or precisely: the image of one of its representatives) is dense in $Y$. 

\subsubsection{Étale map}
\label{etalemap}

A morphism of schemes of finite type $f:X \to Y$ is \textbf{étale} if it is smooth of dimension zero. This is equivalent to $f$ being flat and $\Omega_{X/Y}=0$. This again is equivalent to $f$ being flat and unramified.

\subsubsection{Genus}
\label{genus}
 
The \textbf{geometric genus} of a smooth, algebraic variety, is defined as the number of sections of the \hyperref[canonicalsheaf]{canonical sheaf}, that is, as $H^0(V,\omega_X)$. This is often denoted $p_X$.

\subsubsection{Geometric quotient}
\label{geometricquotient}

Let $X$ be an algebraic variety and $G$ an algebraic group. Then a \textbf{geometric quotient} is a morphism of varieties $\pi:X \to Y$ such that
\begin{enumerate}
\item For each $y \in Y$, the fiber $\pi^{-1}(y)$ is an orbit of $G$.
\item The topology of $Y$ is the quotient topology: a subset $U$ of $Y$ is open if and only if $\pi^{-1}(U)$ is open.
\item For any open subset $U \subset Y$, $\pi^\ast: k[U] \to k[\pi^{-1}(U)]^G$ is an isomorphism of $k$-algebras.
\end{enumerate}
The last condition may be rephrased as an isomorphism of strcture sheaves: $\OO_Y \simeq (\pi_\ast \OO_X)^G$.   

\subsubsection{Hodge numbers}
\label{hodgenumbers}

If $X$ is a complex manifold, then the \textbf{Hodge numbers $h^{pg}$} of $X$ are defined as the dimension of the cohomology groups $H^p(X,\Omega_X^q)$.

\subsubsection{Linear series}
\label{linearseries}

A \textbf{linear series} on a smooth curve $C$ is the data $(\LL,V)$ of a line bundle on $C$ and a vector subspace $V \subseteq H^0(C,\LL)$. We say that the linear series $(\LL,V)$ have \emph{degree} $\deg \LL$ and \emph{rank} $\dim V - 1$. 

\subsubsection{Log structure}
\label{logstructure}

A \textbf{prelog structure} on a scheme $X$ is given by a pair $(X,M)$, where $X$ is a scheme and $M$ is a sheaf of monoids on $X$ (on the \hyperref[etalesite]{Ètale site}) together with a morphisms $\alpha:M \to \OO_X$. It is a \textbf{log structure} if the map $\alpha:\alpha^{-1}\OO_X^\ast \to \OO_X^\ast$ is an isomorphism.

See \cite{kato_log}.

\subsubsection{Normal crossings divisor}
\label{normalcrossingsdivisor}

Let $X$ be a smooth variety and $D \subset X$ a divisor. We say that $D$ is a \textbf{simple normal crossing divisor} if every irreducible component of $D$ is smooth and all intersections are transverse. That is, for every $p \in X$ we can choose local coordinates $x_1,\cdots,x_n$ and natural numbers $m_1,\cdots,m_n$ such that $D=\left( \prod_{i} x_i^{m_i} =0 \right)$ in a neighbourhood of $p$.

Then we say that a divisor is \textbf{normal crossing} (without the ``simple'') if the neighbourhood above can is allowed to be chosen locally analytically or as a formal neighbourhood of $p$.

Example: the nodal curve $y^2=x^3+x^2$ is a a normal crossing divisor in $\C^2$, but not a simple normal crossing divisor.

This definition is taken from \cite{kollar_resolution}. 


\subsubsection{Normal variety}
\label{normalvariety}
A variety $X$ is \textbf{normal} if all its local rings are \hyperref[normalring]{normal rings}.

\subsubsection{Proper morphism}
\label{proper}
A morphism $f:X \to Y$ is \textbf{proper} if it \hyperref[separated]{separated}, of finite type, and universally closed.

\subsubsection{Resolution of singularities}
\label{resolutionsing}

A morphism $f:X \to Y$ is a \textbf{resolution of singularities of $Y$} if $X$ is non-singular and $f$ is birational and \hyperref[proper]{proper}. 

\subsubsection{Separated}
\label{separated}
Let $f:X \to Y$ be a morphism of schemes. Let $\Delta:X \to X \X_Y X$ be the diagonal morphism. We say that $f$ is \textbf{separated} if $\Delta$ is a closed immersion. We say that $X$ is \textbf{separated} if the unique morphism $f:X \to \Spec \Z$ is separated.

This is equivalent to the following: for all open affines $U,V \subset X$, the intersection $U \cap V$ is affine and $\OO_X(U)$ and $\OO_X(V)$ generate $\OO_X(U \cap V)$. For example: let $X=\PP^1$ and let $U_1=\{ [x:1] \}$ and $U_2= \{ [1:y] \}$. Then $\OO_X(U_1)=\Spec k[x]$ and $\OO_X(U_2)=\Spec k[y]$. The glueing map is given on the ring level as $x \mapsto \frac 1y$. Then $\OO_X(U_1 \cap U_2) = k[y,\frac 1y]$. 

\subsubsection{Unirational variety}
\label{unirational}

A variety $X$ is \textbf{unirational} if there exists a generically finite dominant map $\PP^n \rmap X$. 

\subsection{Moduli theory and stacks} 

\subsubsection{Étale site}
\label{etalesite}

Let $S$ be a scheme. Then the \textbf{small étale site over $S$} is the \hyperref[site]{site}, denoted by $\Et S$ that consists of all étale morphisms $U \to S$ (morphisms being commutative triangles). Let $\Cov(U \to S)$ consist of all collections $\{ U_i \to U \}_{i \in I}$ such that
\[
\coprod_{i \in I} U_i \to U
\]
is surjective.

\subsubsection{Grothendieck topology}
\label{grothendiecktopology}

Let $\CC$ be a category. A \textbf{Grothendieck topology} on $\CC$ consists of a set $\Cov(X)$ of sets of morphisms $\{ X_i \to X\}_{i \in I}$ for each $X$ in $\mathrm{Ob}(\CC)$, satisfying the following axioms:
\begin{enumerate}
\item If $V \xrightarrow{\approx} X$ is an isomorphism, then $\{ V \to X\} \in \Cov(X)$.
\item If $\{X_i \to X\}_{i \in I} \in \Cov(X)$ and $Y \to X$ is a morphism in $\CC$, then the fiber products $X_i \times_X Y$ exists and ${\{ X_i \times_X Y \to Y\}_{i \in I} \in \Cov(Y)}$.
\item If $\{X_i \in X\}_{i \in I} \in \Cov(X)$, and for each $i \in I$, $\{ V_{ij} \to X_i\}_{j \in J} \in  \Cov(X_i)$, then
\[
\{ V_{ij} \to X_i \to X\}_{i \in I, j \in J} \in \Cov(X).
\]
\end{enumerate}

The easiest example is this: Let $\CC$ be the category of open sets on a topological space $X$, the morphisms being only the inclusions. Then for each $U \in \mathrm{Ob}(\CC)$, define $\Cov(U)$ to be the set of all coverings $\{ U_i \to U \}_{i \in I}$ such that $U = \bigcup_{i\in I} U_i$. Then it is easily checked that this defines a Grothendieck topology.

\subsubsection{Site}
\label{site}

A \textbf{site} is a category equipped with a \hyperref[grothendiecktopology]{Grothendieck topology}.


\subsection{Results and theorems}
\subsubsection{Adjunction formula}
\label{adjunction}

Let $X$ be a smooth algebraic variety $Y$ a smooth subvariety. Let $i:Y \hookrightarrow X$ be the inclusion map, and let $\mathcal I$ be the corresponding ideal sheaf. Then $\omega_Y = i^\ast \omega_X \otimes_{\OO_X} \det(\mathcal I/\mathcal I^2)^\vee$, where $\omega_Y$ is the \hyperref[canonicalsheaf]{canonical sheaf} of $Y$.

In terms of \hyperref[canonicaldivisor]{canonical classes}, the formula says that $K_D = \restr{(K_X + D)}{D}$. 

Here's an example: Let $X$ be a smooth quartic surface in $\PP^3$. Then $H^1(X,\OO_X)=0$. The divisor class group of $\PP^3$ is generated by the class of a hyperplane, and $\K_{\PP^3}=-4H$. The class of $X$ is then $4H$ since $X$ is of degree $4$. $X$ corresponds to a smooth divisor $D$, so by the adjunction formula, we have that
\[
K_D = \restr{(K_{\PP^3}+D)}{D} = \restr{-4H+4H}{D} = 0.
\]
Thus $X$ is an example of a \hyperref[k3]{K3 surface}. 


\subsubsection{Bertini's Theorem}
\label{bertini}
Let $X$ be a nonsingular closed subvariety of $\PP_k^n$, where $k=\bar k$. Then the set of of hyperplanes $H \subseteq \PP_k^n$ such that $H \cap X$ is regular at every point) and such that $H\not  \subseteq X$ is a dense open subset of the complete linear system $|H|$. See \cite[Thm II.8.18]{hartshorne}.

\subsubsection{Chow's lemma}
\label{chowslemma}

Chow's lemma says that if $X$ is a scheme that is proper over $k$, then it is ``fairly close'' to being projective. Specifically, we have that there exists a projective $k$-scheme $X^\prime$ and morphism $f:X^\prime \to X$ that is birational.

So every scheme proper over $k$ is birational to a projective scheme. For a proof, see for example the Wikipedia page.

\subsubsection{Euler sequence}
\label{eulersequence}
If $A$ is a ring and $\PP_A^n$ is projective $n$-space over $A$, then there is an exact sequence of sheaves on $X$:
\[
0 \to \Omega_{\PP^n_A/A} \to \OO_{\PP^n_A}(-1)^{n+1} \to \OO_{\PP_A^n} \to 0.
\]
See \cite[Thm II.8.13]{hartshorne}.

\subsubsection{Hurwitz' formula}
\label{hurwitzformula}

Let $X,Y$ be smooth curves in the sense of Hartshorne. That is, they are integral $1$-dimensional schemes, proper over a field $k$ (with $\bar k = k$), all of whose local rings are regular.

Then Hurwitz' formula says that if $f:X \to Y$ is a separable morphism and $n= \deg f$, then
\[
2(g_X-1) = 2n(g_Y-1) + \deg R,
\]
where $R$ is the ramification divisor of $f$, and $g_X,g_Y$ are the genera of $X$ and $Y$, respectively.

\subsubsection{Kodaira vanishing}
\label{kodairavanishing}
If $k$ is a field of characteristic zero, $X$ is a smooth and projective $k$-scheme of dimension $d$, and $\LL$ is an \hyperref[amplelinebundle]{ample} invertible sheaf on $X$, then ${H^q(X,{\LL \otimes_{\OO_X} \Omega^p_{X/k}})=0}$ for $p+q > d$. In addition, $H^q(X,\LL^{-1} \otimes_{\OO_X} \Omega_{X/k}^p)=0$ for $p+q < d$. 

\subsubsection{Lefschetz hyperplane theorem}
\label{lefschetz} 
Let $X$ be an $n$-dimensional complex projective algebraic variety in $\PP_\C^n$ and let $Y$ be a hyperplane section of $X$ such that $U=X \bs Y$ is smooth. Then the natural map $H^k(X,\Z) \to H^k(Y,\Z)$ in singular cohomology is an isomorphism for $k<n-1$ and injective for $k=n-1$.

\subsubsection{Riemann-Roch for curves}
\label{riemannroch}

The \textbf{Riemann-Roch theorem} relates the number of sections of a line bundle with the genus of a smooth proper curve $C$. Let $\LL$ be a line bundle $\omega_C$ the canonical sheaf on $C$. Then
\[
h^0(C,\LL) - h^0(C, \LL^{-1} \otimes_{\OO_{C}} \omega_C) = \deg(\LL) +1 -g.
\]
This is \cite[Theorem IV.1.3]{hartshorne}.

\subsubsection{Semi-continuity theorem}
\label{semicontinuity}

Let $f:X \to Y$ be a projective morphism of noetherian schemes, and let $\FF$ be a coherent sheaf on $X$, flat over $Y$. Then for each $i \geq 0$, the function $h^i(y,\FF)=\dim_{k(y)} H^i(X_y,\FF_y)$ is an upper semicontinuous function on $Y$. See \cite[Chapter III, Theorem 12.8]{hartshorne}.

\subsubsection{Serre duality}
\label{serreduality}

Let $X$ be a projective Cohen-Macaulay scheme of equidimension $n$. Then for any locally free sheaf $\mathcal F$ on $X$ there are natural isomorphisms
\[
H^i(X,\mathcal F) \simeq H^{n-i}(X, \mathcal F^\vee \otimes \omega_X^\circ).
\]
Here $\omega_X^\circ$ is a \emph{dualizing sheaf} for $X$. In the case that $X$ is nonsingular, we have that $\omega_X^\circ \simeq \omega_X$, the canonical sheaf on $X$ (see \cite[Chapter III, Corollary 7.12]{hartshorne}). 

\subsubsection{Serre vanishing}
\label{serrevanishing}

One form of Serre vanishing states that if $X$ is a proper scheme over a noetherian ring $A$, and $\LL$ is an \hyperref[amplelinebundle]{ample} sheaf, then for any coherent sheaf $\FF$ on $X$, there exists an integer $n_0$ such that for each $i > 0$ and $n \geq n_0$ the group $H^i(X, \FF \otimes_{\OO_X} \LL^n)=0$ vanishes. See \cite[Proposition III.5.3]{hartshorne}.

\subsection{Sheaves and bundles}
\subsubsection{Ample line bundle}
\label{amplelinebundle}
A line bundle $\LL$ is \textbf{ample} if for any coherent sheaf $\FF$ on $X$, there is an integer $n$ (depending on $\FF$) such that $\FF \otimes_{\OO_X} \LL^{\otimes n}$ is generated by global sections. Equivalently, a line bundle $\LL$ is ample if some tensor power of it is \hyperref[veryample]{very ample}.

\subsubsection{Invertible sheaf}
\label{invertiblesheaf}
A locally free sheaf of rank 1 is called \textbf{invertible}. If $X$ is \hyperref[normalvariety]{normal}, then, invertible sheaves are in $1-1$ correspondence with line bundles.  

\subsubsection{Anticanonical sheaf}
\label{anticanonical}

The \textbf{anticanonical sheaf} $\omega_X^{-1}$ is the inverse of the \hyperref[canonicalsheaf]{canonical sheaf} $\omega_X$, that is $\omega_X^{-1} = \Shom_{\OO_X}(\omega_X,\OO_X)$.

\subsubsection{Canonical class}
\label{canonicaldivisor}

The \textbf{canonical class} $K_X$ is the class of the \hyperref[canonicalsheaf]{canonical sheaf} $\omega_X$ in the divisor class group.

\subsubsection{Canonical sheaf}
\label{canonicalsheaf} 

If $X$ is a smooth algebraic variety of dimension $n$, then the canonical sheaf is $\omega := \wedge^n \Omega^1_{X/k}$ the $n$'th exterior power of the cotangent bundle of $X$.

\subsubsection{Sheaf of holomorphic p-forms}
\label{pforms}

If $X$ is a complex manifold, then the \textbf{sheaf of of holomorphic $p$-forms $\Omega_X^p$} is the $p$-th wedge power of the cotangent sheaf $\wedge^p \Omega_X^1$. 

\subsubsection{Normal sheaf}
\label{normalsheaf}

Let $Y \hookrightarrow X$ be a closed immersion of schemes, and let $\mathcal I \subseteq \OO_X$ be the ideal sheaf of $Y$ in $X$. Then $\mathcal I/\mathcal I^2$ is a sheaf on $Y$, and we define the sheaf $\mathcal N_{Y/X}$ by $\Shom_{\OO_Y}(\mathcal I/\mathcal I^2, \mathcal \OO_Y)$.

\subsubsection{Reflexive sheaf}
\label{reflexivesheaf}

A sheaf $\FF$ is \textbf{reflexive} if the natural map $\FF \to \FF^{\vee\vee}$ is an isomorpism. Here $\FF^\vee$ denotes the sheaf $\Shom_{\OO_X}(\FF,\OO_X)$.

\subsubsection{Very ample line bundle}
\label{veryample}
A line bundle $\LL$ is \textbf{very ample} if there is an embedding $i:X \hookrightarrow \PP_S^n$ such that the pullback of $\OO_{\PP_S^n}(1)$ is isomorphic to $\LL$. In other words, there should be an isomorphism $i^\ast \OO_{\PP^n_S}(1) \simeq \LL$.

\subsection{Toric geometry}

\subsubsection{Chow group of a toric variety}
\label{chowtoric}

The \hyperref[chowgroup]{Chow group} $A_{n-1}(X)$ of a toric variety can be computed directly from its fan. Let $\Sigma(1)$ be the set of rays in $\Sigma$, the fan of $X$. Then we have an exact sequence
\[
 0 \to M \to \Z^{\Sigma(1)} \to A_{n-1}(X) \to 0.
\] 
The first map is given by sending $m \in M$ to $(\langle m,v_p \rangle )_{\rho \in \Sigma(1)}$, where $v_p$ is the unique generator of the semigroup $\rho \cap N$. The second map is given by sending $(a_\rho)_{\rho \in \Sigma(1)}$ to the divisor class of $\sum_\rho  a_\rho D_\rho$. 

\subsubsection{Polarized toric variety}
A toric variety equipped with an \hyperref[amplelinebundle]{ample} $T$-invariant divisor.

\subsubsection{Toric variety associated to a polytope}

There are several ways to do this. Here is one: Let $\Delta \subset M_\R$ be a convex polytope. Embed $\Delta$ in $M_R \times \R$ by $\Delta \times \{1\}$ and let $C_\Delta$ be the cone over $\Delta \times \{ 1\}$, and let $\C[C_\Delta \cap (M \times \Z)]$ be the corresponding semigroup ring. This is a semigroup ring graded by the $\Z$-factor. Then we define $\PP_\Delta=\Proj \C[C_\Delta \cap (M \times \Z)]$ to be the toric variety associated to a polytope.

\subsection{Types of varieties}
\subsubsection{Abelian variety}
\label{abelianvar}

A variety $X$ is an \textbf{abelian variety} if it is a connected and \hyperref[completevariety]{complete} algebraic group over a field $k$. Examples include \hyperref[ellipticc]{elliptic curves} and for special lattices $\Lambda \subset \C^{2g}$, the quotient $\C^{2g}/\Lambda$ is an abelian variety.

\subsubsection{Calabi-Yau variety}

In algebraic geometry, a \textbf{Calabi-Yau} variety is a smooth, proper variety $X$ over a field $k$ such that the \hyperref[canonicaldivisor]{canonical sheaf} is trivial, that is, $\omega_X \simeq \OO_X$, and such that $H^j(X,\OO_X)=0$ for $1 \leq j \leq n-1$. 

\subsubsection{del Pezzo surface}
\label{delpezzo}

A \textbf{del Pezzo} surface is a $2$-dimensional \hyperref[fano]{Fano variety}. In other words, they are complete non-singular surfaces with ample anticanonical bundle. The \emph{degree} of the del Pezzo surface $X$ is by definition the self intersection number $K.K$ of its \hyperref[canonicaldivisor]{canonical class} $K$. 

\subsubsection{Elliptic curve}
\label{ellipticc}

An \textbf{elliptic curve} is a smooth, projective curve of genus $1$. They can all be obtained from an equation of the form $y^2=x^3+ax+b$ such that $\Delta = -2^4(4a^3+27b^2) \neq 0$. 

\subsubsection{Fano variety}
\label{fano}

A variety $X$ is \textbf{Fano} if the \hyperref[anticanonical]{anticanonical  sheaf} $\omega_X^{-1}$ is \hyperref[amplelinebundle]{ample}.  

\subsubsection{Jacobian variety}
\label{jacobianvariety}

Let $X$ be a curve of genus $g$ over $k$. The \textbf{Jacobian variety} of $X$ is a scheme $J$ of finite type over $k$, together with an element $\LL \in \Pic^\circ(X/J)$, with the following universal property: for any scheme $T$ of finite type over $k$ and for any $\mathcal M \in \Pic^\circ(X/T)$, there is a unique morphism $f:T \to J$ such that $f^\ast \LL \simeq \mathcal M$ in $\Pic^\circ (X/T)$. This just says that $J$ represents the functor $T \mapsto \Pic^\circ(X/T)$. 

If $J$ exists, its closed points are in $1-1$ correspondence with elements of $\Pic^\circ (X)$.

It can be checked that $J$ is actually a group scheme. For details, see \cite[Ch. IV.4]{hartshorne}.

\subsubsection{K3 surface}
\label{k3}

A \textbf{K3 surface} is a complex algebraic surface $X$ such that the \hyperref[canonicaldivisor]{canonical sheaf} is trivial, $\omega_X \simeq \OO_X$, and such that $H^1(X,\OO_X)=0$. These conditions completely determine the Hodge numbers of $X$.

\subsubsection{Toric variety}
\label{toricvariety}

A \textbf{toric variety} $X$ is an integral scheme containing the torus $(k^\ast)^n$ as a dense open subset, such that the action of the torus on itself extends to an action $(k^\ast)^n \times X \to X$.

%%%%%%%%%%%%%%
\section{Commutative algebra}
\subsection{Modules}
\subsubsection{Depth}
Let $R$ be a noetherian ring, and $M$ a finitely-generated $R$-module and $I$ an ideal of $R$ such that $IM \neq M$. Then the $I$-depth of $M$ is (see \hyperref[ext]{Ext}): \[\inf\{i \mid \Ext^i_R(R/I,M) \neq 0\}.\]
This is also the length of a maximal $M$-sequence in $I$.

\subsection{Results and theorems}
\subsubsection{The Unmixedness Theorem}
Let $R$ be a ring. If $I=\langle x_1,\cdots,x_n\rangle $ is an ideal generated by $n$ elements such that $\codim I=n$, then all minimal primes of $I$ have codimension $n$. If in addition $R$ is \hyperref[cmring]{Cohen-Macaulay}, then every associated prime of $I$ is minimal over $I$. See the discussion after \cite[Corollary 18.14]{eisenbud} for more details. 

\subsection{Rings}
\subsubsection{Cohen-Macaulay ring}
\label{cmring}
A local Cohen-Macaulay ring (CM-ring for short) is a commutative noetherian local ring with Krull dimension equal to its depth. A ring is Cohen-Macaulay if its localization at all prime ideals are Cohen-Macaulay.

\subsubsection{Depth of a ring}

The depth of a ring $R$ is is its depth as a module over itself.

\subsubsection{Gorenstein ring}

A commutative ring $R$ is Gorenstein if each localization at a prime ideal is a Gorenstein local ring. A Gorenstein local ring is a local ring with finite injective dimension as an $R$-module. This is equivalent to the following: $\Ext_R^i(k,R)=0$ for $i \neq n$ and $\Ext_R^n(k,R) \simeq k$ (here $k = R/\mm$ and $n$ is the Krull dimension of $R$).

\subsubsection{Normal ring}
\label{normalring}

An integral domain $R$ is \textbf{normal} if all its localizations at prime ideals ${\pp \in \Spec R}$ are integrally closed domains.  

\section{Convex geometry}
\subsection{Cones}
\subsubsection{Gorenstein cone}
\label{gorensteincone}

A strongly convex cone $C \subset M_\R$ is \textbf{Gorenstein} if there exists a point $n \in N$ in the dual lattice such that $\langle v,n \rangle = 1$ for all generators of the semigroup $C \cap M$.

\subsubsection{Reflexive Gorenstein cone}
\label{reflexivegorensteincone}

A cone $C$ is \textbf{reflexive} if both $C$ and its dual $C^\vee$ are \hyperref[gorensteincone]{Gorenstein cones}. See for example \cite{mirrorsymalggeo}.

\subsubsection{Simplicial cone}
A cone $C$ generated by $\{v_1, \cdots, v_k \} \subseteq N_\R$ is \textbf{simplicial} if the $v_i$ are linearly independent.

\subsection{Polytopes}

\subsubsection{Dual (polar) polytope}
\label{polarpolyhedron}

If $\Delta$ is a polyhedron, its dual $\Delta^\circ$ is defined by
\[
\Delta^\circ = \left\{ x \in N_\R \mid \langle x,y\rangle \geq -1 \,\forall \, y \in \Delta \right\}.
\]

\subsubsection{Gorenstein polytope of index r}
\label{gorensteinpolytope}
A lattice polytope $P \subset \R^{d+r-1}$ is called a \textbf{Gorenstein polytope of index $r$} if $rP$ contains a single interior lattice point $p$ and $rP-p$ is a \hyperref[reflexivepolytope]{reflexive polytope}.

\subsubsection{Nef partition}
\label{nefpartition}
Let $\Delta \subset M_\R$ be a $d$-dimensional \hyperref[reflexivepolytope]{reflexive polytope}, and let $m=\mathrm{int}(\Delta) \cap M$. A Minkowski sum decomposition $\Delta=\Delta_1+\dotsc +\Delta_r$ where $\Delta_1,\ldots,\Delta_r$ are lattice polytopes is called a \textbf{nef partition of $\Delta$ of length $r$} if there are lattice points $p_i \in \Delta_i$ for all $i$ such that $p_1+\cdots+p_r = m$. The nef partition is called \emph{centered} if $p_i=0$  for all $i$. 

This is equivalent to the toric divisor $D_j=\OO(\Delta_i)=\sum_{\rho \in \Delta_i} D_\rho$ being a Cartier divisor generated by its global sections. See \cite[Chapter 4.3]{mirrorsymalggeo}.  

\subsubsection{Reflexive polytope}
\label{reflexivepolytope}

A polytope $\Delta$ is \textbf{reflexive} if the following two conditions hold:
\begin{enumerate}
\item All facets $\Gamma$ of $\Delta$ are supported by affine hyperplanes of the form $\{ m \in M_{\R} \mid \langle m,v_\Gamma \rangle \}$ for some $v_\Gamma \in N$.
\item The only interior point of $\Delta$ is $0$, that is: $\mathrm{Int}(\Delta) \cap M = \{0\}$.
\end{enumerate}

\section{Homological algebra}
\subsection{Derived functors}
\subsubsection{Ext}
\label{ext}
Let $R$ be a ring and $M,N$ be $R$-modules. Then $\Ext_R^i(M,N)$ is the right-derived functors of the $\Hom(M,-)$-functor. In particular, $\Ext^i_R(M,N)$ can be computed as follows: choose a projective resolution $C_.$ of $N$ over $R$. Then apply the left-exact functor $\Hom_R(M,-)$ to the resolution and take homology. Then $\Ext_R^i(M,N)=h^i(C_.)$.

\subsubsection{Local cohomology}
\label{localcohomology}
Let $R$ be a ring and $I \subset R$ an ideal. Let $\Gamma_I(-)$ be the following functor on $R$-modules:
\[
\Gamma_I(M) = \left\{ f \in M \mid \exists n \in \N, s.t. I^n f = 0 \right \}.
\]
Then $H_I^i(-)$ is by definition the $i$th right derived functor of $\Gamma_I$.  In the case that $R$ is noetherian, we have $H_I^i(M) = \varinjlim \Ext^i_R(R/I_n,M)$.

See \cite{eisenbud} and \cite{weibel} for more details. 

\subsubsection{Tor}
\label{tor}
Let $R$ be a ring and $M,N$ be $R$-modules. Then $\Tor_R^i(M,N)$ is the right-derived functors of the $- \otimes_R N$-functor. In particular $\Tor_R^i(M,N)$ can be computed by taking a projective resolution of $M$, tensoring with $N$, and then taking homology.

\section{Differential and complex geometry}
\subsection{Definitions and concepts}

\subsubsection{Almost complex structure}
\label{almostcomplex}

An \textbf{almost complex structure} on a manifold $M$ is a map $J:T(M) \to T(M)$ whose square is $-1$.

\subsubsection{Connection}
\label{connection}
Let $E \to M$ be a vector bundle over $M$. A \textbf{connection} is a $\R$-linear map $\nabla:\Gamma(E) \to \Gamma(E \otimes T^\ast M)$ such that the Leibniz rule holds:
\[
\nabla(f \sigma) = f \nabla(\sigma) + \sigma \otimes \d f
\]
for all functions $f:M \to \R$ and sections $\sigma \in \Gamma(E)$.

\subsubsection{Hermitian manifold}
\label{hermitianmanifold}

A \emph{Hermitian metric} on a complex vector bundle $E$ over a manifold $M$ is a positive-definite Hermitian form on each fiber. Such a metric can be written as a smooth section $\Gamma(E \otimes \bar{E})^\ast$, such that $h_p(\eta, \bar{\zeta}) = \bar{h_p\left(\zeta,\bar{\eta}\right)}$ for all $p \in M$, and such that $h_p(\eta,\bar{\eta}) > 0$ for all $p \in M$. A \textbf{Hermitian manifold} is a complex manifold with a Hermitian metric on its holomorphic tangent space $T^{(1,0)}(M)$.

\subsubsection{Kähler manifold}
\label{kahlermanifold}

A \textbf{Kahler manifold} is ????

\subsubsection{Symplectic manifold}
\label{symplectic}

A $2n$-dimensional manifold $M$ is \textbf{symplectic} if it is compact and oriented and has a closed real two-form $\omega \in \bigwedge^2 T^\ast(M)$ which is nondegenerate, in the sense that $\restr{\wedge^n \omega}{p} \neq 0$ for all $p \in M$.

\subsection{Results and theorems}

\section{Worked examples}
\subsection{Algebraic geometry}
\subsubsection{Hurwitz formula and Kähler differentials}

Let $X$ be the conic in $\PP^2$ given with ideal sheaf $\langle xz-y^2 \rangle$. Let $Y$ be $\PP^1$, and consider the map $f:X \to Y$ given by projection onto the $xz$-line. $X$ is covered by two affine pieces, namely $X= U_x \cup U_z$, the spectra of the homogeneous localizations at $x,z$, respectively. Let $U_x=\Spec A$ for $A=k[z]$ and $U_z = \Spec B$ for $B=k[x]$. Then the map is locally given by $A \to k[y,z]/(z-y^2)$ where $z \mapsto \bar z$, and similarly for $B$. We have an isomorphism $k[y,z]/(z-y^2) \simeq k[t]$, given by $y \mapsto t$ and $z \mapsto t^2$, so that locally the map is given by $k[z] \to k[t], z \mapsto t^2$.

This is a map of smooth projective curves, so we can apply Hurwitz' formula. Both $X,Y$ are $\PP^1$, so both have genus zero. Hence Hurwitz formula says that
\[
-2 = -n \cdot 2+ \deg R,
\]
where $R$ is the ramification divisor and $n$ is the degree of the map. The degree of the map can be defined locally, and it is the degree of the field extension $k(Y) \hookrightarrow k(X)$. But (the image of) $k(Y) = k(t^2)$ and $k(X)=k(t)$, so that $[k(Y):k(X)]=2$. Hence by Hurwitz' formula, we should have $\deg R = 2$. Since $R = \sum_{P \in X} \length \, {\Omega_{X/Y}}_P \cdot P$, we should look at the sheaf of relative differentials $\Omega_{Y/X}$.

First we look in the chart $U_z$. We compute that $\Omega_{k[t]/k[t^2]} = k[t]/(t)$. This follows from the relation $d(t^2)=2dt$, implying that $dt=0$ in $\Omega_{k[t]/k[t^2]}$. This module is zero localized at all primes but $(t)$, where it is $k$. Thus for $P=(0:0:1)$, we have $\length \, {\Omega_{X/Y}}_P = 1$.

The situation is symmetric with $z \leftrightarrow x$, so that we have $R = (0:0:1) + (1:0:0)$, confirming that $\deg R=2$.  



\bibliographystyle{plain}
\bibliography{bibliografi}

\end{document}

%sagemathcloud={"zoom_width":140}
