\documentclass[11pt, english]{article}
%\usepackage[latin1]{inputenc}
\usepackage[T1]{fontenc}
\usepackage[utf8]{inputenc}
\usepackage[english]{babel}   % S P R A A K
%%
%% husk 
%% git pull origin 
%% git add *
%% git commit -m "..."
%% git push
% \usepackage{graphicx}    % postscript graphics
\usepackage{amssymb, amsmath, amsthm, amssymb} % symboler, osv
\usepackage{mathrsfs,calligra}
\usepackage{url}
\usepackage{thmtools}
\usepackage{enumerate}  % lister $
\usepackage{float}
\usepackage{tikz}
\usepackage[all]{xy}   % for comm.diagram
\usepackage{wrapfig} % for float right
\usepackage[colorlinks=true]{hyperref}
\usepackage{mystyle} % stilfilen      
 
\title{Algebraic Geometry Buzzlist}
\author{Fredrik Meyer}
\date{}
\begin{document}
\maketitle

\section{Algebraic Geometry}
\subsection{General properties}
\subsubsection{Complete variety}
\label{completevariety}
Let $X$ be an  integral, \hyperref[separated]{separated} scheme over a field $k$. Then $X$ is  \textbf{complete} if is \hyperref[proper]{proper}.

\subsubsection{Crepant resolution}
\label{crepantresolution}
A \textbf{crepant resolution} is a resolution of singularities $f:X \to Y$ that does not change the \hyperref[canonicalsheaf]{canonical bundle}, i.e. such that $\omega_X \simeq f^\ast (\omega_Y)$.

\subsubsection{Dominant map}
\label{dominantmap}

A rational map $f:X \rmap Y$ is  \textbf{dominant} if its image (or precisely: the image of one of its representatives) is dense in $Y$. 

\subsubsection{Genus}
\label{genus}
 
The \textbf{geometric genus} of a smooth, algebraic variety, is defined as the number of sections of the \hyperref[canonicalsheaf]{canonical sheaf}, that is, as $H^0(V,\omega_X)$. This is often denoted $p_X$.

\subsubsection{Hodge numbers}
\label{hodgenumbers}

If $X$ is a complex manifold, then the \textbf{Hodge numbers $h^{pg}$} of $X$ are defined as the dimension of the cohomology groups $H^p(X,\Omega_X^q)$.

\subsubsection{Normal variety}
\label{normalvariety}
A variety $X$ is \textbf{normal} if all its local rings are \hyperref[normalring]{normal rings}.

\subsubsection{Proper morphism}
\label{proper}
A morphism $f:X \to Y$ is \textbf{proper} if it \hyperref[separated]{separated}, of finite type, and universally closed.

\subsubsection{Resolution of singularities}
\label{resolutionsing}

A morphism $f:X \to Y$ is a \textbf{resolution of singularities of $Y$} if $X$ is non-singular and $f$ is birational and \hyperref[proper]{proper}. 

\subsubsection{Separated morphism}
\label{separated}
Let $f:X \to Y$ be a morphism of schemes. Let $\Delta:X \to X \X_Y X$ be the diagonal morphism. We say that $f$ is \textbf{separated} if $\Delta$ is a closed immersion. 


\subsection{Results and theorems}
\subsubsection{Adjunction formula}
\label{adjunction}

Let $X$ be a smooth algebraic variety $Y$ a smooth subvariety. Let $i:Y \hookrightarrow X$ be the inclusion map, and let $\mathcal I$ be the corresponding ideal sheaf. Then $\omega_Y = i^\ast \omega_X \otimes_{\OO_X} \det(\mathcal I/\mathcal I^2)^\vee$, where $\omega_Y$ is the \hyperref[canonicalsheaf]{canonical sheaf} of $Y$.

In terms of \hyperref[canonicaldivisor]{canonical classes}, the formula says that $K_D = \restr{(K_X + D)}{D}$. 

Here's an example: Let $X$ be a smooth quartic surface in $\PP^3$. Then $H^1(X,\OO_X)=0$. The divisor class group of $\PP^3$ is generated by the class of a hyperplane, and $\K_{\PP^3}=-4H$. The class of $X$ is then $4H$ since $X$ is of degree $4$. $X$ corresponds to a smooth divisor $D$, so by the adjunction formula, we have that
\[
K_D = \restr{(K_{\PP^3}+D)}{D} = \restr{-4H+4H}{D} = 0.
\]
Thus $X$ is an example of a \hyperref[k3]{K3 surface}. 


\subsubsection{Bertini's Theorem}
\label{bertini}
Let $X$ be a nonsingular closed subvariety of $\PP_k^n$, where $k=\bar k$. Then the set of of hyperplanes $H \subseteq \PP_k^n$ such that $H \cap X$ is regular at every point) and such that $H\not  \subseteq X$ is a dense open subset of the complete linear system $|H|$. See \cite[Thm II.8.18]{hartshorne}.

\subsubsection{Euler sequence}
\label{eulersequence}
If $A$ is a ring and $\PP_A^n$ is projective $n$-space over $A$, then there is an exact sequence of sheaves on $X$:
\[
0 \to \Omega_{\PP^n_A/A} \to \OO_{\PP^n_A}(-1)^{n+1} \to \OO_{\PP_A^n} \to 0.
\]
See \cite[Thm II.8.13]{hartshorne}.

\subsubsection{Kodaira vanishing}
\label{kodairavanishing}
If $k$ is a field of characteristic zero, $X$ is a smooth and projective $k$-scheme of dimension $d$, and $\LL$ is an \hyperref[amplelinebundle]{ample} invertible sheaf on $X$, then ${H^q(X,{\LL \otimes_{\OO_X} \Omega^p_{X/k}})=0}$ for $p+q > d$. In addition, $H^q(X,\LL^{-1} \otimes_{\OO_X} \Omega_{X/k}^p)=0$ for $p+q < d$. 

\subsubsection{Lefschetz hyperplane theorem}
\label{lefschetz} 
Let $X$ be an $n$-dimensional complex projective algebraic variety in $\PP_\C^n$ and let $Y$ be a hyperplane section of $X$ such that $U=X \bs Y$ is smooth. Then the natural map $H^k(X,\Z) \to H^k(Y,\Z)$ in singular cohomology is an isomorphism for $k<n-1$ and injective for $k=n-1$.

\subsubsection{Riemann-Roch for curves}
\label{riemannroch}

The \textbf{Riemann-Roch theorem} relates the number of sections of a line bundle with the genus of a smooth curve $C$. Let $\LL$ be a line bundle $\omega_C$ the canonical sheaf on $C$. Then
\[
h^0(C,\LL) - h^0(C, \LL^{-1} \otimes_{\OO_{C}} \omega_C) = \deg(\LL) +1 -g.
\]
This is \cite[Theorem IV.1.3]{hartshorne}.

\subsubsection{Semi-continuity theorem}
\label{semicontinuity}

Let $f:X \to Y$ be a projective morphism of noetherian schemes, and let $\FF$ be a coherent sheaf on $X$, flat over $Y$. Then for each $i \geq 0$, the function $h^i(y,\FF)=\dim_{k(y)} H^i(X_y,\FF_y)$ is an upper semicontinuous function on $Y$. See \cite[Chapter III, Theorem 12.8]{hartshorne}.

\subsubsection{Serre vanishing}
\label{serrevanishing}

One form of Serre vanishing states that if $X$ is a proper scheme over a noetherian ring $A$, and $\LL$ is an \hyperref[amplelinebundle]{ample} sheaf, then for any coherent sheaf $\FF$ on $X$, there exists an integer $n_0$ such that for each $i > 0$ and $n \geq n_0$ the group $H^i(X, \FF \otimes_{\OO_X} \LL^n)=0$ vanishes. See \cite[Proposition III.5.3]{hartshorne}.

\subsection{Sheaves and bundles}
\subsubsection{Ample line bundle}
\label{amplelinebundle}
A line bundle $\LL$ is \textbf{ample} if for any coherent sheaf $\FF$ on $X$, there is an integer $n$ (depending on $\FF$) such that $\FF \otimes_{\OO_X} \LL^{\otimes n}$ is generated by global sections. Equivalently, a line bundle $\LL$ is ample if some tensor power of it is \hyperref[veryample]{very ample}.

\subsubsection{Invertible sheaf}
\label{invertiblesheaf}
A locally free sheaf of rank 1 is called \textbf{invertible}. If $X$ is \hyperref[normalvariety]{normal}, then, invertible sheaves are in $1-1$ correspondence with line bundles.  

\subsubsection{Anticanonical sheaf}
\label{anticanonical}

The \textbf{anticanonical sheaf} $\omega_X^{-1}$ is the inverse of the \hyperref[canonicalsheaf]{canonical sheaf} $\omega_X$, that is $\omega_X^{-1} = \Shom_{\OO_X}(\omega_X,\OO_X)$.

\subsubsection{Canonical class}
\label{canonicaldivisor}

The \textbf{canonical class} $K_X$ is the class of the \hyperref[canonicalsheaf]{canonical sheaf} $\omega_X$ in the divisor class group.

\subsubsection{Canonical sheaf}
\label{canonicalsheaf} 

If $X$ is a smooth algebraic variety of dimension $n$, then the canonical sheaf is $\omega := \wedge^n \Omega^1_{X/k}$ the $n$'th exterior power of the cotangent bundle of $X$.

\subsubsection{Sheaf of holomorphic p-forms}
\label{pforms}

If $X$ is a complex manifold, then the \textbf{sheaf of of holomorphic $p$-forms $\Omega_X^p$} is the $p$-th wedge power of the cotangent sheaf $\wedge^p \Omega_X^1$. 

\subsubsection{Normal sheaf}
\label{normalsheaf}

Let $Y \hookrightarrow X$ be a closed immersion of schemes, and let $\mathcal I \subseteq \OO_X$ be the ideal sheaf of $Y$ in $X$. Then $\mathcal I/\mathcal I^2$ is a sheaf on $Y$, and we define the sheaf $\mathcal N_{Y/X}$ by $\Shom_{\OO_Y}(\mathcal I/\mathcal I^2, \mathcal \OO_Y)$.

\subsubsection{Reflexive sheaf}
\label{reflexivesheaf}

A sheaf $\FF$ is \textbf{reflexive} if the natural map $\FF \to \FF^{\vee\vee}$ is an isomorpism. Here $\FF^\vee$ denotes the sheaf $\Shom_{\OO_X}(\FF,\OO_X)$.

\subsubsection{Very ample line bundle}
\label{veryample}
A line bundle $\LL$ is \textbf{very ample} if there is an embedding $i:X \hookrightarrow \PP_S^n$ such that the pullback of $\OO_{\PP_S^n}(1)$ is isomorphic to $\LL$. In other words, there should be an isomorphism $i^\ast \OO_{\PP^n_S}(1) \simeq \LL$.

\subsection{Toric geometry}
\subsubsection{Polarized toric variety}
A toric variety equipped with an \hyperref[amplelinebundle]{ample} $T$-invariant divisor.

\subsection{Types of varieties}
\subsubsection{Abelian variety}
\label{abelianvar}

A variety $X$ is an \textbf{abelian variety} if it is a connected and complete algebraic group over a field $k$. Examples include \hyperref[ellipticc]{elliptic curves} and for special lattices $\Lambda \subset \C^{2g}$, the quotient $\C^{2g}/\Lambda$ is an abelian variety.

\subsubsection{Calabi-Yau variety}

In algebraic geometry, a \textbf{Calabi-Yau} variety is a smooth, proper variety $X$ over a field $k$ such that the \hyperref[canonicaldivisor]{canonical sheaf} is trivial, that is, $\omega_X \simeq \OO_X$, and such that $H^j(X,\OO_X)=0$ for $1 \leq j \leq n-1$. 

\subsubsection{del Pezzo surface}
\label{delpezzo}

A \textbf{del Pezzo} surface is a $2$-dimensional \hyperref[fano]{Fano variety}. In other words, they are complete non-singular surfaces with ample anticanonical bundle. The \emph{degree} of the del Pezzo surface $X$ is by definition the self intersection number $K.K$ of its \hyperref[canonicaldivisor]{canonical class} $K$. 

\subsubsection{Elliptic curve}
\label{ellipticc}

An \textbf{elliptic curve} is a smooth, projective curve of genus $1$. They can all be obtained from an equation of the form $y^2=x^3+ax+b$ such that $\Delta = -2^4(4a^3+27b^2) \neq 0$. 

\subsubsection{Fano variety}
\label{fano}

A variety $X$ is \textbf{Fano} if the \hyperref[anticanonical]{anticanonical  sheaf} $\omega_X^{-1}$ is \hyperref[amplelinebundle]{ample}.  

\subsubsection{K3 surface}
\label{k3}

A \textbf{K3 surface} is a complex algebraic surface $X$ such that the \hyperref[canonicaldivisor]{canonical sheaf} is trivial, $\omega_X \simeq \OO_X$, and such that $H^1(X,\OO_X)=0$. These conditions completely determine the Hodge numbers of $X$.


\section{Commutative algebra}
\subsection{Modules}
\subsubsection{Depth}
Let $R$ be a noetherian ring, and $M$ a finitely-generated $R$-module and $I$ an ideal of $R$ such that $IM \neq M$. Then the $I$-depth of $M$ is (see \hyperref[ext]{Ext}): \[\inf\{i \mid \Ext^i_R(R/I,M) \neq 0\}.\]
This is also the length of a maximal $M$-sequence in $I$.

\subsection{Results and theorems}
\subsubsection{The Unmixedness Theorem}
Let $R$ be a ring. If $I=\langle x_1,\cdots,x_n\rangle $ is an ideal generated by $n$ elements such that $\codim I=n$, then all minimal primes of $I$ have codimension $n$. If in addition $R$ is \hyperref[cmring]{Cohen-Macaulay}, then every associated prime of $I$ is minimal over $I$. See the discussion after \cite[Corollary 18.14]{eisenbud} for more details. 

\subsection{Rings}
\subsubsection{Cohen-Macaulay ring}
\label{cmring}
A local Cohen-Macaulay ring (CM-ring for short) is a commutative noetherian local ring with Krull dimension equal to its depth. A ring is Cohen-Macaulay if its localization at all prime ideals are Cohen-Macaulay.

\subsubsection{Depth of a ring}

The depth of a ring $R$ is is its depth as a module over itself.

\subsubsection{Gorenstein ring}

A commutative ring $R$ is Gorenstein if each localization at a prime ideal is a Gorenstein local ring. A Gorenstein local ring is a local ring with finite injective dimension as an $R$-module. This is equivalent to the following: $\Ext_R^i(k,R)=0$ for $i \neq n$ and $\Ext_R^n(k,R) \simeq k$ (here $k = R/\mm$ and $n$ is the Krull dimension of $R$).

\subsubsection{Normal ring}
\label{normalring}

An integral domain $R$ is \textbf{normal} if all its localizations at prime ideals ${\pp \in \Spec R}$ are integrally closed domains.  

\section{Convex geometry}
\subsection{Cones}
\subsubsection{Gorenstein cone}
\label{gorensteincone}

A strongly convex cone $C \subset M_\R$ is \textbf{Gorenstein} if there exists a point $n \in N$ in the dual lattice such that $\langle v,n \rangle = 1$ for all generators of the semigroup $C \cap M$.

\subsubsection{Reflexive Gorenstein cone}
\label{reflexivegorensteincone}

A cone $C$ is \textbf{reflexive} if both $C$ and its dual $C^\vee$ are \hyperref[gorensteincone]{Gorenstein cones}. See for example \cite{mirrorsymalggeo}.

\subsubsection{Simplicial cone}
A cone $C$ generated by $\{v_1, \cdots, v_k \} \subseteq N_\R$ is \textbf{simplicial} if the $v_i$ are linearly independent.

\subsection{Polytopes}

\subsubsection{Dual (polar) polytope}
\label{polarpolyhedron}

If $\Delta$ is a polyhedron, its dual $\Delta^\circ$ is defined by
\[
\Delta^\circ = \left\{ x \in N_\R \mid \langle x,y\rangle \geq -1 \,\forall \, y \in \Delta \right\}.
\]

\subsubsection{Gorenstein polytope of index r}
\label{gorensteinpolytope}
A lattice polytope $P \subset \R^{d+r-1}$ is called a \textbf{Gorenstein polytope of index $r$} if $rP$ contains a single interior lattice point $p$ and $rP-p$ is a \hyperref[reflexivepolytope]{reflexive polytope}.

\subsubsection{Nef partition}
\label{nefpartition}
Let $\Delta \subset M_\R$ be a $d$-dimensional \hyperref[reflexivepolytope]{reflexive polytope}, and let $m=\mathrm{int}(\Delta) \cap M$. A Minkowski sum decomposition $\Delta=\Delta_1+\dotsc +\Delta_r$ where $\Delta_1,\ldots,\Delta_r$ are lattice polytopes is called a \textbf{nef partition of $\Delta$ of length $r$} if there are lattice points $p_i \in \Delta_i$ for all $i$ such that $p_1+\cdots+p_r = m$. The nef partition is called \emph{centered} if $p_i=0$  for all $i$. 

This is equivalent to the toric divisor $D_j=\OO(\Delta_i)=\sum_{\rho \in \Delta_i} D_\rho$ being a Cartier divisor generated by its global sections. See \cite[Chapter 4.3]{mirrorsymalggeo}.  

\subsubsection{Reflexive polytope}
\label{reflexivepolytope}

A polytope $\Delta$ is \textbf{reflexive} if the following two conditions hold:
\begin{enumerate}
\item All facets $\Gamma$ of $\Delta$ are supported by affine hyperplanes of the form $\{ m \in M_{\R} \mid \langle m,v_\Gamma \rangle \}$ for some $v_\Gamma \in N$.
\item The only interior point of $\Delta$ is $0$, that is: $\mathrm{Int}(\Delta) \cap M = \{0\}$.
\end{enumerate}

\section{Homological algebra}
\subsection{Derived functors}
\subsubsection{Ext}
\label{ext}
Let $R$ be a ring and $M,N$ be $R$-modules. Then $\Ext_R^i(M,N)$ is the right-derived functors of the $\Hom(M,-)$-functor. In particular, $\Ext^i_R(M,N)$ can be computed as follows: choose a projective resolution $C_.$ of $N$ over $R$. Then apply the left-exact functor $\Hom_R(M,-)$ to the resolution and take homology. Then $\Ext_R^i(M,N)=h^i(C_.)$.

\subsubsection{Local cohomology}
\label{localcohomology}
Let $R$ be a ring and $I \subset R$ an ideal. Let $\Gamma_I(-)$ be the following functor on $R$-modules:
\[
\Gamma_I(M) = \left\{ f \in M \mid \exists n \in \N, s.t. I^n f = 0 \right \}.
\]
Then $H_I^i(-)$ is by definition the $i$th right derived functor of $\Gamma_I$.  In the case that $R$ is noetherian, we have $H_I^i(M) = \varinjlim \Ext^i_R(R/I_n,M)$.

See \cite{eisenbud} and \cite{weibel} for more details. 

\subsubsection{Tor}
\label{tor}
Let $R$ be a ring and $M,N$ be $R$-modules. Then $\Tor_R^i(M,N)$ is the right-derived functors of the $- \otimes_R N$-functor. In particular $\Tor_R^i(M,N)$ can be computed by taking a projective resolution of $M$, tensoring with $N$, and then taking homology.

\bibliographystyle{plain}
\bibliography{bibliografi}

\end{document}

%sagemathcloud={"zoom_width":140}
