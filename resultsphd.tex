\documentclass[11pt, english]{article}
%\usepackage[latin1]{inputenc}
\usepackage[T1]{fontenc}
\usepackage[utf8]{inputenc}
\usepackage[english]{babel}   % S P R A A K
% \usepackage{graphicx}    % postscript graphics
\usepackage{amssymb, amsmath, amsthm, amssymb} % symboler, osv
\usepackage{mathrsfs}
\usepackage{url}
\usepackage{thmtools}
\usepackage{enumerate}  % lister $  
\usepackage{float}
\usepackage{tikz}
\usepackage{tikz-cd}
\usetikzlibrary{calc}
%\usepackage{tikz-3dplot}
\usepackage{subcaption}
\usepackage[all]{xy}   % for comm.diagram
\usepackage{wrapfig} % for float right
\usepackage{hyperref}
\usepackage{mystyle} % stilfilen      

%\usepackage[a5paper,margin=0.5in]{geometry}


\begin{document}
\title{Results so far}
\author{Fredrik Meyer}
\maketitle 


\section{Deformations of the 5-dimensional toric variety}

Recall that the $5$-dimensional toric variety $T$ had $2$-dimensional singularities (actually two disjoint copies of $dP_6$).

\begin{thm}
There exists a flat deformation of $\PP(\K \ast \Delta^1)$, $\mathfrak X \to S$, such that $\mathfrak X_{t_1} = T$ for some $t_1 \in S$ and such that the general fiber $\widetilde T$ have one-dimensional singularities.
\end{thm}
\begin{proof}
 As per now, the proof is purely by computer. The technique is this: First, consider the monomial degeneration of $T$ to the Stanley-Reisner ring $A(\K \ast \Delta^1)$ (recall that $\K=D_6 \ast D_6$). Choose deformation parameters $t_i$ perturbing the equations ``in the direction of $T$'', meaning that we only choose parameters introducing terms already occuring in the equations of $T$.

Now, using the package \verb|VersalDeformations|, it is possible to produce a flat family $\mathfrak X \to S'$ with $\mathfrak X_0 = \PP(\K \ast \Delta^1)$.

The base space is a union of toric varieties of dimensions $14,13,13,12$, respectively. Call the largest one for $S$. The equations are 
\[
\begin{vmatrix}
 t_1 & t_2 & t_2 & t_3 & \ldots & t_{12} \\
 t_{13} & t_{14} & t_{15} & t_{16} & \ldots & t_{24} 
\end{vmatrix} \leq 1
\]
By restriction, we get the claimed family $\mathfrak X \to T$. It is cheched by setting $t_i=i$ for $i=1,\ldots,12$ and $t_i=2i$ for $i=13,\ldots,24$ that the resulting fiber have one-dimensional singularities. The reason we don't set all $t_i=1$, is that this point lies in the intersection of the components of $S$. 
\end{proof}

\begin{corr}
The Stanley-Reisner scheme $\PP(\K)$ smooths to a smooth Calabi-Yau variety $X$.
\end{corr}
\begin{proof}
The scheme $\PP(\K)$ sits as a complete intersection in $\PP(\K \ast \Delta^1)$. Complete intersections deform together with the ambient variety, so $\PP(\K \ast \Delta^1)$ deforms to a general complete intersection in $\widetilde T$. Since $\widetilde T$ have curve singularities, it follows by two applications of Bertini's theorem \cite[Theorem II.8.18]{hartshorne}, that $X$ is smooth.
\end{proof}

Now what we would really like to do is to compute the Hodge numbers $h^{ij}=\dim_k H^j(X,\Omega^i_X)$ of $X$. 

We can however compute the Hodge numbers of $\widetilde T$. The hope is that there is some sort of Lefschetz theorem giving us the Hodge numbers of $X$.

\begin{thm}
  We have $h^{11}(\widetilde T)=1$ and $h^{12}(\widetilde T)=13$. 
\end{thm}
\begin{proof}
Again, this is purely computational. We use long exact sequences together with sheaf cohomology computations in \verb|Macaulay2|.

Since the ideal of $\widetilde T$ is rather complicated, doing this naïvely does not work. The trick is to choose the right term order. Since we know that $\widetilde T$ has a nice degeneration, we would like to find a term order such that its initial ideal is precisely the Stanley-Reisner ideal.

The \verb|Macaulay2| package \verb|gfanInterface| provides an interface with \verb|gfan|, which is a program that can compute weight vectors given polynomials with prescribed initial terms. The weight vector is 
\[ \omega = (1, 1, 4, 7, 7, 4, 1, 1, 4, 7, 7, 4, 1, 1). \]
With this term order, giving a very small Gröbner basis (18 elements), the computations are much faster than with the standard term order. We are able to compute resolutions of all the relevant modules within a few minutes in total.

We have an exact sequence of sheaves on $\widetilde T$:
\[
0 \to \mathscr T_1 \hookrightarrow \mathcal I/\mathcal I^2 \xrightarrow{d} \Omega_{\PP^{13}}^1 \otimes \OO_{\widetilde T} \to \Omega_{\widetilde T}^1 \to 0.
\]
This sequence can be broken into two short exact sequences. The relevant one is this:
\begin{equation}
\label{eqt1}
0 \to \im d \to \Omega_{\PP^{13}}^1 \otimes \OO_{\widetilde T} \to \Omega_{\widetilde T}^1 \to 0.  
\end{equation}

We also have the restriced Euler sequence:
\begin{equation}
\label{eqeuler}
0 \to \Omega_{\PP^{13}} \otimes \OO_{\widetilde T} \to \OO_{\widetilde T}(-1)^{14} \to \OO_{\widetilde   T} \to 0.
\end{equation}

We first compute $h^{11}$. From \eqref{eqt1}  we get a long exact sequence
\[
\ldots \to H^1(\im d) \to H^1(\Omega_{\PP^{13}}^1 \otimes \OO_{\widetilde T}) \to H^1(\Omega_{\widetilde T}^1) \to H^2(\im d) \to \ldots 
\]
The cohomology of $H^1(\im d)$ and $H^2(\im d)$ was computed with \verb|Macaulay2| to be both zero. Thus $H^1(\Omega_{\widetilde T}^1) \simeq H^1(\Omega_{\PP^{13}}^1 \otimes \OO_{\widetilde T})$. From the Euler sequence we get 
\[
\ldots \to H^0(\OO_{\widetilde T}(-1)^{14}) \to H^0(\OO_{\widetilde T}) \to H^1(\Omega_{\PP^{13}}^1 \otimes \OO_{\widetilde T}) \to H^1(\OO_{\widetilde T}(-1)^{14}) \to \ldots
\]
But the left and right terms are both zero. Hence $h^{11}=1$. We now compute $h^{12}$. 

From \eqref{eqt1} we again get 
\[
0 \to H^2(\Omega_{\PP^{13}}^1 \otimes \OO_{\widetilde T}) \to H^2(\Omega_{\widetilde T}^1) \to H^3(\im d) \to H^3(\Omega_{\PP^{13}}^1 \otimes \OO_{\widetilde T}) \to \ldots ,
\]
where we have used that $H^2(\im d)=0$. But from the Euler sequence we get that the right term is also zero. 
\end{proof}

\bibliographystyle{alpha} 
\bibliography{bibliografi}

\end{document}
