\documentclass[11pt, english]{article}
%\usepackage[latin1]{inputenc}
\usepackage[T1]{fontenc}
\usepackage[utf8]{inputenc}
\usepackage[english]{babel}   % S P R A A K
% \usepackage{graphicx}    % postscript graphics
\usepackage{amssymb, amsmath, amsthm, amssymb} % symboler, osv
\usepackage{mathrsfs}
\usepackage{url}
\usepackage{thmtools}
\usepackage{enumerate}  % lister $  
\usepackage{float}
\usepackage{tikz}
\usepackage{tikz-cd}
\usetikzlibrary{calc}
%\usepackage{tikz-3dplot}
\usepackage{subcaption}
\usepackage[all]{xy}   % for comm.diagram
\usepackage{wrapfig} % for float right
\usepackage{hyperref}
\usepackage{mystyle} % stilfilen      

%\usepackage[a5paper,margin=0.5in]{geometry}


\begin{document}
\title{Results so far}
\author{Fredrik Meyer}
\maketitle 

\section{Introduction}

In the article \cite{gaifullin_triangulation}, it is proven that there exists a triangulation of $\C \PP^2$ with $15$ vertices and other nice properties [[expand]]. The triangulation is constructed by ``glueing'' cones of $3$-spheres along a triangle.  Call this triangulation for $\S$.

A smoothing of the associated Stanley-Reisner scheme $\PP(\S)$ would have interesting properties. In particular, it would be a hyper-Kähler variety [[EXPLAIN]]

The Hilbert polynomial of $\PP(\S)$ is $(9/2) t^{4}+(15/2) t^{2}+3$. The simplicial complex $\S$ have $f$-vector $(15,90,240,270,108)$. In particular, the degree of $\PP(\S)$ is $108$.

Here are a few computations:
\begin{lemma}
The module of first-order deformations of $\PP(\S)$ is $90$-dimensional, i.e. $\dim_k T^1(\PP(\S)) = 90$.
\end{lemma}
\begin{lemma}
The obstruction module have $\dim_k T^2(\PP(\K))=306$.
\end{lemma}

The link of $\mathcal S$ at one of its vertices is a particularly simple $3$-sphere, namely the join of the boundaries of two hexagons. Call this $\K$. Then $\K$ have $f$-vector $(1, 12, 48, 72, 36)$. In particular the Stanley-Reisner scheme $\PP(\K)$ have degree $36$.

Consider now $\K \ast \Delta^1$. This a cone over a ball, so it is topologically a $5$-dimensional ball. In fact, it is the join of two hexagons. So it is a $5$-dimensional ball. Consider now the polytope $P \ast P$ that is the convex hull of the columns of the matrix
\[
\left(
\begin{array}{*{12}c}
 1 & 1 & 0 & -1 & -1& 0 & 0 & 0 & 0 & 0 & 0 & 0 \\
 0 & 1 & 1 & 0 & -1 & -1& 0 & 0 & 0  & 0 & 0 & 0 \\
 0 & 0 & 0 & 0 & 0 & 0 & 1 & 1 & 0  & -1 & -1 & 0 \\
 0 & 0 & 0 & 0 & 0 & 0 & 0 & 1 & 1  & 0 & -1 & -1 \\
 0 & 0 & 0 & 0 & 0 & 0 & 1 & 1 & 1  & 1 & 1 & 1 \\
\end{array}
\right) \]

This polytope is the join of two \emph{actual} hexagons in $\R^5$. Thus, by standard Sturmfels theory, there is a Gröebner degeneration of the associated toric variety (whose fan is the polar polytope of $P$).

\section{Deformations of the 5-dimensional toric variety}

The $5$-dimensional toric variety $T$ had $2$-dimensional singularities (actually two disjoint copies of $dP_6$).

\begin{thm}
There exists a flat deformation of $\PP(\K \ast \Delta^1)$, $\mathfrak X \to S$, such that $\mathfrak X_{t_1} = T$ for some $t_1 \in S$ and such that the general fiber $\widetilde T$ have one-dimensional singularities.
\end{thm}
\begin{proof}
 As per now, the proof is purely by computer. The technique is this: First, consider the monomial degeneration of $T$ to the Stanley-Reisner ring $A(\K \ast \Delta^1)$ (recall that $\K=D_6 \ast D_6$). Choose deformation parameters $t_i$ perturbing the equations ``in the direction of $T$'', meaning that we only choose parameters introducing terms already occuring in the equations of $T$.

Now, using the package \verb|VersalDeformations|, it is possible to produce a flat family $\mathfrak X \to S'$ with $\mathfrak X_0 = \PP(\K \ast \Delta^1)$.

The base space is a union of toric varieties of dimensions $14,13,13,12$, respectively. Call the largest one for $S$. The equations are 
\[
\begin{vmatrix}
 t_1 & t_2 & t_2 &  \cdots & t_{6} \\
 t_7 & t_8 &  t_9 & \cdots & t_{12} 
\end{vmatrix} \leq 1 \qquad 
\begin{vmatrix}
 t_{13} & t_{14} & t_{15} &  \cdots & t_{18} \\
 t_{19} & t_{20}  & t_{21} & \cdots & t_{24} 
\end{vmatrix} \leq 1 
\]
By restriction, we get the claimed family $\mathfrak X \to T$. It is cheched by setting $t_i=i$ for $i=1,\ldots,12$ and $t_i=2i$ for $i=13,\ldots,24$ that the resulting fiber have one-dimensional singularities. The reason we don't set all $t_i=1$, is that this point lies in the intersection of the components of $S$. 
\end{proof}

\begin{corr}
The Stanley-Reisner scheme $\PP(\K)$ smooths to a smooth Calabi-Yau variety $X$.
\end{corr}
\begin{proof}
The scheme $\PP(\K)$ sits as a complete intersection in $\PP(\K \ast \Delta^1)$. Complete intersections deform together with the ambient variety, so $\PP(\K \ast \Delta^1)$ deforms to a general complete intersection in $\widetilde T$. Since $\widetilde T$ have curve singularities, it follows by two applications of Bertini's theorem \cite[Theorem II.8.18]{hartshorne}, that $X$ is smooth.
\end{proof}

Now what we would really like to do is to compute the Hodge numbers $h^{ij}=\dim_k H^j(X,\Omega^i_X)$ of $X$. 

We can however compute the Hodge numbers of $\widetilde T$. The hope is that there is some sort of Lefschetz theorem giving us the Hodge numbers of $X$.

\begin{thm}
  We have $h^{11}(\widetilde T)=1$ and $h^{12}(\widetilde T)=12$. 
\end{thm}
\begin{proof}
Again, this is purely computational. We use long exact sequences together with sheaf cohomology computations in \verb|Macaulay2|.

Since the ideal of $\widetilde T$ is rather complicated, doing this naïvely does not work. The trick is to choose the right term order. Since we know that $\widetilde T$ has a nice degeneration, we would like to find a term order such that its initial ideal is precisely the Stanley-Reisner ideal.

The \verb|Macaulay2| package \verb|gfanInterface| provides an interface with \verb|gfan|, which is a program that can compute weight vectors given polynomials with prescribed initial terms. The weight vector is 
\[ \omega = (1, 1, 4, 7, 7, 4, 1, 1, 4, 7, 7, 4, 1, 1). \]
With this term order, giving a very small Gröbner basis (18 elements), the computations are much faster than with the standard term order. We are able to compute resolutions of all the relevant modules within a few minutes in total.

We have an exact sequence of sheaves on $\widetilde T$:
\[
0 \to \mathscr T_1 \hookrightarrow \mathcal I/\mathcal I^2 \xrightarrow{d} \Omega_{\PP^{13}}^1 \otimes \OO_{\widetilde T} \to \Omega_{\widetilde T}^1 \to 0.
\]
This sequence can be broken into two short exact sequences. The relevant one is this:
\begin{equation}
\label{eqt1}
0 \to \im d \to \Omega_{\PP^{13}}^1 \otimes \OO_{\widetilde T} \to \Omega_{\widetilde T}^1 \to 0.  
\end{equation}

We also have the restriced Euler sequence:
\begin{equation}
\label{eqeuler}
0 \to \Omega_{\PP^{13}} \otimes \OO_{\widetilde T} \to \OO_{\widetilde T}(-1)^{14} \to \OO_{\widetilde   T} \to 0.
\end{equation}

We first compute $h^{11}$. From \eqref{eqt1}  we get a long exact sequence
\[
\ldots \to H^1(\im d) \to H^1(\Omega_{\PP^{13}}^1 \otimes \OO_{\widetilde T}) \to H^1(\Omega_{\widetilde T}^1) \to H^2(\im d) \to \ldots 
\]
The cohomology of $H^1(\im d)$ and $H^2(\im d)$ was computed with \verb|Macaulay2| to be both zero. Thus $H^1(\Omega_{\widetilde T}^1) \simeq H^1(\Omega_{\PP^{13}}^1 \otimes \OO_{\widetilde T})$. From the Euler sequence we get 
\[
\ldots \to H^0(\OO_{\widetilde T}(-1)^{14}) \to H^0(\OO_{\widetilde T}) \to H^1(\Omega_{\PP^{13}}^1 \otimes \OO_{\widetilde T}) \to H^1(\OO_{\widetilde T}(-1)^{14}) \to \ldots
\]
But the left and right terms are both zero. Hence $h^{11}=1$. We now compute $h^{12}$. 

From \eqref{eqt1} we again get 
\[
0 \to H^2(\Omega_{\PP^{13}}^1 \otimes \OO_{\widetilde T}) \to H^2(\Omega_{\widetilde T}^1) \to H^3(\im d) \to H^3(\Omega_{\PP^{13}}^1 \otimes \OO_{\widetilde T}) \to \ldots ,
\]
where we have used that $H^2(\im d)=0$. But from the Euler sequence we get that the right term is also zero. Thus $H^2(\Omega_{\widetilde T}^1) \simeq H^3(\im d)$. This last group can be computed in \verb|Macaulay2| to be $12$-dimensional. 
\end{proof}

By \verb|Macaulay2| computations we find that
\[
h^i(\widetilde{\im d}) = \begin{cases}
12 & i=3\\ 
2 & i = 4  \\
0 & \text{else},
\end{cases}
\]
and
\[
h^i(\widetilde{\im d}(-1)) = \begin{cases}
0 & i=0,1,2\\ 
24 & i = 3  \\
12 & i= 4 \\
18 & i = 5,
\end{cases}
\]
and
\[
h^i(\widetilde{\im d}(-2)) = \begin{cases}
0 & i=0,1,2\\ 
36 & i = 3  \\
24 & i= 4 \\
218 & i = 5,
\end{cases}
\]
In the same manner we find that:

\begin{prop}
We have $H^3(\Omega_Y^1)=2$. The other Hodge groups $H^i(\Omega_Y^1)=0$ (for $i=0,4,5$).
\end{prop}

By twisting all the exaxt sequences above, we can also calculate:
\begin{prop}
 We have
\[
h^i(\Omega_Y^1(-1)) = \begin{cases}
0 & i=0 \\
0 & i=1 \\
24 & i= 2 \\
12 & i = 3, \\
\end{cases}
\]
and also $h^4(\Omega_Y^1(-1))-h^5(\Omega_Y^1(-1))=4$.

Similarly:
\[
h^i(\Omega_Y^1(-2)) = \begin{cases}
0 & i=0 \\
0 & i=1 \\
36 & i= 2 \\
24 & i = 3, \\
\end{cases}
\]
and $h^4(\Omega_Y^1(-2))-h^5(\Omega_Y^1(-2))=23$.
\end{prop}

\begin{remark}
 The reader may wonder we just didn't ask \verb|Macaulay2| to compute the cohomology sheaf $\Omega_{\widetilde T}^1$ directly, e.g. by the command \verb|HH^i(cotangentSheaf Proj A)|. The reason is that \verb|Macaulay2|'s algorithims actually compute the \emph{sheaf}, and not just the dimension, and this is too computationally intensive.
\end{remark}

\begin{remark}[Question]
A computation reveals that $H^i(\mathcal I/\mathcal I^2) \simeq H^i(\im d)$ for $ i\geq 2$. This could be because the singularities are of dimension $1$. Is there a theoretical result to this effect?
\end{remark}

We can compute it \verb|Macaulay2| that:
\begin{lemma}
The first cotangent module of $\widetilde T$ has $\dim_k T^1(\widetilde T/k) = 26$. 
\end{lemma}

We have that $h^0(\PP(\K \ast \Delta^1),\mathcal T)=14$ by Theorem 5.2 in \cite{deforming_christophersen}. Thus these numbers fit in the narrative that we should have $T^1_{X_0} = h^1(\mathcal T_{X_t}) + h^0(\mathcal T_{X_0})$. (IS THERE ANY HEURISTIC FOR THIS??)

\section{Computing the Hodge numbers of X}

Since $X$ is a complete intersection of two hyperplanes in $Y$, we have an exact sequence
\[
0 \to \OO_Y(-2) \to \OO_Y(-1)^2 \to \mathscr I_{X/Y} \to 0,
\]
where $\mathscr I_{X/Y}$ is the ideal sheaf of $X$ in $Y$. We also have the sequence
\begin{equation}
\label{eqox}
0 \to \mathscr I_{X/Y} \to \OO_Y \to i^\ast \OO_X \to 0,  
\end{equation}
where $i:X \to Y$ is the inclusion. This allows us to compute the Hodge numbers of $X$:

\begin{thm}
There exists a non-singular Calabi-Yau with $X$ with $\chi(\Omega_X^1)=36$. 
\end{thm}

\begin{proof}
Since $X$ is a complete intersection in $Y$, we have $\mathscr I_{X/Y}/\mathscr I_{X/Y}^2 \simeq \OO_X(-1)^2$ as $\OO_X$-modules. Hence we have an exact sequence
\[
0 \to \OO_X(-1)^2 \to \Omega_Y^1 \otimes \OO_X \to \Omega_X^1 \to 0.
\]
The first sheaf have cohomology only in $H^3(\OO_X(-1))=H^0(\OO_X(1))=12$, which can be computed from its Stanley-Reisner degeneration. Hence the Euler characteristics are related by $\chi(\Omega_X^1) = \chi(\restr{\Omega_Y^1}X) + 24$.

Now tensor the exact sequence \eqref{eqox} with $\Omega_Y^1$ to get
\[
0 \to \mathscr I_{X/Y} \otimes \Omega_Y^1 \to \Omega_Y^1 \to \Omega_Y^1 \otimes \OO_X \to 0.
\]
Tensoring with $\Omega_Y^1$ is exact because the singularities of $Y$ lie outside $X$ (recall that the sheaf on the right is extended by zero outside $X$). Do the same with the $\OO_Y$-resolution of $\mathscr I_{X/Y}$ to get
\[
0 \to \Omega_Y^1(-2) \to \Omega_Y(-1)^2 \to \mathscr I_{X/Y} \otimes \Omega_Y \to 0.
\]
Taking Euler characteristics, we find that $\chi(\mathscr I_{X/Y} \otimes \Omega_Y)=-3$. Since $\chi(\Omega_Y^1)$ was computed to be $9$, it follows from the first exact sequence that $\chi(\Omega_X^1)=36$.
\end{proof}
\begin{remark}
The standard toric construction used on $T$ gives a Calabi-Yau $X'$ with $\chi(\Omega_{X'}^1,X')=-36$ with Hodge numbers $(44,8)$, so we would \emph{really} want our Calabi-Yau to have Hodge numbers $(8,44)$. In that case, it would be an example of an \emph{extremal transition}, in the sense of Morrison.
\end{remark}

To actually find the Hodge numbers, we need a few lemmas.

\begin{lemma}
\label{lemmanormal}
Let $\mathcal N_{X/\PP^{13}}$ be the normal sheaf of $X$ in $\PP^{13}$. Then $h^3(\mathcal I_X/\mathcal I_X^2) = h^0(\mathcal N_{X/\PP^{13}})$.
\end{lemma}
\begin{proof}
By Serre duality $h^{3-i}(\mathcal I_X/\mathcal I_X^2)=h^{i}((\mathcal I_X/\mathcal I_X^2)^\vee \otimes \omega)$, where $\omega$ is the dualizing sheaf. But $X$ is Calabi-Yau, so $\omega \simeq \OO_X$. The dual of $\mathcal I_X/\mathcal I_X^2$ is by definition the normal bundle.
\end{proof}

Consider the Euler sequence
\[
0 \to \Omega_{\PP^{13}} \otimes \OO_{X} \to \OO_{X}(-1)^{14} \to \OO_X \to 0.
\]

Since $X$ is a deformation of a Stanley-Reisner sphere, we know the cohomology of $\OO_X$. So we can extract the cohomology of $\Omega_{\PP^{13}} \otimes \OO_X$.

\begin{lemma}
We have
\[
H^i(\Omega_{\PP^{13}} \otimes \OO_X) = \begin{cases}
0 & \text{if } i=0 \\
1 & \text{if } i=1 \\
0 & \text{if } i=2 \\
167 & \text{if } i=3. \\
\end{cases}
\]
\end{lemma}
\begin{proof}
The full long exact sequence is:
\begin{align*}
0 & \to H^0(\Omega_{\PP^{13}} \otimes \OO_X) &\to H^0(\OO_X(-1)^{14}) &\to H^0(\OO_X) \to \\
H^1(\Omega_{\PP^{13}} \otimes \OO_X) &\to H^1(\OO_X(-1)^{14}) &\to H^1(\OO_X) \to \\
 H^2(\Omega_{\PP^{13}} \otimes \OO_X) &\to H^2(\OO_X(-1)^{14}) &\to H^2(\OO_X) \to \\
 H^3(\Omega_{\PP^{13}} \otimes \OO_X) &\to H^3(\OO_X(-1)^{14}) &\to H^3(\OO_X) \to 0
\end{align*}
Inserting the dimensions we know, we get:
\begin{align*}
0 & \to H^0(\Omega_{\PP^{13}} \otimes \OO_X) & 0  &\to 1 \to \\
H^1(\Omega_{\PP^{13}} \otimes \OO_X) &\to 0 &\to 0 \to \\
 H^2(\Omega_{\PP^{13}} \otimes \OO_X) &\to 0 &\to 0 \to \\
 H^3(\Omega_{\PP^{13}} \otimes \OO_X) &\to 168  &\to 1 \to 0
\end{align*}
Hence we conclude.
\end{proof}

Since $X$ is smooth, the conormal sequence is exact, so we have
\[
0 \to \mathcal I_X/\mathcal I_X^2 \to \Omega^1_{\PP^{13}} \otimes \OO_X \to \Omega_X^1 \to 0.
\]

\begin{lemma}
\label{lemmah21}
We have
\[
h^{21}(X) = h^0(\mathcal N_{X/\PP^{13}})-167.
\]
where $\chi$ denotes the Euler characteristic.
\end{lemma}
\begin{proof}
Write up the long exact sequence coming from the conormal sequence of $X$ and use Lemma \ref{lemmanormal}.
\end{proof}

\begin{lemma}
There is an exact sequence
\[
0 \to \OO_X(1)^2 \to \mathcal N_{X/\PP^{13}} \to \restr{\mathcal N_{Y/\PP^{13}} }{X} \to 0
\]
\end{lemma}
\begin{proof}
First note that there is an exact sequence of conormal sheaves:
\[
0 \to \restr{ \mathcal I_Y /\mathcal I_Y^2 }{X} \to \mathcal I_X/\mathcal I_X^2 \to \mathscr O_X(-1)^2 \to 0.
\]
The last term is $\mathcal N_{X/Y}$, since $X$ is a complete intersection in $Y$. Dualizing is exact even though $Y$ is not smooth, because by the long exact sequence of $\mathscr {E}xt$ sheaves, we must have $\mathscr{E}xt^1(\OO_X(-1),\OO_X)=0$. But this is true, since both of these are locally free.
\end{proof}

\begin{prop}
 We have 
\[
h^{21}(X) = 39.
\]
Hence $h^{11}(X)=3$.
\end{prop}
\begin{proof}
By lemma \ref{lemmah21}, we need to compute $h^0(\mathcal N_{X/\PP^{13}})$. By the previous lemma, we have $h^0(\mathcal N_{X/\PP^{13}})=\restr{\mathcal N_{Y/\PP^{13}}}{X}+24$. 

We have an exact sequence
\[
0 \to \mathcal I_{X/Y} \otimes \mathcal N_{Y/\PP^{13}} \to  \mathcal N_{Y/\PP^{13}} \to \restr{\mathcal N_{Y/\PP^{13}}}{X} \to 0.
\]
And also an exact sequence:
\[
0 \to \mathcal N_{Y/\PP^{13}}(-2) \to \mathcal N_{Y/\PP^{13}}(-1)^{\oplus 2} \to \mathcal I_{X/Y} \otimes \mathcal N_{Y/\PP^{13}} \to 0.
\]
The cohomology of $\mathcal N_{Y/\PP^{13}}(-i)$ is possible to compute in \verb|Macaulay2|, and it follows from the long exact sequence that $h^0(\mathcal I_{X/Y} \otimes \mathcal N_{Y/\PP^{13}})=36$, and $h^1(\mathcal I_{X/Y} \otimes \mathcal N_{Y/\PP^{13}})=0$. Hence it follows from the same computation that $h^0(\restr{\mathcal N_{Y/\PP^{13}}}{X})=182$ and that $h^0(\mathcal N_{X/\PP^{13}})=206$. We conclude that $h^{21}=206-167=39$.
\end{proof}

\bibliographystyle{alpha} 
\bibliography{bibliografi}

\end{document}
