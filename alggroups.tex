\documentclass[11pt, norsk]{article}
%\usepackage[latin1]{inputenc}
\usepackage[T1]{fontenc}
\usepackage[utf8]{inputenc}
\usepackage[norsk]{babel}   % S P R A A K


% \usepackage{graphicx}    % postscript graphics
\usepackage{amssymb, amsmath, amsthm, amssymb} % symboler, osv
\usepackage{mathrsfs}
\usepackage{url}
\usepackage{thmtools}
\usepackage{enumerate}  % lister $  
\usepackage{float}
\usepackage{tikz}
\usetikzlibrary{calc}
\usepackage{tikz-3dplot}
\usepackage{subcaption}
\usepackage[all]{xy}   % for comm.diagram
\usepackage{wrapfig} % for float right
\usepackage{hyperref}
\usepackage{mystyle} % stilfilen      


\begin{document}
\title{Algebraiske grupper og moduliteori}
\author{Fredrik Meyer}
\maketitle 

\section{Representations of algebraic groups}

Let $G = \Spec A$ be an affine algebraic group over a field $k$. 

\begin{defi}
 An \emph{algebraic representation of $G$} is a pair $(V,\mu_V)$ consisting of a $k$-vector space $V$ and a $k$-linear map $\mu_V: V \to V \otimes_k A$ satisfying the following two conditions:
 \begin{enumerate}
 \item The diagram
\[
\xymatrix{
V \ar[dr]^\id \ar[r]^{\mu_V} & V \otimes_k A \ar[d]^{\id \otimes \epsilon} \\
 & V \simeq V \otimes_k k
}
\]
is commutative.
\item The diagram
\[
\xymatrix{
V \ar[r]^{\mu_V} \ar[d]_{\mu_V} & V \otimes_k A \ar[d]^{\mu_V \otimes \id_A} \\
V \otimes_k A \ar[r]_{\id_V \otimes \mu_A} & V \otimes _k A \otimes_k A
}
\]
is commutative. Here $\mu_A$ is the coproduct in the coordinate ring of $G$.
 \end{enumerate}
\end{defi}

We often drop the subcript from $\mu_V$ unless confusion may arise. The same comment applies to tensor products. They will always be over the ground field unless otherwise stated. We will sometimes refer to a representation $(V,\mu_V)$ sometimes as ``a representation $\mu:V \to V \otimes A$'' and sometimes as just ``a representation $V$''.

\begin{defi}
Let $\mu:V \to V \otimes A$ be a representation of $G = \Spec A$. Then:
\begin{enumerate}
\item A vector $x \in V$ is said to be \emph{$G$-invariant} if $\mu(x) = x \otimes 1$.
\item A subspace $U \subset V$ is called a \emph{subrepresentation} if $\mu(U) \subseteq U \otimes A$. 
\end{enumerate}
\end{defi}

\begin{prop}
Every representation $V$ of $G$ is locally finite-dimensional. Precisely: every $x \in V$ is contained in a finite-dimensional subrepresentation of $G$.
\end{prop}

\begin{proof}
Write $\mu(x)$ as a finite sum $\sum_i x_i \otimes f_i$ for $x_i \in V$ and linearly independent $f_i \in A$. This we can always do, by definition of tensor product and bilinearity. Let $U$ be the subspace of $V$ spanned by the vectors $x_i$. 
\end{proof}

\end{document}
