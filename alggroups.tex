\documentclass[11pt, english]{article}
%\usepackage[latin1]{inputenc}
\usepackage[T1]{fontenc}
\usepackage[utf8]{inputenc}
\usepackage[english]{babel}   % S P R A A K


% \usepackage{graphicx}    % postscript graphics
\usepackage{amssymb, amsmath, amsthm, amssymb} % symboler, osv
\usepackage{mathrsfs}
\usepackage{url}
\usepackage{thmtools}
\usepackage{enumerate}  % lister $  
\usepackage{float}
\usepackage{tikz}
\usetikzlibrary{calc}
\usepackage{tikz-3dplot}
\usepackage{subcaption}
\usepackage[all]{xy}   % for comm.diagram
\usepackage{wrapfig} % for float right
\usepackage{hyperref}
\usepackage{mystyle} % stilfilen      


\begin{document}
\title{Algebraic groups and moduli theory}
\author{Fredrik Meyer}
\maketitle 

\abstract{
These are notes from the course Algebraic Geometry III.  
}

\section{Representation theory in general}

Let $V$ be a vector space. Briefly, a \emph{representation} of any group $G$ on $V$ is just a group homomorphism $\rho:G \to \GL(V)$.

\begin{example}
The \emph{trivial representation} is given by sending every $g \in G$ to the identity transformation.
\end{example}

\begin{example}
Suppose $G$ is a finite group. Then there is an embedding $G \hookrightarrow S_n$, and every element of $S_n$ can be represented by permutation matrices (that is, matrices $M_g$ such that $Me_i=e_{g(i)}$ for all $g \in G$). This defines a representation of $G$ in $k^n$. 
\end{example}

\begin{example}
Suppose $G$ acts on a (finite) set $X$. Let $V$ be the vector space with basis identified with the elements of $X$. Then $G$ acts on $V$ by linearity: for each $g \in G$, $\rho(g)$ is the linear map sending $e_x$ to $e_{gx}$. Such representations are called \emph{permutation representations}.
\end{example}

A \emph{morphism of representations} $(\rho,V)$,$(\rho',W)$ consists of commutative diagrams
\[
\xymatrix{
V \ar[r]^{\psi} \ar[d]_{\rho(g)}& W \ar[d]^{\rho'(g)} \\
V \ar[r]_{\psi} & W
}
\]
for each $g \in G$. Thus, if $\psi$ is invertible, this says that the linear operators $\rho(s),\rho'(s)$ are similar.

\section{Algebraic groups}

Algebraic groups are group objects in the category of affine varieties. More precisely:

\begin{defi}
Let $A$ be a finitely generated $k$-algebra. An \emph{affine algebraic group} is a quadruple $(A,\mu_A,\epsilon,\iota)$ where $\mu_A:A \to A \otimes_k A$ (the \emph{coproduct}), $\epsilon:A \to k$ (the \emph{coidentity}), $\iota: A \to A$ (the \emph{coinverse}) are $k$-algebra homomorphisms, satisfying the following conditions:
\begin{enumerate}
\item Coassociativity. The following diagram commutes:
\[
\xymatrix{
A \ar[r]^{\mu_A} \ar[d]_{\mu_A} & A \otimes_k A \ar[d]^{\id_A \otimes \mu_A} \\
A \otimes_k A \ar[r]_{\mu_A \otimes \id_A} & A \otimes_k A \otimes_k A
}
\]
\item The following diagram commutes:
\[
\xymatrix{
 & & k \otimes_k A  \ar[dr]^{\simeq} \\
A \ar[r]^\mu & A \otimes_k A \ar[ur]^{\epsilon \otimes \id_A} \ar[dr]_{\id_A \otimes \epsilon}  & & A \\
 & & A \otimes_k k \ar[ur]_{\simeq}
 }
\]
and is equal to the identity.
\item Inverse. The following diagram commutes:
\[
\xymatrix{
A \ar[d]^\mu \ar[rr]^\epsilon & & k \ar[d] \\
A \otimes_k A \ar[r]_{\id_A \otimes \iota} & A \otimes_k A \ar[r]_\cdot & A
}
\]
Here the right arrow is the morphism making $A$ a $k$-algebra. The last arrow in the lower sequence is multiplication in $A$.
\end{enumerate}
\end{defi}

\begin{example}
 Let $G$ be any group, and let $k[G]$ be its group ring. Let $A$ be its $k$-linear dual, that is $A=\Hom_k(k[G],k)$. This is a priori just another vector space, but we can give it the structure of a $k$-algebra by defining multiplication as follows: let $\lambda:k[G] \to k$,$\gamma:k[G] \to k$ be $k$-linear maps. It is enough to say what should happen on a basis, and a basis is given by the elements $g$ of $G$. Then, set $(\lambda \cdot \gamma)(g) = \lambda(g)\cdot \gamma(g)$.

Then set $\mu:A \to A \otimes A$ to be the dual of the multiplication map on $k[G]$. Explicitly, let $m:k[G] \otimes_k k[G] \to k[G]$ denoted the multiplication map. Let $\lambda:k[G] \to k$ be an element of $A$. Then we can form $m^\ast \lambda = \lambda \circ m$, which is an element of $(k[G] \otimes k[G])^\vee$. For finite-dimensional vector spaces, this is isomorphic to $A \otimes A$, which gives our multiplication map $\mu$. The coidentity is given by sending $\lambda:k[G]\to k$ to $\lambda(1_G)$, where $1_G \in G \subseteq k[G]$.

For example: let $G=C_n$ be the cyclic group of order $n$. Then $k[G] = k[t]/(t^n-1)$, and since this is finite-dimensional over $k$, we can find an isomorphism $k[G] \approx A$. Unwinding definitions, we see that [????] (I dont see this)
\end{example}

\begin{example}
 Let $A=k[s]$ be the polynomial ring in one variable. This is the coordinate ring of $\Aa_k^1$. We can define 
$$
\mu(s) = s \otimes 1 + 1 \otimes s.
$$
Also, $\epsilon(s)=0$, and $\iota(s)=-s$. 
\end{example}


\begin{defi}
\label{defaction}
An \emph{action} of an affine algebraic group $G = \Spec A$ on an affine variety $X= \Spec R$ is a morphism $G \times X \to X$ defined dually by a $k$-algebra morphism $\mu_R : R \to R \otimes_k A$ satisfying the following two conditions.
\begin{enumerate}
\item  The following diagram is commutative:
\[
\xymatrix{
R \ar[dr]^{\id_R} \ar[r]^{\mu_R} & R \otimes_k A \ar[d]^{\id_R \otimes \epsilon} \\
 & R \simeq R \otimes_k k
}
\]
\item The diagram
\[
\xymatrix{
R \ar[r]^{\mu_R} \ar[d]_{\mu_R} & R \otimes_k A \ar[d]^{\mu_R \otimes \id_A} \\
R \otimes_k A \ar[r]_{\id_R \otimes \mu_A} & R \otimes _k A \otimes_k A
}
\]
\end{enumerate}
\end{defi}


\section{Representations of algebraic groups}

Let $G = \Spec A$ be an affine algebraic group over a field $k$. 

\begin{defi}
 An \emph{algebraic representation of $G$} is a pair $(V,\mu_V)$ consisting of a $k$-vector space $V$ and a $k$-linear map $\mu_V: V \to V \otimes_k A$ satisfying the following two conditions:
 \begin{enumerate}
 \item The diagram
\begin{equation}
\label{eq:algrep1}
\xymatrix{
V \ar[dr]^{\id_V} \ar[r]^{\mu_V} & V \otimes_k A \ar[d]^{\id_V \otimes \epsilon} \\
 & V \simeq V \otimes_k k
}
\end{equation}
is commutative.
\item The diagram
\[
\xymatrix{
V \ar[r]^{\mu_V} \ar[d]_{\mu_V} & V \otimes_k A \ar[d]^{\mu_V \otimes \id_A} \\
V \otimes_k A \ar[r]_{\id_V \otimes \mu_A} & V \otimes _k A \otimes_k A
}
\]
is commutative. Here $\mu_A$ is the coproduct in the coordinate ring of $G$.
 \end{enumerate}
\end{defi}

\begin{remark}
In lieu of Definition \ref{defaction}, we see that any action of an algebraic group $G$ on an affine variety $X=\Spec R$ is a representation of $G$ on the infinite-dimensional $k$-vector space $R=\Gamma(X,\OO_X)$.
\end{remark}

\begin{remark}
Mumford calls this a \emph{dual action of $G$ on $V$}, in his 1965 book ``Geometric Invariant Theory''.
\end{remark}

We often drop the subcript from $\mu_V$ unless confusion may arise. The same comment applies to tensor products. They will always be over the ground field unless otherwise stated. We will sometimes refer to a representation $(V,\mu_V)$ sometimes as ``a representation $\mu:V \to V \otimes A$'' and sometimes as just ``a representation $V$''.

\begin{defi}
Let $\mu:V \to V \otimes A$ be a representation of $G = \Spec A$. Then:
\begin{enumerate}
\item A vector $x \in V$ is said to be \emph{$G$-invariant} if $\mu(x) = x \otimes 1$.
\item A subspace $U \subset V$ is called a \emph{subrepresentation} if $\mu(U) \subseteq U \otimes A$. 
\end{enumerate}
\end{defi}

\begin{prop}
Every representation $V$ of $G$ is locally finite-dimensional. Precisely: every $x \in V$ is contained in a finite-dimensional subrepresentation of $G$.
\end{prop}

\begin{proof}
Write $\mu(x)$ as a finite sum $\sum_i x_i \otimes f_i$ for $x_i \in V$ and linearly independent $f_i \in A$. This we can always do, by definition of tensor product and bilinearity. Let $U$ be the subspace of $V$ spanned by the vectors $x_i$. 

Now, by the commutativity of the diagram \eqref{eq:algrep1} it follows that $$ x = \sum_i \epsilon(f_i) x_i. $$

By the commutativity of the second diagram in the definition, it follows that
$$
\sum_i \mu_V (x_i) \otimes f_i = \sum_i x_i \otimes \mu_A(f_i) \in U \otimes A_k \otimes_k A.
$$

Because each term of the right-hand-side is contained in $U \otimes A \otimes A$, it follows that $\mu_V(x_i)$ is contained in $U$ since the $f_i$ are linearly independent.

Thus $x$ is contained in the finite-dimensional representation $\restr{\mu_V}{U}:U \to U \otimes A$.
\end{proof}

We can classify representations of $\Gr_m$ easily. They are all direct sums of ``weight $m$''-representations, that is, representations of the form $$V \to V \otimes k[t, t^{-1}], v \mapsto v \otimes t^m.$$ 

\begin{prop}
 Every representation $V$ of $\Gr_m$ is a direct sum $V = \oplus_{m \in \Z} V_{(m)}$, where each $V_{(m)}$ is a subrepresentation of weight $m$. 
\end{prop}

\begin{proof}
For each $m \in \Z$, define
\[
V_{(m)} = \{ v \in V \mid \mu(v) = v \otimes t^m \}.
\]
This is a subrepresentation of $V$: we must see that $\mu(V_{(m)}) \subset U \otimes A$, but this is true by construction. It is also clear that is has weight $m$. Next we show that $V = \oplus_{m \in \Z} V_{(m)}$. Write
\[
\mu(v) = \sum_{m \in \Z} v_m \otimes t^m \in V \otimes k[t,t^{-1}].
\]

Using the first condition in the definition of a representation, we get that $v = \sum_{m \in \Z} \epsilon(t^m)v_m$. It remains to check that each $v_m \in V_{(m)}$ (we can forget the scalars $\epsilon(t^m)$). But from definition ii), it follows that
\[
\sum \mu(v_m) \otimes t^m = \sum v_m \otimes t^m \otimes t^m,
\]
so that indeed $\mu(v_m)= v_m \otimes t^m$, as wanted.
\end{proof}

\begin{example}
An action of $\Gr_m$ on $X = \Spec R$ is equivalent to specifying a grading
\[
\xymatrix{
R = \oplus_{m \in \Z} R_{(m)} & R_{(m)}R_{(n)} \subset R_{(m+n)}.
}
\]
The invariants under this action are thus the homogeneous elements of weight zero, that is, the subring $R_{(0)}$. Moreover, we have a special operator. There is a linear endomorphism $E$ of $R$ that sends $f = \sum f_m \mapsto \sum m f_m$, and it is a derivation of $R$, called the Euler operator. We have $R^{\Gr_m}=\ker E$.

To see that $E$ is a derivation, we must check that $E(fg)=fE(g)+gE(f)$. The operator is homogeneous, so it is enough to check on homogeneous elements. So let $f_m,g_n$ be of degree $m,n$, respectively. Then
\[
E(f_mg_n)=(m+n)f_mg_n = g_n(mf_m)+f_m(ng_n)=g_nE(f_m)+f_mE(g_m),
\]
as wanted.
\end{example}


A character in regular representation theory is a homomorphism $G \to \C^\ast$, so do we have a corresponding notion of characters in this ``dual'' world:

\begin{defi}
Let $G=\Spec A$ be an affine algebraic group. A $1$-dimensional character of $G$ is a function $\chi \in A$ satisfying
\[
\xymatrix{
\mu_A(\chi) = \chi \otimes \chi & \iota(\chi)\chi = 1.
}
\]
\end{defi}

\begin{lemma}
The characters of the general linear group $\GL(n) = \Spec k[x_{ij}, \det X]$ are precisely the integer powers of the determinant $(\det X)^n$ for $n \in Z$.   
\end{lemma}

\begin{defi}
  Let $\chi$ be a character of an affine algebraic group $G$, and let $V$ be a representation of $G$. A vector $v \in V$ satisfying $$\mu_V(v) = v \otimes \chi$$ is called a \emph{semi-invariant} of $G$ with weight $\chi$. The semi-invariants of $V$ belonging to a given character $\chi$ form a subrepresentation $V_\chi \subset V$ of $V$.
\end{defi}

\subsection{Linear reductivity}

\begin{defi}
An algebraic group $G$ is said to be \emph{linearly reductive} if, for every epimorphism $\varphi:V \to W$ of $G$-representations, the induced map of $G$-invariants $\varphi^G:V^G \to W^G$ is surjective.
\end{defi}

For the following proposition, assume that $char k$ does not divide $ \lvert G \rvert$.

\begin{prop}
Every finite group $G$ is linearly reductive.
\end{prop}

\begin{proof}
Let $\varphi: V \to W$ be the given epimorphism of representations. Let $ R:V \to V^G \subset V$ be given by $v \mapsto \sum_{g \in G} g\cdot v$. Let $w \in W^G$. Then it is an easy calculation to check that $\varphi(R(v))=R(\varphi(v))$, from which it follows that $\varphi(R(v))=w$ (note that $\restr{R}{W^G}=\id_{W^G}$).
\end{proof}

The homomorphism $R$ above is called the \emph{Reynolds operator}.


\end{document}
