\documentclass[11pt, english]{article}
%\usepackage[latin1]{inputenc}
\usepackage[T1]{fontenc}
\usepackage[utf8]{inputenc}
\usepackage[english]{babel}   % S P R A A K


% \usepackage{graphicx}    % postscript graphics
\usepackage{amssymb, amsmath, amsthm, amssymb} % symboler, osv
\usepackage{mathrsfs}
\usepackage{url}
\usepackage{thmtools}
\usepackage{enumerate}  % lister $  
\usepackage{float}
\usepackage{tikz}
\usetikzlibrary{calc}
\usepackage{tikz-3dplot}
\usepackage{subcaption}
\usepackage[all]{xy}   % for comm.diagram
\usepackage{wrapfig} % for float right
\usepackage{hyperref}
\usepackage{mystyle} % stilfilen      

\usepackage[a5paper,margin=0.5in]{geometry}

\begin{document}
\title{Algebraic groups and moduli theory}
\author{Fredrik Meyer}
\maketitle 

\abstract{
These are notes from the course Algebraic Geometry III. We work over a field of characteristic zero. We start by doing some basic representation theory. Then we introduce algebraic groups. Then we study representations of algebraic groups. Finally we apply this to moduli problems. 
}

\section{Representation theory in general}

Let $V$ be a vector space. Briefly, a \emph{representation} of any group $G$ on $V$ is just a group homomorphism $\rho:G \to \GL(V)$.

\begin{example}
The \emph{trivial representation} is given by sending every $g \in G$ to the identity transformation.
\end{example}

\begin{example}
Suppose $G$ is a finite group. Then there is an embedding $G \hookrightarrow S_n$, and every element of $S_n$ can be represented by permutation matrices (that is, matrices $M_g$ such that $Me_i=e_{g(i)}$ for all $g \in G$). This defines a representation of $G$ in $k^n$. 
\end{example}

\begin{example}
Suppose $G$ acts on a (finite) set $X$. Let $V$ be the vector space with basis identified with the elements of $X$. Then $G$ acts on $V$ by linearity: for each $g \in G$, $\rho(g)$ is the linear map sending $e_x$ to $e_{gx}$. Such representations are called \emph{permutation representations}.
\end{example}

A \emph{morphism of representations} $(\rho,V)$,$(\rho',W)$ consists of commutative diagrams
\[
\xymatrix{
V \ar[r]^{\psi} \ar[d]_{\rho(g)}& W \ar[d]^{\rho'(g)} \\
V \ar[r]_{\psi} & W
}
\]
for each $g \in G$. Thus, if $\psi$ is invertible, this says that the linear operators $\rho(s),\rho'(s)$ are similar.

\section{Algebraic groups}

Algebraic groups are group objects in the category of affine varieties. More precisely:

\begin{defi}
\label{defalggrp}
Let $A$ be a finitely generated $k$-algebra. An \emph{affine algebraic group} is a quadruple $(A,\mu_A,\epsilon,\iota)$ where $\mu_A:A \to A \otimes_k A$ (the \emph{coproduct}), $\epsilon:A \to k$ (the \emph{coidentity}), $\iota: A \to A$ (the \emph{coinverse}) are $k$-algebra homomorphisms, satisfying the following conditions:
\begin{enumerate}
\item Coassociativity. The following diagram commutes:
\[
\xymatrix{
A \ar[r]^{\mu_A} \ar[d]_{\mu_A} & A \otimes_k A \ar[d]^{\id_A \otimes \mu_A} \\
A \otimes_k A \ar[r]_{\mu_A \otimes \id_A} & A \otimes_k A \otimes_k A
}
\]
\item The following diagram commutes:
\[
\xymatrix{
 & & k \otimes_k A  \ar[dr]^{\simeq} \\
A \ar[r]^\mu & A \otimes_k A \ar[ur]^{\epsilon \otimes \id_A} \ar[dr]_{\id_A \otimes \epsilon}  & & A \\
 & & A \otimes_k k \ar[ur]_{\simeq}
 }
\]
and is equal to the identity.
\item Coinverse. The following diagram commutes:
\[
\xymatrix{
A \ar[d]^\mu \ar[rr]^\epsilon & & k \ar[d] \\
A \otimes_k A \ar[r]_{\id_A \otimes \iota} & A \otimes_k A \ar[r]_\cdot & A
}
\]
Here the right arrow is the morphism making $A$ a $k$-algebra. The last arrow in the lower sequence is multiplication in $A$.
\end{enumerate}
\end{defi}

\begin{example}
If $G=\GL_n$, then $A=k[T_{ij},\det T]$. Then $\mu_A$ is given by
\[
T_{ij} \mapsto \sum_{h=1}^n T_{ih} \otimes T_{hj}.
\]
The coinverse is given by the usual Cramer's rule. Also $\epsilon(T_{ij})=\delta_{ij}$.
\end{example}

\begin{example}
If $G=\Gr_a=(\Aa^1,+)=\Spec k[X]$, then $\mu_A(X)=X \otimes 1 + 1 \otimes X$. The coidentity is $\epsilon(X)=0$, and the coinverse is $\iota(X)=-X$.
\end{example}

\begin{example}
 Let $A=k[s]$ be the polynomial ring in one variable. This is the coordinate ring of $\Aa_k^1$. We can define 
$$
\mu(s) = s \otimes 1 + 1 \otimes s.
$$
Also, $\epsilon(s)=0$, and $\iota(s)=-s$. 
\end{example}


\begin{defi}
\label{defaction}
An \emph{action} of an affine algebraic group $G = \Spec A$ on an affine variety $X= \Spec R$ is a morphism $G \times X \to X$ defined dually by a $k$-algebra morphism $\mu_R : R \to R \otimes_k A$ satisfying the following two conditions.
\begin{enumerate}
\item  The following diagram is commutative:
\[
\xymatrix{
R \ar[dr]^{\id_R} \ar[r]^{\mu_R} & R \otimes_k A \ar[d]^{\id_R \otimes \epsilon} \\
 & R \simeq R \otimes_k k
}
\]
\item The diagram
\[
\xymatrix{
R \ar[r]^{\mu_R} \ar[d]_{\mu_R} & R \otimes_k A \ar[d]^{\mu_R \otimes \id_A} \\
R \otimes_k A \ar[r]_{\id_R \otimes \mu_A} & R \otimes _k A \otimes_k A
}
\]
\end{enumerate}
\end{defi}


\section{Representations of algebraic groups}

Let $G = \Spec A$ be an affine algebraic group over a field $k$. 

\begin{defi}
 An \emph{algebraic representation of $G$} is a pair $(V,\mu_V)$ consisting of a $k$-vector space $V$ and a $k$-linear map $\mu_V: V \to V \otimes_k A$ satisfying the following two conditions:
 \begin{enumerate}
 \item The diagram
\begin{equation}
\label{eq:algrep1}
\xymatrix{
V \ar[dr]^{\id_V} \ar[r]^{\mu_V} & V \otimes_k A \ar[d]^{\id_V \otimes \epsilon} \\
 & V \simeq V \otimes_k k
}
\end{equation}
is commutative.
\item The diagram
\[
\xymatrix{
V \ar[r]^{\mu_V} \ar[d]_{\mu_V} & V \otimes_k A \ar[d]^{\mu_V \otimes \id_A} \\
V \otimes_k A \ar[r]_{\id_V \otimes \mu_A} & V \otimes _k A \otimes_k A
}
\]
is commutative. Here $\mu_A$ is the coproduct in the coordinate ring of $G$.
 \end{enumerate}
\end{defi}

\begin{remark}
In lieu of Definition \ref{defaction}, we see that any action of an algebraic group $G$ on an affine variety $X=\Spec R$ is a representation of $G$ on the infinite-dimensional $k$-vector space $R=\Gamma(X,\OO_X)$.
\end{remark}

\begin{remark}
Mumford calls this a \emph{dual action of $G$ on $V$}, in his 1965 book ``Geometric Invariant Theory''.
\end{remark}

We often drop the subcript from $\mu_V$ unless confusion may arise. The same comment applies to tensor products. They will always be over the ground field unless otherwise stated. We will sometimes refer to a representation $(V,\mu_V)$ sometimes as ``a representation $\mu:V \to V \otimes A$'' and sometimes as just ``a representation $V$''.

\begin{defi}
Let $\mu:V \to V \otimes A$ be a representation of $G = \Spec A$. Then:
\begin{enumerate}
\item A vector $x \in V$ is said to be \emph{$G$-invariant} if $\mu(x) = x \otimes 1$.
\item A subspace $U \subset V$ is called a \emph{subrepresentation} if $\mu(U) \subseteq U \otimes A$. 
\end{enumerate}
\end{defi}

\begin{prop}
\label{propfinite}
Every representation $V$ of $G$ is locally finite-dimensional. Precisely: every $x \in V$ is contained in a finite-dimensional subrepresentation of $G$.
\end{prop}

\begin{proof}
Write $\mu(x)$ as a finite sum $\sum_i x_i \otimes f_i$ for $x_i \in V$ and linearly independent $f_i \in A$. This we can always do, by definition of tensor product and bilinearity. Let $U$ be the subspace of $V$ spanned by the vectors $x_i$. 

Now, by the commutativity of the diagram \eqref{eq:algrep1} it follows that $$ x = \sum_i \epsilon(f_i) x_i. $$

By the commutativity of the second diagram in the definition, it follows that
$$
\sum_i \mu_V (x_i) \otimes f_i = \sum_i x_i \otimes \mu_A(f_i) \in U \otimes A_k \otimes_k A.
$$

Because each term of the right-hand-side is contained in $U \otimes A \otimes A$, it follows that $\mu_V(x_i)$ is contained in $U$ since the $f_i$ are linearly independent.

Thus $x$ is contained in the finite-dimensional representation $\restr{\mu_V}{U}:U \to U \otimes A$.
\end{proof}

We can classify representations of $\Gr_m$ easily. They are all direct sums of ``weight $m$''-representations, that is, representations of the form $$V \to V \otimes k[t, t^{-1}], v \mapsto v \otimes t^m.$$ 

\begin{prop}
\label{propgm}
 Every representation $V$ of $\Gr_m$ is a direct sum $V = \oplus_{m \in \Z} V_{(m)}$, where each $V_{(m)}$ is a subrepresentation of weight $m$. 
\end{prop}

\begin{proof}
For each $m \in \Z$, define
\[
V_{(m)} = \{ v \in V \mid \mu(v) = v \otimes t^m \}.
\]
This is a subrepresentation of $V$: we must see that $\mu(V_{(m)}) \subset U \otimes A$, but this is true by construction. It is also clear that is has weight $m$. Next we show that $V = \oplus_{m \in \Z} V_{(m)}$. Write
\[
\mu(v) = \sum_{m \in \Z} v_m \otimes t^m \in V \otimes k[t,t^{-1}].
\]

Using the first condition in the definition of a representation, we get that $v = \sum_{m \in \Z} \epsilon(t^m)v_m$. It remains to check that each $v_m \in V_{(m)}$ (we can forget the scalars $\epsilon(t^m)$). But from definition ii), it follows that
\[
\sum \mu(v_m) \otimes t^m = \sum v_m \otimes t^m \otimes t^m,
\]
so that indeed $\mu(v_m)= v_m \otimes t^m$, as wanted.
\end{proof}

\begin{example}
An action of $\Gr_m$ on $X = \Spec R$ is equivalent to specifying a grading
\[
\xymatrix{
R = \oplus_{m \in \Z} R_{(m)} & R_{(m)}R_{(n)} \subset R_{(m+n)}.
}
\]
The invariants under this action are thus the homogeneous elements of weight zero, that is, the subring $R_{(0)}$. Moreover, we have a special operator. There is a linear endomorphism $E$ of $R$ that sends $f = \sum f_m \mapsto \sum m f_m$, and it is a derivation of $R$, called the Euler operator. We have $R^{\Gr_m}=\ker E$.

To see that $E$ is a derivation, we must check that $E(fg)=fE(g)+gE(f)$. The operator is homogeneous, so it is enough to check on homogeneous elements. So let $f_m,g_n$ be of degree $m,n$, respectively. Then
\[
E(f_mg_n)=(m+n)f_mg_n = g_n(mf_m)+f_m(ng_n)=g_nE(f_m)+f_mE(g_m),
\]
as wanted.
\end{example}


A character in regular representation theory is a homomorphism $G \to \C^\ast$, so do we have a corresponding notion of characters in this ``dual'' world:

\begin{defi}
Let $G=\Spec A$ be an affine algebraic group. A $1$-dimensional character of $G$ is a function $\chi \in A$ satisfying
\[
\xymatrix{
\mu_A(\chi) = \chi \otimes \chi & \iota(\chi)\chi = 1.
}
\]
\end{defi}

\begin{lemma}
The characters of the general linear group $\GL(n) = \Spec k[x_{ij}, \det X]$ are precisely the integer powers of the determinant $(\det X)^n$ for $n \in Z$.   
\end{lemma}

\begin{defi}
  Let $\chi$ be a character of an affine algebraic group $G$, and let $V$ be a representation of $G$. A vector $v \in V$ satisfying $$\mu_V(v) = v \otimes \chi$$ is called a \emph{semi-invariant} of $G$ with weight $\chi$. The semi-invariants of $V$ belonging to a given character $\chi$ form a subrepresentation $V_\chi \subset V$ of $V$.
\end{defi}

We will often change the point of view depending upon the situation. Sometimes we think of a representation of an algebraic group as a $k$-linear map $V \to V \otimes_k A$ satisfying some axioms, and sometimes we think of a representation as a group $G$ acting on a vector space $V$ in the usual fashion.

\begin{prop}
 Let $\mu: V \to V \otimes_k A$ be a representation of an algebraic group $G$. Let $g \in G(k)$ be a $k$-valued point and $\mm_g \subseteq A$ the corresponding maximal ideal. Denote by $\rho(g)$ the composition
\[
V \xrightarrow{\mu} V \otimes_k A \xrightarrow {\mod \mm_g} V \otimes_k k \simeq V.
\]
Then, if $A=\Gamma(G,\OO_G)$ is an integral domain, a vector $v \in V$ such that $\rho(g)(v)=v$ for all $g \in G(k)$ is a $G$-invariant.
\end{prop}
\begin{proof}
We need to check that $\mu(v)=v \otimes 1$.  First, since $G$ is the spectrum of a finitely generated $k$-algebra, we can write $A$ as $k[y_1,\cdots,y_m]/I$ for some prime ideal $I$. Then the same trick as in the proof of Proposition \ref{propfinite} works. Write $\mu(v)=\sum v_i \otimes f_i$ with $f_i \in A$ for all $i$. Since the composition is the identity, we have that $f_i \equiv 1 \pmod{\mm_g}$ for all $g \in G$. This implies that $f_i-1$ is contained in the Jacobson radical of $A$. But $A$ is an integral domain, so $f_i-1=0$.
\end{proof}
Thus, in a sense, the two notions of $G$-invariance coincides.

\subsection{Algebraic groups and their Lie spaces}

[[something about local study]]

\begin{defi}
 Let $R$ be a $k$-algebra and $M$ an $R$-module. An \emph{$M$-valued derivation} is a $k$-linear map $D:R \to M$ satisfying the Leibniz rule $D(xy)=xD(y) + yD(x)$ for $x,y \in R$.
\end{defi}

The set of $M$-valued derivations is also an $R$-module, denoted by $\Der_k(R,M)$. This is used to define tangent spaces in algebraic geometry as follows: Let $p \in \Spec A=X$ be a closed point. Then we have a local ring $\OO_{X,p}$ and a quotient map $\OO_{X,p} \to \OO_{X,p}/\mm_p \simeq  k$. Then the $k$-module $(\mm_p/\mm_p^2)^\vee$ is called the \emph{Zariski tangent space} of $p \in X$. In fact:
\begin{prop}
\label{propzariski}
We have an isomorphism of $\OO_{X,p}$-modules:
$$ \Der_k(\OO_{X,p},k) \simeq (\mm_p/\mm_p^2)^\vee
$$
\end{prop}
\begin{proof}
Send $D \in \Der_k(\OO_{X,p},k)$ to $\restr{D}{\mm_p}$. This is well-defined if $D$ vanishes on $\mm_p^2$. But if $m_1,m_2 \in \mm_p$, then $$D(m_1m_2)=m_1D(m_2)+m_2D(m_1)=0 \in \OO_{X,p}/\mm_p.$$
So the map is well-defined.

We have a map in the opposite direction as well. Let $\ell:\mm_p/\mm_p^2 \to k$ be a linear functional on the $k$-vector space $\mm_p/\mm_p^2$. Define $D_\ell$ to be the $k$-linear map $D_\ell:\OO_{X,p} \to k$ given by
$$
D_\ell(f) = \begin{cases}
0 & \text{if $f \in k$} \\
\ell(f) & \text{if $f \in \mm_p$} \\
0 & \text{if $f \in \mm_p^2$}.
\end{cases}
$$
I know claim that this is a derivation. Write $f,g \in \OO_{X,p}$ as $c+m,c'+m'$ where $c$ is outside the maximal ideal and $m \in \mm_p$. Then
$$
D_\ell(fg)=D_\ell(cc'+cm'+c'm+m'm') = cD_\ell(m')+c'D_\ell(m),
$$
by definition of $D_\ell$.

It is easy to check that these two maps are inverse to each other.
\end{proof}

In fact, the module of derivations is what is called a \emph{corepresentable functor}. There is an $R$-module $\Omega_{R/k}$ such that $\Der_k(R,M)=\Hom_R(\Omega_{R/k},M)$, functorially in $M$. This is the module of \emph{Kähler differentials}.

A derivation is a $k$-linear map vanishing at $\mm_p^2$. The generalization of this are \emph{local distributions}:

\begin{defi}
Let $\mm_p \in Spec X$. A \emph{local distribution with support $\mm_p \in X$} is a $k$-linear map $\alpha:R \to k$ with the property that $\alpha(\mm_p^N)=0$ for some $N \in \N$.

The minimal $N$ such that $\alpha(\mm_p^{N+1})=0$ is called the \emph{degree} of the distribution. Thus the distributions of degree $1$ are the derivations (up to isomorphism).
\end{defi}

We can identify the set of local distributions of degree $\leq d$ with $k$-module $(R/\mm_p^{d+1})^\vee$. The surjections $R/\mm_p^{d+1} \to R/\mm_p^{d}$ induce injections $(R/\mm_p^d)^\vee \hookrightarrow (R/\mm_p^{d+1})^\vee$. We can thus identify the set of distributions supported at $p$ with $\varinjlim R/\mm_p^i = \bigcup_i R/\mm_p^i$.

\subsubsection{The distribution algebra}

If now $G=\Spec A$ is an affine algebraic group with coordinate ring $A$, then denote by $\HH(G)$ the vector space of distributions $\alpha:A \to k$ supported at $e \in G$. The Zariski tangent space at $e \in G$ is called the \emph{Lie space} of $G$ and is denoted by $\g \subset \HH(G)$.

\begin{defi}
 Let $\alpha,\beta \in \HH(G)$. Then we define the \emph{convolution product} $\alpha \star \beta$ to be the composition
$$
A \xrightarrow{\mu_A} A \otimes_k A \xrightarrow{\alpha \otimes \beta} k \otimes_k k \simeq k.
$$
\end{defi}

\begin{lemma}
The convolution product $\alpha \star \beta$ is again a local distribution supported at the identity with
$$
\deg \alpha \star \beta \leq \deg \alpha + \deg \beta.
$$
\end{lemma}
\begin{proof}
 [comes later]
\end{proof}

\begin{lemma}
The structure map $\epsilon:A \to k$ from Definition \ref{defalggrp} (``evaluation at the identity'') is an identity element for the convolution product.
\end{lemma}
\begin{proof}
This follows from the following diagram:
\[
A \xrightarrow{\mu_A} A \otimes_k A \xrightarrow{\epsilon \otimes \id} k \otimes A \xrightarrow{\id \otimes \alpha} k \otimes k \simeq k.
\]
The composition is equal to $\epsilon \star \alpha$, but by Part 2 of Definition \ref{defalggrp}, it is also equal to $\alpha$.
\end{proof}

[...]

\subsubsection{The Casimir operator}



\subsection{Linear reductivity}

\begin{defi}
An algebraic group $G$ is said to be \emph{linearly reductive} if, for every epimorphism $\varphi:V \to W$ of $G$-representations, the induced map of $G$-invariants $\varphi^G:V^G \to W^G$ is surjective.
\end{defi}

\begin{prop}
Every finite group $G$ is linearly reductive.
\end{prop}

\begin{proof}
Let $\varphi: V \to W$ be the given epimorphism of representations. Let $ R:V \to V^G \subset V$ be given by $v \mapsto \sum_{g \in G} g\cdot v$. Let $w \in W^G$. Then it is an easy calculation to check that $\varphi(R(v))=R(\varphi(v))$, from which it follows that $\varphi(R(v))=w$ (note that $\restr{R}{W^G}=\id_{W^G}$).
\end{proof}

The homomorphism $R$ above is called the \emph{Reynolds operator}.

\begin{prop}
\label{proplinred}
The following are equivalent:
\begin{enumerate}[i)]
\item $G$ is linearly reductive.
\item For every epimorphism $V \to W$  of \emph{finite-dimensional} $G$-representations, the induced map $V^G \to W^G$ is surjective.
\item If $V$ is any finite-dimensional representation and $U \subseteq$ is a proper subrepresentation and $\bar v \in V/U$ is $G$-invariant, then the coset $v + U$ (for any lifting of $\bar v$) contains a non-trivial $G$-invariant vector.
\end{enumerate}
\end{prop}
\begin{proof}
$i) \Rightarrow ii)$ is trivial. For $ii) \Rightarrow iii)$, apply $ii)$ to the quotient map $V \to V/U$. Then $V^G \to (V/U)^G$ is surjective. This implies that for every nonzero $\bar v \in (V/U)^G$, there exists a $G$-invariant $v \in \pi^{-1}(v)=U+\bar v$.

$iii) \Rightarrow i)$ is hardest. Suppose $\phi:V \to W$ is an epimorphism of representations (not necessarily finite-dimensional). Suppose $\phi(v)=w \in W^G$ for some $v \in V$. 

By Proposition \ref{propfinite} there exists a finite-dimensional subrepresentation $V_0 \subseteq V$ containing $v$. Now $v \in V_0$ is $G$-invariant modulo $U_0 := V_0 \cap \ker \phi$ (since $V/\ker \phi \simeq W$ as $G$-representations), so by $iii)$, there exists a $G$-invariant vector $v' \in V_0$ such that $v'-v \in U_0$. But $\phi(v')=w$, so $\phi^G:V^G \to W^G$ is surjective.
\end{proof}

\begin{lemma}
Direct products of linearly reductive groups are linearly reductive. If $H \subset G$ is a normal subgroup and $G$ is linearly reductive, then so is $G/H$. Moreover, if both $H$ and $G/H$ are linearly redutive, then so is $G$.
\end{lemma}
\begin{proof}
Suppose given an endomorphism of representation of $G \times H$: $V \to W$. In particular, they are representations of $G,H$ separately, by the rule $g \cdot v = (g,e) \cdot v$. In particular, if an element $w \in W$ is $G \times H$-invariant, it is also $G,H$-invariant. Thus by assumption, there is an $G,H$-invariant $v \in V$ mapping to $w$. But if something is $G,H$-invariant, it is also $G \times H$-invariant, since $G,H$ commute in $G \times H$.

Similarly, every $G/H$-representation gives a $G$-representation, by the rule $g \cdot v = \bar g \cdot v$, where $\bar g$ denotes the class of $g$ in $G/H$. Now if $w \in W^{G/H}$ is $G/H$-invariant, then it is by definition $G$-invariant, and by linearly reductivity of $G$, the map is surjective.

Finally, if both $H$ and $G/H$ are linearly reductive, suppose $\phi:V \to W$ is a surjection of $G$-representations. This is also a surjection of $H$-representations, and since $H$ was linearly reductive, we get that $V^H \to W^H$ is surjective. It follows that the map $\phi$ and the vector spaces $V,W$ splits as $(\phi^H,\phi'):V^H \oplus V' \to W^H \oplus W'$, where $H$ acts trivially on the second factor. This implies that $G/H$ acts on $V',W'$, and it follows that $V' \to W'$ is surjective.
\end{proof}

\begin{prop}
Every algebraic torus $(\Gr_m)^N$ is linearly reductive.
\end{prop}
\begin{proof}
By the lemma, it suffices to prove this for $N=1$. We use Proposition \ref{proplinred} iii). By Proposition \ref{propgm}, we can write a representation $V$ and a subrepresentation $U$ as
\begin{align*}
V = \bigoplus_{m \in \Z} V_{(m)} & &\text{and}& & U = \bigoplus_{m \in \Z} U_{(m)}.
\end{align*}
Here $U_{(m)} \subset V_{(m)}$. An element $v \in V/U$ is $\Gr_m$-invariant if any lifting of $v$ to $V$ lies in $U_{(m)}$ for $m \neq 0$. Thus $v_{(0)}$ is $\Gr_m$-invariant and lies in the coset $v+U$.
\end{proof}

The classical example of a group that is not linearly reductive is the affine line $\Aa^1$ under addition:
\begin{example}
 Consider the $2$-dimensional representation given by
\[
\Gr_a \to \GL_2, \qquad t \mapsto \begin{pmatrix} 1 & t \\ 0 & 1 \end{pmatrix}.
\]
This is a representation by the rules of matrix multiplication. Algebraically, this as follows: let $x,y$ be a basis for $V$. Then we define a $k$-linear map $V \to V \otimes_k k[t]$ by $x \mapsto x \otimes 1$ and $y \mapsto x \otimes t + y \otimes 1$. This extends to a representation of $k[V]=k[x,y]$ in the obvious way. Then we can define an epimorphism of representations by sending $k[x,y] \to k[x,y]/(x) \simeq k[y]$. Taking invariants, we get that $k[x,y]^{\Gr_a} = k[x]$ but $k[y]^{\Gr_a}=k[y]$, but the map sends $x$ to $0$, so is not surjective.
\end{example}

\section{Moduli theory}

The problem of moduli theory is \emph{classification}. This can be made precise in various ways, depending upon the sophistication of the reader.

Let $V$ be an $n$-dimensional vector space. Suppose you want to classify $k$-planes in $V$. Up to isomorphism, this is trivial, since all vector spaces are determined once the integer $k$ is given. In that sense, the moduli space is just the point $\{k \}$. However, let us instead look at the whole set
$$ \Gr(k,n) = \{ L \mid \text{ L is a $k$-plane in $V$} \}.$$
So far, this is just a set. What does it mean to give a $k$-plane? It is certainly sufficient to give a spanning set, i.e. $k$ linearly independent vectors in $V$. These are elements of an open subset $U$ of $\Aa^{kn}$. But this is redundant: we have an action of $G=\GL_d$ on $\Aa^{kn}$. Two elements $v,v' \in \Aa^{kn}$  represent the same $k$-plane if and only if $v =gv'$ for $g \in \GL_d$.

We are thus led to ask for the existence of $U/\GL_d$. It turns out that this object exists. It is a projective variety called the Grassmannian $\Gr(k,n)$.

\end{document}
