\documentclass[11pt, english, a4paper]{article}


\usepackage[T1]{fontenc}
\usepackage[utf8]{inputenc}
\usepackage[norsk]{babel}   % S P R A A K
% \usepackage{graphicx}    % postscript graphics
\usepackage{amssymb, amsmath, amsthm} % symboler, osv
\usepackage{mathrsfs}
\usepackage{url}
\usepackage{thmtools}
\usepackage{enumerate}  % lister $  
\usepackage{float}
\usepackage{subcaption}

\usepackage{tikz}
\usetikzlibrary{cd}


\usepackage{wrapfig} % for float right
\usepackage{hyperref}



%%%% start pakke for å inkludere kode
\usepackage{listings}
\usepackage{color}
\definecolor{dkgreen}{rgb}{0,0.6,0}
\definecolor{gray}{rgb}{0.5,0.5,0.5}
\definecolor{mauve}{rgb}{0.58,0,0.82}

\lstset{frame=tb,
  language=Python,
  aboveskip=3mm,
  belowskip=3mm,
  showstringspaces=false,
  columns=flexible,
  basicstyle={\small\ttfamily},
  numbers=none,
  numberstyle=\tiny\color{gray},
  keywordstyle=\color{blue},
  commentstyle=\color{dkgreen},
  stringstyle=\color{mauve},
  breaklines=true,
  breakatwhitespace=true,
  tabsize=3
}
%%% slutt pakke for å inkludere kode

\usepackage{mystyle} % stilfilen  



\begin{document}
\title{Mal til en slags artikkel}
\author{Fredrik Meyer}
\maketitle 

\begin{abstract}
\noindent
Sammendraget er en kort beskrivelse av hva resten av dokumentet handler om. Her skriver du hva temaet er, og hva du har tenkt å bevise. Det er ikke en roman du skriver, så spoilers skal ikke unngås.
\end{abstract}

\section{Introduksjon}

I introduksjonen skal du motivere resten av artikkelen. Du forteller gjerne litt om hvorfor problemet er interessant, litt om historien bak, og hva andre folk har klart å gjøre før.

Deretter beskriver du i mer detalj hva du skriver om i de resterende seksjonene.  %% sett inn dette selv

\section{Hvordan komme i gang med \LaTeX}

Det første du må gjøre er å skaffe \LaTeX. Er du Linux-bruker, kan du følge oppskriften på \url{http://goo.gl/klYY1C} (en stackoverflow-tråd) for å installere på egen maskin. Kjører du Windows, kan du følge oppskriften som ligger ute på IFI sine nettsider: \url{http://goo.gl/XsF5kf}. Bruker du Mac, kan du laste ned herfra: \url{https://tug.org/mactex/}.

Du vil også trenge en editor. Selv bruker jeg \texttt{TexShop} på Mac, og \texttt{Emacs} på Linux. Her er en fin liste over editorer: \url{http://goo.gl/PQi2nK}. Er du på team \texttt{Vim}, finnes det også \LaTeX-pakker til denne. Det viktigste er at du har en hurtigtast for å kompilere (som du vil komme til å gjøre ofte).

Deretter må du lære deg å skrive dokumenter i \LaTeX. Dette gjøres best ved å hoppe ut i det. Skriv obliger i \LaTeX. Slutt med Word og OpenOffice og hold deg til et ordentlig tekstsystem.

Les andres kode. Det er det dette dokumentet skal hjelpe deg å gjøre.

\section{Litt matematikk}

Matematiske symboler og uttrykk skrives enten innad i en linje, eller på en egen linje. Det anbefales for eksempel ikke å ha lange uttrykk med summetegn og det som verre er innad i en linje.

Kanskje skriver du om analyse, og da snakker man gjerne om en \(\epsilon > 0 \). Nå brukte jeg \verb|\(| og \verb|\)| for å begrense matematikkmodusen. Det går også an å bruke dollartegn: \verb| $\epsilon > 0$|. Det har samme effekt, men førstnevnte er snillere med feilmeldinger.

Noen ganger har du lengre uttrykk:

\begin{equation}
f(z) = \sum_{i=0}^\infty \left( 
\frac{\int_\gamma g(w) dw}{i!}
\right).
\end{equation}

For at parenteser skal få riktig størrelse, skriv \verb|\left(| og \verb|\right)| i stedet for kun \texttt{(} og \texttt{)}.

Noen ganger ønsker du ikke nummerering. Den enkleste måten er da å bruke \verb|\[ ... \]|:

\[
\tr \begin{pmatrix}
x_1^1 & \hdots & x_n^1 \\
\vdots & \ddots &  \vdots \\
x_1^n & \hdots & x_n^n
\end{pmatrix} = \sum_{i=1}^n x_i^i.
\]

Andre ganger ønsker du å skrive opp lemmaer, teoremer og bevis. Jeg bruker kommandoene \verb|\begin{prop}...\end{prop}| for proposisjoner og teoremer (se øverst i kildekoden til dette dokumentet). 


\section{Noen eksempler på nyttige pakker}

Kanskje ønsker du å vise kode eller algoritmer i dokumentet ditt. Dette kan gjøres med pakken \texttt{Listings}. Se kildekoden til dette dokumentet for et eksempel. 

\begin{lstlisting}
def approximateFraction(a,n):
	'''
	Input: a real number a natural number n.
	Output: a pair [k,m], such that the fraction k/m is close 
	to a and  1<= m < n.
	'''
	L = [i*a-floor(i*a) for i in range(0,n+1)]
	mindiff = [1,0,0]
	for i in range(len(L)):
		for j in range(i+1,len(L)):
			if abs(L[i]-L[j]) < abs(mindiff[0]):
				mindiff = [(L[i]-L[j]),i,j]
	i,j = mindiff[1], mindiff[2]
	k = -int(round(L[i]-L[j]-(i-j)*a))
	m = i-j
	return [k,m]
\end{lstlisting}

Andre ganger ønsker du kanskje å tegne kommutative diagrammer. Her anbefales \texttt{TikZ}. 

\[
\begin{tikzcd}
T
\arrow[drr, bend left, "x"]
\arrow[ddr, bend right, "y"]
\arrow[dr, dotted, "{(x,y)}" description] & & \\
& X \times_Z Y \arrow[r, "p"] \arrow[d, "q"]
& X \arrow[d, "f"] \\
& Y \arrow[r, "g"]
& Z
\end{tikzcd}
\]



\section{Videre spørsmål}

Det er ofte vanlig å avslutte en artikkel med en diskusjon av hvilke problemer det gjenstår å løse. Om artikkelen din var mer en introduksjon eller en oversiktsartikkel, er det vanlig å snakke litt om hva du \emph{ikke} snakket om.

For eksempel, i dette dokumentet, har jeg unngått å snakke om en del avanserte deler av \LaTeX, 

\end{document}