\documentclass[11pt, english, a4paper]{article}


\usepackage[T1]{fontenc}
\usepackage[utf8]{inputenc}
\usepackage[norsk]{babel}   % S P R A A K
% \usepackage{graphicx}    % postscript graphics
\usepackage{amssymb, amsmath, amsthm} % symboler, osv
\usepackage{mathrsfs}
\usepackage{url}
\usepackage{thmtools}
\usepackage{enumerate}  % lister $  
\usepackage{float}
\usepackage{subcaption}
\usepackage[all]{xy}   % for comm.diagram
\usepackage{wrapfig} % for float right
\usepackage{hyperref}
\usepackage{mystyle} % stilfilen  



\begin{document}
\title{Res}
\author{Fredrik Meyer}
\maketitle 

\begin{abstract}
\noindent
Dette er en kort beskrivelse av hva resten av dokumentet handler om. Her skriver du hva temaet er, og hva du har tenkt å bevise. Det er ikke en roman du skriver, så spoilers skal ikke unngås.
\end{abstract}

\section{Introduksjon}

I introduksjonen skal du motivere resten av artikkelen. Du forteller gjerne litt om hvorfor problemet er interessant, litt om historien bak, og hva andre folk har klart å gjøre før.

Deretter beskriver du i mer detalj hva du skriver om i de resterende seksjonene. 

\section{Noen ord om \LaTeX}

Det første du må gjøre er å skaffe \LaTeX. Er du Linux-bruker, kan du følge oppskriften på \url{http://goo.gl/klYY1C} (en stackoverflow-tråd) for å installere på egen maskin. Kjører du Windows, kan du følge oppskriften som ligger ute på IFI sine nettsider: \url{http://goo.gl/XsF5kf}. Bruker du Mac, kan du laste ned herfra: \url{https://tug.org/mactex/}.

Du vil også trenge en editor. Selv bruker jeg \texttt{TexShop} på Mac, og \texttt{Emacs} på Linux. Her er en fin liste over editorer: \url{http://goo.gl/PQi2nK}. Er du på team \texttt{Vim}, finnes det også \LaTeX-pakker til denne. Det viktigste er at du har en hurtigtast for å kompilere (som du vil komme til å gjøre ofte).


\section{Litt matematikk}

Matematiske symboler og uttrykk skrives enten innad i en linje, eller på en egen linje. Det anbefales for eksempel ikke å ha lange uttrykk med summetegn og det som verre er innad i en linje.

Kanskje skriver du om analyse, og da snakker man gjerne om en \(\epsilon > 0 \). Nå brukte jeg \verb|\(| og \verb|\)| for å begrense matematikkmodusen. Det går også an å bruke dollartegn: \verb| $\epsilon > 0$ |. Det har samme effekt, men førstnevnte er snillere med feilmeldinger.

Noen ganger har du lengre uttrykk:

\begin{equation}
f(z) = \sum_{i=0}^\infty \left( 
\frac{\int_\gamma g(w) dw}{i!}
\right).
\end{equation}

For at parenteser skal få riktig størrelse, skriv \verb|\left(| og \verb|\right)| i stedet for kun \texttt{(} og \texttt{)}.

Noen ganger ønsker du ikke nummerering. Den enkleste måten er da å bruke \verb|\[ ... \]|:

\[
\tr \begin{pmatrix}
x_1^1 & \hdots & x_n^1 \\
\vdots & \ddots &  \vdots \\
x_1^n & \hdots & x_n^n
\end{pmatrix} = \sum_{i=1}^n x_i^i.
\]

Andre ganger ønsker du å skrive opp lemmaer, teoremer og bevis. Jeg bruker kommandoene \verb|\begin{prop}...\end{prop}| for proposisjoner og teoremer (se øverst i kildekoden til dette dokumentet). 

\section{Litt mer avansert matematikk}

\section{Videre spørsmål}

Det er ofte vanlig å avslutte en artikkel med en diskusjon av hvilke problemer det gjenstår å løse. Om artikkelen din var mer en introduksjon eller en oversiktsartikkel, er det vanlig å snakke litt om hva du \emph{ikke} snakket om.

For eksempel, i dette dokumentet, har jeg unngått å snakke om en del avanserte deler av \LaTeX, 

\end{document}