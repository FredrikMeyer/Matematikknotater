\documentclass[11pt, english, a4paper]{article}


\usepackage[T1]{fontenc}
\usepackage[utf8]{inputenc}
\usepackage[norsk]{babel}   % S P R A A K
% \usepackage{graphicx}    % postscript graphics
\usepackage{amssymb, amsmath, amsthm} % symboler, osv
\usepackage{mathrsfs}
\usepackage{url}
\usepackage{thmtools}
\usepackage{enumerate}  % lister $  
\usepackage{float}
\usepackage{subcaption}
\usepackage[all]{xy}   % for comm.diagram
\usepackage{wrapfig} % for float right
\usepackage{hyperref}
\usepackage{mystyle} % stilfilen  



\begin{document}
\title{Ideer til prosjektoppgaver}
\author{Fredrik Meyer}
\maketitle 

\section{Kvaternionene}

Kvaternionene er et $4$-dimensjonalt tallsystem som har den egenskapen at alle tall har en multiplikativ invers. Dette er sant for reelle og komplekse tall (henholdsvis $1$- og $2$-dimensjonalt). Når vi går opp til $4$ dimensjoner ``mister'' vi kommutativitet.

Denne oppgaven kan gå ut på å først beskrive hva kvaternionene er, samt snakke om anvendelser ...............

\section{Lineære coder/feilrettende koder}

...

\section{???}



\end{document}