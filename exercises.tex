\documentclass[11pt, english]{article}
%\usepackage[latin1]{inputenc}
\usepackage[T1]{fontenc}
\usepackage[utf8]{inputenc}
\usepackage[english]{babel}   % S P R A A K
% \usepackage{graphicx}    % postscript graphics
\usepackage{amssymb, amsmath, amsthm, amssymb} % symboler, osv
\usepackage{mathrsfs}
\usepackage{url}
\usepackage{thmtools}
\usepackage{enumerate}  % lister $  
\usepackage{float}
\usepackage{tikz}
\usepackage{tikz-cd}
\usetikzlibrary{calc}
%\usepackage{tikz-3dplot}
\usepackage{subcaption}
\usepackage[all]{xy}   % for comm.diagram
\usepackage{wrapfig} % for float right
\usepackage{hyperref}
\usepackage{mystyle} % stilfilen      

%\usepackage[a5paper,margin=0.5in]{geometry}


\begin{document}
\title{Exercises}
\author{Fredrik Meyer}
\maketitle 

I solve and type exercises from different places (read \emph{books}). 

\section{Algebraic Geometry - Hartshorne}

\subsection{Chapter I - Varieties}

\begin{exc}[Exercise 1.1]
  \begin{enumerate}[a)]
  \item Let $Y$ be the plane curve $y=x^2$. Show that $A(Y)$ is isomorphic to a polynomial ring in one variable over $k$.
\item Let $Z$ be the plane curve $xy=1$. Show that $A(Z)$ is not isomorphic to a polynomial ring in one variable over $k$.
\item Let $f$ be any irreducible quadratic polynomial in $k[x,y]$, and let $W$ be the conic defined by $f$. Show that $A(W)$ is isomorphic to $A(Y)$ or $A(Z)$. Which one is it when?
  \end{enumerate}
\end{exc}
\begin{sol}
  \begin{enumerate}[a)]
  \item  We have $A(Y)=k[x,y]/(y-x^2)$. An isomorphism $A(Y) \to k[t]$ is given by $x \mapsto t$ and $y \mapsto t^2$. 
\item We have $A(Z) = k[x,y]/(xy-1) \simeq k[x,\frac 1x]$. So we must show that $k[x,\frac 1x] \not \approx k[x]$. It can be computed that the first one has automorphisms given by $x \mapsto cx^n$ for $c$ nonzero and $n \neq 0$. The second has as automorphisms $ax+b$ ($a \neq 0$). So the first one have an abelian automorphism group, the second has not.
\item What is special about $A(Y)$ and $A(Z)$? Staring at pictures, we see that any line in $\Aa^2$ intersects $Y$ in at least one point, but in the case of $Z$, there exist two lines which do not intersect $Z$. We claim that this is the only two things that can happen.

First we claim that if we are in the second situation, that is, if there exist a pair of lines $\ell,\ell'$ such that $W \cap \ell = W \cap \ell' =\emptyset$, then $W \simeq Z$. 

A general quadric can be written as
\[
ax^2+bxy+cy^2+dx+ey+f=0.
\]
Suppose now $\ell \cap W=\emptyset$. This is equivalent to $I(f,\ell^\vee)=(1)$. Without loss of generality, we can assumme $\ell = \{ x = 0 \}$. Then
\[
I(f,\ell) = (cy^2+ey+f,x).
\]
This generates $k[x,y]$ if and only if $c=e=0$ and $f \neq 0$. Thus $f$ must be of the form
\[
ax^2+bxy+dx+f=0
\]
with $f \neq 0$. But this can be written as
\[
x(ax+by+d)+f = 0.
\]
Put $y' = ax+by+d$. Then $I(W)$ takes the form $(xy'+f=0)$, which is clearly isomorphic to $Z$ after a linear change of coordinates. Note that the other line not meeting $W$ is the line given by $y'=ax+by+d=0$.

Assume now that we are in the other situation, namely that \emph{every} line in $\Aa^2$ meets $W$. Now pick a tangent line $\ell$ of $W$. Without loss of generality, we can assume that $\ell$ is $\{ y=0 \}$. This is a tangent line if and only if it meets $W$ doubly, meaning that $I(W) + (\ell^\vee)$ takes the form $(l^2,y)$ for some linear form $l$. We can also assume that $\ell \cap W = (0,0)$, so that $I(W) + (\ell^\vee) =(x^2,y)$. But this means that
\begin{align*}
  I(W) + I(\ell) &= (ax^2+bxy+cy^2+dx+ey+f,y) \\
&= (ax^2+dx+f,y)
\end{align*}
We want $ax^2+dx+f=x^2$. This can happen only if $d=f=0$ and $a \neq 0$. Thus the quadric takes the form
\[
ax^2+bxy+cy^2+ey=0.
\]
Now we claim that there exist one line at each point of $W$ that intersect $W$ transversally in exactly one point. This is the case for $Y$. Consider the pencil of lines through $(0,0)$ defined by $x=\lambda y$. We want to find $\lambda$ such that the intersection is transversal and only one point. We have
\[
( ax^2+bxy+cy^2+ey, x-\lambda y) = \left( (a\lambda ^2+b\lambda+c)y^2+ey,x-\lambda y \right).
\]
This have exactly one solution if and only if $a\lambda^2+b\lambda+c=0$. This is solvable since $a \neq 0$ and since all lines intersect $W$. Thus choose $\lambda$ as above. We can rotate this line such that it becomes $x=0$. Then the equation takes the form
\[
ax^2+bxy+ey=0.
\]
We have still not arrived at $y=x^2$. Let now $y=\lambda x$ be a general line through the origin. We demand that this intersect $W$ twice for every $\lambda$ such that the line is not tangent. We get that the intersection is given by
\[
ax^2+b\lambda x+ex = x((a+\lambda b)x+e) = 0.
\]
For this to have two solutions for every $\lambda$ we must have $a+\lambda b \neq 0$ for all $\lambda$. But this requires $b =0$.  Thus the equation is
\[
ax^2+ey = 0
\]
which is the conic we were looking for.
  \end{enumerate}
\end{sol}

\begin{exc}[Exercise 1.2, the twisted cubic curve]

Let $Y \subseteq \Aa^3$ be the set $\{ (t,t^2,t^3) \mid t \in k\}$. Show that $Y$ is an affine variety of dimension $1$. Find generators for the ideal $I(Y)$. Show that $A(Y)$ is isomorphic to a polynomial ring in one variable over $k$. We say that $Y$ is given by the \emph{parametric equation} $x=t,y=t^2,z=t^3$.  
\end{exc}
\begin{sol}
An affine variety is by definition a closed irreducible subset of $\Aa^3$. So we must find an irreducible ideal $I$ such that $Z(I)=Y$ (forgive the abuse of notation).

I claim that $I(Y)=\langle x^2-y,x^3-z \rangle$. Clearly, every $P \in Y$ satisfies these equations. This shows the inclusion $Y \subset Z(I)$. Now suppose $P \in Z(I)$, that is, $f(P)=0$ for all $f \in I$. In particular $(x^2-y)(P)=0$ and $(x^3-z)(P)=0$. Thus $y=x^2$ and $z=x^3$. So if $P=(a,b,c) \in k^3$, then $P=(a,a^2,a^3)$, so $P \in Y$. This shows that $Z(I)=Y$. If we can show that $I$ is prime, then it follows that $I(Y)=I$ and that $Y$ is a variety.

In fact, we claim that $k[x,y,z]/I \simeq k[t]$, implying that $I$ is prime. The map $\varphi$ is given by $x \mapsto t$, $y \mapsto t^2$, $z \mapsto t^3$. Then clearly $I \subseteq \ker \varphi$. We must show equality. So suppose $\varphi(f)=0$. 

First we claim that any $f \in k[x,y,z]$ can be written as $f=R(x)+S(x)y+T(x)z+i(x,y,z)$ where $i$ is a polynomial in $I$. We prove this by induction on $\deg f$. If $\deg f = 1$, this is trivially true.  The rest of the proof proceeds by tedious induction.
\end{sol}

\subsection{Chapter II - Schemes}

\begin{exc}[Exercise 2.19]
Let $A$ be a ring. The following are equivalent:
\begin{enumerate}
\item $\Spec A$ is disconnected.
\item There exists nonzero elements $e_1,e_2 \in A$ such that $e_1e_2=0$, $e_1^2=e_1$, $e_2^2=e_2$ and $e_1+e_2=1$ (these are called \emph{orthogonal idempotents}).
\item $A$ is isomorphic to a direct product $A_1 \times A_2$ of two nonzero rings.
\end{enumerate}
\end{exc}
\begin{proof}
$1 \Rightarrow 3$: Let $U$ be a nonempty connected compontent of $X=\Spec A$. Let $V = X \bs U$ be its complement, and let $i_1:U \to X$ and $i_2=V \to X$ be the natural inclusions on topological spaces. This can be extended to a map of schemes as well: we need to give a morphism $f^\#:\OO_X \to f_\ast \OO_U$. But $f_\ast \OO_U(W)=\OO_X(W \cap U)$, so $f_\ast \OO_U = \restr{\OO_X}{U}$. Hence we just choose $f^\$:\OO_X \to \OO_U$ to be the natural map provided by the sheaf axioms.

We now have two morphisms $i_1:U \to X$ and $i_2:V \to X$ which are closed immersions, hence the induced ring morphisms $A \to A_1$ and $A \to A_2$ are surjective. Also, the universal property for products hold because the universal property for coproducts hold in the category of affine schemes. Hence $A \simeq A_1 \times A_2$.  (a bit clumsy??)

$2 \Rightarrow 3$: Let $\pi_i: A \to A$ be given by multiplication by $e_i$ and let $A_i$ be its image. Then $\ker \pi_1 = A_2$, because  if $e_1f$ then $f=e_2f$, so $f \in A_2$. The splitting maps are the natural inclusions. 

$3 \Rightarrow 2$: If $A = A_1 \times A_2$, let $e_i=\pi_i(1)$. 

$3 \Rightarrow 1$: Similar to the first argument, just opposite.
\end{proof}



\begin{exc}[Excercise 7.1]
Let $(X,\OO_X)$ be a locally ringed space and let $f:\mathscr L \to \mathscr M$ be a surjective map of invertible sheaves on $X$. Show that $f$ is an isomorphism.  
\end{exc}
\begin{sol}
Since $\mathscr L, \mathscr M$ are invertible, we have isomorphisms $\mathscr L_x \approx \OO_{X,x}$ and $\mathscr M_x \approx \OO_{X,x}$ for each $x \in X$.

But $\Hom_{\OO_{X,x}}(\OO_{X,x},\OO_{X,x})=\OO_{X,x}$, that is, all homomorphisms are given by multiplication by some $h \in \OO_{X,x}$. But since $f$ was surjective, we conclude that $h$ is outside $\mm_x$, the maximal ideal of $\OO_{X,x}$. But then $h$ is a unit, so $f$ is an isomorphism.
\end{sol}

\subsection{Chapter III - Cohomology}

\begin{exc}[Exercise 4.3]
Let $X= \Aa^2_k=\Spec k[x,y]$ and let $U = X \bs \{(0,0)\}$. Use a suitable open cover of $X$ by open affine subsets to show that $H^1(U,\OO_U)$ is isomorphic to the $k$-vector space spanned by $\{ x^i y^j \mid i,j < 0 \}$. In particular, it is infinitedimensional, and so $U$ cannot be affine (not projective either).  
\end{exc}
\begin{sol}
We can cover $U$ by $U_1= \Aa^2 \bs \{ x= 0\}$ and $U_2 = \Aa^2 \bs \{ y = 0\}$. We have $U_1 \cap U_2 = \Aa^2 \bs \{ xy=0\}$. Also, $\OO(U_1)=k[x,y,\frac 1x]$ and $\OO(U_2)=k[x,y,\frac 1y]$ and $\OO(U_1 \cap U_2) = k[x,y,\frac {1}{xy}]$. Then the \v{C}ech complex takes the form
\[
0 \to k[x,y,\frac 1x] \times k[x,y, \frac 1y] \xrightarrow{d} k[x,y,\frac{1}{xy}] \to 0,
\]
the differential being difference. Then $H^1(U,\OO_U)$ can be computed as the homology at the second term. But nothing on the left side can hit anything of the form $x^iy^j$ with $i,j < 0$. Anything else is hit. Thus we have
\[
H^1(U, \OO_U) \simeq \{ x^i y^j \mid i,j < 0 \}
\]
as $k$-vector spaces.
\end{sol}

\begin{exc}[Exercise 4.7]
Let $X$ be the subscheme of $\PP_k^2$ defined by a single homogeneous polynomial $f(x_0,x_1,x_2)=0$ of degree $d$. Assume that $(1,0,0)$ is not on $X$. Then show that $X$ can be covered by the two open affine subsets $U= X \cap \{ x_1 \neq 0\}$ and $V = X \cap \{ x_2 \neq 0\}$. Now calculate the \v Cech complex
\[
\Gamma(U,\OO_X) \oplus \Gamma(V,\OO_X) \to \Gamma(U \cap V, \OO_X)
\]
explicitly, and thus show that
\begin{align*}
  \dim_k H^0(X,\OO_X) &= 1 \\
\dim _k H^1(X,\OO_X) &= \frac 12 (d-1)(d-2).
\end{align*}
\end{exc}

\begin{sol}
  $X$ can be covered by just two open affines since $\PP^2 \bs (U \cup V) = \{(1:0:0)\}$, which was assumed not to lie on the curve.

The open affine subset $\Gamma(U, \OO_X)$ can be identified with the polynomial ring $k[u,v]/\langle f(u,1,v) \rangle$, and $\Gamma(V,\OO_X) = k[x,y]/f(x,y,1)$. The differential is then given by 
\[
\left( g(u,v), h(x,y) \right) \mapsto g(xy^{-1},y^{-1})-h(x,y) \in k[x,y,\frac 1y].
\]

We can assume that $f=x_0^d$, since what really matters is the degree, and we are just doing linear algebra.

We first calculate $H^0(X,\OO_X)$. So suppose $g(xy^{-1}, y^{-1})-h(x,y)=0$ in $k[x,y,y^{-1}]/\langle f(x,y,1) \rangle$. By definition this means that
\[
g(xy^{-1},y^{-1}) - h(x,y) = f(x,y,1) \cdot \tilde f(x,y,\frac 1y)
\]
for some polynomial $\tilde f$. Write $\tilde f$ as $\tilde f_0 + \tilde f_1$, where $\tilde f_0=\sum_{j < 0} a_{ij} x^i y^j$ and $\tilde f_1 \in k[x,y]$. Then we have the equality
\[
g(xy^{-1},y^{-1}) - h(x,y) = \sum_{j < 0} a_{ij}x^{i+d}y^j + \sum_{j \geq 0} x^{i+d} y^j.
\]
First of all, we see that the constant terms of $g$ and $h$ must be equal, because there are no constant terms on the right hand side. Secondly, $g(xy^{-1},y^{-1})$ consists solely of terms with $j < 0$. Thus the non constant terms of  $g(xy^{-1},y^{-1})$ must be equal to the left term of the right hand side above. But both terms of the right hand side are zero modulo $f$, so the constant terms of $g(xy^{-1},y^{-1})$ are also zero mod $f$. The same holds for $h(x,y)$. Thus $H^0(X,\OO_X)= \{ (c,c) \mid c \in k \} \simeq k$.

Now we compute $H^1(X,\OO_X)$. Consider a monomial $x^iy^j$ in the target. If both $i,j \geq 0$, then it is hit by $(0,-x^iy^j)$. Likewise, if $j \geq i$, then $(x^iy^{j-i},0) \mapsto x^i x^{-j}$. Thus all monomials $x^iy^{-j}$ with $j \geq i$ is zero in the cokernel. Further, if $i \geq d$, then $x^i y^j$ is already zero! Thus, we can draw the non-zero monomials in the cokernel as points in the lattice $\Z^2$. This is a triangle of length $d-2$. Thus the dimension of $H^1(X,\OO_X)$ is 
\[
1 + 2 + \ldots+ d-3 + d-2 = \frac 12 (d-2)(d-2+1) = \frac 12 (d-2)(d-1).
\]
\end{sol}

\subsection{Chapter IV - Curves}

\begin{exc}[Exercise 1.1]
Let $X$ be a curve and $P \in X$ a point. Show that there exists a nonconstant rational function $f \in K(X)$ which is regular everywhere except at $P$.
\end{exc}
\begin{sol}
Let $D$ be the divisor $D=nP$. The linear system 
$$
\{ E = D + f \geq 0 \}
$$
consists of all divisors linearly equivalent to $D$. But these are classified by those $f$ with $(f) \geq -nP$, i.e. those $f$ with at most poles of order $n$ at $P$.

By Riemann-Roch we have
$$
l(D)-l(K-D) = \deg D +1 -g = n+1-g.
$$
If $n$ is large enough, $K-D$ will have negative degree, so $l(K-D)=0$. Thus for large $n$, we can get $l(D)$ as big as we want.

\end{sol}

%%%%%%%%%%%%%%%
\section{Commutative Algebra - Eisenbud}

\subsection{Chapter 16 - Modules of Differentials}
\begin{exc}[Exercise 16.1]
Show that if $b \in S$ is an idempotent ($b^2=b$), and $d:S \to M$ is any derivation, then $db=0$.  
\end{exc}
\begin{sol}
This is trivial. $db=d(b^2)=2db$. If $2=0$, then the statement is automatically true. If not, then $db=0$ by subtraction. 
\end{sol}

\section{Deformation Theory - Hartshorne}

\subsection{Chapter I.3 - The $T^i$ functors}

\begin{exc}[Exercise 3.1]
Let $B=k[x,y](xy)$. Show that $T^1(B/k,M)=M \otimes k$ and $T^2(B/k,M)=0$ for any $B$-module $M$.  
\end{exc}
\begin{sol}
Since $B$ is defined by a principal ideal in $P=k[x,y]$, it follows that $L_2=0$ in the cotangent complex. Thus $T^2(B/k,M)$ is automatically zero.

We have that $L_1 = B$ and $L_0 = B dx \oplus B dy$ with $d_1$ being $f \mapsto (fy,fx)$. Applying $\Hom(-,M)$, we get $\Hom(L_0,M)=M \oplus M$ and $\Hom(L_1,M)=M$.

We have $\Hom(B \oplus B,M) \simeq M \oplus M$ by $\phi \mapsto (\phi(1,0),\phi(0,1)$. We have a diagram
\[
\xymatrix{
\Hom(B \oplus B, M) \ar[d]_{\simeq} \ar[r]^{\psi^\ast} & \Hom(B,M) \ar[d]^{\simeq} \\
M \oplus M  \ar[r] & M
}
\]
Under these isomorphisms, it is easy to see that the bottom map is given by
\[
(\phi(1,0),\phi(0,1)) \mapsto y \phi(1,0) + x\phi(0,1).
\]
Thus since $T^1$ is the cokernel of this map, we must have $T^1(B/k,M) = M \otimes k$. 
\end{sol}

\begin{exc}[Exercise 3.3]
Let $B = k[x,y]/(x^2,xy,y^2)$. Show that $T^0(B/k,B) = k^4$, $T^1(B/k,B)=k^4$ and $T^2(B/k,B)=k$.  
\end{exc}

\begin{sol}
Let's compute $L_2$ first. For that we need part of a resolution of $I$. We have in fact
\[
0 \to \im \begin{pmatrix} -y & 0 \\ x & -y \\ 0 & x \end{pmatrix} \to P(-2)^3 \to I \to 0.
\]
The Koszul relations are given by 
\[
\im \begin{pmatrix}
-y^2 & -xy & 0 \\
0 & x^2 & -y^2 \\
x^2 & 0 & xy
\end{pmatrix}.
\]
Let's compute $Q/F_0$ (relations modulo Koszul relations). Since $Q$ is generated in degree $3$, and $F_0$ is of degree $4$, we have $\dim_k (Q/F_0)_3 = 2$. Let's consider degree $4$. As a $k$-vector space $Q_4$ is spanned by the four elements
\[
\begin{pmatrix}
  -y^2 \\ xy \\ 0 
\end{pmatrix},
\begin{pmatrix}
  0 \\ -y^2 \\ xy 
\end{pmatrix},
\begin{pmatrix}
  -yx \\ x^2 \\ 0
\end{pmatrix},
\begin{pmatrix}
  0 \\ -yx \\ x^2
\end{pmatrix}.
\]
The two in the middle are already Koszul relations, so that $(Q/F_0)_4$ have dimension $\leq 2$. But we also have
\[
\begin{pmatrix}
  -y^2 \\ xy \\ 0
\end{pmatrix} = 
\begin{pmatrix}
0 \\ yx \\ -x^2 
\end{pmatrix}
+
\begin{pmatrix}
-y^2 \\ 0 \\ x^2
\end{pmatrix}.
\]
Thus $\dim_k (Q/F_0)_4=1$, since the second term above is a Koszul relation. Similarly we find that $\dim_k (Q/F_0)_5 =0$. Hence, $L_2$ is the $3$-dimensional $k$-vector space spanned by $Q_3$ and one more relation. $L_1$ is $F \otimes B=B^3$, and $L_0$ is $B \oplus B$, spanned by $dx,dy$.

Taking duals, we get that $L_2 = \Hom(Q/F_0,B)$. This set can be identified with
\begin{align*}
\Hom(Q/F_0,B) &= \{ \varphi: Q \to B \mid \restr{\varphi}{F_0} = 0 \} \\
&= \{ \varphi: Q \to P \mid \im \restr{f}{F_0} \subseteq I \}
\end{align*}
Thus, since $I= \mm^2$, we must have that $\varphi$ sends the two generators of $Q$ to something of degree $1$ (degree $0$ is not ok, since then $F_0$ would be sent outside $I$). Thus $\Hom(Q/F_0,B)$ is $2 \times 2=4$-dimensional, spanned by 
\[
\im 
\begin{pmatrix}
  y & x & 0 & 0 \\
0 & 0 & x & y
\end{pmatrix}.
\]
But $d_2$ is the dual of the inclusion $Q \to F$ from the exact sequence above. The dual is given by transposing, and we are left with one column - in conclusion, $T^2(B/k,B)$ is one-dimensional.

The Jacobian of $I$ is given by
\[
\begin{pmatrix}
  2x & y & 0 \\
0 & x & 2y 
\end{pmatrix},
\]
and it is easily seen that the kernel of $\text{Jac} \otimes B$ is given by $\mm \oplus \mm \oplus \mm \subset B^3$. The two relations kill off two dimensions, so $\dim_k T^1(B/k,B) = \dim_k \mm^{\oplus 3} - 2 = 6-2=4$.

Also $T^0(B/k,B)$ is $B^2$ modulo the image of the Jacobian. The constants are left untouched, so $\dim_k T^0(B/k,B) = 2+2+2-3=3$. A basis is given by $(1,0),(0,1)$ and $(x,y)$. (thus Hartshorne is wrong?)
\end{sol}

\section{Introduction to Commutative Algebra - Atiyah-MacDonald}

\subsection{Chapter 1 - Rings and ideals}

\begin{exc}
Let $x$ be a nilpotent element of a ring $A$. Show that $1+x$ is a unit of $A$. Deduce that the sum of a nilpotent element and a unit is a unit.
\end{exc}
\begin{sol}
Suppose $x^{n+1}=0$ and that $x^n \neq 0$. Consider
\[
s = 1-x+x^2-x^3+\ldots+x^n
\]
Then
\[
sx = x-x^2+x^3-x^4+\ldots-x^n
\]
since $x^{n+1}=0$. But then $s+sx=1$, so that $s(1+x)=1$. Hence $1+x$ is a unit. To prove that the sum of any unit and any nilpotent is a unit, note that if $u$ is any unit, then $u^{-1}x$ is still nilpotent. So since $u+x=u(1+u^{-1}x)$ and product of units are units, the claim follows.
\end{sol}

\begin{exc}[Exercise 11]
A ring $A$ is \emph{Boolean} if $x^2=x$ every $x \in A$. In a Boolean ring $A$, show that
\begin{enumerate}[i)]
\item $2x=0$ for all $x \in A$.
\item Every prime ideal $\pp$ is maximal, and $A/\pp$ is a field with two elements.
\item Every finitely generated ideal in $A$ is principal.
\end{enumerate}
\end{exc}
\begin{sol}
  \begin{enumerate}[i)]
  \item We have $4x=4x^2=(2x)^2=2x$, hence $2x=0$.
\item Consider $A/\pp$. This is an integral domain in which $x^2=x$ for all $x \in A/\pp$. But then $x^2-x=x(x-1)=0$. Hence either $x=0$ or $x=1$, hence $A/\pp$ can have only two elements. Thus it is isomorphic to $\Z/2\Z$ which is a field, hence $\pp$ is maximal.
\item Let $I=(a_1,\cdots,a_r)$. Every ideal is contained in a maximal ideal $\mm$. Consider the image of $I$ in $A/\mm$. 
\item By induction we can assume that $I$ is generated by two elements, say $I=(a_1,a_2)$. Then I claim that $I=(a_1+a_2)$. Cleary $(a_1+a_2) \subseteq (a_1,a_2)$. The other direction will follow if we can see that $a_1a_2=0$ (or they can be assumed to satisfy this), because $a_1a_2+a_1 \in (a_1+a_2)$.  [[[[[[[[[[[????]]]]]]]]]]]
\end{enumerate}
\end{sol}

\begin{exc}[Exercise 12]
A local ring contains no nontrivial idempotents.  
\end{exc}
\begin{sol}
Suppose $x \neq 0,1$ and that $x^2=x$. Then $x^2-x=x(x-1)=0$. Both $x$ and $x-1$ cannot be contained in $\mm$ since they generate $A$. Hence one of the is unit. Hence either $x=0$ or $x=1$, contradiction. 
\end{sol}

\begin{exc}[Exercise 15, The prime spectrum of a ring]

Let $A$ be a ring and let $X$ be the set of prime ideals of $A$. For each subset $E$ of $A$, let $V(E)$ denote the set of prime ideals of $A$ which contain $E$. Prove that
\begin{enumerate}
\item If $\ia$ is the ideal generated by $E$, then $V(E)=V(\ia)=V(r(\ia))$\footnote{Here $r(\ia)$ denotes the radical of $\ia$}.
\item $V(0)=X$ and $V(1)=\emptyset$.
\item If $(E_i)_{i \in I}$ is a family of subsets of $A$, then
\[
V\left( \bigcup_{i \in I} E_i \right) = \bigcap_{i \in I} V\left(E_i\right).
\]
\item $V(\ia \cap \ib)=V(\ia \ib)=V(\ia) \cup V(\ib)$ for all ideals $\ia,\ib$ of $A$.
\end{enumerate}
These results show that the sets $V(E)$ satisfy the axioms for closed sets in a topological space. The resulting topolgoy is called the \emph{Zariski topology}. The topological space $X$ is called the \emph{prime spectrum of $A$} and denoted $\Spec A$.
\end{exc}

\begin{sol}
We do these one by one.
\begin{enumerate}
\item Clearly $\pp \supset \langle E \rangle \supset E$, where the brackets denote the ideal generated by $E$. Hence $V(\ia) \subset V(E)$. But if $\pp \supset E$, we must have $\pp \supset \ia$ since $\langle \pp \rangle = \pp$. Thus the first equality is established.

Since $r(\ia) \subset \ia$, we have $V(\ia) \subset V(r(\ia))$. Suppose $\pp \supset r(\ia)$ and suppose $a \in \ia$. We want to show $a \in \pp$. We know that $a^n \in r(\ia)$ for some $n$, hence $a^n \in \pp$. But $\pp$ is a prime ideal, so $a \in \pp$ also. Hence equality is established.
\item Every ideal contains the zero ideal and $(1)$ is not a prime ideal.
\item Suppose $\pp \supset \cup E_i$. Then $\pp \supset E_i$ for all $i$, so $\pp \in \cap V(E_i)$. Thus this is just a formal consequence of the contravariant nature of $V(-)$.
\item Since $\ia \ib  \subset \ia \cap \ib$, we automatically have $V(\ia \cap \ib) \subset V(\ia \ib)$. So suppose $\pp \supset \ia \ib$ and let $a \in \ia \cap \ib$. Then $a^2 \in \ia \ib \subset \pp$, but then $a \in \pp$ since $\pp$ is prime.

Now suppose $\pp \supset \ia$ or $\pp \supset \ib$. Then if $a \in \ia \cap \ib$, we have $a \in \pp$, so $V(\ia) \cup V(\ib) \subset V(\ia \cap \ib)$. Now suppose $\pp \supset \ia \cap \ib$. Then by Proposition 1.11, we have $\pp \supset \ia$ or $\pp \supset \ib$. 
\end{enumerate}
\end{sol}

\begin{exc}[Exercise 17]
For each $f \in A$, let $X_f$ denote the complement of $V(f)$ in $X=\Spec A$. The sets $X_f$ are open. Show that they form a basis for the Zariski topology, and that
\begin{enumerate}
\item $X_f \cap X_g = X_{fg}$.
\item $X_f = \emptyset \Leftrightarrow f$ is nilpotent.
\item $X_f = X \Leftrightarrow f$ is a unit.
\item $X_f = X_g \Leftrightarrow r((f)) = r((g))$.
\item $X$ is quasi-compact.
\item More generally, each $X_f$ is quasi-compact.
\item An open subset of $X$ is quasi-compact if and only if it is a finite union of the sets $X_f$.
\end{enumerate}
The sets $X_f$ are called \emph{basic open sets} of $X=\Spec A$.
\end{exc}
\begin{sol}
We need to show that the sets $X_f$ forms a basis for the Zariski topology on $X$. This means that each open in $X$ can be written as a union of the $X_f$. An open in $X$ have the form 
\[
U(\ia) = \{ \pp \in \Spec A \mid \pp \not \supset \ia \}.
\]
The sets $X_f$ have the form
\[
X_f = \{ \pp \in \Spec A \mid f \not \in \pp \}.
\]
Let $\{ f_i \}_{i \in I}$ generate $\ia$. I claim that $\bigcup X_{f_i} = U(\ia)$. Let $\pp$ be an element of the left hand side. This means by definition that $f_i \not \in \pp$ for some $i$. But $f_i$ is an element of $\ia$, so $\ia \not \subset \pp$, hence $\pp \in U(\ia)$. 

Conversely, suppose $\pp \not \supset \ia$. Then some generator $f_i$ of $\ia$ is not contained in $\pp$. Hence $\pp \in X_{f_i}$. 

\begin{enumerate}
\item We have $$X_f \cap X_g = \{ \pp \mid f,g  \not \in \pp \} = \{ \pp \mid fg \not \in \pp \} ,$$
since $\pp$ is a prime ideal: for suppose $f,g \not \in \pp$, then $fg \not \in \pp$ also, because if $fg \in \pp$, primality implies either $f$ or $g in \pp$. Conversely, suppose $fg \not \in \pp$. Then neither $f,g$ can be in $\pp$ by defintion of ideals.
\item Suppose $X_f$ is empty. Then there are no prime ideals with $f \not \in \pp$. But that means that $f$ is contained in every prime ideal, hence $f$ is nilpotent.
\item Suppose $X_f = X$. Then for all prime ideals, $f \not \in \pp$, hence $f$ generates the unit ideal, hence $f$ is a unit. For if $f$ did not generate the unit ideal, $f$ would be contained in some maximal ideal $\mm$, and maximal ideals are prime.
\item Suppose $X_f=X_g$. By definition, this means that for every prime $\pp$ with $f \not \in \pp$, we have $g \not \in \pp$ (and conversely). The contrapositive of this is $g \in \pp \Leftrightarrow f \in \pp$. Hence we have 
$$r((f)) = \bigcap_{\pp \supset (f)} \pp=\bigcap_{\pp \ni f} \pp = \bigcap_{\pp \ni g} \pp = r((g)).$$ 
\item Let $\{X_f\}_{f \in I}$ be a covering of $X$ by basic opens, that is, $X= \bigcup_{f \in I} X_f$. This means that for every $\pp \in X$, there is some $f \in I$ with $f \not \in \pp$. I claim that the $f_i$ generate the unit ideal: for if not, $\langle f_i \rangle$ would be contained in some prime ideal, but by the above, this is not the case. Hence there is an equation of the form $1=\sum g_if_i$ with $g_i \in A$, which is a \emph{finite} sum. Hence these finitely many $f_i$ suffice.
\item ...
\end{enumerate}
\end{sol}




\subsection{Chapter 2 - Modules}

\begin{exc}[Excercise 1]
Show that $\Z/m \otimes_Z \Z/n = 0$ if $m,n$ are coprime.
\end{exc}
\begin{sol}
Write $1=am+bn$. Then 
\begin{align*}
1 \otimes 1 = (am+bn) \otimes 1 &= am \otimes 1 + bn \otimes 1 \\
&=  0 + bn \otimes 1 = 1 \otimes bn = 1 \otimes 0 = 0.
\end{align*}
And we are done.
\end{sol}

\begin{exc}[Exercise 2]
 Let $A$ be a ring, $\ia$ an ideal, and $M$ an $A$-module. Then $(A/\ia) \otimes_A M$ is isomorphic to $M/\ia M$.
\end{exc}
\begin{sol}
Start with
\[
0 \to \ia \to A \to A/ \ia \to 0.
\]

Tensoring with $M$ gives
\[
\ia \otimes M  \to M \to A/\ia \otimes_A M \to 0.
\]
But $\ia \otimes_A M \simeq \ia M$, so that the sequence reads $A/\ia \otimes M \simeq M/\ia M$.
\end{sol}

\begin{exc}[Exercise 3]
 Let $A$ be a local ring, $M,N$ finitely generated $A$-modules. Prove that if $M \otimes N=0$, then $M=0$ or $N = 0$. 
\end{exc}
\begin{sol}
First a counterexample if $A$ is not a local ring. Let $A=k[x]$ and $M=k[x]/(x-1)$ and $N=k[x]/(x)$. We can write $1 = -(x-1) + x$. Then $M \otimes_A N = 0$ by the same method as in Exercise 1 ($1 \otimes 1 = (-x+1 + x) \otimes 1 = x \otimes 1 = 1 \otimes x = 0$). 

Let $M_k := M \otimes k = M/\mm M$. By Nakayama's lemma, $M_k=0 \Rightarrow M=0$.

So suppose $M \otimes_A N=0$. Then $(M \otimes_A N)_k = 0$. But this is isomorphic to $M_k \otimes_A  N_k$ since $k \otimes_A k = k$. But $M_k \otimes_A N_k \simeq M_k \otimes_k N_k$, as $k$-modules, since everything in $\mm$ acts trivially on $M_k$. But these are vector spaces over a field, now we must have $M_k=0$ or $N_k=0$, and by Nakayama we are done.
\end{sol}

\begin{exc}[Exercise 4]

Let $M_i$ ($i \in I$) be any family of $A$-modules, and let $M$ be their direct sum. Then $M$ is flat if and only if each $M_i$ is flat.  
\end{exc}
\begin{sol}
Let
\[
0 \to N' \to N \to N'' \to 0
\]
be any exact sequence. Then tensoring with $M$ gives
\[
0 \to N' \otimes_A M \to N \otimes_A M \to N'' \otimes_A M \to 0.
\]
We only need to check that the left map is injective. But we have $N' \otimes_A M \simeq \bigoplus_i N' \otimes_A M_i$ and $N \otimes_A M \simeq \bigotimes_i N \otimes_A M_i$, and thus the left map is just the direct sum of all the maps 
\[
0 \to N' \otimes_A M_i \to N \otimes_A M,
\]
which is injective if and only if each $M_i$ is flat.
\end{sol}

\begin{exc}[Exercise 5]
Let $A[x]$ be the ring of polynomials in one indeterminate over a ring $A$. Prove that $A[x]$ is flat $A$-algebra.  
\end{exc}
\begin{sol}
We have $A[x] = \bigoplus_{i=0}^\infty x^i A$ as an $A$-module. Now use Exercise 4.
\end{sol}

\begin{exc}[Exercise 24]
If $M$ is an $A$-module, the following are equivalent:
  \begin{enumerate}[i)]
  \item $M$ is flat.
\item $\Tor_n^A(M,N)=0$ for all $n>0$ and $A$-modules $N$.
\item $\Tor_1^A(M,N)=0$ for all $A$-modules $N$.
  \end{enumerate}
  \begin{sol}

To compute $\Tor_A^n(M,N)$, one takes an $A$-resolution of $N$ and tensor it with $M$ and take homology. But $M$ is flat, so the sequence stays exact, so the homology is zero. This shows $i) \Rightarrow ii)$.

The implication $ii) \Rightarrow iii)$ is trivial.

Now let
\[
0 \to N' \to N \to N'' \to 0
\]
be any exact sequence of $A$-modules. Then by properties of the Tor functor, we have an exact sequence
\[
\Tor_1(M,N'') \to N' \otimes M \to N \otimes M \to N'' \otimes M \to 0.
\]
But $\Tor_1(M,N'')=0$, so the sequence is short exact. Hence $M$ is flat.
  \end{sol}

  \begin{exc}[Exercise 25]

Let 
\[
0 \to N' \to N \to N'' \to 0
\]
be an exact sequence with $N''$ flat. Then $N'$ is flat if and only if $N$ is flat.
  \end{exc}
  \begin{sol}
    We have from the Tor exact sequence
\[
0 \to \Tor_1(N',M) \to Tor_1(N,M) \to 0
\]
since $\Tor_2(N'',M)=\Tor_1(N'',M)=0$. The statement follows.
  \end{sol}
  
\end{exc}

\subsection{Chapter III - Rings and modules of fractions}
\begin{exc}[Exercise 1]

Let $S$ be a multiplicatively closed subset of a ring $A$, and let $M$ be a finitely-generated $A$-module. Prove that $S^{-1}M=0$ if and only if there exists $s \in S$ such that $sM=0$.  
\end{exc}

\begin{sol}
 Suppose there exists such $s$. Let $m/s' \in S^{-1}M$. This is zero if and only if there exists $s \in M$ such that $s(s'm)=0$. But $ss'm=s'sm=s'0=0$. So $m=0$ in $S^{-1}M$.  (note that we did not use finite generation)

Now let $m_1,\ldots,m_r$ be a set of generators for $M$ and suppose that $S^{-1}M=0$. Then for each $i$ ($i=1,\ldots,r$), there exists $s_i$ such that $s_im_i=0$. Since every element of $M$ is an $A$-linear combination of the $m_i$, it follows that the product $s_1s_2\cdots s_r$ makes $sM=0$.
\end{sol}

\subsection{Chapter 5 - Integral dependence and valuations}

\begin{exc}[Exercise 1]

Let $f:A \to B$ be an integral morphism of rings. Show that $f^\ast:\Spec B \to \Spec A$ is a closed mapping.  
\end{exc}
\begin{sol}
The map $f^\ast$ is by definition given by $\pp \mapsto f^{-1}(\pp) = \pp \cap A$. A closed subset of $\Spec B$ is by definition 
\[
V(\ia) = \{ \pp \in \Spec B \mid \pp \supset \ia \}
\]
for some ideal $\ia \subset B$.

Then the image of $V(\ia)$ is the set
\begin{align*}
f^\ast(V(\ia)) &= \{ \pp \cap A \mid \pp \in \Spec B, \quad  \pp \supset \ia \} 
\end{align*}
I claim that this is equal to 
\[
V(\ia \cap A) = \{ \qq \in \Spec A \mid \qq \supset \ia \cap A \},
\]
which clearly is a closed subset of $\Spec A$.

One direction is obvious: let $\pp \cap A$ be an element of $f^\ast(V(\ia))$. This is a point of $\Spec A$, and clearly $\pp \cap A \supset \ia \cap A$ since $\pp \supset \ia$.

The other direction needs the going up Theorem 5.10. Suppose $\qq \in V(\ia \cap A)$. Then by Going Up, there exists $\pp \in \Spec B$ with $\pp \cap A = \qq$. But we need to check that $\pp \supset \ia$. That is, we need to prove the assertion that if $\qq = \pp \cap A$ and $\qq \supset \ia \cap A$, then $\pp \supset \ia$. So suppose $a \in \ia \subset B$. Then $a$ satisfies an equation
\[
a^n + b_{n-1}a^{n-1} + \ldots + b_1a+b_0=0
\]
with $b_i \in A$. Since $a \in \ia$, we see that $b_0 \in \qq = \pp \cap A$. Hence
\[
a^n+b_{n-1}a^{n-1}+\ldots+b_1a = a(a^{n-1}+b_{n-1}a^{n-2}+\ldots+b_1) \in \pp
\]
since $\qq \subset \pp$. But $\pp$ is prime so either $a \in \pp$ and we are done, or $a^{n-1}b_{n-1}a^{n-2}+\ldots+b_1 \in \pp$, and we can continue by induction.

Hence we are done.
\end{sol}

\subsection{Chapter 7 - Noetherian rings}

\begin{exc}[Exercise 11]
Let $A$ be a ring such that $A_{\pp}$ is Noetherian for each $\pp \in \Spec A$. Is $A$ necessarily noetherian?
\end{exc}
\begin{sol}
Consider the ring
\[
A= \Z/2 \times \Z/2 \cdots .
\]
It is a countable product of noetherian rings. The primes are just the coordinate axes, and each localization is isomorphic to $\Z/2$. Thus each $A_\pp$ is Noetherian, but $A$ is not.
\end{sol}

\begin{exc}[Exercise 15]
Let $A$ be a Noetherian local ring, $\mm$ its maximal ideal and $k$ its residue field and let $M$ be a finitely generated $A$-module. Then the following are equivalent:
\begin{enumerate}[i)]
\item $M$ is free.
\item $M$ is flat.
\item The mapping $\mm \otimes M \to A \otimes M$ is injeective.
\item $\Tor_1^A(k,M)=0$.
\end{enumerate}
\end{exc}
\begin{sol}
The implication $i) \Rightarrow ii)$ is trivial. One way is to compute $\Tor_1^A(M,N)$ for any $A$-module $N$. But a free resolution of $M$ is just one-term, so $\Tor_1^A(M,N)$ is automatically zero.

The implication $ii \Rightarrow iii)$  follows by tensoring the incusion $\mm \hookrightarrow A$ with $M$. 

The implication $iii) \Rightarrow iv)$ follows from the $\Tor$ exact sequence
\[
\Tor_1^A(A,M) \to \Tor^A_1(k,M) \to \mm \otimes M \to A \otimes M \to k \otimes M \to 0.
\]
The leftmost term is zero since $A$ is a free $A$-module, and by $iii)$ and exactness we must as well have $\Tor_1^a(k,M)$.

Now for $iv \Rightarrow i)$. Choose element $m_i \in M$ ($0 \leq i \leq r$) such that they form a $k$-basis for $M/\mm M$. Choose a surjection $f:A^r \to M$ and let $E=\ker f$ be its kernel. Then we have an exact sequence
\[
0 \to E \to A^r \to M \to 0.
\]
of finitely-generated $A$-modules ($E$ is finitely generated by Proposition 6.2). Tensor the sequence by $k$, and get
\[
\Tor_1^A(k,M) \to E/\mm E \to k^r \to M/\mm M \to 0.
\]
The left-most term is zero by assumption. The last two spaces are $k$-vector spaces of the same dimension, and it follows that $E/\mm E=0$. But then it follows that $E$ is zero by Nakayama's lemma, hence $M$ is free.
\end{sol}

\begin{exc}[Exercise 16]
 Let $A$ be a Noetherian ring, $M$ a finitely-generated $A$-module. Then the following are equivalent:
 \begin{enumerate}[i)]
 \item $M$ is a flat $A$-module.
\item $M_\pp$ is a free $A_\pp$-module for each $\pp \in \Spec A$.
\item $M_\mm$ is a free $A_\mm$-module for each maximal ideal $\mm$.
 \end{enumerate}

So flatness is the same as being locally free.
\end{exc}
\begin{sol}
 The implications $i \Rightarrow ii)$ and $ii) \Rightarrow iii)$ follows trivially from the previous exercise. We prove $iii) \Rightarrow i)$. 

Applying the $\Tor$ functor commutes with localization, hence we have $\Tor_1^A(M,N)_\mm = \Tor_1^{A_\mm}(M_\mm, N_\mm)=0$ for all $\mm$. But being zero is a local property, so it follows that $\Tor_1^A(M,N)=0$ for all $A$-modules $N$. Hence $M$ is flat.
\end{sol}


\section{Representation Theory - Fulton, Harris}

\subsection{Representations of Finite Groups}

\begin{exc}[Exercise 1.1]
Verify that the relation 
\[
\langle g\cdot v^\ast , g \cdot v \rangle = \langle \rho^\ast(g)(v^\ast),\rho(g)(v) \rangle = \langle v^\ast, v\rangle 
\]
is satisfied when we define
\[
\rho^\ast(g) = \rho(g^{-1})^t :V^\ast \to V^\ast, 
\]
that is, $(\rho^\ast g)(v^\ast)(w)= \langle (\rho^\ast g)(v^\ast),w\rangle=\langle v^\ast, (\rho g^{-1})(w) \rangle$.
\end{exc}

\begin{sol}
This is a matter of calculation.
\[
\langle g v^\ast, gv \rangle = \langle v^\ast, (\rho g^{-1})(gv) \rangle = \langle v^\ast, v \rangle.
\]
So the definition is ok.
\end{sol}


\begin{exc}[Exercise 1.2]
Verify that in general the vector space of $G$-linear maps between two representations $V$ and $W$ of $G$ is just the subspace $\Hom(V,W)^G$ of elements of $\Hom(V,W)$ fixed under the action of $G$. This subspace is often denoted $\Hom_G(V,W)$.
\end{exc}
\begin{sol}	
A map $\varphi:V \to W$ is $G$-linear when $\varphi(gv)=g \varphi(v)$. The action of $G$ on $\varphi$ is given by $g\varphi(v)=g \varphi(g^{-1}v)$. But by $G$-linearity, this is
$$
\varphi(gv)=g g^{-1} \varphi(gv)=gg^{-1}\varphi(v)=\varphi(v).
$$
Hence a map is $G$-linear if and only if it is fixed by the action of $G$. 
\end{sol}

\begin{exc}[Exercise 1.3]
Let $\rho:G \to \GL(V)$ be any representation of the finite group $G$ on an $n$-dimensional vector space $V$ and suppose that for any $g \in G$, the determinant if $\rho(g)$ is $1$. Show that the spaces $\wedge^k V $ and $\wedge^{n-k} V^\ast$ are isomorphic as representations of $G$.
\end{exc}

\begin{sol}
This is (again) just a matter of writing out the definitions. First we define the isomorphism, and then we check that it is actually an isomorphism of representations.

\begin{align*}
\bigwedge ^k V &\to \bigwedge^{n-k} V^\ast \\
v_1 \wedge \cdots \wedge v_k &\mapsto \left( w_1 \wedge \cdots \wedge w_{n-k} \mapsto v_1 \wedge \cdots \wedge v_k \wedge w_1 \wedge \cdots \wedge w_{n-k} \right)
\end{align*}
Being a map of representations is equivalent to $g^{-1}\varphi(gv)=\varphi(v)$, so we just need to check that all the $g$'s disappear from the left hand side. 
\begin{align*}
g^{-1}\varphi(gv) &= g^{-1}(w_1\cdots w_{n-k} \mapsto gv_1\cdots gv_k w_1 \cdots w_{n-k}) \\
&= (gv_1\cdots gv_k gw_1 \cdots gw_{n-k}) \\
&= \det \rho(g) v_1 \wedge \cdots \wedge w_{n-k}.
\end{align*}
Hence $\varphi$ is a map of representations if and only if $\det \rho(g)=1$ for all $g \in G$. 

(it is an isomorphism because it has zero kernel: because what would the kernel be? Every subspace is the same, and this is a basis free description)
\end{sol}

\begin{exc}[Exercise 1.4]
The permutation representation $R$ of $G$ acting on a finite set $X$ have two descriptions: one is given by letting $V$ be the vector space with basis $\{ e_x \mid x \in X \}$ and letting $g$ act on $V$ by $ge_x = e_{gx}$. 

Alternatively $R$ is the set of functions $f:X \to \C$ with action $(g\alpha)(h)=\alpha(g^{-1}h)$.

\begin{enumerate}[a)]
\item Show that these two decriptions agree by identifying $e_x$ with the characteristic function which takes the value $1$ on $x$ and $0$ elsewhere. 
\item The space of functions on $G$ can also be made into a $G$-module by the rule $(g\alpha)(h)=\alpha(hg)$. Show that this is an isomorphic representation.
\end{enumerate}
\end{exc}
\begin{sol}
a). Clearly the vector space dimensions agree (since the characteristic functions are a basis). So we need to check that this is a map of representations. Denote the characteristic function by $\chi_x$. Then $\varphi(ge_x)(h) = \varphi(e_{gx})(h)=\chi_{gx}(h)$. Similarly $g \varphi(e_x)(h) = g \chi_x(h) = \chi_x(g^{-1}h)$, The first function is $1$ if $gx=h$, and the second function is $1$ if $g^{-1}h=x$, and these are equivalent.

b). Send $\alpha$ to the function $g \mapsto \alpha(g^{-1})$. Call this assignment $\psi$. We need to check that $\psi(g\alpha)=g \psi(\alpha)$. 

First the left hand side. We have: $\psi(g\alpha)(h)=\psi(h \mapsto \alpha(g^{-1}h))(h)=\alpha(g^{-1}h^{-1})$.

And similarly: $g \psi(\alpha)(h)=g (h \mapsto \alpha(h^{-1}))(h) = g \alpha(h^{-1})=\alpha(g^{-1}h^{-1})$. 

And these are equal.
\end{sol}

\begin{exc}[Exercise 1.10]
$G=S_3$. Verify that with $\sigma=(12)$, $\tau=(123)$, the standard representation has a basis $\alpha=(\omega, 1, \omega^2)$, $\beta=(1,\omega,\omega^2)$, with
\[
\tau \alpha = \omega \alpha, \qquad \tau \beta = \omega^2 \beta, \qquad \sigma \alpha = \beta, \qquad \sigma \beta = \alpha.
\]
\end{exc}
\begin{sol}
The standard representation $V$ is the subspace $\{ x_1+x_2+x_3=0 \}$ of $\C^3$. Since $1+\omega+\omega^2=0$, and $\alpha \cdot \beta = 3\omega \neq 0$, these two span $V$.

The identities are easy.
\end{sol}


\begin{exc}[Exercise 1.11]
Use this approach to find the decomposition of the representations $\Sym^2 V$ and $\Sym^3 V$.
\end{exc}
\begin{sol}
The elements $\{ \alpha^2, \alpha \beta, \beta^2 \}$ are a basis of $\Sym^2 V$, and the eigenvalues are $\omega^2, 1$ and $\omega$, respectively. Thus $\langle \alpha \beta \rangle$ span a representation isomorphic to $U$, the trivial representation, and $\langle \alpha^2, \beta^2 \rangle$ span a representation isomorphic to $V$, the standard representation. Hence $\Sym^2 V = U \oplus V$.

The elements $\{ \alpha^3, \alpha^2 \beta, \alpha \beta^2, \beta^3 \}$ are a basis of $\Sym^3 V$. The eigenvalues are $1, \omega, \omega^2$ and $1$, respectively. Looking at the action of $\sigma=(12)$, we see that $U \simeq \langle \alpha^3+\beta^3 \rangle$, and $U' \simeq \langle \alpha^3-\beta^3 \rangle$. The remaining $\langle \alpha^2 \beta, \alpha \beta^2 \rangle$ span a representation isomorphic to $V$. Hence $\Sym^3 V = U \oplus U' \oplus V$.
\end{sol}

\begin{exc}[Exercise 2.2]
For $\Sym^2 V$, verify that
\[
\chi_{\Sym^2 V}(g) = \frac 12 \left[ \chi_V(g)^2 + \chi_V(g^2)\right].
\]
Note that this is compatible with the decomposition $V \otimes V = \Sym^2 V \oplus \wedge^2 V$.
\end{exc}
\begin{proof}
The eigenvalues of $g$ acting on $\Sym^2 V$ are $\{\lambda_i \lambda_j \}$. Hence
\begin{align*}
\chi_{\Sym^2 V}(g) &= \sum_{i \leq j} \lambda_i \lambda_j \\
&= \sum_{i < j} \lambda_i \lambda_j + \sum_i \lambda_i^2 \\
&= \frac 12 \left( \chi_V(g)^2 - \chi_V(g^2) \right) + \chi_V(g^2) \\
&= \frac 12  \left( \chi_V(g)^2 + \chi_V(g^2) \right).
\end{align*}
\end{proof}

\begin{exc}[Exercise 2.5, The original fixed point formula]
If $V$ is a permutation representation associated to the action of a group $G$ on a finite set $X$, show that $\chi_V(g)$ is the number of elements fixed by $g$.
\end{exc}
\begin{sol}
This is easy. The matrix associated to $g$ is a permutation matrix with a $1$ in row $j$ if element number $i$ is sent to $j$. Then number of fixed points is the number of ones on the diagonal, and this is $\chi_V(g)$.
\end{sol}


\begin{exc}[Exercise 2.34]
Let $V,W$ be irreducible representations of $G$ and $L_0:V \to W$ any linear mapping. Define $L:V \to W$ by 
$$
L(v) = \frac{1}{\lvert G \rvert} \sum_g g^{-1} L_0(gv).
$$
Show that $L=0$ if $V$ and $W$ are not isomorphic, and that $L$ is multiplication by $\tr(L_0)/\dim (V)$ if $V=W$.
\end{exc}
\begin{sol}
We want to apply Schur's lemma. We check that $L$ is a $G$-module homomorphism. We have
\begin{align*}
L(hv) &= \frac{1}{\lvert G \rvert} \sum_g g^{-1}L_0(ghv) \\
&= \frac{1}{\lvert G\rvert} \sum_{gh} h {gh}^{-1}L_0(ghv) \\
&= \frac{1}{\lvert G\rvert} \sum_{g'} h {g'}^{-1}L_0(g'v)
\end{align*}
Hence $L$ is a $G$-module homomorphism. Hence by Schur's lemma, $L$ is either the zero map or an isomorphism. In particular, if they are not isomorphic, $L=0$. Now suppose $V=W$. 
\end{sol}





\end{document}
