\documentclass[11pt, english]{article}
%\usepackage[latin1]{inputenc}
\usepackage[T1]{fontenc}
\usepackage[utf8]{inputenc}
\usepackage[english]{babel}   % S P R A A K
% \usepackage{graphicx}    % postscript graphics
\usepackage{amssymb, amsmath, amsthm, amssymb} % symboler, osv
\usepackage{mathrsfs}
\usepackage{url}
\usepackage{thmtools}
\usepackage{enumerate}  % lister $  
\usepackage{float}
\usepackage{tikz}
\usepackage{tikz-cd}
\usetikzlibrary{calc}
%\usepackage{tikz-3dplot}
\usepackage{subcaption}
\usepackage[all]{xy}   % for comm.diagram
\usepackage{wrapfig} % for float right
\usepackage{hyperref}
\usepackage{mystyle} % stilfilen      

%\usepackage[a5paper,margin=0.5in]{geometry}


\begin{document}
\title{Exercises}
\author{Fredrik Meyer}
\maketitle 

I solve and type exercises from different places (read \emph{books}). 

\section{Algebraic Geometry - Hartshorne}

\subsection{Chapter I - Varieties}

\begin{exc}[Exercise 1.1]
  \begin{enumerate}[a)]
  \item Let $Y$ be the plane curve $y=x^2$. Show that $A(Y)$ is isomorphic to a polynomial ring in one variable over $k$.
\item Let $Z$ be the plane curve $xy=1$. Show that $A(Z)$ is not isomorphic to a polynomial ring in one variable over $k$.
\item Let $f$ be any irreducible quadratic polynomial in $k[x,y]$, and let $W$ be the conic defined by $f$. Show that $A(W)$ is isomorphic to $A(Y)$ or $A(Z)$. Which one is it when?
  \end{enumerate}
\end{exc}
\begin{sol}
  \begin{enumerate}[a)]
  \item  We have $A(Y)=k[x,y]/(y-x^2)$. An isomorphism $A(Y) \to k[t]$ is given by $x \mapsto t$ and $y \mapsto t^2$. 
\item We have $A(Z) = k[x,y]/(xy-1) \simeq k[x,\frac 1x]$. So we must show that $k[x,\frac 1x] \not \approx k[x]$. It can be computed that the first one has automorphisms given by $x \mapsto cx^n$ for $c$ nonzero and $n \neq 0$. The second has as automorphisms $ax+b$ ($a \neq 0$). So the first one have an abelian automorphism group, the second has not.
\item It is seen by tedious calculations that any quadric in $\Aa^2$ can be reduced to one of the form
\[
x^2+bxy+y^2-c=0
\]
by linear transformations. [[[[[[[[[[HOW TO DO THE REST??]]]]
  \end{enumerate}
\end{sol}

\subsection{Chapter II - Schemes}

\begin{exc}[Excercise 7.1]
Let $(X,\OO_X)$ be a locally ringed space and let $f:\mathscr L \to \mathscr M$ be a surjective map of invertible sheaves on $X$. Show that $f$ is an isomorphism.  
\end{exc}
\begin{sol}
Since $\mathscr L, \mathscr M$ are invertible, we have isomorphisms $\mathscr L_x \approx \OO_{X,x}$ and $\mathscr M_x \approx \OO_{X,x}$ for each $x \in X$.

But $\Hom_{\OO_{X,x}}(\OO_{X,x},\OO_{X,x})=\OO_{X,x}$, that is, all homomorphisms are given by multiplication by some $h \in \OO_{X,x}$. But since $f$ was surjective, we conclude that $h$ is outside $\mm_x$, the maximal ideal of $\OO_{X,x}$. But then $h$ is a unit, so $f$ is an isomorphism.
\end{sol}

\subsection{Chapter III}

\begin{exc}[Exercise 4.3]
Let $X= \Aa^2_k=\Spec k[x,y]$ and let $U = X \bs \{(0,0)\}$. Use a suitable open cover of $X$ by open affine subsets to show that $H^1(U,\OO_U)$ is isomorphic to the $k$-vector space spanned by $\{ x^i y^j \mid i,j < 0 \}$. In particular, it is infinitedimensional, and so $U$ cannot be affine (not projective either).  
\end{exc}
\begin{sol}
We can cover $U$ by $U_1= \Aa^2 \bs \{ x= 0\}$ and $U_2 = \Aa^2 \bs \{ y = 0\}$. We have $U_1 \cap U_2 = \Aa^2 \bs \{ xy=0\}$. Also, $\OO(U_1)=k[x,y,\frac 1x]$ and $\OO(U_2)=k[x,y,\frac 1y]$ and $\OO(U_1 \cap U_2) = k[x,y,\frac {1}{xy}]$. Then the \v{C}ech complex takes the form
\[
0 \to k[x,y,\frac 1x] \times k[x,y, \frac 1y] \xrightarrow{d} k[x,y,\frac{1}{xy}] \to 0,
\]
the differential being difference. Then $H^1(U,\OO_U)$ can be computed as the homology at the second term. But nothing on the left side can hit anything of the form $x^iy^j$ with $i,j < 0$. Anything else is hit. Thus we have
\[
H^1(U, \OO_U) \simeq \{ x^i y^j \mid i,j < 0 \}
\]
as $k$-vector spaces.
\end{sol}

\subsection{Chapter IV}

\begin{exc}[Exercise 1.1]
Let $X$ be a curve and $P \in X$ a point. Show that there exists a nonconstant rational function $f \in K(X)$ which is regular everywhere except at $P$.
\end{exc}
\begin{sol}
Let $D$ be the divisor $D=nP$. The linear system 
$$
\{ E = D + f \geq 0 \}
$$
consists of all divisors linearly equivalent to $D$. But these are classified by those $f$ with $(f) \geq -nP$, i.e. those $f$ with at most poles of order $n$ at $P$.

By Riemann-Roch we have
$$
l(D)-l(K-D) = \deg D +1 -g = n+1-g.
$$
If $n$ is large enough, $K-D$ will have negative degree, so $l(K-D)=0$. Thus for large $n$, we can get $l(D)$ as big as we want.

\end{sol}

%%%%%%%%%%%%%%%
\section{Commutative Algebra - Eisenbud}

\subsection{Chapter 16 - Modules of Differentials}
\begin{exc}[Exercise 16.1]
Show that if $b \in S$ is an idempotent ($b^2=b$), and $d:S \to M$ is any derivation, then $db=0$.  
\end{exc}
\begin{sol}
This is trivial. $db=d(b^2)=2db$. If $2=0$, then the statement is automatically true. If not, then $db=0$ by subtraction. 
\end{sol}

\section{Deformation Theory - Hartshorne}

\subsection{Chapter I.3 - The $T^i$ functors}

\begin{exc}[Exercise 3.1]
Let $B=k[x,y](xy)$. Show that $T^1(B/k,M)=M \otimes k$ and $T^2(B/k,M)=0$ for any $B$-module $M$.  
\end{exc}
\begin{sol}
Since $B$ is defined by a principal ideal in $P=k[x,y]$, it follows that $L_2=0$ in the cotangent complex. Thus $T^2(B/k,M)$ is automatically zero.

We have that $L_1 = B$ and $L_0 = B dx \oplus B dy$ with $d_1$ being $f \mapsto (fy,fx)$. Applying $\Hom(-,M)$, we get $\Hom(L_0,M)=M \oplus M$ and $\Hom(L_1,M)=M$.

We have $\Hom(B \oplus B,M) \simeq M \oplus M$ by $\phi \mapsto (\phi(1,0),\phi(0,1)$. We have a diagram
\[
\xymatrix{
\Hom(B \oplus B, M) \ar[d]_{\simeq} \ar[r]^{\psi^\ast} & \Hom(B,M) \ar[d]^{\simeq} \\
M \oplus M  \ar[r] & M
}
\]
Under these isomorphisms, it is easy to see that the bottom map is given by
\[
(\phi(1,0),\phi(0,1)) \mapsto y \phi(1,0) + x\phi(0,1).
\]
Thus since $T^1$ is the cokernel of this map, we must have $T^1(B/k,M) = M \otimes k$. 
\end{sol}

\begin{exc}[Exercise 3.3]
Let $B = k[x,y]/(x^2,xy,y^2)$. Show that $T^0(B/k,B) = k^4$, $T^1(B/k,B)=k^4$ and $T^2(B/k,B)=k$.  
\end{exc}

\begin{sol}
Let's compute $L_2$ first. For that we need part of a resolution of $I$. We have in fact
\[
0 \to \im \begin{pmatrix} -y & 0 \\ x & -y \\ 0 & x \end{pmatrix} \to P(-2)^3 \to I \to 0.
\]
The Koszul relations are given by 
\[
\im \begin{pmatrix}
-y^2 & -xy & 0 \\
0 & x^2 & -y^2 \\
x^2 & 0 & xy
\end{pmatrix}.
\]
Let's compute $Q/F_0$ (relations modulo Koszul relations). Since $Q$ is generated in degree $3$, and $F_0$ is of degree $4$, we have $\dim_k (Q/F_0)_3 = 2$. Let's consider degree $4$. As a $k$-vector space $Q_4$ is spanned by the four elements
\[
\begin{pmatrix}
  -y^2 \\ xy \\ 0 
\end{pmatrix},
\begin{pmatrix}
  0 \\ -y^2 \\ xy 
\end{pmatrix},
\begin{pmatrix}
  -yx \\ x^2 \\ 0
\end{pmatrix},
\begin{pmatrix}
  0 \\ -yx \\ x^2
\end{pmatrix}.
\]
The two in the middle are already Koszul relations, so that $(Q/F_0)_4$ have dimension $\leq 2$. But we also have
\[
\begin{pmatrix}
  -y^2 \\ xy \\ 0
\end{pmatrix} = 
\begin{pmatrix}
0 \\ yx \\ -x^2 
\end{pmatrix}
+
\begin{pmatrix}
-y^2 \\ 0 \\ x^2
\end{pmatrix}.
\]
Thus $\dim_k (Q/F_0)_4=1$, since the second term above is a Koszul relation. Similarly we find that $\dim_k (Q/F_0)_5 =0$. Hence, $L_2$ is the $3$-dimensional $k$-vector space spanned by $Q_3$ and one more relation. $L_1$ is $F \otimes B=B^3$, and $L_0$ is $B \oplus B$, spanned by $dx,dy$.

Taking duals, we get that $L_2 = \Hom(Q/F_0,B)$. This set can be identified with
\begin{align*}
\Hom(Q/F_0,B) &= \{ \varphi: Q \to B \mid \restr{\varphi}{F_0} = 0 \} \\
&= \{ \varphi: Q \to P \mid \im \restr{f}{F_0} \subseteq I \}
\end{align*}
Thus, since $I= \mm^2$, we must have that $\varphi$ sends the two generators of $Q$ to something of degree $1$ (degree $0$ is not ok, since then $F_0$ would be sent outside $I$). Thus $\Hom(Q/F_0,B)$ is $2 \times 2=4$-dimensional, spanned by 
\[
\im 
\begin{pmatrix}
  y & x & 0 & 0 \\
0 & 0 & x & y
\end{pmatrix}.
\]
But $d_2$ is the dual of the inclusion $Q \to F$ from the exact sequence above. The dual is given by transposing, and we are left with one column - in conclusion, $T^2(B/k,B)$ is one-dimensional.

The Jacobian of $I$ is given by
\[
\begin{pmatrix}
  2x & y & 0 \\
0 & x & 2y 
\end{pmatrix},
\]
and it is easily seen that the kernel of $\text{Jac} \otimes B$ is given by $\mm \oplus \mm \oplus \mm \subset B^3$. The two relations kill off two dimensions, so $\dim_k T^1(B/k,B) = \dim_k \mm^{\oplus 3} - 2 = 6-2=4$.

Also $T^0(B/k,B)$ is $B^2$ modulo the image of the Jacobian. The constants are left untouched, so $\dim_k T^0(B/k,B) = 2+2+2-3=3$. A basis is given by $(1,0),(0,1)$ and $(x,y)$. (thus Hartshorne is wrong?)
\end{sol}

\section{Introduction to Commutative Algebra - Atiyah-MacDonald}

\subsection{Chapter 1 - Rings and ideals}

\begin{exc}
Let $x$ be a nilpotent element of a ring $A$. Show that $1+x$ is a unit of $A$. Deduce that the sum of a nilpotent element and a unit is a unit.
\end{exc}
\begin{sol}
Suppose $x^{n+1}=0$ and that $x^n \neq 0$. Consider
\[
s = 1-x+x^2-x^3+\ldots+x^n
\]
Then
\[
sx = x-x^2+x^3-x^4+\ldots-x^n
\]
since $x^{n+1}=0$. But then $s+sx=1$, so that $s(1+x)=1$. Hence $1+x$ is a unit. To prove that the sum of any unit and any nilpotent is a unit, note that if $u$ is any unit, then $u^{-1}x$ is still nilpotent. So since $u+x=u(1+u^{-1}x)$ and product of units are units, the claim follows.
\end{sol}

\subsection{Chapter 2 - Modules}

\begin{exc}[Excercise 1]
Show that $\Z/m \otimes_Z \Z/n = 0$ if $m,n$ are coprime.
\end{exc}
\begin{sol}
Write $1=am+bn$. Then 
\begin{align*}
1 \otimes 1 = (am+bn) \otimes 1 &= am \otimes 1 + bn \otimes 1 \\
&=  0 + bn \otimes 1 = 1 \otimes bn = 1 \otimes 0 = 0.
\end{align*}
And we are done.
\end{sol}

\begin{exc}[Exercise 2]
 Let $A$ be a ring, $\ia$ an ideal, and $M$ an $A$-module. Then $(A/\ia) \otimes_A M$ is isomorphic to $M/\ia M$.
\end{exc}
\begin{sol}
Start with
\[
0 \to \ia \to A \to A/ \ia \to 0.
\]

Tensoring with $M$ gives
\[
\ia \otimes M  \to M \to A/\ia \otimes_A M \to 0.
\]
But $\ia \otimes_A M \simeq \ia M$, so that the sequence reads $A/\ia \otimes M \simeq M/\ia M$.
\end{sol}

\begin{exc}[Exercise 3]
 Let $A$ be a local ring, $M,N$ finitely generated $A$-modules. Prove that if $M \otimes N=0$, then $M=0$ or $N = 0$. 
\end{exc}
\begin{sol}
First a counterexample if $A$ is not a local ring. Let $A=k[x]$ and $M=k[x]/(x-1)$ and $N=k[x]/(x)$. We can write $1 = -(x-1) + x$. Then $M \otimes_A N = 0$ by the same method as in Exercise 1 ($1 \otimes 1 = (-x+1 + x) \otimes 1 = x \otimes 1 = 1 \otimes x = 0$). 

Let $M_k := M \otimes k = M/\mm M$. By Nakayama's lemma, $M_k=0 \Rightarrow M=0$.

So suppose $M \otimes_A N=0$. Then $(M \otimes_A N)_k = 0$. But this is isomorphic to $M_k \otimes_A  N_k$ since $k \otimes_A k = k$. But $M_k \otimes_A N_k \simeq M_k \otimes_k N_k$, as $k$-modules, since everything in $\mm$ acts trivially on $M_k$. But these are vector spaces over a field, now we must have $M_k=0$ or $N_k=0$, and by Nakayama we are done.
\end{sol}
\end{document}
