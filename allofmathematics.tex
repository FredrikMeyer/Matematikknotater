\documentclass[11pt, norsk]{book}
%\usepackage[latin1]{inputenc}
\usepackage[T1]{fontenc}
\usepackage[utf8]{inputenc}
\usepackage[norsk]{babel}   % S P R A A K


% \usepackage{graphicx}    % postscript graphics
\usepackage{amssymb, amsmath, amsthm, amssymb} % symboler, osv
\usepackage{mathrsfs}
\usepackage{url}
\usepackage{thmtools}
\usepackage{enumerate}  % lister $  
\usepackage{float}
\usepackage{tikz}
\usetikzlibrary{calc}
\usepackage{tikz-3dplot}
\usepackage{subcaption}
\usepackage[all]{xy}   % for comm.diagram
\usepackage{wrapfig} % for float right
\usepackage{hyperref}
\usepackage{mystyle} % stilfilen      


\begin{document}
\title{All matematikk}
\author{Fredrik Meyer}
\maketitle 

\chapter{Sets}
\section{Axioms}

\chapter{Relations}

\begin{defi}
A  \emph{binary relation} is a map $S \times S \to $
\end{defi}

\chapter{Groups}

\begin{defi}
A \emph{group} is a triple $(G,\mu,\iota)$ where $G$ is a set and $\mu$ is a binary operation $G \times G \to G$ and $\iota$ is a function $G \to G$, satisfying the following axioms:
\begin{itemize}
\item Associativity: The following diagram commutes:
\[
\xymatrix{
G \times G \times G \ar[r]^{\id \times \mu} \ar[d]_{\mu \times \id} & G \times G \ar[d]^\mu \\
G \times G \ar[r]^\mu & G
}
\]
\item Identity element. There exist an element $e \in G$ such that the following two diagrams commute:
\[
\xymatrix{
\{ e \} \times G \ar[r]^{e \times \id} \ar[dr]_{\pi_2} & G \times G \ar[d]^{\mu} \\
& G 
}
\]
and
\[
\xymatrix{
G \times \{e \} \ar[r]^{\id \times e} \ar[dr]_{\pi_2} & G \times G \ar[d]^{\mu} \\
& G 
}
\]
where the top maps are the natural inclusions.
\item Inverse element. The following diagram is commutative:
\[
\xymatrix{
G \ar[r]^{\id \times \iota} \ar[dr] & G \times G \ar[r]^\mu & G \\
& \{e \} \ar@{^(->}[ur]
}
\]
The same should also hold with $\id \times \iota$ replaces with $\iota \times \id$.
\end{itemize}
\end{defi}
We will never write a group $G$ as a triple, but only refer to the group $(G,\mu,\iota)$ as just $G$, the maps being implicit. We will write $\iota(g)$ as $g^{-1}$, and $\mu(g,h)$ as $gh$.

\begin{lemma}
The identity element $e \in G$ is unique.
\end{lemma}
\begin{proof}
Suppose $e^\prime$ is another identity element. Then $e e^\prime = e^\prime$ by the first diagram. But also $e e^\prime = e$ by the second diagram. Then $e=e^\prime$. Alternatively, the following is a commutative diagram:
\[
\xymatrix{
& \{e \} \times \{ e^\prime \} \ar@{^(->}[dl] \ar[d]^\Delta \ar@{_(->}[dr] & \\
\{ e\} \times G \ar[r]^{e \times \id} \ar[dr]_{\pi_2} & G \times G  \ar[d]^{\mu}& \{ e'\}  \times G  \ar[l]_{e^\prime \times \id} \ar[dl]^{\pi_2}\\
& G
} 
\]
\end{proof}

\chapter{Rings}

\chapter{Fields}

\begin{defi}
 A \emph{field} is a commutative ring $k$ with only one ideal.
\end{defi}

\section{Ordered fields}

\chapter{The real numbers}

\end{document} 
