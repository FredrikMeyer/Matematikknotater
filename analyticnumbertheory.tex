\documentclass[11pt, english]{article}
%\usepackage[latin1]{inputenc}
\usepackage[T1]{fontenc}
\usepackage[utf8]{inputenc}
\usepackage[english]{babel}   % S P R A A K


% \usepackage{graphicx}    % postscript graphics
\usepackage{amssymb, amsmath, amsthm, amssymb} % symboler, osv
\usepackage{mathrsfs}
\usepackage{url}
\usepackage{thmtools}
\usepackage{enumerate}  % lister $  
\usepackage{float}
\usepackage{tikz}
\usepackage{tikz-cd}
\usetikzlibrary{calc}
%\usepackage{tikz-3dplot}
\usepackage{subcaption}
\usepackage[all]{xy}   % for comm.diagram
\usepackage{wrapfig} % for float right
\usepackage{hyperref}
\usepackage{mystyle} % stilfilen      

\usepackage[a5paper,margin=0.5in]{geometry}

\begin{document}
\title{Number theoretic functions}
\author{Fredrik Meyer}
\maketitle 

\section{Summary}

\begin{itemize}
\item \textbf{The prime counting function:}
$$
\pi(x) = \# \{ p \leq x \mid \text{p prime} \}.
$$
\item \textbf{The von Mangoldt function:}
$$
\Lambda(n) = \begin{cases} \log p & \text{if $n=p^k$ for $p$ prime} \\
0 & \text{else}
\end{cases}
$$
\item \textbf{The first Chebyshev function} (or the \textbf{summatory von Mangoldt function:})
$$
\vartheta(x) = \sum_{n \leq x} \log p
$$
\item \textbf{The second Chebyshev function}:
$$
\psi(x) = \sum_{n \leq x} \Lambda(n).
$$

\end{itemize}


\end{document}