\documentclass[11pt, english]{article} 
\usepackage[T1]{fontenc}
\usepackage[utf8]{inputenc}

\usepackage[english]{babel}   % S P R A A K
%\usepackage{showlabels} 
   
% \usepackage{graphicx}    % postscript graphics
\usepackage{mathrsfs} 

\usepackage{amssymb, amsmath, amsthm, amssymb} % symboler, osv
\usepackage[all]{xy}
\usepackage{url}
\usepackage{thmtools}  
\usepackage{mystyle} % stilfilen
\usepackage{enumerate}  % lister $ 
\usepackage{float} 
\usepackage{tikz}
%\usepackage{hyperref} 
%\hypersetup{
%    colorlinks=true,
%    linkcolor=black, 
%    citecolor=black,
%    filecolor=black,  
%    urlcolor=black,
%}    
 
  

\title{Notes on algebraic stacks}
\author{Fredrik Meyer}
\date{} 


\begin{document}  
\maketitle 
\begin{abstract}
These are lecture notes from a course on algebraic stacks I attended during the fall semester 2013. They follow very closely the lectures, which in turn followed a draft version of the book ``Algebraic spaces and stacks'' by Martin Olsson. The lectures were held by Paul Arne Østvær. All mistakes are my own.
\end{abstract}

\tableofcontents

\section{Moduli problems, spaces, and stacks. Vector bundles and K-theory}

We all know of varieties.  In this course we will learn of wide generalizations of varieties. Here's some containments: varieties $\subset$ schemes $\subset$ algebraic spaces $\subset$ Deligne-Mumford stacks '69 $\subset$ Artin stacks '74.

So, at the end of this course, you should be able to answer the question, when stopped on the street, ``what's an algebraic stack?''.

Algebraic stacks were introduced by Deligne in his SGA4 and by Giroud in his thesis [[source needed]].

The need for stacks arose in the study of moduli spaces in algebraic geometry. Roughly, a \emph{moduli space} is a geometric space (scheme or stack) whose points represents algebro-geometric data of some kind, or isomorphism classes of such. Some examples:

\begin{itemize}
\item $\R\PP^1$ is the moduli space of lines through the origin in $\R^2$.
\item The Grassmannian $\Gr(n,V)$ is the moduli space of all $n$-dimensional linear subspaces of the vector space $V$.
\end{itemize}

Then, what is a \emph{moduli problem}?

\begin{defi}
A \emph{moduli problem} is a contravariant functor
\[ \mathcal F : \Sch^{op} \to \Set.
\]
\end{defi}

We say that a scheme $M \in \Sch$ is a \emph{fine moduli space} for $\mathcal F$ if it represents $\mathcal F$, that is, if
\[
\mathcal F (X) \approx \Hom(X,M)
\]
for all schemes $X$.

\subsection{Some category theory}

Here, $\mathcal C$ is a category and $X \in ob(\mathcal C)$. We define the functor $h_X: \mathcal C^{op} \to \Set$. On objects: $Y \mapsto \mathcal C (Y,X)$, and on morphisms: by composing. One checks that this defines a contravariant functor. We then say that a functor $\mathcal F:\mathcal C^{op} \to \Set$ is \emph{representable} if there exists $X \in ob(\mathcal C)$ with $\mathcal F \approx h_X$ (isomorphism of functors).

Now, let $X \xrightarrow{g} Y$ be a morphism in $\mathcal C$. This induces a morphism $h(g):h_X \to h_Y$ by sending a morphism $f$ to the composition $g \circ f \in h_Y(Z)$. Given $h(g)$, we can recover $g$, as the image of $\id_X \in h_X(X)$ under $h(g)(X)$.

More generally, suppose we are given a morphism of functors (or a natural transformation, same thing) $\eta:h_X \to \mathcal F$. This gives us a distinguished element $\tau_\eta \in \mathcal F (X)$, that is the image of $\id_X \in h_X(X)$ under $\eta_X$. In fact, this is a bijection.

\begin{lemma}[Yoneda's lemma]
There's a bijection between natural transformations
\[
\eta: h_X \to \mathcal F
\]
and the set $\mathcal{F}(X)$, and it is given by $\eta \mapsto \tau_\eta$.
\end{lemma}

\begin{example}
When $\mathcal F$ is represented by $Y$, that is $\mathcal F = h_Y$, we get a bijection
\[
\Hom_\CC (X,Y) \approx \catname{Nat}(h_X,h_Y) 
\]
\end{example}

\begin{exc}
Prove Yoneda's lemma, plus the following statement: If $\mathcal F : \mathcal C^{op} \to \Set$ is given, then a pair $(X,\eta)$, $X \in ob(\mathcal C)$ and $\eta:h_X \xrightarrow{\approx} \mathcal F$ is unique up to unique isomorphism.
\end{exc}
\begin{sol}

\end{sol}

\subsection{Back to moduli spaces}

The category of affine schemes $\AffSch$ is equivalent to the category $\CR^{op}$, the opposite category of the category of commutative rings. Here are some moduli problems for commutative rings.

\begin{example}
Let $\mathcal F:\CR \to \Set$ be defined by sending $R$ to $R^\times$, that is, sending a ring to its set of units. Then one can check that $\Z[x,\frac 1x]$ is a fine moduli space for units, because $\Hom_\CR(\Z[x,\frac 1x],R) \cong R^\times$.
\end{example}

\begin{example}
Let $\CR \to \Set$ be defined by sending a ring $R$ to the isomorphism class of $(M;m_1,\dotsc,m_n)$, where $M$ is a locally free $R$-module of rank $r$ generated by $m_1,\dotsc,m_r$. Two such pairs are isomorphic if there exists an isomorphism $\alpha$ with $\alpha(m_i)=m_i^\prime$. On maps, one just sends a ring map $R \to S$ to the map of modules $M \ \mapsto M \otimes_R S$.

There is a fine moduli space for this problem as well. It is the Grassmannian. $\Gr(r,n)= \Proj(\Z[x_I]/I_{r,n})$, where $I_{r,s}$ is the Plücker ideal. 
\end{example}

\begin{example}
Fix a numerical polynomial $P(z) \in \Q[z]$. Recall that a numerical polynomial is a polynomial that for large enough integer values of $\Z$, outputs only integers. Define the moduli functor $\mathsf{Hilbert}_{P,n}:\Sch^{op} \to \Set$ by sending a scheme $X$ to the set of all schemes $Y \subseteq \PP_X^n = \PP_\Z^n \times_\Z X$ flat over $X$ and such that the fibres of $x \in X$ all have Hilbert polynomial $P$. We define what happens to morphisms by base change.

Then there is a theorem of Grothendieck, that there exists a fine moduli space for $\mathsf{Hilbert}_{P,n}$, called the \emph{Hilbert scheme}. It is projective, and in fact contained in a Grassmannian.
\end{example}

\begin{example}
\[
\mathsf{Hilb}_1(\PP^n) = \PP ^n
\]
\[
\mathsf{Hilb}_n(\PP ^n) = \PP ^n
\]
The last equality because $\PP^n$ is the n'th symmetric product of $\PP ^1$.
\end{example}

However, a fine moduli space doesn't always exist! 

\begin{defi}
We say that $X \in \SchS k$ is a \emph{coarse moduli space} for a moduli problem $\mathcal F :\Sch^{op} \to \Set$ if the following two conditions hold:
\begin{enumerate}
\item There exists a \emph{universal} morphism $\rho_X : \mathcal{F} \to h_X$:
\[
\xymatrix{
\mathcal F \ar[r]^{\rho_X} \ar[dr]_f & h_X \ar[d]^{\exists ! h_f } \\
 & h_Y
}
\]
\item
For every algebraically closed field $\bar{k} \supset k$, the morphism
\[
\rho_X(\bar{k}):\mathcal F (\bar k) \to h_X(\bar k)
\]
is a bijection.
\end{enumerate}
\end{defi}

\begin{remark} Fine implies coarse. \end{remark}

Now for a rather lengthy example to motivate why coarse moduli spaces are necessary. Define the moduli functor $\mathcal F_{ell}:\SchS \Q^{op} \to \Set$ by sending $X$ to isomorphism classes of elliptic curves over $X$.

We recall what an elliptic curve is. In general, an elliptic curve over $S$ is an $S$-scheme together with a map $p:\E \to S$ satisfying the three conditions below:
\begin{itemize}
\item $p$ is proper and smooth of relative dimension $1$.
\item The geometric fibers of $p$ are connected curves of genus $1$.
\item There exists a section $0 \in E(s) = \Hom_\Sch(S,\E)$.
\end{itemize}
The first condition is explained in Hartshorne, chapter 3, paragraph ten. In particular, it means that for all $s \in S$, the fiber $p^{-1}(s)=E \times_S \Spec(k(s)) = E_s$ is geometrically regular over $k(s)$, i.e. the base change of $E_s$ to $\Spec (\bar{k(s)})$ is non-singular.

The second condition means that the curve $E_k$ defined by the pullback 
\[
\xymatrix{
E_k \ar[r] \ar[d] & \E \ar[d]^p \\
\Spec(k) \ar[r] & S
}
\]
has arithmetic genus $1$, meaning that $\dim_k H^1(E_k, \OO_{E_k}) = 1$.

\textbf{Claim:} The moduli problem $\mathcal F_{ell}$ has no fine moduli space. This is seen like this: the two curves defined by the equations $y^2=x^3-x$ and $2y^2=x^3-x$ are non-isomorphic over $\Q$, but become isomorphic over $\bar Q$ (via $(x,y) \mapsto (x,\sqrt{2}y)$). This implies that $\mathcal F_{ell} (\Q) \not \hookrightarrow \mathcal F_{ell} (\bar Q)$ is not an injection, but injectivity would have held if $\mathcal F_{ell}$ were representable.

Question: Is the obstruction for a fine moduli space for $\mathcal F_{ell}$ trivial if we work over an algebraically closed field? Turns out: NO!

The reason is because elliptic curves have non-trivial automorphisms. Because of the equation $y^2=f(x)$, we see that every elliptic curve has at least one non-trivial automorphism.

Suppose $\MM_{ell}$ is a fine moduli space for $\mathcal F_{ell}$. Then $\mathcal F_{ell}(\MM_{ell})=\Hom_\Sch(\MM_{ell},\MM_{ell})$, which contains the special element $\id_{\MM_{ell}}$, which corresponds to ``universal elliptic curve'' $\E_{univ} \to \MM_{ell}$. Then one checks that with these definitions, every elliptic curve is the pullback via a \emph{unique} map $X \to \MM_{ell}$:
\[
\xymatrix{
\E \ar[r] \ar[d] & \E_{univ} \ar[d] \\
X \ar[r] & \MM_{ell}
}
\]
But if $\E$ admits automorphisms, this map is not unique! (why?)

Luckily, there is a way forward by realizing that moduli problems live in a ``2-category of fibered categories over $\Sch$''.

A stack is what you get by starting with a groupoid and adding geometry. A groupoid is a category where all morphisms are isomorphisms.

\subsection{The way out of the problem}

Let $\Gpd$ be the collection of groupoids. That is, it consists of categories in which every morphism is an isomorphism.

\begin{example}
Every set $S$ can be regarded as a groupoid by setting
\[
\Hom(X,Y) = \begin{cases} \id_X & \text{if } X=Y \\ \emptyset & \text{otherwise}\end{cases}
\]
In this way we can identify $\Set$ as a full subcategory of $\Gpd$.

We want to redefine a moduli problem as a (pseudo-)functor
\[
\mathcal F : \Sch^{op} \to \Gpd.
\]
If we regard $h_X(Y)$ as a groupoid, then every such object has a trivial automorphism group.
\end{example}
Thus we need to abandon the category of schemes. We'll replace it by the ``2-category of stacks'' which accomodates automorphisms.

\subsection{Algebraic stacks and moduli of vector bundles}
\label{subsec:vb}

In this course we will study the moduli functor
\[
\VBundles_X^n: \SchS k^{op} \to \Set
\]
given by sending a scheme $Y$ to the set of isomorphism classes of rank $n$ vector bundles over $X \times_\Z Y$. On maps, base change.

\begin{example}
Now for a rather lengthy example. We will construct an example of a rank $2$ vector bundle on $\PP_k^2$. Cover $\PP_k^2$ with the standard affine patches, $U_0,U_1, U_2$ with affine coordinates $(\frac{x_1}{x_0},\frac{x_2}{x_0})$ on $U_0$ etc.

We want to construct the transition functions $\rho_{10}:U_0 \cap U_1 \to U_0 \cap U_1$. Let $\lambda \in k^\times$. Then we have the relation:
\[
\begin{pmatrix}
\frac{x_1}{x_0} \\
\frac{x_2}{x_0}
\end{pmatrix}
=
\begin{pmatrix}
(\frac{x_1}{x_0})^2 & 0 \\
(1-\lambda) \frac{x_1x_2}{x_0^2} & \lambda \frac{x_1}{x_0} 
\end{pmatrix}
\begin{pmatrix}
\frac{x_0}{x_1} \\
\frac{x_2}{x_1}
\end{pmatrix}
\]
The entries in the $2 \times 2$-matrix are regular functions on $U_0 \cap U_1$. Similarly, there is a relation $\rho_{21}:U_1 \cap U_2 \to U_1 \cap U_2$:
\[
\begin{pmatrix}
\frac{x_0}{x_1} \\
\frac{x_2}{x_1}
\end{pmatrix}
=
\begin{pmatrix}
\lambda^\prime \frac{x_2}{x_1} & (1-\lambda^\prime)\frac{x_0x_2}{x_1^2} \\
0 & \frac{x_2^2}{x_1^2}
\end{pmatrix}
\begin{pmatrix}
\frac{x_0}{x_2} \\
\frac{x_1}{x_2}
\end{pmatrix}
\]
Combining these two transition maps gives us the transition map $\rho_{20}=\rho_{21} \circ \rho_{10}$:
\[
\begin{pmatrix}
\frac{x_1}{x_0} \\
\frac{x_2}{x_0}
\end{pmatrix}
=
\begin{pmatrix}
\lambda^\prime \frac{x_1x_2}{x_0^2} & (1-\lambda^\prime)\frac{x_2}{x_0} \\
\lambda^\prime (1-\lambda) \frac{x_2^2}{x_0^2} & (\lambda \lambda^\prime -\lambda^\prime + 1) \frac{x_2^2}{x_0x_1}
\end{pmatrix}
\begin{pmatrix}
\frac{x_0}{x_2} \\
\frac{x_1}{x_2}
\end{pmatrix}
\]
To avoid division of zero on $U_2 \cap U_0$ we must thus have $\lambda \lambda^\prime -\lambda^\prime + 1=0$, that is, $\lambda^\prime = \frac{1}{1-\lambda}$. Hence if $\lambda \neq 0,1$, we can glue the patches $U_i \times \Aa_k^2$. Map to $\PP_k^2$ via the evident projection. This gives a vector bundle over $\PP_k^2$.
\end{example}

For convenience we repeat the definition of a vector bundle:
\begin{defi}
Let $X$ be a scheme over a field $k$. Then a \emph{vector bundle over $X$} is a Zariski trivial map $\pi:\E \to X$. That is, there exists an open covering of $X$ and isomorphisms $\varphi_i:\pi^{-1}(U_i) \xrightarrow{\simeq} U_i \times \Aa_k^n$ such that for all $i,j$ there exists transition maps $\varphi_{ij}:U_i \cap U_j \to \GL_n(k)$ such that $\varphi_i \circ \varphi_j^{-1}(x,v)=(x,\varphi_{ij}(x)v)$ for all $x \in U_i \cap U_j$ and $v \in \Aa_k^n$. We call the number $n$ the \emph{rank} of $\E$.
\end{defi}

The next proposition is homework:
\begin{prop}
There is an equivalence between the category $\VBundles_k$ of vector bundles of rank $n$ over $X$ and the category of locally free sheaves of rank $n$ on $X$.
\end{prop}
\begin{proof}
Also see \cite{neumann_stacks}, Chapter I. 
\end{proof}

Also this: Exercise II.5.18 in Hartshorne and exercise 6.10 and 6.11 in Hartshorne, the latter two are about $K$-theory of schemes.

\subsection{K-theory of schemes}

We introduce some of the main characters of K-theory of schemes. Here we assume that $X$ is a Noetherian scheme.

Let $\catname{Vb}(X)$ denote the set of isomorphism classes of vector bundles of finite rank on $X$. Let $\catname{Coh}(X)$ denote the set of isomorphism classes of coherent sheaves on $X$.

Now, define $K_0^\prime(X)=\Z\{ \catname{Coh}(X) \} / \sim$, that is the free abelian group on the set $\catname{Coh}(X)$ under the following equivalence relation: for every exact sequence
\[
0 \to \mathcal F ^\prime \to \mathcal F \to \mathcal F^{\prime \prime} \to 0
\]
we identify $[\mathcal F]$ with $[\mathcal F^\prime] + [\mathcal F^{\prime \prime}]$. Similarly, we define $K_0(X)$ via $\catname{Vb}(X)$.

\begin{lemma}
For $K_0$ and $K_0^\prime$, the following is true. 
\begin{itemize}
\item
Any scheme map $X \xrightarrow{f} Y$ induces a map on $K$-theory:
\[
f^*:K_0(Y) \to K_0(X)
\]
given by $[\E] \mapsto [f^*\E]$.
\item
A flat map $f:X \to Y$ induces a map $f^*:K_0^\prime(Y) \to K_0^\prime(X)$, defined in the same way as the above map.
\item
Tensor products of locally free sheaves turn $K_0(X)$ into a ring and $K_0^\prime(X)$ into a $K_0(X)$-module.
\item 
If $f:X \to Y$ is a proper map, then it induces a pushforward map $f_*:K^\prime(X) \to K_0^\prime(Y)$ given by
\[
f_*([\mathcal F]) = \sum_{i=0}^\infty (-1)^i [R^if_*\mathcal F],
\]
where the $R^if_*\mathcal F$ are the higher derived images of $\mathcal F$.
\item
If $f:X \to Y$ is proper and $\alpha \in K_0^\prime (X)$ and $\beta \in K_0(Y)$, we have a projection formula:
\[
f_*(f^*(\beta) \cdot \alpha) = \beta \cdot f_*(\alpha).
\]
Here $f^*(\beta) \in K_0(X)$ and $f_*(\alpha) \in K_0^\prime(Y)$, the equality taking place in $K_0^\prime(Y)$.
\end{itemize}
\end{lemma}
\begin{proof}
The first three statements are obvious from the defintions.

By standard scheme theory (for example Hartshorne III, section 8.8) there is an induces long exact sequence of coherent sheaves:
\[
0 \to f_*\mathcal F^\prime \to f_*\mathcal F \to f_*\mathcal F^{\prime \prime} \to R^1f_*\mathcal F^\prime \to \cdots \to R^nf_*\mathcal{F}^{\prime \prime} \to 0
\]
The statement now follows by breaking this long exact sequence into short exact sequences.

The last statement follows from the formula \[R^if_*(f^*(\E) \otimes_{\OO_X} \mathcal F) = \E \otimes_{\OO_Y} R^if_*(\mathcal F),\] which is Exercise 8.3 in Chapter III in Hartshorne.
\end{proof}

Here's a remarkable theorem:
\begin{thm}
If $X$ is regular, the two $K$-theories coincide, that is, we have an isomorphism
\[
K_0(X) \xrightarrow{\approx} K_0^\prime(X).
\]
\end{thm}
The theorem says that even though one category (the category of coherent sheaves) is much larger than the other category (the category of vector bundles), their $K$-theories coincide.

For more on $K$-theory, see for example \cite{rosenberg_ktheory}.


%%%%%%%%%%%%%%%%%%
%%%%%%%%%%%%%%%%%
\pagebreak 
\section{Flatness, limits of schemes and Kähler differentials}

We start by recalling definitions of flatness. Let $R$ be a ring and $M$ and $R$-module. Then $T(-)=(-) \otimes_R M$ is an endofunctor of $\Mod{R}$. We say that $M$ is \emph{flat} over $R$ if $T(-)$ is an exact functor.

\begin{defi}
We say that an $R$-module $M$ is \emph{faithfully flat} if for all $R$-modules $N$ and $N^\prime$ the induced homomorphism \[\Hom_R(N,N^\prime) \to \Hom_R(N \otimes_R M, N^\prime \otimes_R M)\]  is injective.
\end{defi}

We have a long list of equivalences:
\begin{prop}
The following are equivalent:
\begin{enumerate}
\item
$M$ is faithfully flat.
\item
$M$ is flat and for all $R$-modules $N^\prime$, the map
\begin{align*}
N^\prime &\to \Hom_R(M, N^\prime \otimes_R M) \\
y &\mapsto (m \mapsto y \otimes m)
\end{align*}
is injective.
\item
A sequence of $R$-modules
\[
E: N^\prime \to N \to N^{\dprime}
\]
is exact if and only if
\[
E \otimes M: N^\prime \otimes_R M \to N \otimes_R M \to N^\dprime \otimes_R M
\]
is exact.
\item
A homomorphism $N \hookrightarrow N^\prime$ is injective if and only if \[N^\prime \otimes_R M \hookrightarrow N \otimes_R M\] is injective.
\item
$M$ is flat and $N \otimes_R M=0$ implies $N=0$.
\item
$M$ is flat and for all maximal ideals $\mm \in \max(R)$, we have $M/\mm M \neq 0$.
\end{enumerate}
\end{prop}
\begin{proof}
$1 \Leftrightarrow 2:$ Assume 1). Then put $N=R$ in the definition of faithfully flat. This implies 2). For the other direction, let $F=\oplus_{i \in I} R$ be some free $R$-module surjecting onto $N$. Then we have a commutative diagram:
\[
\xymatrix{
\Hom_R(N,N^\prime) \ar[r] \ar@{^(->}[d] & \Hom_R(N \otimes_R M, N^\prime \otimes_R M) \ar@{^(->}[d] \\
\Hom_R(F,N^\prime) \ar[r] & \Hom_R(F \otimes_R M, N^\prime \otimes M)
}
\]
But $\Hom_R(F,N^\prime) \simeq \prod_{i \in I} N^\prime$ and $\Hom_R(F \otimes_R M, N^\prime \otimes_R M) \simeq \prod_{i \in I} \Hom_R(M,N^\prime \otimes_R M)$, and so by assumption, the bottom row is injective. Then by commutativity, the top arrow is injective, hence $M$ is faithfully flat.

$2 \Rightarrow 4:$ One direction is obvious, so assume $N^\prime \otimes_R M \hookrightarrow N \otimes_R M$ is injective. We have a commutative diagram:
\[
\xymatrix{
N^\prime \ar[d] \ar@{^(->}[r] & \Hom_R(M, N^\prime \otimes_R M) \ar[d] \\
N \ar@{^(->}[r] & \Hom_R(M, N \otimes_R M) 
}
\]
If we assume the right arrow is injective, it follows by commutativity that the left arrow is commutative also.

$4 \Rightarrow 5:$ Put $N=0$.

$5 \Rightarrow 6:$ Put $N=R/\mm$.

$5 \Leftrightarrow 3:$ 3) can be rephrased as ``$M$ is flat and for any short exact sequence $E$ (as above), the sequence $E \otimes M$ is also exact''. This implies 5) by looking at $0 \to N \to 0$. 

We want to show 3) using 5). Let $H= \ker (N \to N^\dprime) / \im(N^\prime \to N)$. Tensoring with $M$ and using flatness, we get
\[
H \otimes_R  M = \frac{\ker(N \otimes_R \to N^\dprime \otimes_R M)}{\im(N^\prime \otimes_R M \to N \otimes_R M)}
\]
The bottom sequence $E \otimes M$ in 3) is exact if and only if $H \otimes_R M=0$. By 5), this implies $H=0$, which implies that $E$ is exact.

It remains to show $6 \Rightarrow 2$. Assume that $M$ is flat, but that the map in 2) is not injective.  So let $x \in N^\prime$ be some non-zero element such that its image under $N^\prime \to \Hom_R(M,N^\prime \otimes_R M)$ is zero, that is, the map $x \mapsto x \otimes m$ is zero for all $x$. Let $\rho_x:R \xrightarrow{\cdot x} N^\prime$ be multiplication by $x$ and let $\ia = \ker \rho_x$, so we have an exact sequence of $R$-modules:
\[
0 \to \ia \to R \xrightarrow{\rho_x} N^\prime.
\]
Tensoring this with $M$ we get the exact sequence
\[
0 \to \ia M \to M \xrightarrow{\rho_x \otimes \id} N^\prime \otimes_R M.
\]
But the last map is the zero map, so by exactness $\ia M =M$. Let $\mm$ be some maximal ideal containing $\ia$. Then also $\mm M = M$, so $M/\mm M = 0$, so $6)$ is wrong.
\end{proof}

\begin{defi}
We say that a morphism of schemes $X \xrightarrow{f} Y$ is \emph{flat} if $f^\#:\OO_{Y,f(x)} \to \OO_{X,x}$ is flat for all $x \in X$. We say that $f$ is \emph{faithfully flat} if it is flat and surjective.
\end{defi}

\begin{remark}
Let $R \to R^\prime$ be a ring map. Then the map of affine schemes $\Spec R^\prime \to \Spec R$ is flat if and only if $R^\prime$ is a flat $R$-module. The same for faithfully flat.

To see this: flatness is a local property, so the first statement is trivial. For the second statement, assume the map of affine schemes is not surjective. Then there is some prime ideal $\pp \in \Spec R$ such that no $\qq$ in $R^\prime$ maps to $\pp$ as the inverse image of $f^\#$. Then $f^\#(p)$ is a unit in $R^\prime$ for all $p \in \pp$. This implies that the map

\begin{align*}
0 \to \pp \otimes_R R^\prime &\to R^\prime \\
p \otimes r &\mapsto f^\#(p)r
\end{align*}

is surjective, and so $R^\prime$ is not faithfully flat over $R$. The argument can be reversed.
\end{remark}

\begin{prop}
\label{prop:flatopen}
La $X \xrightarrow{f} Y$ be a flat map between locally noetherian schemes of locally finite type. Then $f$ is an open map.
\end{prop}
\begin{proof}
This is Theorem 2.12 in \cite{milne_etale}.  
\end{proof}

Here is an important corollary that will be used repeatedly in the next lectures.

\begin{corr}
\label{corr:fppfcover}
Let $f$ be as above. Let $\{U_i \}$ be an affine cover of $Y$. Then for every index $i$ there exists a Zariski open cover $\{ V_{ij} \}$ of $f^{-1}(U_i)$ where $V_{ij}$ is quasi-compact and $f(V_{ij})=U_i$.
\end{corr}
\begin{proof}
Let $p \in f^{-1}(U_i)$ and let $W_{ip} \subseteq f^{-1}(U_i)$ be an open affine neighbourhood of $p$. By the proposition, the image $f(W_{ip})$ is open in $Y$, and it is contained in the affine set $U_i$ which is quasi-compact. Letting $p$ vary, we see that $U_i = \cup_p f(W _{ip})$, and by quasi-compactness, we can choose a finite number of points $p_j$ such that $U_i= \cup_{p_j} f(W_{ip_j})$. 

Set $V_{ip} = W_{ip} \cup \bigcup_q W_{ip_q}$. Then $V_{ip}$ is a finite union of quasi-compact sets, so is itself quasi-compact, and clearly $f(V_{ip})=U_i$. 

Letting $p$ vary, we have exhibited an open cover $\{V_{ip}\}$ of $f^{-1}(U_i)$ for every $i$, satisfying the conclusion in the statement.
\end{proof}

\begin{defi}
Let $A$ be a ring and $M$ and $A$-module. We say that $M$ is \emph{finitely presented} if there is an exact sequence
\[
A^r \to A^s \to M \to 0
\]
for some natural numbers $r,s$\footnote{Zero is a natural number.}. Note that if $A$ is Noetherian then this is equivalent to $A$ being just finitely generated.

If $A \xrightarrow{f} B$ is a ring map, we say that $B$ is a finitely presented $A$-algebra if there is a surjection
\[
\pi: A[x_1, \dotsc, x_n] \to B
\]
with finitely generated kernel. Again: if $A$ is noetherian, this is equivalent to $B$ being just a finitely generated $A$-algebra (by Hilbert's basis theorem).
\end{defi}

These definitions have of course their analogues in the category of schemes and sheaves:

\begin{defi}
We say that a quasi-coherent sheaf $\mathcal F$ on a scheme $X$ is \emph{locally finitely presented} if for all open affines $\Spec(B) \subseteq X$, the $B$-module $\Gamma(\Spec(B),\mathcal F)$ is a finitely presented $B$-module.

We say that a map $X \xrightarrow{f} Y$ of schemes is \emph{locally of finite presentation} if for each affine open set $\Spec B$ of $Y$, and affine cover $f^{-1}(\Spec B)=\cup \Spec A_i$ of the inverse image, each ring map $B \to A_i$ is finitely presented as $B$-algebras. We say that the morphism is of \emph{finite presentation} if it is quasi-compact and quasi-separated.
\end{defi}

\begin{remark}
Recall that a morphism $f:X \to Y$ is \emph{quasi-separated} if for every affine open subset $U$ of $Y$, the inverse image $f^{-1}(U)$ is a quasi-separated scheme. Recall further that this means that the intersection of any quasicompact open sets are quasicompact. For more on numerous properties of schemes, see Ravi Vakil's ever-developing notes \cite{ravi_vakil}.    
\end{remark}

\begin{lemma}
Locally finitely presented maps are closed under composition.
\end{lemma}
\begin{proof}
This follows from the corresponding facts of ring maps: If $A \to B$ (resp. $B \to C$) are finitely presented $A$-algebras (resp. $B$-algebras), then $C$ is a finitely presented $A$-algebra. This is easy to see.
\end{proof}

Here's a useful observation:
\begin{lemma}
\label{lemma:pullpush}
Let $X \xrightarrow{f} Y$ be a locally finitely presented faithfully flat map (fppf). Let $E$ be a sheaf on $X$. Then the following is true: $f_\ast f^\ast E \approx E$.
\end{lemma}
\begin{proof}
Both are sheaves on $X$, so it is enough to compare stalks. We have:
\begin{align*}
(f_\ast f^\ast E)_x &:= \lim_{U \ni x} (f_\ast(f^\ast E))(U) \\
&= \lim_{U \ni x} (f^\ast E)(f^{-1}(U))\\
&= \lim_{U \ni x} \lim_{V \supseteq f(f^{-1}(U))} E(V) \\
&= \lim_{U \ni x} E(U) = E_x
\end{align*}
The last equality follows because $f$ is surjective and open.
\end{proof}

\subsection{Limits of schemes}

Fix a partially ordered set $(I,\geq)$. Assume that $I$ is filtered, meaning that for any pair of elements, there is some element that is greater than both of them. We can think of $I$ as a category, in which there is an arrow $\lambda \to \mu$ if and only if $\lambda \geq \mu$. 

Fix a scheme $B$. A \emph{projective system}/\emph{inverse system} of $B$-schemes, indexed by $I$, is a functor $S_. : I \to \SchS{B}$. That is, for all $\lambda \in I$, we have a $B$-scheme $S_\lambda$. For all $\lambda \geq \mu$, some $\mu \in I$, we have maps 
\[
\theta_{\lambda \mu}:S_\lambda \to S_\mu.
\]
We say that $S_.$ have affine transition maps if each $\theta_{\lambda\mu}$ is an affine map.

\begin{example}
Let $B= \Spec \Z$ and $A$ a ring. Let $I$ consist of all finitely generated subrings of $A$ and order them by inclusion. We get a projective system in which each $S_\lambda = \Spec(A_\lambda)$. All the transition maps are inclusion maps. 
\end{example}

\begin{lemma}
Let $S_.$ be a projective system of $B$-schemes with affine transition maps. Then the inverse limit $\lim_{\lambda \in I} S_\lambda$ exists in the category of $B$-schemes. For all $\lambda \in I$, the map
\[
\lim_{\lambda \in I} S_\lambda \to S_\lambda
\]
is affine.
\end{lemma}
\begin{proof}[``Proof'']
Since $I$ is directed, we can assume $I$ has a least element $\lambda_0$, since the direct limit is unaffected by this. Then for every $\lambda$, the map $S_\lambda \to S_{\lambda_0}$ is affine. Then for each of these we have a quasi-coherent sheaf of algebras, $\mathcal A_\lambda := \theta_{\lambda \lambda_0*}\OO_{S_{\lambda_0}}$, and we can form the relative spectrum of this, we have equalities $S_\lambda = \SSpec(\mathcal A_\lambda)$ for all $\lambda$. Now, the inverse limit exists in the category of sheaves, and so $\mathcal A = \colim_{\lambda \in I} \mathcal A_\lambda$ exists. Define $S =\SSpec(\mathcal A)$. We get maps $\rho_\lambda:S \to S_\lambda$ and properties of $\SSpec$ shows that $S = \lim_{\lambda \in I} S_\lambda$. 
\end{proof}

\begin{prop}[EGA, IV 8.14.2] A map $X \xrightarrow{f} Y$ is of locally finite presentation if and only if for every projective system of $Y$-schemes $\{S_\lambda\}_{\lambda \in I}$, where each $S_\lambda$ is affine, the map
\[
\colim_{\lambda \in I} \Hom_Y(S_\lambda, X)  \to \Hom_Y(\lim_{\lambda \in I} S_\lambda, X)
\]
is a bijection.
\end{prop}
\begin{proof}
This is Proposition 31.5.1 in \cite[Tag 01ZB]{stacks-project}. 
\end{proof}

\begin{remark}The bijection in the proposition can be more compactly written as $\colim h_X(S_\lambda)  = h_X(\lim S_\lambda)$.
\end{remark}

\begin{example}
Let $Y= \Spec k$. Let $X=\Spec A$, where $A$ is a $k$-algebra, \emph{not} finitely generated over $k$. Let $I$ be the set of finitely generated sub-algebras of $A$, ordered by inclusion. Then $X = \lim S_\lambda$, where $S_\lambda = \Spec A_\lambda$, $\lambda \in I$. Then $\id_X \in \Hom_Y(\lim_{\lambda \in I} S_\lambda, X)=\Hom_Y(X,X)$ is \emph{not} in the image of $\colim_{\lambda \in I} \hom_Y(S_\lambda, X)$.
\end{example}

\subsection{Kähler differentials}

Let $A \to B$ be a ring map. Let $M$ a $B$-module. An $A$-derivation of $B$ is an $A$-linear map, satisfying the Leibniz rule. That is, for all $f,g \in B$, we have
\[
d(fg) = fd(g) + gd(f).
\]
This implies that $d(a)=0$ for all $a \in A$. We denote by $\Der_A(B,M)$ the set of all such derivations.

\begin{defi}
Let $B$ be an $A$-algebra. The module of Kähler differentials of $B$ over $A$ is the $B$-module $\Omega_{B/A}^1$ with an $A$-derivation $d:B \to \Omega_{B/A}^1$ satisfying the universal property:
\[
\xymatrix{
B \ar[r]^{d^\prime} \ar[dr]_d & M \\
& \Omega_{B/A}^1 \ar[u]^{\exists ! }
}
\]
Given $d^\prime$ a derivation, it factors uniquely through $\Omega_{B/A}^1$.
\end{defi}

\begin{prop}
The pair $(\Omega_{B/A}^1,d)$ exists and is unique up to unique isomorphism.
\end{prop}
\begin{proof}
Standard hocus pocus.
\end{proof}

By the universal property, we have an identification:
\[
\Hom_B(\Omega_{B/A}^1,M) \xrightarrow{\approx} \Der_A(B,M).
\]
So in fancy language, the module of differentials is represented by the module of Kähler differentials (or \emph{corepresented}, depending on your taste).

\begin{exc}
\label{ex_isquared}
Let $B$ be an $A$-module, and let $\rho:B \otimes_A B \to B$ be the diagonal map, $b_1 \otimes b_2 \mapsto b_1b_2$. View $B \otimes_A B$ as a $B$-module via $b\cdot (b_1 \otimes b_2) = b_1 \otimes bb_2$. Let $I = \ker \rho$. Then also $I/I^2$ is a $B$-module. Define
\begin{align*}
d:B &\to I/I^2 \\
b &\mapsto db = b \otimes 1 - 1 \otimes b.
\end{align*}
Then $d$ is an $A$-derivation and $I/I^2$ satisfies the universal property of $\Omega_{B/A} ^1$. So we have an isomorphism $(\Omega_{B/A}^1,d) \approx (I/I^2,d)$.
\end{exc}
\begin{sol}
That $I/I^2$ is a $B$-module is obvious. We check that $d:B \to I/I^2$ is a $A$-derivation. Let $f,g \in B$. First note that
\begin{equation}
\label{eq:prod}
(1 \otimes f - f\otimes 1)(1 \otimes g - g \otimes 1) = 1 \otimes fg - g \otimes f -f \otimes g + fg \otimes 1 
\end{equation}
is equal to zero modulo $I ^2$.  This is the same as saying
\[
1 \otimes fg + fg \otimes 1 = f \otimes g + g \otimes f
\]
modulo $I ^2$. Thus:
\begin{align*}
fd(g)+gd(f) &= (1 \otimes fg - g \otimes f) + (1 \otimes fg - f \otimes g) \\
&= 1 \otimes fg - 1 \otimes fg + 1 \otimes fg - fg \otimes 1 \\
&= 1 \otimes fg - fg \otimes 1 \\
&= d(fg) 
\end{align*} 

That $d(A)=0$ follows by setting $g=1$ and $f=a$ for $a \in A$ and applying the just proven Leibniz rule.

It remains to prove the universal property. We have a canonical map $h: \Omega_{B/A}^1 \to I/I^2$ defined by $df = 1 \otimes f - f \otimes 1$. First off, it is a surjection, since $I$ is generated as an ideal by exactly the expressions of the form $1 \otimes f - f \otimes 1$. Too see this: clearly all such expressions lie $I$. Conversely, suppose $\sum f_i \otimes g_i$ is in $I$. This means that $\sum f_ig_i = 0$ so that $\sum f_i \otimes g_i = \sum f_i \otimes g_i + \sum f_ig_i \otimes 1 = \sum f_i(1 \otimes g_i + g_i \otimes 1)$. 

So the map $h: \Omega_{B/A} ^1 \to I/I^2$ is surjective. We now show that there is a map $s:I/I^2 \to \Omega_{B/A}^1$ such that $s \circ f = \id$, proving that $f$ is injective as well. Namely, define $s:I/I^2 \to \Omega^1_{B/A}$ by $f \otimes g \mapsto f dg$. There is at least such a map from $I$ by properties of the tensor product and the bilinearity of $(f,g) \mapsto f dg$. Now, let $1 \otimes f - f \otimes 1$ be a generator of $I$, and resp. for $1 \otimes g - f \otimes g$. Their product is as in in \eqref{eq:prod} and is mapped to 
\[
d(fg) - gd(f) - fd(g) + fgd(1).
\]
This is mapped to zero by the Leibniz rule. Hence we have a well-defined map $s:I/I^2 \to \Omega_{B/A}^1$. Also:
\[
s \circ h(df) = s(1 \otimes f - f \otimes 1) = df - fd(1) = df.
\]
Thus $h$ is a bijection, and we have established the universal property of $(I/I^2,d)$. 
\end{sol}

\begin{example}
\label{ex:om}
Let $B=A[T_1,\dotsc, T_n]$ be the polynomial ring in $n$ variables. Then $\Omega_{B/A}^1$ is the free $B$-module generated by the symbols $dT_i$ for $i=1,\dotsc,n$.
\end{example}

\begin{example}
If $B$ is a localization or a quotient of $A$ then $\Omega_{B/A}^1 = 0$.
\end{example}

Let $B \xrightarrow{\rho} C$ be a map of $A$-algebras. In particular, we have a map $\gamma:A \to B$.  Then we have natural maps of $C$-modules:
\begin{align*}
\alpha: \Omega_{B/A}^1 \otimes_A C &\to \Omega_{C/A}^1 \\
db \otimes c &\mapsto cd(\rho(b))
\end{align*}
and
\begin{align*}
\beta:\Omega_{C/A}^1 &\to \Omega_{C/B} ^1  \\
a dc &\mapsto \gamma(a) dc
\end{align*}

Then we have:
\begin{prop}
\label{prop:omega}
Let $B$ be an $A$-algebra. Then the following holds:
\begin{enumerate}
\item
\emph{Base change}. For $A^\prime$ an $A$-algebra, let $B^\prime = B \otimes_A A^\prime$. Then there is an isomorphism of $B^\prime$-modules $\Omega_{B^\prime/A^\prime}^1 \approx \Omega_{B/A}^1 \otimes_B B^\prime$. 
\item
If $B \to C$ is a map of $A$-algebras with $\alpha,\beta$ as above, then the sequence
\[
\Omega_{B/A} ^1 \otimes_B C \xrightarrow{\alpha} \Omega_{C/A}^1 \xrightarrow{\beta} \Omega_{C/B}^1 \to 0
\]
is exact.
\item
\emph{Localization}. If $S$ is a multiplicatively closed subset of $B$, then
\[
S^{-1}\Omega_{B/A} ^1 \approx \Omega_{S^{-1}B/A} ^1.
\]
\item
\emph{Conormal sequence}. If $C=B/I$, we have an exact sequence
\[
I/I^2 \xrightarrow{\delta} \Omega_{B/A}^1 \otimes_B C \xrightarrow{\alpha} \Omega_{C/A} ^1 \to 0
\]
where $\delta(b)=db \otimes 1$.
\end{enumerate}
\end{prop}
\begin{proof}
For 1), the derivation $d:B \to \Omega_{B/A}^1$ induces an $A^\prime$-derivation
\[
d^\prime := d \otimes \id_{A^\prime}: B^\prime \to \Omega_{B/A}^1 \otimes_A A^\prime .
\]
Now use the fact that $\Omega_{B/A}^1 \otimes_A A^\prime = \Omega_{B/A} ^1 \otimes_B B^\prime$, and check that the pair $(\Omega_{B/A}^1 \otimes_B B^\prime, d^\prime)$ satisfies the universal property. Or else just compute an explicit isomorphism.

Number 2) is Proposition 16.2 in Eisenbud \cite{eisenbud}. Number 3) is Proposition 16.9. Number 4) is Proposition 16.3.
\end{proof} 

\begin{corr}
Let $B$ be a finitely generated $A$-algebra, or a localization of one. Then also $\Omega_{B/A}^1$ is finitely generated over $B$.
\end{corr}
\begin{proof}
This follows from 4) in Proposition \ref{prop:omega} and Example \ref{ex:om}.
\end{proof}  

\begin{example}
Let $B=A[T_1, \dotsc, T_n]$ og let $F \in B$ and let $C=B/(F)$. Then
\[
\Omega_{C/A} ^1 = \oplus_{i=1}^n CdT_i/CdF
\]
where $dF = \sum_{i=1}^n \frac{\partial F}{\partial T_i} dT_i$.
\end{example}

\begin{exc}
Let $A$ be a ring. Show the two following statements:
\begin{enumerate}
\item Let $B_1$ og $B_2$ be $A$-algebras and $R=B_1 \otimes_A B_2$. Then
\[
\varphi:(\Omega_{B_1/A}^1 \otimes_{B_1} R) \oplus (\Omega_{B_2/A}^1 \otimes_{B_2} R) \to \Omega_{R/A}^1
\]
is an isomorphism, where $\varphi(db_1 \otimes r_1 + db_2 \otimes r_2) = r_1 d(b_1 \otimes 1) + r_2d(b_2 \otimes 1)$.  So ``$\Omega$ turns products into sums''.
\item 
Let $B$ be an $A$-algebra and set $C=B[T_1, \dotsc, T_n]/I$ for some ideal $I$. If $\alpha$ and $\delta$ is defined as above, show that there is a surjection of $B$-modules $\ker \delta \to \ker \alpha$.
\end{enumerate}
\end{exc}
\begin{sol}
We prove that it is an isomorphism by giving an inverse. A general element of $\Omega_{R/A}^1$ is a sum of elements of the form $d(b_1 \otimes b_2)$, where $b_1 \in B_1$ and $b_2 \in B_2$. Let $\psi(b_1 \otimes b_2)=d(b_1) \otimes (1 \otimes b_2) + d(b_2) \otimes (b_1 \otimes 1)$. I claim that this is an inverse of $\varphi$.

Write $r_1 = f_1 \otimes f_2$ for a pure tensor in $R$. Such tensors generate $R$. We first check that $\psi \circ \varphi = \id$:

\begin{align*}
&\,  \psi \circ \varphi(db_1 \otimes (f_1 \otimes f_2) + db_2 \otimes (g_1 \otimes g_2) ) \\  &= \psi((f_1 \otimes f_2)d(b_1 \otimes 1)  + (g_1 \otimes g_2)d(1 \otimes b_2)) \\
&= (f_1 \otimes f_2)( db_1 \otimes (1 \otimes 1)) + (g_1 \otimes g_2)(db_2 \otimes (1 \otimes 1)) \\
&= db_1 \otimes (f_1 \otimes f_2) + db_2 \otimes (g_1 \otimes g_2)
\end{align*}
For the equality on the second line, we used that $d(1)=0$, and the third line used that $R$-multiplication in $\Omega_{B_i/A}^1 \otimes B_i R$ is done on the second factor. So $\psi \circ \varphi = \id$.

We check that $\varphi \circ \psi = \id$, using the same notation as above:
\begin{align*}
\varphi \circ \psi(d(b_1 \otimes b_2)) &= \varphi( db_1 \otimes (1 \otimes b_2) + db_2 \otimes (b_1 \otimes 1)) \\
&= (1 \otimes b_2) d(b_1 \otimes 1) + (b_1 \otimes 1) d(1 \otimes b_2) \\
&= d((b_1 \otimes 1)(1 \otimes b_2)) = d(b_1 \otimes b_2)
\end{align*}
The last line used the Leibniz rule. The proposition is proved.
\end{sol}

Here is a lemma that will be useful later when we talk about étale maps.
\begin{lemma}
Let $k$ be a field and $E$ a field extension of $k$. Put $K=E[T]/(P(t))$ for a polynomial $P$. Then if $E$ is separable over $k$, the Kähler differentials vanish: $\Omega_{K/E}^1=0$ and $\Omega^1_{E/k} \otimes_E K \xrightarrow{\simeq} \Omega_{K/k}^1$ is an isomorphism.
\end{lemma}
\begin{proof}
Let $P^\prime(t)$ be the derivative of $P$. Then $\Omega_{K/E} ^1 = K dt / P^\prime(t) dt \simeq E[T]/(P^\prime(T), P(T))$. Separability implies that $P(t)$ and $P^\prime(t)$ have no common factor, hence they generate the unit ideal, hence $\Omega_{K/E}=0$.
\end{proof}

\subsection{Kähler differentials on schemes}

The ``obvious'' construction yields what we want. By abuse of notation, we also denote the resulting sheaf by $\Omega_{X/Y}^1$. 

\begin{prop}
Let $X \xrightarrow{f}Y$ be a map of schemes. Then there exists a unique quasi-coherent sheaf $\Omega_{X/Y}^1$ on $X$ such that for any open affine $V \subseteq Y$ and $U \subseteq f^{-1}(V)$ and $x \in U$, we have $\Omega_{X/Y}^1{|U} \simeq \widetilde{\Omega_{\OO_X(U)/\OO_Y(U)}^1}$. At each stalk, we have
\[
(\Omega_{X/Y}^1)_x \simeq \Omega_{\OO_{X,x}/\OO_{Y,f(x)}}^1.
\]
\end{prop}

\begin{proof}
This is a standard construction. We define $\Omega_{X/Y}^1(U)$ to be the set of functions
\[
s:U \to \coprod_{x \in U} \Omega_{\OO_{X,x}/\OO_{Y,f(x)}}
\]
such that for every $x \in U$, there exists an affine open neighbourhood $V_y$, where $y=f(x)$ and an open affine $U_x \subseteq f^{-1}(V_y)$ of $x$ and an $\omega$ in $\Omega_{\OO_X(U_x)/\OO_Y(V_y)}$ such that $\omega_{x^\prime} = s(x^\prime)$ for all $x^\prime \in U_x$.

Then one sees immediately that $\Omega_{X/Y}^1$ is a sheaf of $\OO_X$-modules, and that the statements in the proposition holds. 
\end{proof}

Another, more coordinate-independent way to define $\Omega_{X/Y}^1$ is the following: let $\Delta:X \to X \times_Y X$ be the diagonal map and let $\mathcal I = \ker (\OO_{X \times_Y X} \to \Delta_\ast \OO_X)$ be the kernel of the associated map of sheaves. Then it turns out that $\Omega_{X/Y}^1 \simeq \Delta^\ast(\mathcal I/\mathcal I^2)$. This holds essentially by Exercise \ref{ex_isquared}. See \cite[Chapter II, §8]{hartshorne} for details.

\begin{example}
For any scheme $Y$, let $X=\mathbb A_Y^n$. Then $\Omega_{X/Y}^1 \approx \OO_X^n$.
\end{example}

\begin{example}
Let $A$ be a ring and $X=\PP_A^1$. Then $\Omega_{X/\Spec A} \approx \OO_X(-2)$. Compare with \cite[Chapter II, §8]{hartshorne}, Thm 8.13. Why? We compute that on the open set $D_+(x_0) \cap D_+(x_1)$ we have
\[
x_0^2d(\frac{x_1}{x_0}) = -x_1^2 d(\frac{x_0}{x_1}).
\]
The coordinate on $D_+(x_0)$ can be identified with $X:=\frac{x_1}{x_0}$ and the coordinate on $D_+(x_1)$ with $Y:=\frac{x_0}{x_1}$. Then the calculation above says just that on the intersection, we have $dX=\frac{1}{X^2} d(\frac{1}{X})$, but applying the transition map $X \mapsto \frac 1X=Y$, this says $dX = Y^2 dY$, so $\OO_X \simeq \OO_X(2) \otimes \Omega_{X/\Spec A}^1$. Untwisting gives $\Omega_{\PP^1/A}^1=\OO_{\PP^1}(-2)$.
\end{example}

The following properties holds for Kähler differentials on schemes. Note that they are completely analogous to the results for Kähler differentials of modules from the previous lecture.

\begin{prop}
Let $X \xrightarrow{f} Y$ be a scheme map. Then:
\begin{enumerate}
\item 
\emph{Base change:} If $Y^\prime$ is a $Y$-scheme, let $X^\prime = X \otimes_Y Y^\prime$. Then
\[
\Omega_{X^\prime/Y^\prime}^1 = p^\ast \Omega_{X/Y}^1,
\]
where $p:X^\prime \to X$ is the natural projection.
\item \emph{Exact sequence:} Let $Y \to Z$ be a scheme map. Then there is an exact sequence of sheaves on $X$:
\[
f^\ast \Omega_{Y/Z}^1 \to \Omega_{X/Z}^1 \to \Omega_{X/Y}^1 \to 0.
\]
\item
\emph{Conormal sequence}: Let $j:Z \hookrightarrow X$ be a closed subscheme defined by a quasi-coherent sheaf of ideals $\mathcal I$. Then there is an exact sequence:
\[
j^\ast(\mathcal I/\mathcal I^2) \to \Omega_{X/Y} \otimes \OO_Z \to \Omega_{Z/Y} \to 0
\]
of sheaves on $Z$.
\end{enumerate}
\end{prop}

The proofs follows directly from Proposition \ref{prop:omega} by definition of $\Omega_{X/Y}^1$. 


%%%%%%%%%%%%%%%%%%%
%%%%%%%%%%%%%%%%%%%
\pagebreak
\section{Étale maps}

Étale maps are like local homeomorphisms in topology. In particular, they are what let us define the fundamental group of an algebraic variety. 

\subsection{Étale and smooth maps}

We define étale and smooth maps of schemes.  Compare with Section 10 in Chapter III in \cite{hartshorne}.

\begin{defi}
A map of schemes $X \xrightarrow{f} Y$ is \emph{formally smooth} (resp. \emph{formally unramified}/\emph{formally étale}) if for every affine $Y$-scheme $Y^\prime$ and every closed embedding $j: Y_0^\prime \hookrightarrow Y^\prime$ defined by a nilpotent ideal, the map
\begin{align*}
\Hom_Y(Y^\prime, X) &\to \Hom_Y(Y_0^\prime , X) \\
f &\mapsto f \circ j
\end{align*}
is surjective (resp. injective/bijective).

If in addition $f$ is of locally finite presentation, then $f$ is called \emph{smooth} (resp. \emph{unramified}/\emph{étale}).
\end{defi}

\begin{remark}
Assume $Y_0^\prime \to Y^\prime$ is defined by $\mathcal I \subseteq \OO_{Y^\prime}$ with $\mathcal I^n=0$ for some $n$. Set $Y_i^\prime \subseteq Y^\prime$ to be the closed subscheme defined by $\mathcal I^{i+1}$. This gives us a series of injections:
\[
Y_0^\prime \hookrightarrow Y_1^\prime \hookrightarrow \dotsc \hookrightarrow Y_{n-1}^\prime = Y^\prime
\]
with $Y_i^\prime$ defined in $Y_{i+1}^\prime$ by a square-zero ideal. Then $\Hom_Y(Y^\prime,X) \to \Hom_Y(Y_0^\prime,X)$ factors as
\[
\Hom_Y(Y^\prime, X) \to \Hom_Y(Y^\prime_{n-2}, X) \to \cdots \to \Hom_Y(Y_0^\prime, X),
\]
so in the definition, we could replace the word ``nilpotent'' by ``square zero''.
\end{remark}

Let $S$ be a set and $G$ a group acting on it. Then we say that $S$ is a \emph{$G$-torsor} if the action of $G$ is transitive.

Consider the set of liftings (dotted arrows) of $y_0$ in the diagram below:
\[
\xymatrix{
Y_0^\prime \ar[r]^{y_0} \ar@{_(->}[d]&  X \\
 \ar@{.>}[ru]Y^\prime 
}
\]
\begin{prop}
\label{prop:liftings}
The set of dotted liftings as above is either empty or a torsor under the action of the group $\Hom_{Y_0^\prime}(y_0^\ast \Omega_{X/Y}^1, \mathcal I)$.
\end{prop}
\begin{proof}
After some translation, I think this is just a global version of Lemma 4.5 in \cite{hartshorne_def}.
\end{proof}

Here are some properties that are easy to check:

\begin{prop}
The following holds: 
\begin{enumerate}
\item
If $X \xrightarrow{f} Y$ is smooth (resp. unramified/étale) and $g:Y^\prime \to Y$ any scheme map, then $f^\prime: X \times_Y Y^\prime \to Y^\prime$ is smooth (resp. unramified/étale).
\item
Smooth/unramified/étale maps are preserved under composition.
\item
Suppose given a composite of maps that are locally of finite presentation $X \xrightarrow{f} Y \xrightarrow{g} Z$ with $gf$ and $g$ smooth and such that $f^\ast \Omega_{Y/Z}^1 \to \Omega_{X/Z}^1$ is an isomorphism. Then $f$ is étale. 
\end{enumerate}
\end{prop}

We also have:
\begin{prop}
\begin{enumerate}[a)]
\item
If $f:X \to Y$ is smooth then $\Omega_{X/Y}^1$ is a locally free sheaf of finite rank on $X$.
\item
If $X \xrightarrow f Y$ is étale, then $\Omega_{X/Y}^1= 0$. 
\item
If $X \xrightarrow g Y$ is smooth and $i:Z \hookrightarrow X$ is a closed embedding, then the composite $f = i \circ g: Z \to Y$ is smooth if and only if the sequence
\[
0 \to i^\ast(\mathcal I_Z/\mathcal I^2_Z) \to i^*(\Omega_{X/Y}^1) \to \Omega_{Z/Y}^1 \to 0
\]
is exact and locally split.
\end{enumerate}
\end{prop}
\begin{proof}

\begin{enumerate}[a)]
\item We may assume that $X = \Spec B$ and $Y= \Spec A$. We want to show that $\Omega_{B/A}^1$ is projective. We define an operation, \emph{``bracketing''}: If $M$ is a $B$-module let $B[M]$ be the $B$-algebra $B \otimes M$ where $(b,m) \cdot (b^\prime,m^\prime) = (bb^\prime,bm^\prime+b^\prime m)$. This gives an evident surjection $\pi_M:B[M] \to B$ with square zero kernel.

Then it is an easy exercise to check that we have a bijection \[ \Hom_{\Mod B}(\Omega_{B/A},M) \simeq \Hom_{\catname{A-alg}}(B,B[M]).\] So we have found an adjunction, and it is easy to check that it is functorial. Since showing that $\Omega_{B/A}$ is projective is, by definition, equivalent to showing that there exists a dotted map below for any $B$-module $M^\prime$:
\[
\xymatrix{
 & \Omega_{B/A} \ar[d] \ar@{.>}[dl]_?  \\
M \ar[r] & M^\prime \ar[r] & 0
}
\]
But by the adjunction, this is equivalent to filling in the diagram below:
\[
\xymatrix{
 & B \ar[d] \ar@{.>}[dl]_? \\
B[M] \ar[r] & B[M^\prime]  \ar[r] & 0
}
\]
But since $X$ is smooth, the $?$-map exists by definition, so $\Omega_{B/A}^1$ is projective.

\item If $f$ is étale, the set of liftings is trivial, so by Proposition \ref{prop:liftings} $\Hom_{R}(\Omega_{X/Y}^1,M)=0$, hence $\Omega_{X/Y}^1=0$.

\item This was not proven.
\end{enumerate}
\end{proof}

Part c) of the proposition gives a ``geometric meaning of smooth/étale maps''. We will now try to explain what this means.

Let $f: \Spec B \to \Spec A$ be smooth. Fix $x \in \Spec B$. Assume $\Omega_{B/A}^1$ is free of rank $r$. Choose a surjection 
\[
A[x_1,\dotsc, x_n] \surj B
\]
with kernel $I$. We than have a split exact sequence of $B$-modules:
\[
0 \to I/I^2 \xrightarrow{\bar{d}} \Omega_{A[x_1,\dotsc,x_n]/A}^1 \otimes_A B \to \Omega_{B/A}^1 \to 0 .
\]

This implies that there exist $n-r$ functions $f_1,\dotsc, f_{n-r} \in I$ that map to a basis of $I/I^2$. Note that we have $\bar{d}(f_j) = \sum_{i=1}^n \frac{\partial f_j}{\partial x_i} dx_i$. Because the classes of $f_j$ form a basis for $I/I^2$ and the differentials $\{ dx_i \}_{i=1}^n$ form a basis for $\Omega_{A[x_1,\dotsc,x_n]/A}$, the condition that $\bar{d}$ is a split injection is equivalent to the $(n-r) \times (n-r)$ minors of the matrix $(\frac{\partial f_j}{\partial x_i}) \in M_{n \times (n-r)}(B)$ generating the unit ideal of $B$.

This tells us that there exist $g \in A[x_1,\dotsc,x_n]$ such that the $(n-r)\times(n-r)$ minors generate the unit ideal in $A[x_1,\dotsc,x_n][1/g]$ and such that the ideal in $A[x_1,\dotsc,x_n][1/g]$ generated by $I$ is $\langle f_1, \dotsc, f_{n-r} \rangle$.

This proves:

\begin{prop}
Let $X \xrightarrow{f} Y$ be of locally finite presentation. Then $f$ is smooth if and only if for every $x \in X$ there exists an affine neighbourhood $\Spec B \subseteq X$ containing $x$ and an affine neighbourhood $f(x) \in \Spec A \subseteq Y$ with $f(\Spec B) \subseteq \Spec A$ and such that 
\[
B \simeq A[x_1,\dotsc, x_n]/\langle f_1, \dotsc f_s \rangle  [1/g]
\]
for $f_1, \dotsc, f_s,g \in A[x_1,\dotsc, x_n]$ with $s \leq n$ and such that the $s \times s$-minors of the Jacobian generate the unit ideal of $B$.

If in addition $f$ is étale, then the above holds with $s=n$. 
\end{prop} 

So a smooth map has at least some similarity with the concept in differential geometry with the same name.

\begin{corr}
Let $A$ be a ring and $f \in A[x]$ a monic polynomial. Then the finite ring map $A \to A[x]/(f)$ is étale if and only if $f^\prime$ maps to a unit in $A[x]/(f)$.
\end{corr}
\begin{proof}
The Jacobian in this case is just the $1 \times 1$-matrix $(f^\prime)$, and the minors is just the single element $f^\prime$, and the proposition says that $A \to A[x]/(f)$ is étale if and only if $f^\prime$ generate the unit ideal in $A[x]/(f)$. But since $(f^\prime)$ is principal, this happens if and only if $f^\prime$ is a unit in the quotient ring. 
\end{proof}

\begin{example}
Let $K$ be a field and $F \in K[x]$ an irreducible polynomial. Then $L := K[x]/(F)$ is a finite field extension of $K$. Then the corollary says that $L/K$ is étale if and only if $F^\prime \neq 0$. 
\end{example}

\begin{example}
  Let $\Aa_k^1 \xrightarrow{x \mapsto x^n} \Aa_k^1$ be the map sending a point $x$ to its $n$'th power. Then the map is étale at $x \neq 0$ if and only if $(n, char( k)) = 1$. It is not étale at any point if $n = char (k)$.
\end{example}

Here's an important corollary. It says that smooth morphisms always have sections locally in the étale topology.
\begin{corr}
\label{corr:homotopy}
Let $f:X \to Y$ be smooth and let $x \in X$. Then there exists an étale map $\pi:Y^\prime \to Y$ with image containing $f(x)$ and a map $s:Y^\prime \to X$ such that $f \circ s = \pi$. 
\[
\xymatrix{
 & X \ar[d]^f \\
Y^\prime \ar@{.>}[ru]^s  \ar[r]^\pi &  Y
}
\]
\end{corr}
\begin{proof}[Proof sketch]

After shrinking $X$, we may assume there exists an étale map $\alpha:X \to \Aa_Y^n$ for some $n$. The reference is SGA. Let $Y^\prime$ be the fiber product of the diagram
\[
\xymatrix{
 & X \ar[d]^\alpha \\
Y \ar[r]^0 & \Aa_Y^n
}
\]
Then $Y^\prime$ fits into a diagram $Y \leftarrow Y^\prime \xrightarrow{s} X$ as desired. 
\end{proof}

\begin{remark}
So for example, locally $\sqrt[n]{x}$ exists: Just consider the smooth map $x \mapsto x^n$. This will be useful in a later example.
\end{remark}

\subsection{Invariance of étale sites under infinetesimal thickenings}

Let $S_0 \hookrightarrow S$ be a closed embedding, defined by a nilpotent ideal. Let $\Et S$ be the category having objects étale $S$-schemes and morphisms $S$-maps.

\begin{thm}
The functor
\begin{align*}
F: \Et S &\to \Et {S_0} \\
(Z \to S) &\mapsto (Z \X_S S_0 \to S_0)
\end{align*}
is an equivalence of categories.
\end{thm}
\begin{proof} 
This is Theorem 3.23 \cite{milne_etale}.
\end{proof}

We skip the proof for now, but philosophically it says that the étale topology on $S$ is equivalent to the étale topology on $S_0$, hence when dealing with sheaves in the étale topology, we can always assume that $S$ is reduced.

%%%%%%%%%%%%%%%%%%
%%%%%%%%%%%%%%%%555%%
\pagebreak
\section{Functors, Hilbert Polynomial, Grassmannians}

We begin by recalling something about representable functors.

Recall that a functor $F:\mathcal C^{op} \to  \Set$ is \emph{representable} if $F \approx h_X$ for some $X \in \mathcal C$.

\begin{defi}
Let $F,G$ be functors $\mathcal C^{op} \to \Set$. We say that a morphism $F \xrightarrow{f} G$ is \emph{relatively representable} if for all $X \in \mathcal C$ and $g:h_X \to G$, the fiber product
\[
h_X \times _G F: \mathcal C^{op} \to \Set
\]
is representable.
\end{defi}

Often we will just say that a morphism is \emph{representable}. In the future, we will often not distinguish between 

Now let $\mathcal C$ be the category $\catname{Aff}_S$ of affine schemes over an affine scheme $S$. Any affine $S$-scheme $X$ does of course define a functor $h_X$ as usual.

\begin{defi}
A morphism of functors $F \xrightarrow{f} G$ is an affine open (closed) embedding if the following holds: 
\begin{enumerate}
\item $f$ is relatively representable.
\item For all affine schemes $X$ and natural transformations $h_X \to G$, the map $F \times_G h_X \to h_X$ is an open (closed) embedding.
\end{enumerate}
\end{defi}
Note that the second condition makes sense only because of the first condition. 

\begin{defi}A \emph{(big) Zariski-sheaf} on $\catname{Aff}_S$ is a functor
\[
F:\catname{Aff}_S^{op} \to \Set
\]
such that for all $U \in \catname{Aff}_S$, and an open cover $\{U_i\}$ of $U$ of affine schemes, the sequence
\[
\xymatrix{
F(U) \ar[r] &  \prod_{i \in I} F(U_i) \ar@/^1ex/[r] \ar@/_1ex/[r]  & \prod_{i,j \in I} F(U_i \cap U_j)
}
\]
is an equalizer.
\end{defi}

\begin{defi}
A morphism of \emph{(big) Zariski sheaves} $g:F \to G$ is \emph{surjective} if for every affine scheme $U$ and $u \in G(U)$, there exists an open cover of $U$, $\{U_i\}$, such that $u_{|U_i} \in G(U_i)$ is in the image of $F(U_i)$ for all $i \in I$.
\end{defi}

Here's a criterion to check when a functor is represantable in schemes.

\begin{prop}
A functor $F:\catname{Aff}_S \to \Set$ is represented by a separated $S$-scheme if and only if the following three conditions are satisifed:
\begin{enumerate}
\item $F$ is a big Zariski sheaf.
\item The diagonal map  $\Delta: F \to F \times_S F$ is relatively representable and is a closed embedding.
\item There exists affine schemes $\{X_i \}$ and morphisms $\pi_i:h_{X_i} \to F$ that are relatively representable and affine open embeddings such that $\coprod_{i \in I} h_{X_i} \to F$ is a surjection of Zariski sheaves.
\end{enumerate}
In this case we have an equivalence of categories induced by the Yoneda embedding $h_{\_}$. On the left side: the category of separated $S$-schemes. On the right side: the category of functors $F:\catname{Aff}_S \to \Set$ that satisfies 1)-3).
\end{prop}

\begin{proof}
Suppose that $X$ is a separated $S$-scheme, and let $F=h_X$. We want to show that 1)-3) is satisified.
\begin{enumerate}
\item Let $U=\cup U_i$ where $U, U_i$ are affine. Need to show that a map of schemes $F:U \to X$ is equivalent to a collection of maps $\{ f_i:U_i \to X \}$ such that $f_{i|U_i \cap U_j} = f_{j|U_i \cap U_j}$. But this is obvious.

\item The Yoneda map commutes with products, that is, $h_{X \times_S X} \approx h_X \times_S h_X$. Thus we can identify the diagonal map $\Delta:h_X \to h_X \times_S h_X$ with $\Delta:h_X \to h_{X \times_S X}$, which is induced by $\Delta: X \to X \times_S X$, which is a closed embedding since $X/S$ is separable.

\item
Let $X = \cup X_i$ be an affine open cover. We get maps $h_{X_i} \to h_X$, induced by the inclusions. If $T$ is an affine $S$-scheme and $h_T \to h_X$ corresponds to a map of schemes $T \to X$, then the fiber product $h_T \times_{h_X} h_{X_i}$ is represented by $f^{-1}(X_i)$, because the Yoneda embedding commutes with the fiber product.

Since $X/S$ is separable and $S$ is affine, it follows that each $f^{-1}(X_i)$ is an open affine subset of $T$, thus the $\pi_i$ constitutes an open affine embedding. Then it is clear that $\coprod_{i \in I} h_{X_i} \to F$ is surjective.
\end{enumerate}
We now try to prove the other direction. 

So suppose $F:\catname{Aff}_S \to \Set$ is a functor satisfying 1-3). Choose $\pi_i:h_{X_i} \to F$ as in 3). We get a diagram
\[
\xymatrix{
 & h_{X_i} \ar[d]^{\pi_i} \\
h_{X_j} \ar[r]_{\pi_j} & F
}
\]
Because of the assumptions, the fiber product is represented by an affine scheme $V_{ij}$, so get open embeddings $X_j \hookleftarrow V_{ij} \hookrightarrow X_i$. There is an obvious isomorphism of functors $h_{X_j} \times_F h_{X_i} \simeq h_{X_i} \times_F h_{X_j}$ obtained by switching factors. This induces an isomorphism $\varphi:V_{ij} \to V_{ji}$ satisfying the glueing conditions of schemes. So we can glue to get a scheme $X$. Then $F \simeq h_X$. [[<- The last statement was an exercise...]]

For the equivalence in the proposition, one must show that $h_\_$ is fully faithful, which is part of the statement of the Yoneda lemma.
\end{proof}


\subsection{The Hilbert polynomial}

Let $k$ be a field and $j: X \hookrightarrow \PP_k^r$ a closed embedding. Let $\OO_X(1)$ be the inverse image of the Serre twist $\OO_{\PP_k^r}(1)$ under $j$. Let $\mathcal F$ be a coherent sheaf on $X$.

\begin{defi}
The \emph{Euler characteristic} of $\mathcal F$ is the integer
\[
\chi(X,\mathcal F) = \sum_{i=0}^\infty (-1)^i \dim_k H^i(X, \mathcal F).
\]
\end{defi}
By a theorem of Grothendieck (see \cite[Chapter III, Theorem 2.7]{hartshorne}), the higher cohomomology of $\mathcal F$ vanishes, so the sum is finite. In fact, $H^i(X,\mathcal F)=0$ for $i > \dim X$. We often abbreviate $\chi(X, \mathcal F)$ by $\chi(\mathcal F)$.

By Ex 5.1 in \cite{hartshorne}, we have the following:
\begin{prop}
Let $\mathcal F, \mathcal G, \mathcal H$ be coherent sheaves on $X$, fitting into a short exact sequence:
\[
0 \to \mathcal F \to \mathcal G \to \mathcal H \to 0.
\]
Then $\chi(G) = \chi(F) + \chi(H)$.
\end{prop}
\begin{proof}
A sort exact sequence of sheaves induces a long exact sequence on homology. After sorting the terms, we get the result.
\end{proof}

\begin{corr}
Let
\[
0 \to \mathcal F_1 \xrightarrow{d_1} \mathcal F_2 \xrightarrow{d_2} \dotsb \xrightarrow{d_{n-1}} \mathcal F_n \to 0
\]
be an exact sequence. Then $\sum_{i=1}^n (-1)^i \chi(\mathcal F_i)=0$.
\end{corr}
\begin{proof}
Split the long exact sequence into many short exact sequences:
\[
\xymatrix{
& & & 0 \ar[dr] & & 0  \\
& & & & \im d_{i+1} \ar[ur] \ar[dr] \\
\dotsc \ar[r] & \mathcal F_i \ar[dr]^{d_i}  \ar[rr]^{d_i} & & \mathcal F_{i+1} \ar[ur]^{d_{i+1}} \ar[rr]^{d_{i+1}} & & \mathcal F_{i+2} \ar[r] \cdots &
 \\
 & & \im d_i \ar[dr] \ar[ru] \\
& 0 \ar[ur] && 0
}
\]
Then we have
\begin{align*}
\sum (-1)^i \chi (\FF_i) &= \sum (-1)^i(\chi(\im d_i) + \chi(\im d_{i+1})) \\
&= \sum (-1)^i \chi(\im d_i) + \sum (-1)^{i-1} \chi(\im d_i) = 0 
\end{align*} 
Hence the corollary is proved.
\end{proof}

The next proposition is Exercise 5.2 in Chapter III in \cite{hartshorne}. 

\begin{prop}
There exists a polynomial $P(z) \in \Q[z]$ such that
\[
\chi(\mathcal F(n)) = P(n)
\]
for all $n \in \Z$. In addition, the degree of $P$ is bounded above by the dimension of $X$. 
\end{prop}
\begin{proof}
We start by simplifying the situation. We can substitute $\mathcal F$ with $j_\ast \mathcal F$ and $X$ with $\PP_k^r$ because $H^i(X, \mathcal F)=H^i(\PP_k^r, j_\ast \mathcal F)$. So we can assume $\mathcal F$ is a coherent sheaf on projective space.

By flat base change we can assume that $k$ is algebraically closed.

We proceed by induction on the dimension of the support of $\mathcal F$. Recall that $\supp(\mathcal F)= \{ x \in X \, | \, \mathcal F_x \neq 0 \}$.

If $\supp(\mathcal F)=\emptyset$, then $\mathcal F = 0$, so $P(n)=0$ works.  If however $\mathcal F \neq 0$, we have an exact sequence
\[
0 \to \mathcal G \to \mathcal F(-1) \xrightarrow{\cdot x_r} \mathcal F \to \mathcal H \to 0.
\]
If $x_r$ is chosen so that the hyperplane $x_r=0$ does not contain any components of $\supp(\mathcal F)$, it follows that $\supp(\mathcal G)$ and $\supp(\mathcal H)$ have dimension strictly lower than $\dim \supp(\mathcal F)$. Hence by the induction hypothesis, the difference
\[
\chi(\PP_k^r, \mathcal F(n)) - \chi(\PP_k^r, \mathcal F(n-1))
\]
is a polynomial of degree $\leq \dim \supp(\mathcal F)$. The rest follows like in Section 7, Chapter I, in Hartshorne.
\end{proof}

The polynomial $P(n)$ is called the \emph{Hilbert polynomial} of the $\OO_X$-module $\mathcal F$. If $\mathcal F= \OO_X$, then we say that $P(n)$ is the \emph{Hilbert polynomial} of $X$.

By Serre's vanishing theorem (III, Thm 5.2 in \cite{hartshorne}), there exists a natural number $n_0 \in \N$ such that for all $i > 0$, $H^i(X, \mathcal F(n)) = 0$ for $n > n_0$. So $P(n)= \dim_k H^0(X, \mathcal F(n))$ for large $n$. This is the ``old'' definition of the Hilbert polynomial.

\begin{exc}
Verify that the Hilbert polynomial of $\mathbb P_k^r$ is $P(n) = \binom{n+r}{r}$. Also, the Hilbert polynomial of $\Spec(k[t]/(t^2))$ is $P(n)=2$. Note that this is also the Hilbert polynomial of two distinct points. 
\end{exc}
\begin{sol}
For the first part, note that $\dim_k H^0(X, \OO_{\PP_k^r}(n))$ is just the number of monomials of degree $n$. A counting argument shows that there are $\binom{n+r}{r}$ of these.

Since $X= \Spec(k[t]/(t^2))$ is affine, all higher cohomology vanish. Hence $P(n) = \dim_k H^0(X,\OO_X)$, which is just the $k$-dimension of $k[t]/(t^2) \simeq k \oplus tk$, so $P(n)=2$. 
\end{sol}

The next theorem says that in a flat family, the Hilbert polynomial is constant. In fact, the converse is also true. 
 
\begin{thm}
Let $T$ be an integral noetherian scheme. Let $X \subseteq \PP_T^r$ be a closed subscheme and let $\mathcal F$ be a coherent sheaf on $X$. For each $t \in T$, let $P_t \in \Q[t]$ be the Hilbert polynomial of $\mathcal F_t$.

Then $\mathcal F$ is flat relative to $X \xrightarrow{f} T$ if and only if $P_t$ is independent of $t$ (i.e. all fibers have the same Hilbert polynomial).
\end{thm}
Here's a diagram of the situation:
\[
\xymatrix{
 &  \ar[dd] \ar[ld] \ar[r] X_t  \pullbackcorner & X \ar[dd]_f \ar[dr] & \\
\PP_{k(t)}^r  \ar[rd] && & \PP_T^r \ar[dl] \\
 & \Spec(k(t)) \ar[r] & T 
}
\]
\begin{proof}
This is proved in Hartshorne, Chapter II, Section 9, as Theorem 9.9. 
\end{proof}

The theorem says, philosophically at least, that the Hilbert scheme, parametrizing flat families with Hilbert polynomial $P(t)$, is universal among families parametrizing flat families.

\subsection{The Grassmannian}
\label{subsec:grassmannian}

For more about the Grassmannian (more than will be needed), take a look at the article \cite{kleiman_grassmannians}.  

Let $k$ be a field. We will define the Grassmannian $\Gr(d,n)$ for $n \geq d$ as a projective $k$-scheme. We have the the wedge map 
\[
f: \Aa_k^{dn} =  \Spec (k[x_{{ij}_{i \leq d, j \leq n}}]) \to \Spec([x_J]) = \Aa_k^{\binom nd}
\]
The coordinates on $\Aa_k^{\binom nd}$ are $\{x_J\}$, where $J$ is an ordered $d$-tuple, and the coordinates on $\Aa_k^{dn}$ are $x_{ij}$, elements of a generic $d \times n$-matrix.

The map $f$ is defined dually by $f^\#(x_J)=\det(x_{ij})$, which is to be interpreted as taking the determinant of the $d \times d$ submatrix of $(x_{ij}^J)$ involving only the columns $J$.

This gives us a rational map
\[
\phi: \Aa_k^{dn} \rmap \PP_k^{\binom nd - 1}
\]
via composition of $f$ with the standard rational map from $\Aa_k^{\binom nd}$ to $\PP_k^{\binom nd - 1}$. The locus of indeterminacy is $V(\det(x_{ij}))$, the zero set of all subdeterminants. So we have a well-defined map $\varphi:\Aa_k^{dn} \bs V(\det(x_{ij})) \to \PP_k^{\binom nd - 1}$.

If we set $x_I = (-1)^\sigma x_J$ if they differ by a permutation, we see that each $x_I$ gives a well-defined element of $H^0(\PP_k^{\binom dn-1}, \OO(1))$.

The \emph{Plücker quadric} associated to $I,J$ (both $m$-tuples) is the zero set of the following set of equations, for $1 \leq s \leq n$:
\[
Q_{I,J,s} := x_I x_J - \sum_{l=1}^d x_{i_1\dotsc i_{s-1} j_l i_{s+1} \cdots i_d}x_{j_1 \cdots j_{l-1} i_s j_{l+1} \cdots j_d }.
\]

Then we \emph{define} the Grassmannian $\Gr(d,n)$ to be the zero set of all the Plücker quadrics inside $\PP_k^{\binom nd -1}$.

\begin{example}
Set $d=2$ and $n=4$. Then there is only one Plücker quadric, and it is given by
\[
x_{12}x_{34}-x_{13}x_{24}+x_{14}x_{23}
\]
inside $\PP^5$. So $\Gr(2,4)$ is a quadric hypersurface of dimension $4$ in $\PP^5$.
\end{example}

Here comes the geometrical meaning of the Grassmannian:

\begin{prop}
Denote by $U$ the open set $\Aa_k^{dn} \bs V(\det(x_{ij}))$. Let $\varphi$ be the map $\varphi:U \to \PP_k^{\binom nd -1}$ as above.
\begin{enumerate}
\item $\varphi$ factorizes through $\Gr(d,n)$. That is, there is a commutative diagram
\[
\xymatrix{
U \ar[r] \ar[dr]_\varphi & \Gr(d,n) \ar@{^(->}[d] \\
 & \PP_k^{\binom nd - 1}
}
\]
\item The map $\varphi:U \to \Gr(d,n)$ is a locally trivial fiber bundle with fiber $\GL_d(k)$, and it gives a bijection between $d$-dimensional vector subspaces of $k^n$ and $k$-rational points of $\Gr(d,n)$.
\item Let $J$ be a sorted $d$-tuple and let $U_{x_J} := \PP_k^{\binom nd - 1} \bs V(x_J)$. Then $\Gr(d,n) \cap U_{x_J} \simeq \Aa_k^{d(n-d)}$. It follows that $\Gr(d,n)$ is a smooth, projective variety of dimension $d(n-d)$.
\end{enumerate}
\end{prop}
\begin{proof}
The first conditions say that the Plücker quadrics define the image of $\varphi$. This is a standard, see for example the references in my master's thesis \cite{masteroppgaven}.

For number 2), observe that $\varphi$ is equivariant with respect to left multiplication from $\GL_k(m)$, the group of  invertible $d \times d$-matrices. That is, there is a commutative diagram:
\[
\xymatrix{
GL_k(d) \times U \ar[r]^\mu \ar[d]_{pr_2} & U \ar[d]^\varphi \\
U \ar[r]^\varphi & \PP_k^{\binom nd - 1}
}
\]
This almost proves 2). The rest was an exercise.
\end{proof}
 
Thus the Grassmannian is a parameter space, parametrizing $d$-dimensional subspaces of the affine space $k^n$. 

\subsection{The Grassmannian as a functor}

Let $T$ be any scheme. We consider the set of all exact sequences 

\begin{equation}
\label{eq:eks}
0 \to K \to \OO_T^n \to Q \to 0
\end{equation}
where $K$ and $Q$ are locally free sheaves on $T$ of rank $d$ and $n-d$, respectively. We say that two such sequences are equivalent $\sim$ if there is a commutative diagram:
\[
\xymatrix{
0\ar[r]  & K \ar[r] \ar[d]^\simeq & \OO_T^n \ar[d]^= \ar[r] & Q \ar[r] \ar[d]^\simeq & 0 \\
0\ar[r]  & K^\prime \ar[r] & \OO_T^n  \ar[r] & Q^\prime \ar[r] & 0 \\
}
\]

\begin{defi}
We define a contravariant functor $F_{d,n}:\SchS k ^{op} \to \Set$. On objects:
\[
T \mapsto \{ 0 \to K \to \OO_T^n \to Q \to 0 \} / \sim,
\]
where the sequence is as above. On maps, just define via pullback. 
\end{defi}

 
\begin{thm}
The Grassmannian $\Gr(d,n)$ represents the functor $F_{d,n}$. 
\end{thm}
\begin{proof}
This is proven in the first chapter of János Kollár's book, ``Rational Curves on Algebraic Varieties'', which were handed out in the lecture.
\end{proof}

We can of course generalize this. We can substitute $k$ with an arbitrary commutative ring $A$ with unity. We can even generalize to the Grassmannian over any scheme $S$ by glueing on affines. We get $\Gr_S(d,n)$. We have a ``twisted'' version of the functor: Let $F_{m,\E}(T)$ consist of equivalence classes of sequences
\[
0 \to K \to \E_T \to Q \to 0
\]
of locally free sheaves on $T$, where the rank of $K$ is $d$ and $\E_T$ is the pullback $\E$ via $T \to S$ (so $\E$ is a sheaf on $S$).

\subsection{Quot and Hilbert schemes}

Let $\E$ be a coherent sheaf on $S$, where $S$ is a noetherian scheme. Let $X$ be a projective $S$-scheme and let $P(z)$ be a polynomial. Consider an exact sequence $\E_T \to Q \to 0$. We say that two such sequences are equivalent if there is a commutatie diagram:
\[
\xymatrix{
\E_T \ar[r] \ar[d]_= & Q \ar[r]\ar[d] & 0 \\
\E_T \ar[r] & Q^\prime \ar[r] & 0 
}
\]
Then we define a functor $F_{\E, P(z)}$:
\[
F_{\E,P(z)}(T) = \begin{cases}
\text{equivalence classes of sequences as above, where $Q\to T$}\\
\text{is flat, and each $Q_t$ have Hilbert polynomial $P(z)$.}
\end{cases}
\]
On maps, we define $F_{\E, P(z)}(f)=\tilde{f}$, where $\tilde f$ comes from the commutative diagram
\[
\xymatrix{
X \times_S U \ar[r]^{\tilde f} \ar[d] & X \times_S T \ar[d] \\
U \ar[r]^f & T
} 
\]


\begin{example}\textbf{Exercise as well:} Let $X=S$, $P(z)=n-m$ and let $\E$ be locally free of rank $n$. Then $F_{\E,P(z)}=F_{m,\E}=h_{G(m,n)}$. This follows from the fact that $Q$ flat over $T=X \times_S T$ with Hilbert polynomial $n-m$ is equivalent to $Q$ being locally free of rank $n-m$.
\end{example}

\begin{thm}[Grothendieck]
The functor $F_{\E,P(z)}$ is represented by a projective scheme, called $\mathfrak{Quot}(\E,P(z))$.
\end{thm}
\begin{proof}
See for example the nice article \cite{nitsure_hilbert}. 
\end{proof}  

So what is the Hilbert scheme? The Hilbert scheme is the scheme we get if we let $X=\PP_k^n$ and $\E = \OO_X$. Since sub-$\OO_X$-modules are in $1-1$ correspondence with closed subschemes of $X$, we see that the Hilbert scheme parametrizes closed subschemes of $\PP_k^n$ with Hilbert polynomial $P(z)$. So the theorem says that there is a projective scheme parametrizing closed subschemes of $\PP_k^n$. This is quite remarkable, as this says that in principle, we could write down a finite set of equations, such that the solution set correspond to all subschemes with a given Hilbert polynomial.



%%%%%%%%%%%%%%%%%%
%%%%%%%%%%%%%%%%%%
\pagebreak
\section{Topologies, sites, sheaves}

In this section we introduce the notion of a \emph{topology} on a category. This is motivated by the insufficiency of the Zariski topology.
 
\subsection{Grothendieck topologies and sites}

We must generalize the notion of topology. The reason is that we need more ``open sets'' to do interesting things with schemes. Instead of considering open sets, we consider ``coverings'', which are certain classes of maps.

\begin{defi}
Let $\CC$ be a category. A \emph{Grothendieck topology} on $\CC$ consists of a set $\Cov(X)$ of sets of morphisms $\{ X_i \to X\}_{i \in I}$ for each $X$ in $\CC$, satisfying the following axioms:
\begin{enumerate}
\item If $V \xrightarrow{\approx} X$ is an isomorphism, then $\{ V \to X\} \in \Cov(X)$.
\item If $\{X_i \to X\}_{i \in I} \in \Cov(X)$ and $Y \to X$ is a morphism in $\CC$, then the fiber products $X_i \times_X Y$ exists and $\{ X_i \times_X Y \to Y\}_{i \in I} \in \Cov(Y)$.
\item If $\{X_i \in X\}_{i \in I} \in \Cov(X)$, and for each $i \in I$, $\{ V_{ij} \to X_i\}_{j \in J} \in  \Cov(X_i)$, then
\[
\{ V_{ij} \to X_i \to X\}_{i \in I, j \in J} \in \Cov(X).
\]
\end{enumerate}
\end{defi}
\begin{remark}
Grothendieck/SGA calls this a \emph{pretopology}. 
\end{remark}
A \emph{site} is just a category equipped with a Grothendieck-topology. So in a sense, a site is just a very general ``space''. 

\begin{example}
Let $X$ be a topological space and let $\catname{Op}(X)$ be the category of open sets of $X$ and inclusions. If we let $\Cov(U)$ consist of all $U_i \to U$ such that $\cup U_i=U$, then $\catname{Op}(X)$ is a site, called the \emph{small classical site} of $X \in \Top$. 

The axioms are fullfilled, because in this case, if $U_1,U_2$ are two open sets, the fiber product $U_1 \times_U U_2$ is just the intersection $U_1 \cap U_2$.
\end{example}

\begin{example}[The global classical site]
Now let $\CC$ be the category $\Top$ of all topological spaces. If $U$ is a topological space, then a covering of $U$ will be a jointly surjective collection of open embeddings $\{ U_i \to U\}$. Here ``open embedding'' must mean ``open continuous injective map'', or else condition 1) is not satisfied.
\end{example}

\begin{example}[Big Zariski site]
Let $S$ be a scheme and $\SchS S$ the category of $S$-schemes. For $(U \to S) \in \SchS S$, let $\Cov(U)$ be the set $\{ U_i \to U\}$ of maps $U_i \to U$ that are open embeddings and such that $U = \cup U_i$. Since open immersions are stable under base change, this is a site.

It is called the \emph{big Zariski-site of $S$}. Again, open embedding must be interpreted as a morphism that gives an \emph{isomorphism} between an open subscheme of $U$. 
\end{example}

\begin{example}[Small étale site]
Let $\Et S$ be the full subcategory of $\SchS S$ consisting of étale maps $(U \to S)$. We say that a collection of morphisms $\{ U_i \to U \}_{i \in I}$ is in $\Cov(U)$ if
\[
\coprod_{i \in I} U_i \to U
\]
is surjective. It is clear from earlier results that this defines a Grothendieck-topology. We call this the \emph{small étale site on $S$}. 
\end{example}

\begin{example}[Big étale site]
Now consider the category $\SchS S$. In the small case, we worked only in the subcategory $\Et S$, but now we allow all morphism $(U \to S)$. However, the coverings $\Cov(U) = \{ U_i \to U \}$ are now étale maps. 
\end{example}

\begin{example}[The fppf site]
On $\SchS{S}$, let $ \{ U_i \to U\}_{i \in I} \in \Cov(U)$ if each $U_i \to U$ is flat and locally of finite presentation and such that $\coprod U_i \to U$ is surjective. We call this site the \emph{fppf site of $S$} (from french \emph{fidèlement plat de présentation finie}, ``faithfully flat and of finite presentation'').
\end{example}

\begin{example}
Let $\catname{Lis\mathrm{-}\acute{e}t}(X)$ be the full subcategory of $\SchS{S}$ with objects smooth maps $U \to X$, and let $\{U_i \to U\} \in \Cov(U)$ if every $U_i \to U$ is étale and $\coprod U_i \to U$ is surjective. This is the \emph{lisse-étale site}\footnote{``\emph{Lisse}'' is french for ``smooth''.}.
\end{example}

\begin{example}
We also have the \emph{lisse site}, which is like the previous example, except that we demand that each $U_i \to U$ to be smooth.
\end{example}

Note that Zariski-coverings are étale coverings, and étale coverings are fppf coverings. So we have natural inclusions
\begin{center}
big Zariski site $\subseteq$ big étale site $\subseteq$ fppf site.
\end{center}

The two most important topologies for us will be the étale topology and the fppf topology. 

\subsection{Presheaves and sheaves}

Let $F:\CC^{op} \to \Set$ be a presheaf. Let $\hat \CC$ denote the category of presheaves on $\CC$.

\begin{defi}[Sheaf]
\label{def:sheaf}
Let $\CC$ be a site and $F \in \hat \CC$ be presheaf. 
\begin{enumerate}
\item We say that $F$ is \emph{separated} if for all $U \in \CC$, and coverings $\{ U_i \to U\} \in \Cov(U)$, the map
\[
F(U) \to \prod F(U_i)
\]
is injective.
\item
We say that $F$ is a \emph{sheaf} if the diagram
\[
\xymatrix{
F(U) \ar[r] & \prod F(U_i) \ar@<0.8ex>[r] \ar@<-0.8ex>[r] & F(U_i \times_U U_j)
}
\]
is an equalizer for all coverings $\{ U_i \to U\} \in \Cov(U)$.
\end{enumerate}
\end{defi}

If we have a presheaf, we can sheafify to get a sheaf, just as we did as kids.
\begin{thm}
The inclusion $(\text{sheaves on } \CC) \hookrightarrow \hat \CC$ has a left adjoint.
\end{thm}
\begin{proof}
The proof proceeds in two steps. First one shows that the inclusion (separated presheaves on $\CC) \subseteq$ (presheaves on $\CC$) has a left adjoint, and secondly one shows that the inclusion (sheaves on $\CC) \subseteq$ (separated presheaves on $\CC$) has a left adjoint.

So let $F \in \hat \CC$ be a presheaf. We want to construct a separated presheaf. This is the easy step. We just let $F^s$ be the presheaf that for each $U \in \CC$ associates the set $F(U)/\sim$ where $a,b \in F(U)$ are equivalent if there exists a cover $\{ U_i \to U\} \in \Cov(U)$ such that $a,b$ have the same image under the map
\[
F(U) \to \prod F(U_i).
\]
Now if $V \to U$ is a morphism in $\CC$, then, by definition, $\{ U_i \times_ U V \to V\} \in \Cov(V)$. We have a commutative diagram:
\[
\xymatrix{
F(U) \ar[r] \ar[d] & \prod F(U_i) \ar[d] \\
F(V) \ar[r] & \prod F(U_i \X_U V)
}
\]
Thus we see that the composition
\[
F(U) \to F(V) \to F(V)/\sim
\]
factorizes through $F(U)/\sim$, and so $F^s$ is indeed a functor, which is obviously a separated presheaf.

Now let $F$ be a separated presheaf. Let $F^a$ be the presheaf that to each $U \in \CC$ assoicates the set of pairs $(\{U_i \to U\}_{i \in I} , \{a_i\}_{i \in I})/\sim$ where $\{ U_i \to U\} \in \Cov(U)$ and $\{ a_i \} \in \ker(\prod F(U_i) \rightrightarrows \prod_{i,j \in I} F(U_i \times_U U_j))$. We say that
\[
(\{ U_i \to U\}_{i \in I}, \{a_i\}) \sim (\{ V_j \to U\}_{j \in J}, \{ t_j\} )
\]
if $a_i$ and $b_j$ have the same image in $F(U_i \times_U V_j)$ for all $(i,j) \in I \times J$.


It is an exercise in patience to check that this gives us a sheaf.
\end{proof}

Now, in good Yoneda-Grothendieck spirit, one realizes that the basic object of study is not the space itself, but sheaves on it. 

\begin{defi}[Topoi]
A \emph{topos} is a category equivalent to the category of sheaves on a site \footnote{Plural ``topoi''}.
\end{defi}

\begin{defi}[Morphisms of topoi]
\label{def:toposmorphism}
A morphism of topoi $f:T \to T^\prime$ is an isomorphism class of triples $(f_\ast,f^\ast, \phi)$ where $f_\ast:T \to T^\prime$ and $f^\ast:T^\prime \to T$ are covariant/contravariant functors and $\phi:\Hom_T(f^\ast-,-) \xrightarrow{\sim} \Hom_{T^\prime} (-, f_\ast -)$ is an isomorphism of bifunctors, that is, $f_\ast$ and $f^\ast$ are adjoint functors. In addition, we demand that $f^\ast$ must commute with finite projective limits (in particular, $f^\ast$ preserve fiber products). 
\end{defi}

There is also the notion of a \emph{continuous} map of topoi, but we won't mention that here.

\begin{lemma}
A morphism $X \xrightarrow{f} Y$ in a site $\CC$ induces a morphism of topoi $\widetilde{\CC/X} \to \widetilde{\CC/Y}$.
\end{lemma}
\begin{proof}
First of all, a morphism $X \to Y$ induces a morphism $\CC/Y$ to $\CC/X$ given by sending an object $(Z \to Y)$ to the fiber product $(Z \times_Y X \to X)$. 

Given a sheaf $E$ on $X$, we define $f_\ast E(Z \to Y) := E(Z \times_Y X \to X)$. Similarly, for a sheaf $F$ on $Y$, we define $f^\ast$ as $f^\ast(Z \to X) = F(Z \to X \to Y)$. 

One can check that is functorial.

It remains to check that $f^\ast$ and $f_\ast$ are adjoint to each other.
\end{proof}

We will denote the different toposes by $X_{\acute et}$ (sheaves on the small étale site), $X_{ET}$ (sheaves on the big étale site), $X_{Zar}$ (sheaves on the small Zariski site), $X_{ZAR}$ (sheaves on the big Zariski site), $X_{fppf}$ (sheaves on the big fppf site of $X$). 

The great thing is that different sites can give rise to equivalent topoi.

\begin{example}
The category of sheaves on $\Sch$ is equivalent to the category of sheaves on $\AffSch$, the category of affine schemes, since sheaves can be defined locally on affine opens.
\end{example}

The main theorem of this section is the following:

\begin{thm}
\label{thm:hxsheaf}
Let $X \to Y$ be a morphism of schemes. Then $h_X$ is a sheaf in the fppf-topology on $\SchS{Y}$.
\label{thm:fppf}
\end{thm}

The theorem says that morphisms of schemes can be glued, even in topologies much finer than the Zariski topology. In particular, all representable functors are sheaves.

This is a highly non-trivial theorem, and the proof will need several lemmas. First, we consider the affine case. So let $f:A \to B$ be a ring map and $M$ an $A$-module. Then $M_B := M \otimes_A B$ is a $B$-module. Similarly, $M_{B \otimes _AB} = M \otimes_A (B \otimes _A B)$. By abuse of notation, write $f:M \to M_B$ for the map induced by $f$. Let $p_1,p_2:M_B \to M_{B \otimes_A B}$ be the maps induced by $b \mapsto b \otimes 1$ and $b \mapsto 1 \otimes b$, respectively.

\begin{lemma}
\label{lemma:mb}
If $f:A \to B$ is faithfully flat then the sequence
\[
\xymatrix{
0 \ar[r] & M \ar[r]^f & M_B \ar@<1ex>[r]^{p_2} \ar@<-1ex>[r]_{p_1} & M_{B \otimes_A B}
}
\]
is exact.
\end{lemma}

\begin{proof}
Since $B$ is faithfully flat over $A$, the lemma is equivalent to the sequence
\[
\xymatrix{
0 \to M_B \ar[r]^{f^\prime} & M_{B\otimes_A B} \ar@<1ex>[r]^{p_2^\prime } \ar@<-1ex>[r]_{p_1^\prime } & M_{(B \otimes_A B) \otimes_A B}
}
\]
being exact. We first show that $f^\prime$ is injective. Multiplication $B \otimes_A B \to B$ induces a map $\eta: M_{B \otimes_A B} \to M_B$. Then $\eta \circ f^\prime = \id_{M_B}$, which implies that $f^\prime$ is injective.

Now assume $\alpha \in M_{B \otimes_A B}$ satisfies $p_1^\prime(\alpha)=p_2^\prime(\alpha)$. Define $\tau:M_{B \otimes_A B \otimes_A B} \to M_{B \otimes_A B}$ by multiplying the last two factors. Then $\tau \circ p_1^\prime = \id$ and $\tau \circ p_2 ^\prime = f^\prime \circ \eta$. Then 
\[
\alpha = \tau(p_1^\prime(\alpha)) = \tau(p_2^\prime(\alpha)) = f^\prime(\eta(\alpha)).
\]
So $\alpha$ is in the image of $f^\prime$, that is, the sequence is exact.
\end{proof}

We get as a corollary:
\begin{corr}
\label{corr:affineexact}
Let $X$ be an affine scheme and assume $V \to U$ is a faithfully flat map between affine schemes. Then the sequence
\[
h_X(U) \to  h_X(V)\rightrightarrows  h_X(V \times_U V) 
\]
is exact.
\end{corr}
\begin{proof}
Let $X = \Spec R$ and $V = \Spec B$ and $U = \Spec A$. Then the statement is equivalent to the exactness of the sequence
\[
\Hom_{\catname{A-alg}}(R,A) \to \Hom_{\catname{A-alg}}(R,B) \rightrightarrows \Hom_{\catname{A-alg}}(R, B \otimes_A B).
\]
Let $M$ be the $A$-module $\Hom(R,A)$. Then $\Hom(R,B) = M \otimes_A B$ and $\Hom(R,B \times_A B) = M_{B \otimes_A B}$, so the exactness of the above sequence is equivalent to the statement of Lemma \ref{lemma:mb}.
\end{proof}

We need another lemma. It says that when checking if a functor is a fppf sheaf, it is enough to consider the case of a single morphism.

\begin{lemma}
\label{lemma:singlemorphism}
Let $F:\Sch^{op} \to \Set$ be a Zariski sheaf. Then $F$ is a fppf sheaf if and only if for any faithfully flat map of locally finite presentation $V \to U$, the sequence
\[
F(U) \to F(V) \rightrightarrows F(V \times _U V)
\]
is exact.
\end{lemma}
\begin{proof}
Suppose that $\{ U_i \to U\} \in \Cov(U)$ in the fppf topology. Let $V = \coprod U_i$. Then $V \to U$ is a fppf map as in the lemma. We have a commutative diagram:
\[
\xymatrix{
F(U) \ar[r] \ar@{=}[d] & F(V) \ar@<1ex>[r] \ar@<-1ex>[r] \ar[d]^\approx  & F(V \times_U V) \ar[d]^\approx\\ 
F(U) \ar[r]&  \prod F(U_i) \ar@<1ex>[r] \ar@<-1ex>[r] & \prod_{i,j} F(U_i \times_U U_j)
}
\]
The exactness of the top row is equivalent to the exactness of the bottom row.
\end{proof}

Here's another lemma. It says that when checking if $F$ is an fppf sheaf, it is enough to consider single \emph{affine} morphisms.

\begin{lemma}
\label{lemma:sheaf_fppf}
Let $F$ be a functor $\Sch^{op} \to \Set$. Then if the two conditions
\begin{enumerate}
\item $F$ is a sheaf in the Zariski topology and
\item If $V \to U$ is an fppf map of affine schemes, then the sequence
\[
F(U) \to  F(V) \rightrightarrows F(V \times_U V)
\]
is exact.
\end{enumerate}
are satisfied, $F$ is a sheaf in the fppf topology.
\end{lemma}
\begin{proof}
Let $f:V \to U$ be fppf covering consisting of a single morphism $f$. By Corollary \ref{corr:fppfcover}, we can choose a cover $\{ V_i\}$ of $V$ such that each $f(V_i)$ is affine and each $V_i$ is quasi-compact. Write each $V_i$ as a finite union $\cup_a V_{ia}$ of affine schemes\footnote{We need a finite union here, because if $U_i$ are affine schemes, then $\coprod U_i$ is only affine if the index set is finite.}. Then we have a diagram:
\[
\xymatrix{
F(U) \ar[r] \ar[d] & F(V) \ar@<1ex>[r]\ar[d] \ar@<-1ex>[r] & F(V \times_U V) \ar[d] \\
\prod_i F(U_i) \ar[r] \ar[d] & \prod_{i,a} F(V_{ia}) \ar[d] \ar@<1ex>[r] \ar@<-1ex>[r] & \prod_{i,a,b} F(V_{ia} \times_U V_{ib}) \\
\prod_{i,j} F(U_i \cap U_j) \ar[r] & \prod_{i,j,a,b} F(V_{ia} \cap V_{jb})
}
\]
We need to show that the top row is exact. The first column is exact since the $U_i$ cover $U$ in the Zariski topology, and the second column is exact since the $V_{ia}$ cover $V$ in the Zariski topology. The middle row is exact because for each $i$, the $V_{ia}$ are a finite number of affines covering $U_i$, and by Lemma \ref{lemma:singlemorphism}, this is equivalent to the having just a single covering, but we know that in the case of affines, this row is exact, by Corollary \ref{corr:affineexact}.

Now a diagram chase shows that the top row is exact.
\end{proof}

We need one more lemma before we can prove Theorem \ref{thm:fppf}. 
\begin{lemma}
\label{lemma:pointsfiber}
Let $f_1:X_1 \to Y$ and $f_2:X_2 \to Y$ be morphisms of schemes. If $x_1,x_2$ are points of $X_1,X_2$, respectively, satisfying $f_1(x_1)=f_2(x_2)$, there exists a point $z$ in the fibered product $X_1 \times_Y X_2$ such that $\pr_1(z)=x_1$ and $\pr_2(z)=x_2$.
\end{lemma}
\begin{proof}
For a point $x$ in a scheme $X$, write $k(x)$ for the quotient field $\OO_{X,x}/\mm_x$. Set $y=f_1(x_1)=f_2(x_2) \in Y$. Then $k(y) \hookrightarrow k(x_1)$ and $k(y) \hookrightarrow k(x_2)$ are two field extensions. The tensor product $k(x_1) \otimes_{k(y)} k(x_2)$ is not $0$ because the tensor product of two non-zero vector spaces is never zero. Hence $k(x_1) \otimes_{k(y)} k(x_2)$ has a maximal ideal $\mm$. The quotiend field $K := (k(x_1) \otimes_{k(y)} k(x_2))/\mm$ is an extension of $k(y)$ containing both $k(x_1)$ and $k(x_2)$. 

Let $U$ be an affine open in $Y$ containing $y$.  The two composites
\[
\Spec K \to \Spec k(x_i) \to X_i \xrightarrow{f_i} U
\]
for $i=1,2$ coincide, so we get a morphism $\Spec K \to X_1 \times_Y X_2$. We take $z$ to be the image of $\Spec K$ in $X_1 \times_Y X_2$.
\end{proof}

Now we are in position to prove Theorem \ref{thm:fppf}.

\begin{proof}[Proof of Theorem \ref{thm:fppf}]
We have already proven the statement in the case $X$ is affine.  This is the content of Lemma \ref{lemma:sheaf_fppf}.

So write $X$ as a union $\cup_i X_i$ of open affine schemes. We first show that $h_X$ is separated (cf. Definition \ref{def:sheaf}). Let $h:V \to U$ be a covering, and take two morphisms $f,g:U \to X$ such that the two composites $V \to U \to X$ are equal.

Since $V \to U$ is surjective, the inverse images of $X_i$ coincide, so we can set $U_i=f^{-1}(X_i)=g^{-1}(X_i)$, and let $V_i=h^{-1}(U_i)$. Then the two composites
\[
\xymatrix{
V_i \ar[r] & U_i \ar@<1ex>[r]^{f|_{U_i}} \ar@<-1ex>[r]_{f|_{U_i}} & X_i
}
\]
coincide, and since $X_i$ is affine, we conclude $f|_{U_i} = g|_{U_i}$ by Lemma \ref{lemma:sheaf_fppf}. This holds for all $i$, hence $f=g$. 

Now suppose that $g:V \to X$ is a morphism such that the two composites 
\[
\xymatrix{
V \times_U V   \ar@<1ex>[r]^{\pr_1} \ar@<-1ex>[r]_{\pr_2} &  V \ar[r]^g \ar@{.>}[d]^? & X \\
  & U \ar[ur] 
}
\]
coincide. We need to show that $g$ factors through $U$. From Lemma \ref{lemma:pointsfiber} we get that $g$ factors through $U$ set-theoretically, and by Proposition \ref{prop:flatopen}, $U$ has the quotient topology induced by $V \to U$, so the function $f:|U| \to |X|$ (of topological spaces) is continuous.

Set $U_i = f^{-1}(X_i)$ and $V_i = g^{-1}(V_i)$ for all $i$. Then the composites
\[
\xymatrix{
V_i \times_U V_i \ar@<1ex>[r]^{\pr_1} \ar@<-1ex>[r]_{\pr_2} & V_i \ar[r]^{g|_{V_i}} & V_i \ar[r] & X_i
} 
\] 
coincide. Since $X_i$ is affine, it follows from the affine case of the theorem we are trying to prove that $g|_{V_i}$ factors uniquely through a morphism $f:U_i \to X_i$. We have
\[
f_i|_{U_i \cap U_j} = f_j|_{U_i \cap U_j} : U_i \cap U_j \to X
\]
because $h_X$ is separated, hence the $f_i$ glue to give the desired factorization $V \to U \to X$.
\end{proof}

%%%%%%%%%%%%%
%%%%%%%%%%%%%
\pagebreak
\section{Fibered categories}
\label{sec:fibered}

Fibered categories are constructs designed to handle all the different (yet isomorphic) fiber products. We will see later that stacks are a special kind of fibered category. In particular, it is a category fibered in groupoids satisfying a descent condition.

\subsection{Fibered categories}  

Let $\CC$ be any category. For (almost) all examples here, $\CC$ will just be the category $\SchS S$ of $S$-schemes in some topology (e.g. the Zariski, étale, fppf topology). 

\begin{defi}
A category \emph{over} $\CC$ is a couple $(F,p)$ where $F$ is a category and $p:F \to \CC$ is a functor.
\end{defi}

We say that a morphism $\phi:\alpha \to \beta$ in $F$ is \emph{cartesian} if for all $\gamma \in F$ with a morphism $\psi: \gamma \to \beta$ and a factorization
\[
\xymatrix{
p(\gamma)\ar@/_1pc/[rr]_{p(\psi)}  \ar[r]^h & p(\alpha) \ar[r]^{p(\phi)} & p(\beta) 
}
\]
in $\CC$, of $p(\psi)$, then there exists an \emph{unique} morphism $\lambda:\gamma \to \alpha$ such that $\phi \circ \lambda = \psi$ and $p(\lambda)=h$. We draw a diagram to illustrate this:

\[
\xymatrix{
\mathcal F: & \gamma \ar@/^2pc/[rr]_\psi \ar@{-->}[r]_{\exists ! \lambda} \ar@{|->}[d] & \alpha \ar@{|->}[d] \ar[r]^\phi & \beta \ar@{|->}[d] \\
\CC: & p(\gamma) \ar[r] ^h & p(\alpha) \ar[r]^{p(\phi)} & p(\beta)
}
\]

If $\phi:\alpha \to \beta$ is cartesian, then $\alpha$ is called a \emph{pullback} of $\beta$ along $p(\phi)$.

\begin{exc}
\label{ex:unique}
If $\alpha \xrightarrow{\phi} \beta$ and $\alpha^\prime \xrightarrow{\phi^\prime} \beta$ are two pullbacks of $\beta$ along $p(\phi)$, then there exists an \emph{unique} isomorphism $\lambda:\alpha \to \alpha^\prime$ with $p(\lambda)=\id_{p(\alpha)}$ and $\phi^\prime \circ \lambda = \phi$. 
\end{exc}
\begin{sol}
This follows from the usual trickery with universal properties.
\end{sol}


\begin{defi}
For $U \in \CC$, let $F(U)$, \emph{the fiber of $p_F$ over $U$}, be the category with objects $\alpha \in F$ with $p(\alpha)=U$ (actual equality) and morphisms $\alpha^\prime \to \alpha$ in $F$ such that $p(f)=\id_U$.
\end{defi}

\begin{defi}[Fibered category] We define fibered categories in 1), morphisms between them in 2), and natural transformations between morphisms in 3).
\label{def:fibered}
\begin{enumerate}
\item A fibered category over $\CC$ is a category over $\CC$ such that for all morphisms $f: U \to V$ in $\CC$ and $\beta \in F(V)$, there exists a cartesian morphism $\phi:\alpha \to \beta$ such that $p_F(\phi)=f$. (in particular, $\alpha \in F(U)$).

\item A morphism between fibered categories $p_F:F \to \CC$ and $p_G:G \to \CC$ is a functor $g:F \to G$ such that
\begin{enumerate}
\item $p_G \circ g = p_F$.
\item $g$ maps cartesian morphisms to cartesian morphisms.
\end{enumerate}
\item Let $g,g^\prime:F \to G$ be morphisms of fibered categories. Then a \emph{base-preserving natural transformation} $\pi:g \to g^\prime$ is a natural transformation of functors such that for all $\alpha \in F$, the morphism $\pi_\alpha:g(\alpha) \to g^\prime(\alpha)$ in $G$ projects to the identity in $\CC$ (under $p_G$). (i.e. $\pi_\alpha$ is a morphism in $G(p_F(\alpha))$)
\end{enumerate}
\end{defi}

This gives us a new category, $\HOM_\CC(F,G)$ with objects morphisms of fibered categories and morphisms base-preserving natural transformations.

\begin{remark}
The category of categories fibered over $\CC$ is an example of a $2$-category (which can be thought of as a category where the $\Hom$-sets are categories themselves).
\end{remark}

It will be useful to single out \emph{unique} pullbacks. This way there will be no ambiguity when talking about the fiber products.

\begin{defi}[Cleavage] A \emph{cleavage} of a fibered category $F \to \CC$ consists of a class $K$ of cartesian arrows such that for every $f:U \to V$ in $\CC$ and $\eta \in F(V)$, there exists a unique morphism to $\eta$ that maps to $f$ in $\CC$.
\end{defi}

A cleavage always exists by the axiom of choice.

\begin{example}
Let $f:X \to Y$ be a map of schemes. For $T \to Y$, consider $X \times_Y T$. Let $\CC = \SchS{Y}$. Then define the category $F$ to have objects quadruples $(t,P,a,b)$ where $t:T \to Y$ is a map, $P$ is a scheme, $a:P \to T$ is a map and $b:P \to X$ is a map. A morphism $(t^\prime, P^\prime, a^\prime, b^\prime) \to (t,P,a,b)$ is a couple of morphisms $(\alpha,\beta)$ where $\alpha:T^\prime \to T$ and $\beta:P^\prime \to P$ such that $a \circ \beta = \alpha \circ a^\prime$ and $b \circ \beta = b^\prime$. The quadruple $(t,P,a,b)$ should also be a cartesian square:
\[
\xymatrix{
P \ar[r]^b \pullbackcorner  \ar[d]^a & X \ar[d]^f \\
T \ar[r]_t & Y 
}
\] 

Then we have a functor $p:F \to \CC$ that sends a quadruple $(t,P,a,b)$ to the morphism $t:T \to Y$. Thus, for $t:T \to Y$, the category $F(t:T \to Y)$ is the category of cartesian squares as above (with a fixed base). Two objects in $F(t:T \to Y)$ are uniquely isomorphic, and a choice of one object corresponds to a choice of fiber product $T \times_Y X$. 

The axioms of a fibered category corresponds to the existence of fiber products.
\end{example}

\begin{lemma}
\label{lemma:factorcartesian}
Let $p:F \to \CC$ be a fibered category. Then every morphism $\psi:\gamma \to \beta$ can be factorized as 
\[
\gamma \xrightarrow{\lambda} \alpha \xrightarrow{\phi} \beta
\]
where $\phi$ is cartesian, and $\lambda$ is a morphism in $F(p(\gamma))$.
\end{lemma}
\begin{proof}
Apply $p$ to $\psi$ to get $p(\psi):p(\gamma) \to p(\beta)$. Then $\beta \in F(p(\beta))$, so by definition of a fibered category, there exists a cartesian arrow $\phi:\alpha \to \beta$ such that $p(\phi)=p(\psi)$. In particular, $p(\alpha)=p(\gamma)$, so that $\alpha \in F(p(\gamma))$. Thus we have a diagram:
\[
\xymatrix{
F: & \gamma \ar@/^2pc/[rr]_\psi \ar@{-->}[r]_{\exists ! \lambda} \ar@{|->}[d] & \alpha \ar@{|->}[d] \ar[r]^\phi & \beta \ar@{|->}[d] \\
\CC: & p(\gamma) \ar@{=}[r] ^{\id} & p(\alpha) \ar[r]^{p(\phi)} & p(\beta)
}
\]
So that $\psi=\phi \circ \lambda$ where $\lambda$ projects down to the identity on $p(\gamma)$. 
\end{proof}


\begin{lemma}
Let $F \xrightarrow{g} G$ be a morphism of fibered categories over $\CC$ such that for all objects $U \in \CC$, the functor $g_U:F(U) \to G(U)$ is full and faithful. Then $g$ is full and faithful as a functor between $F$ and $G$.
\end{lemma}

\begin{proof}
Let $x,y \in F$ be two objects. Then we have a diagram:
\[
\xymatrix{
\Hom_F(x,y) \ar[rr]^g \ar[dr]_{p_F} && \Hom_G(g(x),g(y)) \ar[dl]^{p_G} \\
& \Hom_\CC(p_F(x),p_F(y))
}
\]
We need to show that $g$ induces a bijection on the $\Hom$-sets on the top.

By Lemma \ref{lemma:factorcartesian} every morphism $\psi:g(x) \to g(y)$ in $\Hom_G(g(x),g(y))$ factors as
\[
\xymatrix{
G: &g(x) \ar@/^2pc/[rr]_\psi \ar@{|->}[dr] \ar[r]^\lambda & \alpha \ar@{|->}[d]\ar[r]^\phi & g(y) \ar@{|->}[d]\\
\CC: & & x \ar[r] & y
}
\]
Since $g$ was a morphism of \emph{fibered} categories, the cartesian morphism $\phi$ comes from a morphism $\alpha^\prime \to y$, up to unique isomorphism, by Exercise \ref{ex:unique}. So we can write $\alpha=g(\alpha^\prime)$. Thus by the assumption $\lambda=g(\gamma)$ for a \emph{unique} $\gamma \in \Hom_\F(x,y)$. Thus $g$ is full, and it is faithful because any other $\phi$ differs by a unique isomorphism from $\Hom_F(x,x)$.
\end{proof}

\begin{defi}
A morphism $F \xrightarrow{g} G$ of fibered categories over $\CC$ is an \emph{equivalence} if there exists $G \xrightarrow{h} F$ and base-preserving isomorphisms such that
\[
\xymatrix{
h \circ g \simeq \id_F & \text{ and }& g \circ h \simeq \id_G
}
\]
\end{defi}

\begin{prop}
\label{prop:fiberedequivalence} 
A morphism $F \xrightarrow{g} G$ over $\CC$ is an equivalence if and only if for all $U \in \CC$, the functor $g_U:F(U) \to G(U)$ is an equivalence.
\end{prop}
\begin{proof}
A morphism is an equivalence if it is fully faithful and essentially surjective. Since morphisms over $\CC$ are base-preserving, it is enough to chech on fibers.
\end{proof}
\begin{remark}
You might say that equivalence can be checked ``pointwise'' or ``fiberwise''.
\end{remark}

\subsection{2-Yoneda-lemma}

Let $p_F:F \to \CC$ be a fibered category. Let $X \in \CC$. Consider $\CC/X$ (the slice category of $X$ in $\CC$). We have an evident forgetful functor $\CC/X \to \CC$.

We have a map
\begin{align*}
\alpha:\HOM_\CC(\CC/X, F) &\to F(X) \\
(\CC/X \xrightarrow{g} F) &\mapsto g(\id_X)
\end{align*}
that is a morphism of fibered categories.

\begin{lemma}[2-Yoneda] The map $\alpha$ is an equivalence of categories.
\end{lemma}
\begin{proof}
We want to define a map in the other direction: Let $\eta:F(X) \to \HOM_\CC(\CC/X,F)$ be defined as follows. For $x \in F(X)$, let
\[
\eta_x:\CC/X \to F
\]
be given by $(\varphi:Y \to X) \mapsto \varphi^\ast x$. Here $\varphi \in \Hom_\CC(Y,X)$ and $\varphi^\ast x$ is a \emph{choice} of pullback of $x \in F(X)$ along $p_F$. This follows from the first property of fibered categories in Definition \ref{def:fibered}. See the diagram below:
\[
\xymatrix{
F: & \varphi^\ast x \ar[r]^{\exists \psi} \ar@{|->}[d]_{p_F} & x \ar@{|->}[d]^{p_F}\\
\CC: & Y \ar[r]^\varphi  & X
}
\]
Now, given a morphism
\[
\xymatrix{
\ar[dr]_{\varphi^\prime} Y^\prime \ar[r]^\epsilon & Y \ar[d]^\varphi \\
& X
}
\]
in $\CC/X$, define $\eta_x(\epsilon):\varphi^{\prime \ast} x \to \varphi^\ast x$ to be the uniquely defined morphism in the diagram
\[
\xymatrix{
\ar@/^1pc/[rr] \varphi^{\prime \ast} x \ar@{-->}[r]_{\exists !} & \varphi^\ast x \ar[r] & x
}
\]
Thus we have defined $\eta_x$. It is a morphism of fibered categories. Compositions of maps need to be checked, though.

\emph{Note:} A morphism $x^\prime \xrightarrow{f} x$ in $F(X)$ induces a morphism of functors $\eta_f:\eta_{x^\prime} \to \eta_x$: If $\varphi:Y \to X \in \Hom_\CC(Y,X)$ and $\varphi^\ast x^\prime \to \varphi^\ast x$ is a pullback, let $\eta_f(\varphi)$ be the \emph{unique} (since $\varphi^\ast x$ is a pullback) morphism $\varphi^\ast x^\prime \to \varphi^\ast x$ such that the diagram
\[
\xymatrix{
\varphi^\ast x^\prime \ar[d] \ar[r]^{\eta_f(\varphi)} & \varphi^\ast x \ar[d] \\
x^\prime \ar[r]^f & x  
}
\]
commutes. This gives the desired morphism of functors $\eta(f)$. Thus we have defined $\eta$ on objects ($\eta_x$) and on morphisms ($x^\prime \xrightarrow{f} x$). This defines $\eta$.

Now consider the composition
\[
F(X) \xrightarrow{\eta} \HOM_\CC(\CC/X, F) \xrightarrow{\alpha} F(X)
\]
It sends $x \in F(X)$ to $\id_X^\ast$, which is canonically isomorphic to $x$. This gives an isomorphism of functors $\alpha \circ \eta \simeq \id_{F(X)}$. 

On the other hand, consider $\eta \circ \alpha$. This composition sends a function $f:\CC/X \to F$ to the functor $\CC/X \to F$ that for $(\varphi:Y \to X) \in \Hom_\CC(Y,X)$ associates $\varphi^\ast f(\id_X)$. This functor is canonically isomorphic to $f$. (check this!) 
\end{proof}

Thus if $F$ is some moduli problem, the lemma says that the elements of $F(X)$ (i.e. the solutions of the moduli problem), correspond to maps \emph{to} the moduli problem.

So when we have defined stacks, this will imply that a stack is a fine moduli space for the moduli problem it represents.

\begin{corr}
Let $X,Y \in \CC$ be objects. Then the functor
\[
\HOM_\CC(\CC/X, \CC/Y) \to \Hom_\CC(X,Y)
\]
that maps $f$ to $f(\id_X)$ is an equivalence of categories.
\end{corr}

\begin{remark} If $X \in \CC$ and $F \to \CC$ is a fibered category, we will often write $X \to F$ to denote an objects in $F(X)$. (``in good Yoneda lemma tradition'')
\end{remark}



\subsection{Categories fibered in $\Set$}

The objects of $\Set$ can be identified with categories with no non-trivial structure. That is, one can identify a set $X$ with the category $X$ whose only morphisms are the identity morphisms.

\begin{defi}
A category fibered in $\Set$ is a fibered category $p:F \to \CC$ such that for all $U \in \CC$, the category $F(U)$ is a set.
\end{defi}

\begin{lemma}
Let $q:G \to \CC$ be a fibered category and let $p:F \to \CC$ be a category fibered in $\Set$. Then the category $\HOM_\CC(G,F)$ is a set.
\end{lemma}
\begin{proof}
Let $f,g:G \to F$ be morphisms of fibered categories and $\alpha:g \to f$ a morphism in $\HOM_\CC(G,F)$. Then, for all $x \in G$ over $X \in \CC$, we have a morphism $\alpha_x:f(x) \to g(x)$ (in $F(X)$). 

Now, if $F$ is fibered in $\Set$, $\alpha_x$ must be the identity. But then $f(x)=g(x)$ for all $x \in G$. But then $f=g$. So any two objects with a morphism between them are equal, hence $\HOM_\CC(G,F)$ is a set.
\end{proof}


Let $F:\CC^{op} \to \Set$ be a presheaf over $\CC$. We will construct a fibered category $\mathcal F$ over $\CC$. The category $\FF$ has objects pairs $(U,x)$ with $U \in \CC$ and $x \in F(U)$. The morphisms  $g:(U^\prime, x^\prime) \to (U,x)$ are given as a morphism $g:U^\prime \to U$ such that $g^\ast x := F(g)(x)=x^\prime$ in $F(u^\prime)$. The morphism $\mathcal F \to \CC$ is given by $(U,X) \mapsto U$. 

It is an easy exercise in definition-hunting to see that this is indeed a fibered category. 

\begin{prop}
We have an equivalence of categories
\[
\Gamma:\PreSheaves{\CC} \to \{ \text{categories fibered in sets over } \CC \}
\]
defined by $F \mapsto \FF$. 
\end{prop}
\begin{proof}
Now given a category $p:\FF \to \CC$ fibered in $\Set$, define
\begin{align*}
F:\CC^{op} &\to \Set \\
U &\mapsto \FF(U)
\end{align*}

Thus we have defined a natural transformation $\Sigma$ (things need to be checked though) from categories fibered in $\Set$ over $\CC$ to $\PreSheaves \CC$ such that $\Sigma \circ \Gamma = \id_{\PreSheaves \CC}$. 

On the other hand, it is easy to see that $\Sigma$ is an equivalence in each fiber (there are no morphisms to check on!), so by Proposition \ref{prop:fiberedequivalence} it follows that $\Sigma$ is an equivalence.


\end{proof}

\begin{remark} Schemes give rise to fibered categories because of the embedding $\Sch \hookrightarrow \PreSheaves{\Sch}$ given by $X \mapsto h_X$. In particular, schemes give rise to fibered categories with no automorphisms in the fibers.
\end{remark}

So a fibered category $p:\catname{Groupoids} \to \Sch$ without automorphisms is the same as a scheme, in the same way that a scheme without gluing data is the same as an affine scheme.


%%%%%%%%%%%%%%%%%%%%%%
%%%%%%%%%%%%%%%%%%%%%%
\pagebreak
\section{Groupoids and descent}

We define groupoids and show that fiber products of groupoids exist. After that we define what we mean by descent, prove some lemmas, and state the main theorem.

\subsection{Groupoids}

\begin{defi}
A \emph{groupoid} is a category where all morphisms are isomorphisms.
\end{defi}

\begin{example}[The fundamental groupoid] Let $X$ be a topological space. Then $\Pi_1(X)$ is the category with objects the points of $X$, and morphisms the paths betweens points of $X$, up to homotopy equivalence. This is a groupoid because any path has an inverse path. Then the usual fundamental group $\pi_1(X,x)$ based at $x \in X$ is just $\Aut_{\Pi_1(X)}(x)$.
\end{example}

\begin{defi}
A fibered category $p:F \to \CC$ is \emph{fibered in groupoids} if every fiber $F(U)$ is a groupoid.
\end{defi}
\begin{remark}
In particular, every scheme give rise to a category fibered in groupoids, because each fiber is just a set, and identity morphisms are isomorphisms.
\end{remark}

\begin{example}
Let $\CC$ be the category $\Top$ of topological spaces and let $F$ be the category consisting of triples $(X,x,\pi_1(X,x))$ where $X \in \Top$ is a topological space, $x \in X$ and $\pi_1(X,x)$ is the fundamental group based at $x$. Define maps between $(X,x,\pi_1(X,x))$ and $(Y,y,\pi_1(Y,y))$ to consist of a map $X \to Y \in \Hom_\Top(X,Y)$ such that $f(x)=y$. This induces a corresponding map on the fundamental groups by composition: $f_*:\pi_1(X,x) \to \pi_1(X,y)$.

Define $p: F \to \CC$ by $(X,x, \pi_1(X,x)) \mapsto X$. Then the fiber $F(X)$ over $X$ is just the fundamental groupoid of $X$, so we have an example of a fibered category fibered in groupoids.
\end{example}

\begin{lemma}
Let
\[
\xymatrix{
F \ar[dr] \\
F^\prime \ar[r] & \CC
}
\]
be categories fibered in categories. Assume further that $F^\prime$ is fibered in groupoids. Then the category $\HOM_\CC(F,F^\prime)$ is also a groupoid.
\end{lemma}
\begin{proof}
Let $f,g:F \to F^\prime$ be morphisms of fibered categories and let $\alpha:f \to g$ be a morphism in $\HOM_\CC(F,F^\prime)$. To show that $\alpha$ is an isomorphism, it is enough to do it on each fiber, by Proposition \ref{prop:fiberedequivalence}, but $\alpha_x:f(x) \to g(x)$ is a morphism in $F^\prime$, hence an isomorphism. So $\alpha$ is an isomorphism, so $\HOM_\CC(F,F^\prime)$ is a groupoid by definition.
\end{proof}

\begin{defi}[The ``slice groupoid'']
Let $p:F \to \CC$ be fibered in groupoids. For $X \in \CC$, we get a new category fibered in groupoids as follows: The notation is $p/X:F/X \to \CC/X$. The objects of $F/X$ are pairs $(y,p(y) \xrightarrow{f} X)$  with $y \in F$ and $f$ a morphism in $\CC$. The morphisms are morphisms $g:y^\prime \to y$ in $F$ such that $f \circ p(g) = f^\prime$.

Then $p/X$ sends $(y, p(y) \xrightarrow{f} X)$ to $f:p(y) \to X$. For a morphism $Y \to X$, the fiber $(F/X)(Y \to X)$ is just $F(X)$. This is a groupoid since $F$ was fibered in groupoids.
\end{defi}

The next definition will be important in the definition of stacks. 

\begin{defi}[The isomorphism presheaf] Let $p:F \to \CC$ be a category fibered in groupoids. For $X \in \CC$, let $x, x^\prime \in F(X)$. Then we  get a presheaf $\Isom(x,x^\prime):\CC^{op}/X \to \Set$ defined as follows. For an object $Y \in \CC/X$, send  For each morphism $Y \xrightarrow{f} X$, choose pullbacks $f^\ast x$ and $f^\ast x^\prime$ (elements in $F(Y)$) and define
\[
\Isom(x,x^\prime)(Y \xrightarrow{f} X) := \Hom_{F(Y)}(f^\ast x, f^\ast x^\prime) \in \Set
\]
For a morphism $g:Z \to Y$ in $\CC/X$, the restriction map is defined by
\[
\Hom_{F(Y)}(f^\ast x, f^\ast x^\prime) \xrightarrow{g^\ast} \Hom_{F(Z)}(g^\ast f^\ast x, g^\ast f^\ast, y) \simeq \Hom_{F(Z)}((fg)^\ast x, (fg)^\ast y).
\]
The isomorphism to the right comes from $F$ being a fibered category and pullbacks being unique up to unique isomorphism.
\end{defi}

\begin{exc}
For a composition of $Z \xrightarrow{g} Y$ and $Y \xrightarrow{f} X$ we have a natural map $g^*:\Isom(x,x^\prime)(Y \to X) \to \Isom(x,x^\prime)(Z \to X)$.
\end{exc}
\begin{sol}
\emph{Hint:} By choice of cleavage there are unique isomorphisms $(fg)^\ast x \simeq g^\ast f^\ast x$.
\end{sol}

Note that when $x=x^\prime$ we get a presheaf of groups: $\Aut_x: (\CC/X)^{op} \to \Grps$, and furthermore, the set $\Isom(x,x^\prime)(Y \to X)$ is a torsor under the action of $\Aut_x$.

\subsection{Fiber products of categories fibered in groupoids}

Suppose we have a diagram of groupoids.
\[
\xymatrix{
 & \mathcal G_1 \ar[d]^f  \\
\mathcal G_2 \ar[r]^g & \mathcal G
}
\]
Then the fiber product $\mathcal G_1 \X_{\mathcal G} \mathcal G_2$ of the diagram exists, and it is a groupoid. Its object are triples $(x,y,\sigma)$ where $x \in \mathcal G_1$ and $y \in \mathcal G_2$ and $\sigma:f(x) \to g(x)$ is a morphism in $\mathcal G$. The morphism $(x^\prime,y^\prime, \sigma^\prime) \to (x,y,\sigma)$ are pairs $(a,b)$ where $a:x^\prime t\ x$ is an isomorphism and $b:y^\prime \to y$ is an isomorphism, such that the diagram 
\[
\xymatrix{
f(x^\prime) \ar[r]^{\sigma^\prime} \ar[d]^{f(a)} & g(y^\prime) \ar[d]^{g(b)} \\
f(x) \ar[r]^\sigma & g(y) 
}
\]
commutes. There are evident functors $p_j:\mathcal G_1 \times_{\mathcal G} \mathcal G_2 \to \mathcal G_j$ for $j=1,2$ and a natural isomorphism of functors $\Sigma:f \circ p_1 \to g \circ p_2$. (check this!)

In addition, the triple $(\mathcal G_1 \X_{\mathcal G} \mathcal G_2, p_i, \Sigma)$ has a universal property. Suppose that $\mathcal H$ is another groupoid and $\alpha:\mathcal H \to \mathcal G_1$, $\beta:\mathcal H \to \mathcal G_2$ and $\gamma:f \circ \alpha \to g \circ \beta$ is an isomorphism of functors. Then there exists a (unique) triple $(h: \mathcal H \to \mathcal G_1 \times_G \mathcal G_2, \lambda_1, \lambda_2)$ such that $\lambda_1: \alpha \to p_1 \circ h$ and $\lambda_2: \beta \to p_2 \circ h$ are isomorphisms of functors, and such that the diagram
\[
\xymatrix{
f \circ \alpha \ar[r]^{f(\lambda_1)} \ar[d]_\gamma & f \circ p_1 \circ h \ar[d]^{\Sigma \circ h} \\
g \circ \beta \ar[r]^{g(\lambda_2)} & g \circ p_2 \circ h 
}
\]
commutes.

Now let $\CC$ be a site, and let
\[
\xymatrix{
F_1 \ar[dr]_d && F_2 \ar[dl]^c \\
 & F_3
}
\]
be categories fibered in groupoids over $\CC$. Suppose further that $\mathcal G$ is a category fibered in groupoids over $\CC$ and that we have functors $\alpha:\mathcal G \to F_1$ and $\beta:\mathcal G \to F_2$ and an isomorphism $\gamma:c \circ \alpha \to d \circ \beta$. 

The triple $(\alpha,\beta,\gamma)$ is equivalent with an object in
\[
\HOM_\CC(\mathcal G, F_1) \times_{\HOM_\CC(\mathcal G, F_3)} \HOM_\CC(\mathcal G, F_2).
\]
Such a triple defines in turn a for any category fibered in groupoids $ H$ a morphism of groupoids
\begin{align*}
\ast: \HOM_\CC(H,G) &\to \HOM_\CC(H,F_1) \times_{\HOM_\CC(H,F_3)} \HOM_\CC(H,F_2) \\
(h:H \to \mathcal G) &\mapsto (\alpha \circ h, \beta \circ h, \gamma \circ h)
\end{align*}

\begin{prop}
\begin{enumerate}
\item There exists a quadruple $(G,\alpha,\beta,\gamma)$ such that for all $H$ fibered in groupoids, the map $\ast$ above is an isomorphism.
\item If $(G^{\prime}, \alpha^\prime, \beta^\prime, \gamma^\prime)$ is as above, then there exists a triple consisting of $(F,u,v)$ where $F:G \to G^\prime$ is an equivalence of fibered categories and $u:\alpha \to \alpha^\prime \circ f$ and $v:\beta \to \beta^\prime \circ f$ are isomorphisms of fibered categories. In addition, the diagram
\[
\xymatrix{
c \circ \alpha \ar[r]^{c \circ u} \ar[d]_\gamma &  c \circ \alpha^\prime \circ F \ar[d]^{\gamma^\prime} \\
d \circ \beta \ar[r]^{d \circ v} & d \circ \beta^\prime \circ F
}
\]
commutes. Further, if $(F^\prime,u^\prime,v^\prime)$ is another such triple, there exists a unique isomorphism $\sigma:F^\prime \to F$ such that
\[
\xymatrix{
\alpha \ar[r]^{u^\prime} \ar[dr]_u & \alpha^\prime \circ F^\prime \ar[d]^\sigma & \beta \ar[r]^{v^\prime} \ar[dr]_{v} & \beta^\prime \circ F^\prime \ar[d]^\sigma \\
& \alpha^\prime \circ F && \beta^\prime \circ F
}
\]
\end{enumerate}
\end{prop}
We will write $F_1 \times_{F_3} F_2$ for the fibered category in part 1 of the proposition. We skip the proof.


\subsection{Descent theory}

A \emph{stack} is morally a sheaf of categories fibered in groupoids over $\CC$. What do we mean by a sheaf? The answer lies in \emph{descent theory} á la Grothendieck. The situation is that we want some properties to be ``local'' in the some ``topology'', but for this to make sense, we have to prove that they are local, which is the point of descent theory.

Let $\CC$ be a category with fiber products and let $p:F \to \CC$ be a fibered category, and choose a cleavage\footnote{we say $p$ is \emph{cloven}.}.

We will define a \emph{category of descent data}. We will say (see definitions below), that a morphism $X \to Y$ is of \emph{effective descent} if giving the fibered category on ``open sets'' in a covering plus some glueing data satisfying a cocycle condition is equivalent to giving the fibered category on the space being covered (here $Y$).

For a morphism $X \to Y$ in $\CC$, we define the  $F(X \to Y)$: The objects are pairs $(E,\sigma)$ with $E \in F(X)$ and $\sigma:\pr_1^\ast E \xrightarrow{\simeq} \pr_2^\ast E$ an isomorphism in $F(X \times_Y X)$ such that in $F(X \times_Y X \times_Y X)$ we have a commutative diagram
\[
\xymatrix{
 & \pr_{13}^\ast \pr_1^\ast E \ar@{=}[r] \ar[dl]_{\pr_{13}^\ast \circ \sigma} & \pr_{12}^\ast \pr_1^\ast E \ar[dr]^{\pr_{12}^\ast \circ \sigma} & \\
\pr_{13}^\ast \pr_2^\ast E \ar@{=}[dr]  & &  & \pr_{12} ^\ast \pr_2 ^ \ast E \ar@{=}[dl]  \\
 & \pr_{23}^\ast \pr_2^\ast E & \ar[l]^{\pr_{23}^\ast \circ \sigma} \pr_{23} ^\ast \pr_1 ^\ast E & & 
} 
\]
,which we call the \emph{cocycle condition}. The equality signs are the canonical isomorphisms coming from the choice of cleavage. Remember to think of the fiber fiber products as intersections.

A morphism between pairs $(E^\prime, \sigma^\prime)$ and $(E,\sigma)$ consists of a morphism $g:E^\prime \to E$ in $F(X)$ such that
\[
\xymatrix{
\pr_1^\ast E^\prime \ar[r]^{\pr_1^\ast \circ g} \ar[d]^{\simeq}_{\sigma^\prime} & \pr_1^\ast E \ar[d]^\sigma_\simeq \\
\pr_2^\ast E^\prime \ar[r]_{\pr_2^\ast \circ g} & \pr_2^\ast E
}
\]
The isomorphism $\sigma$ is called \emph{descent data} for $E$.

\begin{remark}
Notice that the cocycle condition is a technical way of writing 
\[
(\pr_{|23}^\ast \circ \sigma) \circ (\pr_{12}^\ast \circ \sigma) = (\pr_{13}^\ast \circ \sigma).
\]
Or better:
\[
\sigma_{|23} \circ \sigma_{|12} = \sigma_{|13}.
\]
Which is just the usual cocycle condition.
\end{remark}

We have a natural functor
\begin{equation}
\label{eq:epsilon}
\epsilon:F(Y) \to F(X \xrightarrow{f} Y)
\end{equation}
For $E_0 \in F(Y)$, choose two pullbacks $\pr_1^\ast f^\ast E_0$ and $\pr_2 ^\ast f^\ast E_0$ along the projection $X \times_Y X \to Y$. Then, because of the chosen cleavage, there is a unique isomorphism $\sigma_{can}:\pr_1^\ast f^\ast E_0 \to \pr_2 ^\ast f^\ast E_0$. Then we define $\epsilon$ on objects by sending $E_0$ to the pair $(f^\ast E_0, \sigma_{can})$. 

It is left to the reader to check that the pair $(E,\sigma_{can})$ actually satisfy the cocycle condition.

\begin{remark} One can similarly define the descent category $F(\{ X_i \to Y\})$ for a collection $\{X_i \to Y\}_{i \in I}$. This is done completely analogously. However, they are very often equivalent, as the following lemma shows:
\end{remark}

\begin{lemma}
Suppose that coproducts exist in $\CC$ and that the natural functor $F(\coprod_i X_i) \to \prod
_i F(X_i)$ is an equivalence of categories. Let $\{X_i \to Y\}$ be morphisms in $\CC$. Let $Q = \coprod X_i$. Then the map
\[ \epsilon: F(Y) \to F(\{ X_i \to Y\}) \]
is an equivalence if and only if
\[ \epsilon: F(Y) \to F(Q \to Y) \]
is an equivalence.
\end{lemma}
\begin{proof}
Exercise. (do this later!!)
\end{proof}

\begin{defi}
For an object $(E,\sigma) \in F(X \to Y)$, we say that the descent data $\sigma$ is \emph{effective} if $(E,\sigma)$ is in the essential image of $\epsilon$.
\end{defi}

\begin{defi}
We say that a morphism $X \to Y$ is of \emph{effective descent} for $F$ if $\epsilon:F(Y) \to F(X \to Y)$ is an equivalence of categories.
\end{defi}

Algebraic geometers like to prove descent theorems. In one case it is easy, and this if the map $X \to Y$ has a section. So let $p:F \to \CC$ be a fibered category and $X \xrightarrow{f} Y$ a morphism in $\CC$.

\begin{prop}
If there is a section $s:Y \to X$, then $f$ is an effective descent morphism.\footnote{Recall: a section $s:Y \to X$ of $f:X \to Y$ is a morphism such that $f \circ s = \id_X$}
\end{prop}
\begin{proof}
We must show that the functor
\[
\epsilon:F(Y) \to F(X \to Y)
\]
is an equivalence of categories. There are three things to see. That it is faithful, that it is full, and that it is essentially surjective.

Let $\eta:F(X \to Y) \to F(Y)$ be defined by sending an object $(E,\sigma)$ to the pullback $s^\ast E$. 

Showing faithfullness is easy. One sees that the composition $\eta \circ \epsilon(E) = s^\ast f^\ast E$, but this is canonically isomorphic to $(fs)^\ast E = \id_X^\ast E \approx E$. So $\epsilon$ has an essential left inverse, hence is faithful.

[[Remains to show fullness and essential surjectivity.]]
\end{proof}


%%%%%%%%%%%%%%%%%%%%%%%
%%%%%%%%%%%%%%%%%%%%%%%
\pagebreak
\section{Descent theorems}
\label{sec:descent}
In this lecture we will prove various descent theorems, which roughly guarantee that many of the objects we work with in the future actually will be stacks.

\subsection{Descent for sheaves on a site}

Let $\CC$ be a site. For $X \in \CC$, we have an associated topos $\widetilde{\CC/X}$, the category of sheaves on $X$.

We define a new fibered category over a site $\CC$, and call it $\Sh$. Its objects are pairs $(X,E)$, where $X \in \CC$ and $E \in \widetilde{\CC/X}$.

A morphism $(X,E) \to (Y,F)$ is given by a pair $(f,\epsilon)$ where $f:X \to Y$ is a morphism in $\CC$ and $\epsilon: E \to f^\ast F$ is a morphism in $\widetilde{\CC/X}$. 

Notice that if we have a composition
\[
(X,E) \xrightarrow{(f,\epsilon)} (Y,F) \xrightarrow{(g,\rho)} (Z,G)
\]
we have that the composition is defined as the pair $(g \circ f):X \to Z$ and the $f^\ast \rho \circ \epsilon$, i.e. by the diagram
\[
E \xrightarrow{\epsilon} f^\ast F \xrightarrow{f^\ast \rho} f^\ast g^\ast G \overset{!}{=} (g \circ f)^\ast G.
\]

Let $p:\Sh \to \CC$ be defined by $(X,E) \mapsto X$.
\begin{exc}
This give us a fibered category with fiber over $X \in \CC$ given by  $\Sh(X)=\widetilde{\CC/X}$.
\end{exc}
\begin{sol}
Easy. 
\end{sol}

Now we are able to state the main theorem of this section. It is also time to recall the definition of a morphism of topoi, i.e. Definition \ref{def:toposmorphism}.

\begin{thm}
\label{thm:desc}
Any covering in $\CC$ is an effective descent morphism for $\Sh$. More precisely, the functor $\epsilon:\Sh(Y) \to \Sh(X \to Y)$ is an equivalence of categories.
\end{thm} 

\begin{proof}
Hopefully to be understood and TeXed some day. The proof was long and technical.
\end{proof}

The theorem has a nice corollary, which -- according to the Østvær -- is difficult to prove my other means:

\begin{corr}
Let $X,Y$ be $S$-schemes. Let $S^\prime \to S$ be a fppf-cover, and put $S^\dprime = S^\prime \times_S S^\prime$, $X^\prime = X \times_S S^\prime$ and $X^\dprime = X \times_S \times S^{\dprime}$. And also $Y^\prime = Y \times_S S^\prime$ and $Y^\dprime = Y \times_S S^\dprime$. We have two projections $\pr_1, \pr_2 : S^\dprime \to S^\prime$.

Suppose $f^\prime : X^\prime \to Y^\prime$ is an $S^\prime$-morphism such that $\pr_1^\ast f^\prime = pr_2^\ast f^\prime : X^\dprime \to Y^\dprime$. Then $f^\prime$ is induced by a \emph{unique} morphism $f:X \to Y$.
\end{corr}
\begin{proof}
We have fppf-sheaves $h_X, h_Y$ induced by $X,Y$. Let $\CC$ be the category of $S$-schemes with the fppf-topology. 

A morphism $f^\prime$ such that $\pr_1^\ast f^\prime = \pr_2^\ast f^\prime$ is equivalent to a morphism 
\[
(h_{X^\prime}, \sigma_{can}) \to (h_{Y^\prime}, \sigma_{can})
\]
in $\Sh(S^\prime \to S)$. But the theorem says that such a morphism is induced by a unique morphism $h_X \to h_Y$. Now apply the Yoneda lemma.
\end{proof} 


\subsection{Descent for quasi-coherent sheaves}

Let $S$ be a scheme and let $\CC$ be the fppf site of $S$. That is, the objects of $\CC$ are $S$-maps and the coverings are jointly surjective morphisms $\{U_i \to U\}$ that are fppf.

Then the sheaf $\OO$ is a presheaf of rings on $\CC$. It assigns for every $S$-scheme $T$ the set of global sections $\Gamma(T, \OO_T)$. This is a sheaf because it is represented by a scheme $\Aa_S^1$ (and all schemes are sheaves, by Theorem \ref{thm:hxsheaf}). 

Now let $\Qcoh S$ be the category of quasi-coherent sheaves on $S$. We will show (among other things) that the category of quasi-coherent sheaves in the Zariski-topology is equivalent to the category of quasi-coherent sheaves in the fppf topology.

Let $F \in \Qcoh S$. This gives us a presheaf $F_{big}$ of $\OO$-modules on $\CC$ (recall that this is just a functor $\CC^{op} \to \Set$\footnote{Well, not really $\Set$, but $\Mod {}$.}). It sends an $S$-scheme $T \xrightarrow{f} S$ to the set $\Gamma(T,f^\ast F) = f^\ast F(T)$. 

\begin{lemma}
The presheaf $F_{big}$ is a fppf sheaf.
\end{lemma}
\begin{proof}
By Lemma \ref{lemma:sheaf_fppf} we need to check two things: 1) That $F_{big}$ is a Zariski sheaf. This is clear by childhood knowledge. And 2), that the familiar equalizer is true for all fppf maps $V \to U$ of affine schemes. 

So let $\Spec B \to \Spec A$ be fppf. Let $M$ be the $A$-module corresponding to pullback of $F$ to $\Spec A$. Then the sheaf condition on $F_{big}$ is equivalent to the exactness of the diagram
\[
\xymatrix{
0 \ar[r] &  M \ar[r] & M \otimes_A B \ar@<1ex>[r] \ar@<-1ex>[r] & M \otimes_A (B \otimes_A B).
}
\]
But this was shown in Lemma \ref{lemma:mb}.
\end{proof}

Now let $G$ be a sheaf of $\OO$-modules on $\CC$. Thist just means that for every scheme $T$, the set $G(T)$ is an $\OO(T)$-module. 

Given a scheme map $T \to S$ we get an element $G_T := G(T) \in \Qcoh T$ by applying $T$ to $G$. In particular, for $F \in \Qcoh S$ we get an isomorphism $F \simeq (F_{big})_{S}$. 

In fact we have an adjunction:
\begin{lemma}
Restriction induces an isomorphism:
$$
\Hom_{\OO}(F_{big}, G) \xrightarrow{\simeq} \Hom_{\OO_S}(F, G_S)
$$
So ``biggifying'' is left adjoint to ``localizing''.\footnote{Anyone have a better name for the operation $?_{big}$?}
\end{lemma}
\begin{proof}
Exercise with hint. We can assume that $S$ is affine. Write $F$ as a cokernel
\[
F_2 \to F_1 \to F \to 0,
\]
where the $F_i$'s are direct sums of copies of $\OO_S$. Claim: $F \mapsto F_{big}$ is right-exact (this is probably because tensoring is right-exact). So we get a commutative diagram:

$$
\xymatrix{
0 \ar[r] & \Hom_{\OO}(F_{big}, G) \ar[r] \ar[d] & \Hom_{\OO}(F_{1,big}, G) \ar[d] \ar[r] & \Hom_{\OO}(F_{2,big}, G) \ar[d] \\ 
0 \ar[r] & \Hom_{\OO_S}(F,G_S) \ar[r] & \Hom_{\OO_S}(F_1,G_S) \ar[r] & \Hom_{{\OO}_S}(F_2,G_s)
}
$$

The vertical arrows are restriction maps (i.e. applying $S$ to the sheaves $F_{big},G$, etc). If the two right arrows are isomorphisms, then the left arrow is also. So we can assume that $F$ is a direct sum of copies of $\OO_S$, but if the statement is true for a direct sum, it is also true if $F=\OO_S$. So we assume $F=\OO_S$. 

So we are reduced to proving that there is a bijection 
\[
\Hom_{\OO}(\OO,G) \xrightarrow{\simeq} \Hom_{\OO_S}(\OO_S,G_S).
\]
The right hand side is equal to the set $\Gamma(S,G_S)$ of global sections of $G_S$. The left hand side is equal to $\varprojlim \Gamma(X,G_X)$. But $\CC$ has a terminal object, namely the identity morphism $S \to S$, so the left hand side is in fact also equal to $\Gamma(S,G_S)$.
\end{proof}

\begin{defi}
A \emph{big quasi-coherent sheaf} on $S$ is a sheaf of $\OO$-modules $F$ on $\CC$ such that:
\begin{enumerate}
\item For all $S$-schemes $T \in \SchS S$, the sheaf $F_T$ on $T_{Zar}$ is quasi-coherent.
\item For any morphism $T^\prime \xrightarrow{g} T$ in $\CC$, the morphism $g^\ast F_T \to F_{T^\prime}$ is an isomorphism.
\end{enumerate}
\end{defi}
\begin{remark}
The map in 2) is the familiar pullback of quasi-coherent map. 
\end{remark}

This gives us a category $\Qcoh {S_{fppf}}$ of big fppf sheaves on $S$. This is in contrast to the situation in Hartshorne, where we worked in the category $\Qcoh {S_{Zar}}$ of quasi-coherent sheaves on the \emph{small} Zariski site of $S$.

Our discussions above showed that if $F \in \Qcoh {S_{Zar}}$, then $F_{big} \in \Qcoh{S_{fppf}}$. By definition, we have an equivalence of categories:

\begin{lemma}
We have an equivalence of categories:
\begin{align*} 
\Qcoh{S_{Zar}} &\to \Qcoh{S_{fppf}} \\
F &\mapsto F_{big} \\
G_S &\text{ \reflectbox{$\mapsto$}} \, G
\end{align*}
\end{lemma} 
\begin{proof}
One direction follows since $F = (F_{big})_S$. The other direction follows by point 2) in the definition above, namely, that $(G_S)_{big} \simeq G$.
\end{proof}

Let us define a new category $\QCOH$: Its objects are pairs $(T,E)$ where $T$ is a scheme and $E$ is quasi-coherent sheaf in $\Qcoh{T_{Zar}}$. A morphism between two pairs $(T^\prime, E^\prime) \to (T,E)$ is a pair $(f,\epsilon)$ where $f:T^\prime \to T$ and $\epsilon: E^\prime \to f^\ast E$ is a morphism between the quasi-coherent sheaves.

We have an evident forgetful functor $p:\QCOH \to \Sch$. 
\begin{exc}[Exercise-Proposition]
This gives us a category $\QCOH$ fibered over schemes and the fiber over $T \in \Sch$, $\QCOH(T)$ is $\Qcoh{T_{Zar}}$. 
\end{exc}
\begin{sol}
This mounts down to saying that fiber products of quasicoherent sheaves exist. It does.
\end{sol}

\begin{thm}[Descent for quasi-coherent sheaves]
\label{desc_qcoh}
Let $X \xrightarrow{f} Y$ be a fppf cover of schemes. Then $f$ is an effective descent morphism for $\QCOH$. 
\end{thm}
\begin{proof}
If $f$ is quasi-compact and quasi-separated, then $f_\ast$ preserves quasi-coherence. In that case, the functor $\eta:\Sh(X \xrightarrow{f} Y) \to \Sh(Y)$ given by $(E, \sigma) \mapsto f_\ast E$ will be an inverse for $\epsilon:\QCOH(Y) \to \QCOH(X \xrightarrow{f} Y)$. This is because fppf implies that $f_\ast f^\ast $ is an isomorphism, i.e. Lemma \ref{lemma:pullpush}.

So we want to reduce to that case. So let $(E,\sigma) \in \QCOH(X \to Y)$. We first show that $(E,\sigma)$ is in the essential image of $\epsilon$. Since $f$ is fppf we can choose a Zariski cover $Y = \cup Y_i$ of $Y$ such that $f^{-1}(Y_i) = \cup_j X_{ij}$ where each $Y_i$ is affine and $X_{ij}$ is quasi-compact with $f(X_{ij})=Y_i$ for all $j$. Thus for all $i,j$ we have a quasi-separated and quasi-compact cover $X_{ij} \to Y_i$. So the restriction of $(E,\sigma)$ to $X_{ij}$ is induced by pullback from a unique object $F_{ij} \in \QCOH(Y_i)$.

For indices $i,j$ we have a commutative diagram
\[
\xymatrix{
 & \QCOH(Y_i) \ar@/^/[ddr] \ar@/_/[ldd]\ar[d] \\
 & \QCOH(X_{ij} \coprod X_{ij} \to Y_i) \ar[dr]_{res} \ar[dl]^{res} \\
\QCOH(X_{ij}  \to Y_i)  & & \QCOH(X_{ij} \to Y_i)
}
\]
There is a unique isomorphism $\sigma_{ijj^\prime}:F_{ij} \to F_{ij^\prime}$, so $F_{ij}$ is not dependent upon $j$, up to isomorphism. So we get an equivalence of categories
\[
\epsilon_i: \QCOH(Y_i) \xrightarrow{(\ast)} \QCOH(f^{-1}(Y_i) \to Y_i)
\]
Now, an element in $\QCOH(X \to Y)$ is, in this cover, equivalent to the following data: a pair $(\{(E_i,\sigma_i)\}, \{\alpha_{ii^\prime}\})$, where $(E_i, \sigma_i) \in \QCOH(f^{-1}(Y_i) \to Y_i)$ and for two indices $i,i^\prime$  \[\alpha_{ii^\prime}:(E_i, \sigma_i)_{|Y_i \cap Y_{i^\prime}} \xrightarrow{\simeq} (E_{i^\prime}, \sigma_{i^\prime})_{|Y_i \cap Y_{i^\prime}}\]
is an isomorphism in $\QCOH(f^{-1}(Y_i \cap Y_{i^\prime}) \to Y_i \cap Y_{i^\prime})$ that satisfies the cocycle condition for triples, but by the equivalence $(\ast)$ above, this is equivalent to the data of a pair $(\{E_i\}, \{\beta_{ii^\prime} \})$ where $E_i$ is a quasi-coherent sheaf on $Y_i$ and for indices $i,i^\prime$, we have an isomorphism in $\QCOH(Y_i \cap Y_{i^\prime})$ between $\beta_{ii^\prime}: E_i \xrightarrow{\simeq} E_{i^\prime}$. But by standard glueing of sheaves (see \cite{hartshorne}), we get a sheaf in $\QCOH(Y)$.
\end{proof}


\subsection{Descent for closed subschemes}

Let $X \xrightarrow{f} Y$ be a fppf cover. Let $\pr_i:X \times_Y X \to X$ for $i=1,2$ be the two projections. Let $Z \hookrightarrow X$ be a closed subscheme and let $\pr_i^\ast Z \hookrightarrow X \times_Y X$ be $Z \times_{X,\pr_i} (X \times_Y X)$.

Then:
\begin{prop}
We have a bijection between closed subschemes $W \hookrightarrow Y$ and closed subschemes $Z \hookrightarrow X$ such that $\pr_1^\ast Z = \pr_2^\ast Z$.

The bijection is given by sending $W$ to its inverse image $f^{-1}(W)$.
\end{prop} 
\begin{proof}
There is a 1-1 correspondance between closed subschemes and quasi-coherent ideal sheaves, \cite[Chapter II, §5]{hartshorne}. So it is enough to show that the pullback map of quasi-coherent sheaves induces a bijection between the set of quasi-coherent ideal sheaves $\subseteq \OO_Y$ and quasicoherent ideal sheaves $\mathcal J \subseteq \OO_X$ such that $\pr_1^\ast \mathcal J = \pr_2^\ast \mathcal J$. But this follows from descent for quasi-coherent sheaves, i.e. Theorem \ref{desc_qcoh}. 
\end{proof}

This has the consequence that the fibered category $F$ consisting of pairs $(Y,X)$ where $Y$ is a closed subscheme of $Y$ and the forgetful morphism is $(Y,X) \mapsto X$, is a stack. The fibers $F(X)$ are the set of all closed subschemes of $X$ (notice that this is a set, because there are no non-identity automorphisms of objects of $F(X)$. 

\subsection{Descent for open embeddings}

This turns out to be almost trivial. We don't need to apply any heavy theorems.

So let $\catname {Op}$ be the category with objects pairs $(X,U)$ where $X$ is a scheme and $U$ is an open subscheme of $X$. A morphism in $\catname {Op}$ is just a scheme map $X^\prime \xrightarrow{f} X$ such that $U^\prime \subseteq f^{-1}(U)$. 
 
\begin{exc}
The evident map $p:\catname{Op} \to \Sch$ gives us a fibered category.
\end{exc}
\begin{sol}
Note first that the fibers $\catname{Op}(X)$ are just the set of open subschemes of $X$. Let $f:X \to Y$ be a morphism of schemes, and let $U$ be an open subscheme of $Y$. Then the ordinary fiber product of schemes gives us a scheme $U \times_Y X$. But topologically this is just $f^{-1}(U)$, which is open (see Exercise 7.1B in \cite{ravi_vakil}).

So for every map in $\Sch$ we get a fiber product, and this is just the condition in Definition \ref{def:fibered}.
\end{sol}

Here's the statement:
\begin{prop}
Any fppf cover $S^\prime \xrightarrow{f} S$ is an effective descent morphism for $\catname {Op}$. 
\end{prop}

\begin{proof}
Let $S^{\prime\prime} = S^\prime \times_S S^\prime$. We must show that if $U^\prime \subseteq S^\prime$ is an open subset with $\pr_1^{-1}(U^\prime) = \pr_2^{-1}(U^\prime)$, in $S^\dprime$, then $U^{\prime} = f^{-1}(U)$ for a unique open $U \subseteq S$. 

We first observe that uniqueness is clear since $f$ is surjective. This is just the set-theoretic fact that $f$ surjective implies $f(f^{-1}(U))=U$ for all $U \subseteq S$.

So we prove existence. Since $f$ is fppf, it is an open map, so $U := f(U^\prime)$ is an open subset of $S$, so at least $U ^\prime \subseteq f^{-1}(U)$. Suppose now $f^{-1}(U) \bs U^\prime$. Then there is some $y \in U^\prime$ with $f(x)=f(y)$. But then the pair $(x,y) \in S^\dprime$ will be in $\pr_2^{-1}(U^\prime)$ but not in $\pr_1^{-1}(U^\prime)$. But this contradicts the hypothesis! So $U^\prime = f^{-1}(U)$.
\end{proof}
\begin{remark}
Note that the only property of $f:S^\prime \to S$ we used was that it was an open map. So we could have gotten away with assuming, for example, that $f$ was just flat and locally finitely presented.
\end{remark}



\subsection{Descent for affine morphisms}

We have a category $\Aff$: its objects are affine scheme maps $X \xrightarrow{f} Y$ and the morphisms are commutative diagrams
\[
\xymatrix{
X^\prime \ar[d]_{f^\prime} \ar[r] & X \ar[d]^f \\
Y^\prime \ar[r] & Y
}
\]
There is an obvious functor $p:\Aff \to \Sch$, given by sending a morphism $(X \to Y)$ to its target $Y$. 

\begin{prop}
Let $S^\prime \xrightarrow{s} S$ be a fppf cover. Then $s$ is of effective descent for $\Aff$.
\end{prop}
\begin{proof}[Sketch proof]
This follows from effective descent for modules and noting that tensor products of modules are preserved (where?).
\end{proof}

\subsection{Descent for quasiaffine morphisms}

Recall that a morphism $X \xrightarrow{f} Y$ is \emph{quasi-affine} if there exists a factorization
\[
\xymatrix{
X \ar[dr]_f \ar@{^(->}[r] & W \ar[d]^g \\
 & Y
}
\]
where the top arrow is an open embedding and $g$ is affine. We can then form the category $\catname {QAff}$ of quasi-affine morphisms. The morphisms are commutative squares of such. Again, it is easy to see that we get a fibered category $p:\catname{QAff} \to \Sch$.

\begin{prop}
Any fppf cover $S^\prime \to S$ is of effective descent for $\catname{QAff}$.
\end{prop}
\begin{proof}[Sketchy proof]

This follows from descent for open embeddings and for affine morphisms. 
\end{proof}

\subsection{Descent for polarized schemes}

Let $\catname {Pol}$ be the category consisting of pairs $(X \xrightarrow{f} Y, \LL)$ where $f$ is a proper morphism and $\LL$ is relatively ample invertible sheaf on $X$. This means that for all affine opens $V \subseteq Y$, the restriction $\LL_{|f^{-1}(V)}$ is ample on $X$. This means that for all coherent sheaves $\FF$ on $X$, there exists an integer $n_0 > 0$ such that for all $n \geq n_0$, the sheaf $\FF \otimes \LL^{\otimes n}$ is generated by global sections. For more on this, see \cite{hartshorne}.

The morphisms $(X^\prime \xrightarrow{f^\prime} Y^\prime, \LL^\prime) \to (X \xrightarrow{f} Y, \LL)$ in $\catname {Pol}$ are triples $(\epsilon, a,b)$ where $a,b$ are morphisms of schemes and $\epsilon$ is a sheaf isomorphism such that
\[
\xymatrix{
X^\prime \ar[r]^b \ar[d]_{f^\prime} & X \ar[d]^f \\
Y^\prime \ar[r]^a & Y
}
\]
and $\epsilon:b^\ast \LL \xrightarrow{\simeq} \LL^\prime$ is an isomorphism of sheaves. This way we get a fibered category $p:\catname {Pol} \to \Sch$ given by $(X \to Y, \LL) \mapsto Y$.

\begin{prop}
Any fppf cover $S^\prime \to S$ is of effective descent for $\catname {Pol}$.
\end{prop}
\begin{proof}
The proof was given as ``self study''.
\end{proof}

%%%%%%%%%%%%%%%
%%%%%%%%%%%%%%%
\pagebreak
\section{Example, torsors, principal homogeneous spaces}

We start with an example. Then we define torsors and principal homogeneous spaces, and end with another example. 


\subsection{The moduli stack of curves of genus $ g \geq 2$}
\label{subsec:mg}

We define a category $\MM_g$ that should parametrize families of curves of genus $g$. We also show that it is in fact a stack.

Its objects are maps $C \xrightarrow{f} S$ such that for each $s \in S$, the fiber $C_s$ is a geometrically connected, proper, smooth, curve of genus $g$.

The morphisms are just commutative diagrams
\[
\xymatrix{
C^\prime \ar[r] \ar[d] & C \ar[d] \\
S^\prime \ar[r] & S
}
\]
We get a functor $p:\MM_g \to \Sch$ for free: It just sends a morphism $C \to S$ to the codomian $S$. Clearly this makes $\MM_g$ into a fibered category. 

We have a morphism of fibered categories from $\MM_g$ to $\mathsf{Pol}$  (the category of polarized schemes) by sending $C \to S$ to the pair $(C \to S, \Omega_{C/S} ^1)$.

For a fppf cover $S^\prime \to S$ we have a commutative diagram comparing the descent data for $\MM_g$ and $\mathsf{Pol}$:
\[
\xymatrix{
\MM_g(S) \ar[r] \ar[d] & \mathsf{Pol}(S) \ar[d]^{\simeq} \\
\MM_g(S^\prime \to S) \ar[r] & \mathsf{Pol}(S^\prime \to S)
}
\]
The right arrow is an equivalence since every descent data for $\mathsf{Pol}$ is effective (proven last lecture). A diagram stare shows that every object in $\MM_g(S^\prime \to S)$ lies in the essential image of $\MM_g(S)$. 

Also, the map $\MM_g(S) \to \MM_g(S^\prime \to S)$ is fully faithful by the effective descent theorem of morphisms of sheaves on an arbitrary site, i.e. Theorem \ref{thm:desc}.  

\subsection{Torsors / principal homogeneous spaces}

Let $\CC$ be a site and $\mu$ a sheaf of groups on $\CC$. A $\mu$-torsor on $\CC$ is a sheaf $\mathcal P$ on $\CC$ together with a left action $\rho: \mu \times \mathcal P \to \mathcal P$ such that the following two conditions hold:
\begin{enumerate}
\item [T1.] For all $x \in \CC$ there exists a cover $\{ X_i \to X\}$ such that $\mathcal P(X_i) \neq \emptyset$ for all $i$.
\item [T2.] The map 
\begin{align*}
\mu \times \mathcal P &\to \mathcal P \times \mathcal P \\
(g,p) &\mapsto (p,gp)
\end{align*}
is an isomorphism.
\end{enumerate}

\begin{exc}
The condition T2 is equivalent with: If $\mathcal P(X) \neq \emptyset$, then the action of $\mu(X)$ on $\mathcal P(X)$ is transitive. 
\end{exc}
\begin{sol}
Let $x,y \in \mathcal P(X)$. Then by T2, the pair $(x,y) \in \mathcal P(X) \times \mathcal P(X)$ is is equivalent to the pair $(x,gx)$ for some $g \in \mu(X)$. But then $y=gx$, so $\mu(X)$ acts transitively.
\end{sol}

A morphism of $\mu$-torsors $(\mathcal P,p) \to (\mathcal P^\prime, p^\prime)$ is a morphism $f:\mathcal P \to \mathcal P^\prime$ such that 
\[
\xymatrix{
\mu \times \mathcal P \ar[r]^{\id_\mu \times f} \ar[d]^p & \mu \times \mathcal P^\prime \ar[d]^{p^\prime} \\
\mathcal P \ar[r]^f & \mathcal P^\prime
} 
\]

\begin{example}[Example of torsor]
This example is from linear algebra. Let $T:V \to W$ be a linear transformation and let $\vec w \in W$. Then the solution set of $T(\vec w)=\vec w$ is either empty or it is a torsor under the action of $\ker T$.
\end{example}

Now suppose that $\mu$ is represented by $G$, a flat finite type group scheme (over $X$). 

\begin{defi}
A \emph{principal $G$-bundle} over $X$ is a pair $(\pi:P \to X, \rho)$ where $\pi$ is a smooth surjective morphism of schemes, and $\rho:G\X_X P \to P$ is a map such that the following three conditions are true:
\begin{enumerate}
\item The following diagram commutes:
\[
\xymatrix{
G \X_X(G \X_X  P) \ar[rr]^{\id_G \times \rho} \ar[d]_{m_g \times \id_{ P}} && G \X_X  P \ar[d]^\rho \\
G \X_X  P \ar[rr]^\rho &&  P
}
\]
\item If $e:X \to G$ is the identity section, then
\[
\xymatrix{
 P \ar[rr]_{e \times \id_{ P}} \ar@/^2pc/[rrr]^{\id_{ P}} && G \times_X  P \ar[r]_e &  P
}
\]
commutes. 
\item
There is an isomorphism $p \times \pr_2: G \times_X  P \to  P \times_X  P$.
\end{enumerate}
A morphism $( P,\rho) \to ( P^\prime, \rho^\prime)$ of principal $G$-bundles consists of an $X$-morphism $f: P \to  P^\prime$ such that
\[
\xymatrix{
G \times_X  P \ar[r]^{\id_G \times f} \ar[d]_\rho & G \times_X  P \ar[d]^{\rho^\prime} \\
 P \ar[r]_f &  P^\prime 
}
\]
commutes.
\end{defi}

Now let $( P, \sigma)$  be a principal $G$-bundle. We define a $\mu$-torsor $(\mathcal P,\rho)$ as follows: \footnote{Recall that $\mu$ is represented by $G$.} The sheaf $\mathcal P$ is the sheaf represented by $P$, namely $\mathcal P = h_P$, with action $\rho$ induced by $\sigma$. 

Condition T2 of the definition of a $\mu$-torsor follows easily. It is the first condition that need some big theorems: Since $\pi: P \to X$ is smooth and surjective, there exists locally an étale section of $X$, by Corollary \ref{corr:homotopy}. This implies in turn that there exists an fppf cover $\{ X_i \to X\}$ with $\mathcal P(X_i) \neq \emptyset$ for all $i$. [[I don't see how...]]

This discussion gives us a fully faithful functor $\ast$ from principal $G$-bundles on $X$ to $\mu$-torsors on $X$, given by sending a pair $(P,\sigma)$ to the functors they represent: $(h_p,h_\sigma)$. 

\begin{prop}
If the structure map $G \to X$ is affine, then $\ast$ is an equivalence of categories.
\end{prop}
\begin{proof}[Proof sketch]
We need to show essential surjectivity (fully faithfullness is free by Yoneda). That is, if $(\mathcal P, \rho)$ is a torsor, we want to see that $\mathcal P$ is represented by a smooth $X$-scheme $P$.

Choose a fppf cover $\{ X_i \to X\}$ with $\mathcal P(X_i) \neq \emptyset$ for all $i$. Then the restriction of $\mathcal P$ to $X_i$ is represented by an affine $X_i$-scheme $P_i$, namely $G \X_X X_i$ (I don't see why).

We can glue this together to get a scheme $P$ by descent for affine morphisms.
\end{proof}

Here's a long example:
\begin{example}
Let $X$ be a scheme, and suppose that $n$ is invertible on $X$. This just means that $n$ is a unit in $\OO_X(X)^\ast$. Then we define the group scheme of roots of unity $\mu_n$ by $\mu_n(Y) = \{ f \in \OO_Y^\ast \, | \, f^n = 1\}$. We will consider the category of $\mu_n$-torsors on $X_{\acute et}$.

Let $\Sigma_n$ be the category defined as follows: Its objects are pairs $(\LL, \sigma)$, where $\LL$ is an invertible sheaf on $X$ and $\sigma: \LL^{\otimes n } \xrightarrow{\approx} \OO_X$ is a trivialization of the $n$th power of $\LL$.\footnote{Compare with the discussion of root stacks in the last section.} The morphisms $(\LL,\sigma) \to (\LL^\prime, \sigma^\prime)$ is a morphism of invertible sheaves $\rho:\LL \to \LL^\prime$ such that 
\[
\xymatrix{
\LL^{\otimes n} \ar[r]^\sigma \ar[d]_{\rho^{\otimes n}} & \OO_X \\
\LL^{\prime \otimes n} \ar[ur]_{\sigma^\prime}
} 
\]
Then we claim that we have an equivalence of categories $F:\Sigma_n \to \mathrm{Tors}(\mu_n)_{\acute e t}$. Given a pair $(\LL,\sigma) \in \Sigma_n$, let $\mathcal P_{(\LL,\sigma)}$ be the étale sheaf associated to the presheaf that maps a $X$-scheme $U$ to the set of trivializations $\lambda:\OO_U \to \LL_{|U}$ such that 
\[
\xymatrix{
\OO_U \ar[r]^{\lambda^{\otimes n}} \ar@/_2pc/[rr]_{\id_{\OO_U}} & \LL_{|U}^{\otimes n} \ar[r]^\sigma & \OO_U
}
\]
commutes. We have an action of $\mu_n(U)$ on $\mathcal P_{(\LL,\sigma)}(U)$ as follows: given $\zeta \in \mu_n(U)$, we send $(\zeta,\lambda)$ to $\zeta \lambda$, which makes sense because we have a $\OO_X$-module structure on $\LL$.

This makes $\mathcal P_{(\LL,\sigma)}$ into a $\mu_n$-torsor.

Now we see why we needed the étale topology: Given $(\LL,\sigma) \in \Sigma_n$, we have no guarantee that $\mathcal P_{(\LL,\sigma)}$ is non-empty in the Zariski-topology. However, we can always find such a $\lambda$ in the étale locally. [[Anyone sees why?]]

On the other hand, we have a natural functor $G:\mathrm{Tors}(\mu_n)_{\acute et}\to \Sigma_n$. Given $\mathcal P$ a $\mu_n$-torsor... (I don't understand my notes...)
\end{example}

Here's another example:
\begin{example}
Let $X = \Spec k$, where $k$ is a field. In this case all line bundles are trivial, hence $\Sigma_n$ can be identified with a category having objects $\sigma \in k^\ast$ and morphisms $\sigma \to \sigma^\prime$ are just given by a $\lambda \in k^\ast$ such that $\sigma^\prime = \lambda^n \sigma$. 

Then we get that isomorphism classes of $\sigma_n$-torsors on $\Spec k_{\acute et}$ are in bijection with $k^\ast / (k^\ast)^n$, the $n$th roots of $k$.
\end{example}






%%%%%%%%%%%%%%%%
%%%%%%%%%%%%%%%%
\pagebreak
\section{Stacks and algebraic spaces} 

We are finally able to define a stack. Furthermore, we define algebraic spaces and \emph{algebraic stacks}.

\subsection{Definition of stacks}

We have finally arrived at the definition of stacks. Let $\CC$ be a site.

\begin{defi}
A category fibered in groupoids $p:F \to \CC$ is a \emph{stack} if for all $X \in \CC$ and coverings $\{ X_i \to X\}_{i \in I}$, the functor
\begin{align*}
\epsilon:F(X) &\to F(\{X_i \to X\}) \\
E &\mapsto (E, \sigma_{can})
\end{align*}
is an equivalence of categories.
\end{defi}

So, a stack is a fibered category in which the fibers can be defined locally, i.e. by coverings.

\begin{lemma}
A category fibered in groupoids $p:F \to \CC$ is a stack if and only if the following two conditions hold:
\begin{enumerate}
\item For all $X \in \CC$ and $x,y \in F(X)$, the presheaf $\Isom(x,y)$ on $\CC/X$ is a sheaf.
\item Any  covering $\{ X_i \to X\}$ is of effective descent.
\end{enumerate}
\end{lemma}
\begin{proof}
Condition 1) says that the functor $\epsilon$ in the defintion is fully faithful, and condition 2) says that it is essentially surjective.
\end{proof} 

Fibered categories only satisfying condition 1) are called \emph{prestacks}. 

\begin{lemma}
Let 
\[
\xymatrix{
 & F_1 \ar[d]^c \\
F_2 \ar[r]_d & F_3
}
\]
be a diagram of stacks fibered in groupoids over $\CC$. Then the fiber product $F_1 \times_{F_3} F_2$ is also a stack.
\end{lemma}
\begin{proof}[Proof hint] 
For any covering $\{ X_i \to X\}$ in $\CC$, we have maps
\[
(F_1 \X_{F_3} F_2)(X) \to F_1(X) \X_{F_3(X)} F_2(X)
\]
and
\[
(F_1 \X_{F_3} F_2)(\{X_i \to X\}) \to F_1(\{X_i \to X\}) \times_{F_3(\{X_i \to X\})} F_2(\{ X_i \to X\}).
\]
Both of these are equivalences of groupoids. 
\end{proof}

Given a category $p:F \to \CC$ fibered in groupoids, we can ``stackify''.

\begin{thm}[Stackification]
Let $p:F \to \CC$ be a category fibered in groupoids. Then there exists a stack $F^a$ over $\CC$ and a morphism of fibered categories $q:F \to F^a$ such that for all stacks $G$ over $\CC$, the induced functor
\[
\HOM_\CC(F^a, G) \to \HOM_\CC(F,G)
\]
is an equivalence of categories.
\end{thm}
\begin{proof}[Proof sketch]

There are two main steps. In Step 1 one constructs $F \to F^\prime$ that is universal among morphism to prestacks.

In Step 2 one constructs a morphism $F^\prime \to F^a$ that is universal for morphisms to stacks. 

The construction is essentially set-theoretical and consists of a lot of book-keeping.
\end{proof}

\subsection{Algebraic spaces}

Let $S$ be a base-scheme and give $\SchS{S}$ the étale topology (i.e. we are working in the big étale site of $S$), and let $F \xrightarrow{f} G$ be a morphism of sheaves.

\begin{defi}
We say that $f$ is \emph{schematic} or \emph{represented by schemes} if for all $T \in \SchS S$ and morphisms $T \to G$, the fiber product $F \times_G T$ is a scheme.
\end{defi}

\begin{defi}
Let $P$ be a stable property of morphisms of schemes, that is, it is stable under pullbacks. If $f$ is represented by schemes, we say that $f$ has property $P$ if for all $S$-schemes $T$, the morphism
\[
\pr_2: F \times_G T \to T
\]
have the property $P$.
\end{defi}

\begin{exc}
If $F,G$ are representable sheaves, then any morphism $f:F \to G$ is represented by schemes.
\end{exc}
\begin{sol}
Let $F=h_X$ and $G=h_Y$, and let $f:F \to G$ be a morphism. By the weak Yoneda lemma, this corresponds to a unique morphism $f:X \to Y$. Since the fiber product of schemes over schemes is a scheme, of course $F \times_G T$ is a scheme. (this exercise was so trivial it wasn't even fun)
\end{sol}
\begin{exc}Let $X \xrightarrow{f} Y$ be a map of schemes and let $P$ be some stable property. Then $f$ have the property $P$ if and only if $h_X \to h_Y$ have property $P$.
\end{exc}
\begin{sol}
This is equally trivial.
\end{sol}

\begin{lemma}
Suppose $\Delta:F \to F \times F$ is represented by schemes. Then any morphism $f:T \to F$ where $T$ is an $S$-scheme is represented by schemes. 
\end{lemma}
\begin{proof}
Let $T,T^\prime$ be $S$-schemes, and consider the diagrams below:
\[
\xymatrix{
\circ & T^\prime \ar[d]^g & \circ & T \times T^\prime \ar[d]^{f \times g} \\
T \ar[r]^f & F & F \ar[r]^\Delta & F \times F
}
\]
The fiber products in these two diagrams are isomorphic. To see this, just write down the corresponding maps with elements.
\end{proof}

Now we come to the definition of algebraic spaces:
\begin{defi}
An \emph{algebraic space} over $S$ is a functor $X:\SchS S^{op} \to \Set$ satisfying the following three conditions:
\begin{enumerate}
\item $X$ is a sheaf (in the étale topology).
\item The diagonal mapping $\Delta:X \to X \times X$ is represented by schemes.
\item There exists an $S$-scheme $U$ and a surjective étale morphism $U \to X$, called an \emph{atlas} of $X$.
\end{enumerate}
\end{defi}

\begin{remark}
In lieu of the last lemma, the third condition means the following: for any $S$-scheme $T$ and morphism $T \to X$ the morphism $U \times_X T \to T$ is étale and surjective. So in the way schemes are covered by affine schemes in the Zariski topology, algebraic spaces are covered by schemes in the étale topology.
\end{remark}

\subsection{Algebraic spaces as sheaf quotients}

We're still working in the étale topology.

\begin{defi}
Let $X \in \SchS S$. An étale equivalence relation on $X$ is a subscheme $R \hookrightarrow X \times_S X$ such that:
\begin{enumerate}
\item For all $S$-schemes $T$, the inclusion of sets
\[
R(T) \subset X(T) \times_{S(T)} X(T)
\]
is an equivalence relation (reflexive, symmetric, transitive).
\item The maps $s,t:R \to X$ induced by the projections are étale maps.
\end{enumerate}
\end{defi}

By taking quotiens of $R$ we get a presheaf $\SchS S ^{op} \to \Set$ given by sending an $S$-scheme $T$ to the set-theoretic quotient $X(T)/R(T)=X(T)/\sim$. We can sheafify to get a sheaf $X/R$ on the site $(\SchS S)_{\acute et}$.

\begin{prop}

The construction above makes $X/R$ into an algebraic space.  

Conversely, if $Y$ is an algebraic space over $S$ and $X \to Y$ is an étale surjection where $X$ is a scheme, then $R:= X \times_Y X$ is a scheme, and the inclusion $R \hookrightarrow X \times_S X$ is an étale equivalence relation. In addition $X/R \to Y$ is an isomorphism.
\end{prop}
\begin{proof}
We will only say a little bit about the proof.

For the first part, let $Y=X/R$. The hard part is showing that the diagonal $\Delta$ is representable, so we assume this has already been shown. We will show that there exists a surjective étale atlas for $Y$. It is enough to consider morphisms $T \to Y$ that factor through $X$ (why?). Given such a factorization, we get a commutative diagram
\[
\xymatrix{
T \X_Y X \ar[r] \ar[d] & R \ar[d]^t \ar[r]^s & X \ar[d] \\
T \ar[r] & X \ar[r] & Y
}
\]
Since $t$ by assumption is surjective étale, by base change, $T \times_Y X \to T$ is also surjective étale.

Finally, we say something about $R = X \times_Y X$: It follows from the commutative diagram
\[
\xymatrix{
R \ar[r] \ar[d] \pullbackcorner & X \times_S X \ar[d] \\
Y \ar[r]^\Delta & Y \times_S Y
}
\]
that $R$ is a scheme, since $Y$ is an algebraic space.
\end{proof}

\subsection{Examples of algebraic spaces}

\begin{example}
Let $X$ be a scheme and let $G$ be a finite group acting on $X$ via $\rho:G \times X \to X$. We say that the action is \emph{free} if the map $G \times X \to X \times X$ given by $(g,x) \mapsto (x,\rho(g,x))$ is injective.

Let $X/G$ be the sheaf associated to the presheaf $T \mapsto X(T)/G$. Then $X/G$ is an algebraic space.
\end{example}

Here is an example of an algebraic space that is not a scheme:

\begin{example}
This example comes from Hironaka, and is described in Appendix B of \cite{hartshorne}. We start with $\PP^3$ with coordinates $x_0,x_1,x_2,x_3$. And we define two curves by
\begin{align*}
C_1 &: x_0x_1+x_1x_2+x_2x_0=x_3=0 \\
C_2 &: x_0x_1+x_1x_3+x_3x_0=x_2= 0
\end{align*}
They intersect in two points, namely $p_1=(1:0:0:0)$ and $p_2=(0:1:0:0)$. Let $i=0,1$. On $\PP^3 \bs \{p_i\}$,  blow up $C_i$ and then blow up $C_{1-i}$. We get two blowups which we can glue along the inverse image of $\PP^3 \bs \{p_1,p_2\}$. Call the resulting scheme for $Z$.

There is an involution $\sigma:Z \to Z$ induced by $x_0 \mapsto x_1$, $x_1 \mapsto x_0$ and $x_2 \mapsto x_3$ and $x_3 \mapsto x_2$. The automorphism $\sigma$ have fix points, so we restrict to the open subscheme $Z^\prime \subset Z$ where $\sigma$ acts freely. Here we make the quotient $Z^\prime/\sigma$. By the above result, this is an algebraic space, but it is not a scheme. The proof uses intersection theory.
\end{example}

\subsection{More on algebraic spaces}

Let $P$ be a property of schemes stable in the étale topology. That is, such that $V \in \SchS S$ have $P$ if and only if for every étale covering $\{ V_i \to V\}$ all the $V_i$'s have $P$. 

\begin{defi}
We say that an algebraic space $X$ have a property $P$ if there is an étale surjection $U \to X$ such that $U$ have $P$.
\end{defi}

We say that a property $P$ of morphisms of schemes is stable in the étale topology if for any covering $\{ V_i \to V\}$, a morphism $U \to V$ have $P$ if and only if $U \times_V V_i \to V_i$ have $P$ for all $i$.

\begin{defi}
Suppose $P$ is a property of morphisms of schemes that is stable in the étale topology. Let $X \xrightarrow{f} Y$ be a morphism of algebraic spaces represented in schemes. Then we say that $f$ has property $P$ if there is an étale surjection $V \to Y$ such that $V \times_X Y \to V$ have property $P$.
\end{defi}

In particular, we can talk about open and closed embeddings of algebraic spaces.

\begin{defi}
We say that an algebraic space $X$ is \emph{quasi-separated} if $\Delta:X \to X \times X$ is quasi-compact.
\end{defi}

We say that a property $P$ of morphisms of schemes is stable and local on the domain if for any scheme map $U \xrightarrow{f} V$ and covering $\{ U_i \xrightarrow{\varphi_i} U\}$, the morphism $f$ have $P$ if and only if $f \circ \varphi_i$ have $P$ for all $i$.

\begin{defi}
Let $P$ be a propery of morphisms of schemes that is stable and local on the domain in the étale topology. Let $X \xrightarrow{f} Y$ be a morphism of algebraic spaces. Then we say that $f$ has $P$ if there are étale surjections $v: V \to Y$ and $u:U \to X$ such that the projection $U \times_Y V \to V$ have $P$.
\end{defi}

So we can talk about morphisms of algebraic spaces that are étale, flat, smooth, surjective, etc.

\begin{exc}
Let $f:X \to Y$ be a representable morphism of algebraic spaces and $P$ a property that is local on the source. If $f$ have $P$, then for any étale surjection $V \to Y$, the morphism $X \times_Y V \to V$ have $P$.
\end{exc}
\begin{sol}
\end{sol}

\begin{exc}
If $P$ is a property that is stable and local on the source, and $X \xrightarrow{f} Y$ is a morphism of algebraic spaces that have $P$, then if we have a commutative diagram,
\[
\xymatrix{
U \ar@{->>}[rr]^u && X \X_Y V \ar[r] \ar[d] & V \ar[d] \\
& & X \ar[r]^f & Y
}
\]
where $u$ is an étale surjection, then the composition $U \to V$ will have property $P$.
\end{exc}
\begin{sol}
\end{sol}

\begin{prop}
The full subcategory of ${\Sch/ S}_{\acute et}$ consisting of algebraic spaces is closed under finite limits.
\end{prop}
\begin{proof}
It is enough to show that it is closed under taking fiber products.

[[Long proof]]
\end{proof}

A priori, an algebraic space $X$ is only a sheaf in the étale topology. The next lemma shows that it is a sheaf in the fppf topology. 

\begin{lemma}
Let $X$ be an algebraic space over $S$ with quasi-compact diagonal $\Delta_X$. Then $X$ is a sheaf in the fppf topology.
\end{lemma}

\begin{exc}
Let $\CC$ be a site and let $X,R$ be sheaves on $\CC$. Let $s\times t: R \hookrightarrow X \times X$ be an inlusion such that for all $U \in \CC$, the inclusion $R(U) \subseteq X(U) \times X(U)$ is an equivalence relation. Let $X/R$ be the presheaf on $\CC$ associated to the presheaf $U \mapsto X(U)/R(U)$. Then the diagram
\[
\xymatrix{
R \ar[r]^s \ar[d]_t & X \ar[d] \\
X \ar[r] & X/R
}
\]
is cartesian.
\end{exc}
\begin{sol}
Easy exercise.
\end{sol}

The next exercise will be used in a later lecture:
\begin{exc}
\label{ex:star}
Let $Y$ be an algebraic space over $S$. Let $F$ be a sheaf on $\SchS S$ in the étale topology. Let $g$ be a morphism $g:F \to Y$. Show that if there exists an étale surjection $U \to Y$ with $U$ a scheme such that $F \times_Y U$ is an algebraic space, then $F$ is an algebraic space.
\end{exc}
\begin{sol}
\end{sol}


 

%%%%%%%%%%%%%%%%%%%%%%
%%%%%%%%%%%%%%%%
\pagebreak
\section{Algebraic stacks}

Now we have defined both stacks and algebraic spaces. Algebraic stacks are something in between.

\subsection{First results on algebraic stacks}

Recall, that from now an, all stacks will be over $\SchS S$ in the étale topology.

\begin{defi}
We say that a morphism of stacks $f:\mathfrak X \to \mathfrak Y$ is representable if for all  $S$-schemes $U$ and morphisms  $y:U \to \mathfrak Y$, the fiber product $\mathfrak X \times_{\mathfrak Y}  U$ is an algebraic space.\footnote{The gothic letter $\mathfrak Y$ is math-speak for ``Y''.}
\end{defi}

\begin{lemma}
If $\mathfrak X \xrightarrow{f} \mathfrak Y$ is a morphism of stacks, then for any algebraic space $V$ and morphism $y:V \to \mathfrak Y$, the fiber product $\mathfrak X \times_{\mathfrak Y} V$ is an algebraic space.
\end{lemma}
\begin{proof}
This should follow from Exercise \ref{ex:star}.
\end{proof}

\begin{defi}[Algebraic stack]
\label{def:algstack}
A stack $\mathfrak X$ over $S$ is \emph{algebraic} if 
\begin{enumerate}
\item The diagonal $\Delta:\mathfrak X \to \mathfrak X \times_S \mathfrak X$ is representable and
\item there exists a smooth surjective morphism $\pi:X \to \mathfrak X$ where $X$ is a scheme.
\end{enumerate}
\end{defi}

An algebraic stack is often called an \emph{Artin stack}.

\begin{lemma}
\label{lemma11.1}
Let $\XX/S$ be a stack. Then the diagonal $\Delta$ is representable if and only if for all $S$-schemes $U$ and objects $u_1,u_2 \in \XX(U)$, the sheaf $\Isom(u_1,u_2)$ on $\SchS U$ is an algebraic space.

Furthermore, this happens if and only if for all algebraic spaces $X$ and maps $x,y: X \to \XX$, the sheaf $\Isom(x,y)$ is an algebraic space.
\end{lemma}
\begin{proof}
Only a hint was given: The following diagram is cartesian.
\[
\xymatrix{
\Isom(u_1, u_2) \ar[r] \ar[d] & U \ar[d]^{u_1 \times u_2} \\
\XX \ar[r]^\Delta & \XX \X \XX
}
\]
\end{proof}

\begin{lemma}
Suppose $\XX / S$ is an Artin stack. For all diagrams
\[
\xymatrix{
 & X \ar[d]^x \\
Y \ar[r]_y & \XX
}
\]
where $X,Y$ are algebraic spaces, the fiber product $X \times_\XX Y$ is an algebraic space.

In particular, every morphism $X \to \XX$ where $X$ is an algebraic space is representable.
\end{lemma}
\begin{proof}[``Proof'']
Observe that $X \times_{\mathfrak X} Y \simeq \Isom(\pr_1^\ast x, \pr_2^\ast y)$ over $X \times_S Y$.
\end{proof}

Here's a lengthy example:

\begin{example}
Let $X$ be an algebraic space. Let $G/S$ be a smooth group scheme acting on $X$. We can define a stack $[X/G]$ as follows: The objects are triples $(T,P,\pi)$, where:
\begin{enumerate}
\item $T$ is a scheme over $S$. 
\item $P$ is a $G_T=G \times_S T$-torsor on the big étale site of $T$.
\item $\pi$ is a morphism $\pi:P \to X \X_S T=X_T$ that is a $G_T$-equivariant morphism of sheaves on $\SchS T$.
\end{enumerate}
The morphisms $(T^\prime, P^\prime, \pi^\prime) \to (T,P,\pi)$ are pairs $(f,f^b)$ where $f:T^\prime \to T$ is a map of $S$-schemes and $f^b:P^\prime \xrightarrow{\simeq} f^\ast P$ is an isomorphism of $G_T$-torsors on $\SchS T$ such that the diagram
\[
\xymatrix{
P^\prime \ar[r]^{f^\ast} \ar[dr]^{\pi} & \ar[d]^{f^\ast \pi} f^\ast P^\prime \ar@{.>}[r] & P \ar@{.>}[d]^\pi \\
 & X \X_S T^\prime  \ar@{.>}[r]_{\id \X f}  & X_T 
}
\]
commutes. The dotted arrows automatically commute.

Now descent for sheaves on sites shows that $[X/G]$ is a stack. It is actually an Artin stack:  We first have to check if $\Delta$ is representable. Let $T \in \SchS S$ and let $(P_i,\pi_i)$ be objects in $[X/G]$ over $T$ (we're omitting the $T$ from the notation here since we're looking over the fibers anyway).

To show that $\II:= \Isom((P_1,\pi_1),(P_2,\pi_2))$ is an algebraic space, we can replace $T$ by an étale cover (this was an exercise from last time). Thus we can assume that $P_1$ and $P_2$ are trivial torsors. Fix isomorphisms $\sigma_i:P_i \to G_T=G \X_S T$ such that  we can identify the $\pi$ in the definition with a map $\pi_i: G_T \to X_T$. Now the sheaf $\II$ can be identified with the sheaf that to $T^\prime$ over $T$ associates the set of elements $g \in G(T^\prime)$ such that the diagram
\[
\xymatrix{
G_{T^\prime} \ar[dr]_{\pi_1} \ar[r]^{m_g} & G_{T^\prime} \ar[d]^{\pi_2} \\
& X_T
}
\]
commutes (the upper arrow is just multiplication by $g$). Now, since we are dealing with group schemes, commutativity is equivalent with $\pi_1(e)=\pi_2(g)$, where $e \in G_T(T)$ is the identity section.

Thus we can identify $\II$ with the fiber product of the diagram:
\[
\xymatrix{
\II \ar[r] \ar[d]  & G_T \ar[d] \\
X_T \ar[r]^\Delta & X_T \X_T X_T
}
\]
This implies that $\II$ is a scheme (and in particular, an algebraic space), by representability of the diagonal morphism, and it follows by Lemma \ref{lemma11.1} that the diagonal morphism of $[X/G]$ is representable. 
\end{example}

\begin{exc}
 Find a smooth surjection onto $[X/G]$.
\end{exc}
\begin{sol}

\end{sol}

\begin{defi}
The \emph{classifying stack} of $G$ is the stack quotient $[S/G]$, and is denoted by $BG$.
\end{defi}
\begin{remark}
If we had come so far as to define dimension of stacks, we would have seen that the dimension of $[S/G]$ is $\dim S - \dim G$, so we see that arbitrary large \emph{negative} dimensions of stacks occur (!!).
\end{remark}

\begin{lemma}
Artin stacks have fiber products. That is, if $\XX_1,\XX_2, \XX$ are Artin stacks, then the fiber product $\XX_1 \times_\XX \XX_2$ is also an Artin stack.
\end{lemma}
The proof is not difficult but very long, so we skip it.

\begin{defi}[Inertia stacks]
Let $\XX/S$ be an Artin stack. The \emph{inertia stack}
 (norwegian: \emph{treghetsstacken}) $\II_X$ of $X$ is the fiber product $\XX \times_{\XX \times_S \XX} \XX$.
\end{defi}

Explicitly, the objects are pairs $(x,g)$ where $x \in \XX(T)$ for some scheme $T$ and $g$ is an automorphism of $x \in \XX(T)$. The morphisms $(x^\prime,g^\prime) \to (x,g)$  consists of a morphism $f:x^\prime \to x$ in $\XX$ such that 
\[
\xymatrix{
x^\prime \ar[r]^f \ar[d]^{g^\prime}_\simeq & x \ar[d]^g_\simeq \\
x^\prime \ar[r]_f & x
}
\]
commutes. The map $\II_X \to \XX$ sends $(x,g) \to x$. 

\begin{example}
Let $\XX=[X/G]$. Then $\II_X$ is the stabilizer of $G$ acting on $X$. 
\end{example}

\subsection{Properties of Artin stacks}

Let $P$ be a property of $S$-schemes that is stable in the smooth topology.

\begin{defi}
We say that an Artin stack $\XX/S$ has property $P$ if there is a surjective smooth map $\pi: X \to \XX$ where $X$ is a scheme that have the property $P$.
\end{defi}

\begin{lemma}
Let $P$ be a property as above and $\XX/S$ an Artin stack with $P$. Then for any morphism $y:Y \to \XX$ where $Y$ is an algebraic space, $Y$ also have $P$.
\end{lemma}
\begin{proof}
The proof was given as an ``exercise with hint''. Consider the following diagram:
\[
\xymatrix{
Y \times_{\mathfrak X} X \ar[r] \ar[d] & X \ar[d] ^\pi \\
Y \ar[r]_y & X
}
\]
It follows (why?) that the left vertical arrow is also smooth and surjective.
\end{proof}

Let $\XX$ and $\mathfrak Y$ be maps of Artin stacks. A \emph{map\footnote{Not ``map'' as in ``function'', but ``map'' as in ``atlas''.} for $f$} is a commutative diagram
\begin{equation}
\label{eq:map}
\xymatrix{
X \ar@/^1pc/[rr]^h \ar[r]^g \ar[dr]_q  & \XX^\prime \ar[d]^{p ^\prime}  \ar[r] & Y \ar[d]^p \\
 & \XX \ar[r]^f & \mathfrak Y
} 
\end{equation}
where $g$ and $p$ are smooth og surjective and $X$ and $Y$ are algebraic spaces.

If $X,Y$ are schemes, we say that it is a \emph{map of schemes}.\footnote{This can't be right? It sounds terrible.}

\begin{defi}
Let $P$ be a property of morphisms between schemes that is stable and local on the source in the smooth topology (for example smooth maps, locally of finite presentation, etc.). We say that a morphism $f:\XX \to \mathfrak Y$ has the property $P$ if there is a map for $f$ of schemes where $h$ has the property $P$.
\end{defi}

Having a property $P$ is independent of the map:
\begin{prop}
Let $P$ be as above. Then the map $\XX \xrightarrow{f} \mathfrak Y$ have $P$ if and only if for all maps for $f$, $h$ has $P$. 
\end{prop}
\begin{proof}
Fix one map as in Equation \ref{eq:map}, and let $Y^\prime \to Y$ be a smooth surjection of algebraic spaces. We get a commutative diagram
\[
\xymatrix{
X \times_Y Y^\prime \ar@/^2ex/[rr]^{h^\prime} \ar[r]_{g^\prime} \ar[d] & \mathfrak X \X_{\mathfrak Y} Y^\prime \ar[r] \ar[d] & Y^\prime \ar[d] \\
X \ar[r]^g & \mathfrak X^{\prime} \ar[r] \ar[d] & Y \pushoutcorner  \ar[d] \\
 & \mathfrak X \ar[r]^f & \mathfrak Y \pushoutcorner 
}
\]
Consider the ``outer'' diagram:
\[
\xymatrix{
X \times_Y Y^\prime \ar[r] & \mathfrak X \times_{\mathfrak Y} Y^\prime \ar[r] \ar[d] & Y^\prime \ar[d] \\
& \mathfrak X \ar[r] & \mathfrak Y
}
\]
This is also a map for $f$. Thus we see that the morphism $h$ in Equation \ref{eq:map} have property $P$ if and only if $f^\prime: X \times_Y Y^\prime \to Y^\prime$ have $P$ (the projection $X \times_Y Y^\prime \to X$ is smooth and surjective). 

Now fix two maps ($i=1,2$):
\[
\xymatrix{
X_i \ar@/^1pc/[rr]^{h_i} \ar[r]^{g_i} \ar[dr]_q  & \XX^\prime \ar[d]^{{p_i}^\prime}  \ar[r] & Y_i \ar[d]^{p_i} \\
 & \XX \ar[r]^f & \mathfrak Y
} 
\]
We want to show that $h_1$ have $P$ if and only if $h_2$ have $P$. Let $Y^\prime = Y_1 \times_{\mathfrak Y} Y_2$. By the above discussion, and by considering the maps $p_j: Y^\prime \to Y_j$ for $j=1,2$, one sees that it is enough to consider the case where $Y_1=Y_2$ and $p_1=p_2$. So write $\mathfrak X^\prime$ for the fiber product $\mathfrak X \times_{\mathfrak Y} Y_1 = \mathfrak X \X_{\mathfrak Y} Y_2$. Then we have a commutative diagram, where $X:= X_1 \times_{\mathfrak X} X_2$:
\[
\xymatrix{
X \ar[r]^{\pr_2} \ar[dr]_{\pr_1} & X_2 \ar@/^2ex/[rr]^{h_2} \ar[r]_{g_2} & \mathfrak X^\prime \ar[dd] \ar[r] & Y \ar[dd] \\
& X_1 \ar[ru]_{g_1} \ar@/_2ex/[rru]_{h_1}  \\
& & \mathfrak X \ar[r] & \mathfrak Y 
}
\]
Now since $P$ is local on the source in the smooth topology, the morphism $h_1$ will have $P$ if and only if $h_1 \circ \pr_1:X \to Y$ have $P$, but this happens if and only $h_2 \circ \pr_2$ have $P$ and this happens if and only if $h_2$ have $P$.
\end{proof}

\begin{defi}
Let $P$ be a property of morphisms of algebraic spaces that is stable in the smooth topology on the category of algebraic spaces over $S$.

We say that a representable morphism of Artin stacks $f: \XX \to \mathfrak Y$ have $P$ if for alle morphisms $Y \to \mathfrak Y$ where $Y$ is an algebraic space, the morphism $\XX \times_{\mathfrak Y} Y \to Y$ have $P$
\end{defi}


Thus it makes sense to say that representable morphisms are étale, smooth, separated, proper, etc.

\begin{defi}
An Artin Stack is a \emph{Deligne-Mumford stack} if there is an representable étale surjection $X \to \XX$ where $X$ is a scheme.
\end{defi}
We abbreviate ``Deligne-Mumford stack'' by just saying ``DM stack''. 


Recall that a morphism of scheme $g:Z \to W$ is \emph{formally unramified} if for all closed embedding $S_0 \hookrightarrow S$ of affine schemes defined by nilpotent ideals, the map
\[
Z(S) \to W(S) \times_{W(S_0)} Z(S_0)
\]
is an injection, and this in turn is equivalent to $\Omega_{Z/W}^1 = 0$. Also, being formally unramified is stable and local on the source in both the étale and the smooth topology. Thus it makes sense to say that maps between Artin stacks are formally unramified. 

\begin{thm}
Let $\XX/S$ be an Artin stack. Then $\XX/S$ is DM if and only if the diagonal $\Delta: \XX \to \XX \times_S \XX$ is formally unramified.
\end{thm}

This is a big theorem and we won't prove it. In some sense it says this: $\XX$ is DM if and only if none of the objects of $\XX$ allow infinetesimal automorphisms. We will try to explain what this means in the rest of the lecture.

Here's a fact: The diagonal $\Delta$ is formally unramified if and only if for all algebraically closed field $k$ and objects $x \in \XX(k)$, the automorphism group $\Aut_x$ is a finite, reduced $k$-group scheme. 

It fits into a cartesian diagram (? why??)
\[
\xymatrix{
\Aut_x \ar[r] \ar[d] & \Spec k \ar[d] \\
\XX \ar[r]^\Delta & \XX \times \XX
}
\]
[[ the lecturer goes on to prove the converse but I don't understand it ]]

\subsection{The moduli stack of curves of genus $\geq 2$}
\label{subsec:mg2}

Recall the fibered category $\MM_g$ from Subsection \ref{subsec:mg}. 

\begin{thm}
The stack $\MM_g$ is a Deligne-Mumford stack.
\end{thm}
This is a hard theorem and proven by Deligne and Mumford in the 70's. 

\begin{lemma}
Let $(S, C \xrightarrow{f} S)$ be an object in $\MM_g$ and $\LL_{C/S}$ the invertible sheaf ${\Omega^1_{C/S}}^{\otimes 3}$. Then:
\begin{enumerate}
\item The pullback $f_\ast \LL_{C/S}$ is a locally free sheaf of rank $5g-5$ on $S$. 
\item The map (``the counit of adjunction'') $f^\ast f_\ast \LL_{C/S} \to \LL_{C/S}$ is surjective, and the associated map $C \to \PP(f_\ast \LL_{C/S})$ is a closed embedding. 
\item For any morphism $g^\prime: S^\prime \to S$, the map 
\[
g^\ast f_\ast \LL_{C/S} \to f_\ast^\prime g^\prime  \LL_{C/S} 
\]
is an isomorphism. 
\end{enumerate}
\end{lemma}
\begin{proof}
For 1) and 2) we can assume that $S=\Spec k$ and that $k= \bar k$, by flat base change \cite[Chapter III, §12, Thm 12.11]{hartshorne}. We claim that $\dim_k H^0(C, \LL_{C/S}) = 5g-5$ and that $H^i(C, \LL_{C/S}) = 0$ for all $i> 0$. The last statement follows by Serre duality (I don't see how?). For the first part, note that the degree of the canonical divisor on $C$ is $2g-2$ by Example 1.3.3 in Chapter V in Harthorne. Hence $\dim_k H^0(C, \LL_{C/S}) = 2 \deg(\Omega_{C/S}^1) + 1 -g = 3(2g-2)+1-g=5g-5$. 

For 2), somehow use that $k = \bar k$ and that $\LL_{C/S}$ is very ample.
\end{proof}

Let $\overset{\sim}{M_g}$ be the following functor on $\Sch$: For each $S$-scheme it associates the set of isomorphism classes of pairs $(C \xrightarrow{f} S, \sigma:\OO_S^{5g-5} \xrightarrow{\simeq} f_\ast \LL_{C/S})$. What is an isomorphism of such pairs? Let $(C^\prime \xrightarrow{f^\prime} S^\prime, \sigma^\prime:\OO_S^{5g-5} \to f_\ast^\prime \LL_{C^\prime/S})$ be another such pair. Then an isomorphism is an isomorpism of curves $\alpha:C^\prime \to C$ such that the following diagram commutes:
\[
\xymatrix{
\OO_S^{5g-5} \ar[dr]^{\sigma} \ar[r]^{\sigma^\prime} & f_\ast^\prime \LL_{C^\prime/S} \ar[d] \\
& f_\ast \LL_{C/S}
} 
\]

We have that $G:=\GL_{5g-5}$ acts on $\overset{\sim}{M_g}$: On an $S$-point it works like this: $g \cdot (C \to S, \sigma) = (C \to S, \sigma \circ g)$ for $g \in G(S)$.

We have a map $\pi:\overset{\sim}{M_g} \to \MM_g$ defined by sending a pair $(C/S, \sigma)$ to the pair $(S,C)$. 

It turns out that we have an isomorphism $[\overset{\sim}{M_g}/\GL_{5g-5}] \simeq \MM_g$. In particular, $\MM_g$ is an Artin stack!

To show that the diagonal $\Delta:\MM_g \to \MM_g \times \MM_g$ is formally unramified, we use the isomorphism. It is enough to show that for any algebraically closed field $k$ and smooth genus $g$ curve $C/k$, the $k$-group scheme $\Aut_k(C)$ is reduced. The proof uses that $g \geq 2$.



%%%%%%%%%%%%%%%
%%%%%%%%%%%%%%%
\pagebreak
\section{Root stacks and the Weil conjecures}

In this last lecture we give another example of a stack, and in the last part we prove state the Weil conjectures and define a Weil cohomology theory.
\subsection{Root stacks}

Let $X$ be a scheme. Recall that a \emph{Cartier divisor} on $X$ is given by global sections $f_i \in \Gamma(U_i, \mathcal K^\ast)$ for a (Zariski) covering $\{U_i\}$, such that $f_i/f_j \in \Gamma(U_i \cap U_j, \mathcal O^\ast)$, where $\mathcal K$ is the \emph{sheaf of total quotient rings on $\OO$}. A Cartier divisor is \emph{effective} if already all the $f_i$'s are global sections of $\Gamma(U_i, \mathcal O^\ast)$. Stating things more compactly, giving a Cartier divisor is the same as giving a global section of $\mathcal K^\ast / \OO^\ast$.

Note that effective Cartier divisors are in 1-1 correspondence with locally principal closed subschemes of $X$.

Let $D$ be a given effective Cartier-divisor on $X$ and fix a natural number $n$. Then one can ask: does there exist an effective Cartier-divisor $E$ on $X$ such that $D = nE$?\footnote{ Recall that given a Cartier divisor $D$, one can form an associated invertible sheaf $\mathcal L(D)$, and we have a group operation by $E + E^\prime \leftrightarrow \mathcal L(E) \otimes \mathcal L(E^\prime)$.}

More generally, one can ask: does there exists a scheme $Y$ and a map $f:Y \to X$ and a Cartier divisor $E$ on $Y$ such that $n E = f^\ast D$? The problem is that divisors doesn't always pull back! For example, for curves, divisors only pull back for finite maps! (see \cite{hartshorne}, Chapter II, Proposition 6.8).

So, as one does in mathematics, when something doesn't work, we generalize!

\begin{defi}
A \emph{generalized effective Cartier-divisor} (\emph{geCd}) on $X$ is a couple $(\mathcal L, \rho)$ where $\mathcal L$ is an invertible sheaf on $X$ and $\rho$ is a morphism $\mathcal L \to \mathcal O_X$ of $\OO_X$-modules. A morphism of couples $(\mathcal L^\prime, \rho^\prime) \to (\mathcal L, \rho)$ is a commutative diagram:
\[
\xymatrix{
\LL^\prime \ar[rr]  \ar[dr]_{\rho^\prime} && \LL \ar[dl]^\rho \\
& \OO_X
}
\]
\end{defi}

Here are some examples.
\begin{example}
Let $D \subseteq X$ be a locally principal subscheme of $X$ and let $\mathcal I_D$ be the associated ideal sheaf. Then the inclusion $i:\mathcal I_D \hookrightarrow \OO_X$ makes $(\mathcal I_D,i)$ a geCd.
\end{example}

\begin{example}
Any invertible sheaf $\mathcal L$ give rise to a geCd given by $(\mathcal L, 0)$, where $0$ is the zero map $0:\mathcal L \to \OO_X$.
\end{example}

We can multiply geCd's:
\begin{defi}
Given two geCd's, $(\mathcal L, \rho)$ and $(\mathcal L^\prime, \rho^\prime)$, we define their product to be 
\[
(\mathcal L, \rho) \cdot (\mathcal L^\prime, \rho^\prime) := (\mathcal L \otimes \mathcal L^\prime, \rho \otimes \rho^\prime).
\]
Here $\rho \otimes \rho^\prime$ is the morphism $\LL \otimes \LL^\prime \to \OO_X \otimes \OO_X \approx \OO_X$.
\end{defi}
But behold! Now we can pull back divisors! For let $g:Y \to X$ be some scheme map, and let $(\LL,\rho)$ be a geCd on $X$. This gives \[(g^\ast \LL, g^\ast \rho:g^\ast \LL \to g^\ast \OO_X = \OO_Y),\]
a geCd on $Y$.

We want to define a fibered category $\DD$. Its objects are pairs $(T,(\LL,\rho))$ where $T$ is a scheme and $(\LL, \rho)$ is a geCd. A morphism in $\DD$ consists of a scheme map $g:T \to T^\prime$ and an isomorphism of geCd's $(\LL^\prime, \rho^\prime) \xrightarrow{\approx} (g^\ast \LL, g^\ast \rho)$.

Via the forgetful functor $p:\DD \to \Sch$ given by $(T,(\LL,\rho)) \mapsto T$, we get a category over $\Sch$, where the fibers over $X$ are the category of geCd's on $X$. If we work in the étale topology (or the fppf topology), then it follows that $\DD$ is a stack by descent for invertible sheaves. 

\begin{prop}
We have an isomorphism of stacks
\[
\DD \approx [\Aa^1 / \Gr_m ] 
\]
In particular, $\DD$ is an algebraic stack.
\end{prop}
\begin{proof}[Sketch of proof]

We consider the prestack $\{ \Aa^1 / \Gr_m \}$ whose objects are pairs $(T, f \in \Gamma(T, \OO_T))$, where $T$ is a scheme, and a morphism to $(T^\prime, f^\prime \in \Gamma(T^\prime, \OO_{T^\prime}))$ is given by a pair $(g,u)$ where $g:T^\prime \to T$ og $u \in \Gamma(T^\prime, \OO_{T^\prime}^\ast)$ such that $f^\prime = u \cdot g^\#(f)$ in $\Gamma(T^\prime, \OO_{T^\prime})$.\footnote{Recall that $\Aa^1$ represents the functor $T \mapsto \Gamma(T, \OO_T)$ and $\Gr_m$ represents the functor $T \mapsto \Gamma(T, \OO_T^\ast)$}

We have a morphism of fibered categories $\{ \Aa^1 / \Gr_m \} \to \DD$ given by sending an object $(T,f) \to (T, (\OO_T, \cdot f))$ and a morphism $(g,u)$ to the morphism $(T^\prime,(\OO_{T^\prime},\cdot f^\prime)) \to (T, (\OO_T, \cdot f))$ given by multiplication by $f$.

One can show that this induces an equivalence of stacks after stackifying (but I don't have a reference for this). 
\end{proof}

Now we consider the morphism $\Aa^1 \to \Aa^1$ given by $t\ mapsto t^n$ and the morphism $\Gr_m \to \Gr_m$ given by $u \mapsto u^n$. This induces an endomorphism $\rho_n:[\Aa^1 / \Gr_m] \circlearrowleft$. This endomorphism corresponds to taking $n$th powers of sheaves:
\[
(T,(\LL,\rho)) \mapsto (T, (\LL^{\otimes n}, \rho^{\otimes n})).
\]

Now fix a geCd $(\LL,\rho)$ on $X$ and a natural number $n$. Let $\mathfrak X_n$ be the fibered category over $\Sch$ with objects triples $(T \xrightarrow{f} Xk,(M,\lambda),\sigma)$ where $(M,\lambda)$ is a geCd on $T$ and $\sigma:(M^{\otimes n}, \lambda) \xrightarrow{\approx} (f^\ast \mathcal L, f^\ast \rho)$ is an isomorphism of invertible sheaves.

A morphism $(T^\prime \xrightarrow{f^\prime} X^\prime, (M^\prime, \lambda^\prime), \sigma^\prime) \to (T \xrightarrow{f} X, (M,\lambda), \sigma)$ is given by a pair $(h,h^b)$ where $h:T^\prime \to T$ is an $X$-morphism, and $h^b:(M^\prime, \lambda^\prime \xrightarrow{\approx} (h^\ast M, h^\ast \lambda)$ is an isomorphism of geCd's. In addition, we demand that the diagram
\[
\xymatrix{
M^{\prime, \otimes n} \ar[rr]^{h^{b \otimes n}} \ar[dr]_{\lambda^\prime} && h^\ast M^{\otimes n} \ar[dl] ^{h^\ast \lambda} \\ 
 & f^{\prime \ast} \LL \approx h^\ast f^\ast \LL
}
\]
commutes.
\begin{defi}
The \emph{ $n$th root stack of $(\LL,\rho)$} is the stack $\mathfrak X_n$ defined above.
\end{defi}

That $\mathfrak X_n$ actually is a stack follows from the next theorem:
\begin{thm}
The following holds:
\begin{enumerate}
\item $\mathfrak X_n$ is an algebraic stack.
\item If $\LL = \OO_X$ and $\rho$ is given by $ f \in \Gamma(X, \OO_X)$, then $\mathfrak X_n$ is isomorphic with the quotient stack of $\SSpec_X(\OO_X[T]/(T^n-f))$ by the action of $\mu_n$ given by $\zeta \cdot T = \zeta T$.\footnote{$\mu_n$ is the group of unit roots.}
\item If $n$ is invertible on $X$, then $\mathfrak X_n$ is a DM-sheaf.
\end{enumerate}
\end{thm}

\begin{proof}
We only prove number 1). This follows by noting that $\mathfrak X_n$ is the fiber product of the diagram
\[
\xymatrix{
\mathfrak X_n \pullbackcorner  \ar[r] \ar[d] & \DD \ar[d]^{\rho_n} \\
\DD \ar[r]^{(\LL,\rho)} & \DD
}
\]
The other parts are harder.
\end{proof}

\subsection{The Weil conjectures}

All that was said in this lecture can be found in the Appendix of \cite{hartshorne}, except the definition of a Weil cohomology theory, which we state here for completeness.

A \emph{Weil cohomology theory} for schemes that are reduces and of finite type over a field $k$ with coefficients in a field $K$ with $char K =0$ is a contravariant functor from such schemes to graded commutative $K$-algebras. We write
\[
X \mapsto H^\ast(X) = \oplus H^i(X).
\]
This graded $K$-algebra should also have a cup product.

We also have a \emph{trace map} $\mathrm{Tr}_X:H^{2\dim X}(X) \to K$, and for every closed subscheme $Z \hookrightarrow X$ a \emph{class} $c(Z) \in H^{2c}(X)$, where $c= \mathrm{codim } Z$. This should satisfy the following list of axioms:
\begin{enumerate}[A]
\item[A1. ] \emph{Finite-dimensionality}. Every $H^i(X)$ should be a finite-dimensional $K$-vector space. And furthermore, $H^i(X)=0$ for $i < 0$ and $i > 2 \dim X$. 
\item[A2. ] \emph{Künneth formula}. We have an isomorphism
\begin{align*}
H^\ast (X) \otimes_K H^\ast(Y)  &\to H^\ast(X \times Y) \\
\alpha \otimes \beta &\mapsto \pr_X^\ast \alpha \cup \pr_Y^\ast \beta
\end{align*}
\item[A3. ] \emph{Poincaré-duality}. The trace map $\mathrm{Tr}_X$ is an isomorphism, and for $0 \leq i \leq 2 \dim X$, the $K$-linear map
\begin{align*}
H^i(X) \otimes_K H^{2\dim X-i} &\to K \\
\alpha \otimes \beta &\mapsto \alpha \cup \beta
\end{align*}
is a perfect pairing of vector spaces.

\item[A4. ] \emph{Multiplicativity of the trace map}. For $\alpha \in H^{2 \dim X}(X)$ and $\beta \in H^{2 \dim Y}(Y)$, we have
\[
\mathrm{Tr}_{X \times Y}(\pr_X^\ast \alpha \cup \pr_Y^\ast \beta) = \mathrm{Tr}_X(\alpha) \mathrm{Tr}_Y(\beta).
\]

\item[A5. ] \emph{Yoneda product}. For closed subschemes $Z \hookrightarrow X$ and $W \hookrightarrow Y$, we have $c(Z \times W) = \pr_X^\ast (c(Z)) \cup \pr_Y^\ast(c(W))$. 

\item[A6. ] \emph{Pushforward}. Given a map $f:X \to Y$, there is map $f_\ast:H^\ast(X) \to H^\ast(Y)$.

\item[A7. ] \emph{Pullback}. Given a map $f:X \to Y$, there is a map $f^\ast:H^\ast(Y) \to H^\ast(X)$.

\item[A8. ] \emph{Normalization}. For $X = \Spec k$, we have $c(X)=1$ and $\mathrm{Tr}_X(1)=1$.
\end{enumerate}

For more on Weil cohomology theories, the reader is adviced to do a Google search.

We concluded the course with two theorems:
\begin{thm}[Lefschetz' trace formula]
Let $\varphi: X \to X$ be smooth, projective of dimension $N$. Then
\[
\Delta \cdot \Gamma_\varphi = \sum (-1)^n \mathrm{Tr}(\varphi^\ast |H^n(X))
\]
If $\Delta$ and $\Gamma_\varphi$ intersects transversely (meaning that any irreducible component of $\Delta \cap \Gamma_\varphi$ have codimension $\codim \Delta + \codim \Gamma_\varphi$), then
\[
\Delta \cdot \Gamma_\varphi = \# \{ x \in X | \varphi(x) = x \}.
\]
\end{thm}

\appendix

\pagebreak 
\section{Exam} 

Here are the five topics to be talked about on the exam:
\begin{itemize}
\item \emph{Moduli problems}. See Section 1. Examples includes the Grassmannian (subsection \ref{subsec:grassmannian} and onwards), genus $g$ curves (subsection \ref{subsec:mg} and subsection \ref{subsec:mg2}.
\item \emph{Vektor bundles}. See Section 1, subsection \ref{subsec:vb}. Also see the book by Frank Neumann \cite{neumann_stacks}, especially the first chapter.
\item \emph{Faithfully flat descent}. This is in Section \ref{sec:descent}. 
\item \emph{Fibered categories}. This is Section \ref{sec:fibered}. Also see examples.
\item \emph{The moduli stack of curves of genus $\geq 2$, $\MM_g$.} Subsections \ref{subsec:mg} and \ref{subsec:mg2}.
\end{itemize}

\section{References to the litterature}

Stacks are still so obscure that there are no canonical literature unless one wants to read the french EGA/SGA.

Much of the problem with studying stacks is that they are so abstract that intuition and motivation for the concepts are hard to grasp.

There are several good articles and sections in books that tries to explain and give motivation for the concepts introduced here. For a really good one, see ``Picard groups of moduli problems'' \cite{mumford_picard} by David Mumford, in which he motivates the introduction of other topologies, and uses them to compute the Picard group of $\MM_g$.

In Hartshorne \cite{hartshorne_def}, Remark 27.7.1, there are two pages explaining why stacks are needed.

There is also the note by Barbara Fantechi \cite{fantechi_stacks}, ``Stacks for Everybody'', using concrete examples. A more technical, but more motivating article is ``Algebraic stacks'' by Tomás Gómez \cite{gomez_stacks}.

Of course there's also the ``Stacks Project'' \cite{stacks-project}, but unless you're already an expert in the field, most of the stuff is unreadable. 

\subsection{Background litterature}

For background in algebraic geometry, there is of course \cite{hartshorne}, but also sometimes \cite{hartshorne_def}. Milne has a lot of stuff on properties of étale and flat maps, and also a lot on sheaves, see \cite{milne_etale}. 

The last chapter of \cite{eisenbud_harris} has a lot to say about the functorial point of view in algebraic geometry. 

\bibliographystyle{alpha}
\bibliography{bibliografi} 

\end{document}  
   
